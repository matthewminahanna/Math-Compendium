Much like how substitution is the inverse of the chain rule, we have an inverse of the product rule. 

\begin{align*}
    \dv{x} \left[ f(x) g(x) \right] = &f'(x)g(x)+ f(x)g'(x)\\
     \dv{x} \left[ f(x) g(x) \right]-f(x)g'(x) = &f'(x)g(x)\\
     \int \dv{x} \left[ f(x) g(x) \right] \dd{x} - \int f(x)g'(x) \dd{x} =& \int f'(x)g(x) \dd{x}
\end{align*}

This gives us 

\begin{equation}
    \int f'(x)g(x) \dd{x} = f(x)g(x)-  \int f(x)g'(x) \dd{x} \label{eqn:6/5/25/1}
\end{equation}

We will often use a more compact version of \Cref{eqn:6/5/25/1} 

\begin{equation}
    \int u \dd{v} = uv - \int v \dd{u} \label{eqn:6/5/25/2}
\end{equation}

This is the \vocab{integration by parts} formula. 

\begin{example}
    Find \( \int x \sin(x) \dd{x} \). \\
    Using \Cref{eqn:6/5/25/2}, we set \( u =x \), \( \dd{v}= \sin(x) \dd{x} \) ,so \( v= - \cos(x) \) and \( \dd{u}= \dd{x} \). This gives us 
    \begin{align*}
        \int x \sin(x) \dd{x} &= -x \cos(x)- \int - \cos(x) \dd{x}\\
        &= \sin(x) -x \cos{ \left( x \right) } +C
    \end{align*}
    
\end{example}

\begin{example}
    Find \( \int \ln(x) \dd{x} \).\\
    Setting \( u = \ln(x) \) and \( \dd{v}= 1 \dd{x} \), we get \( \dd{u}= \frac{1}{x} \dd{x} \) and \( v = x \) so 
    \begin{align*}
        \int \ln(x) \dd{x} &= x \ln(x) - \int 1 \dd{x}\\
        &=  x \ln(x) - x +C
    \end{align*}
    
\end{example}

\begin{example}
    Find \( \int x^{2} e^{x} \dd{x} \).\\
    Set \(  u = x^{2} \) so \( \dd{u} = 2x \dd{x} \). This gives us \( \dd{v}= e^{x} \dd{x} \) and \( v = e^{x} \).
    \[ \int x^{2} e^{x} \dd{x} = x^{2} e^{x} - \int x e^{x} \dd{x}. \]
    \( \int x e^{x} \dd{x} \) needs to be tackled again through integration by parts. So we reassign \( u =x, \dd{u}= \dd{x}, \dd{v}= e^{x} \dd{x}, v= e^{x} \) so 
    \begin{align*}
        \int x e^{x} \dd{x}&= x e^{x} - \int e^{x} \dd{x} \\
         &= x e^{x}- e^{x} + C.
    \end{align*}
Substitution into our earlier expression, we get 
\begin{align*}
    \int x^{2} e^{x} \dd{x} &= x^{2} e^{x} - \int x e^{x} \dd{x}\\
    &=  x^{2} e^{x} - \left( x e^{x}- e^{x} \right) +C\\
    &=  x^{2} e^{x} - x e^{x}+ e^{x} +C
\end{align*}

\end{example}

\begin{example}
    Find \( \int \sin (x) e^{x} \dd{x} \).\\
    For reasons that will become clear, let us set \( I = \int \sin{ \left( x  \right) } e^{x} \dd{x} \). We also set \( u = \sin{ \left( x \right) } , \dd{u} = -\cos{ \left( x \right) } \dd{x}, \dd{v} = e^{x } \dd{x}, v = e^{x} \). This gives us 
    \begin{align*}
        I &= \sin{ \left( x  \right) } e^{x} - \int - \cos{ \left( x \right) } e^{x } \dd{x}\\
        &=\sin{ \left( x  \right) } e^{x} + \int  \cos{ \left( x \right) } e^{x } \dd{x}\\
        &= \sin{ \left( x  \right) }e^{x }+ J
    \end{align*}
where  \( J =\int  \cos{ \left( x \right) } e^{x } \dd{x} \) and similarly let \( u = \cos{ \left( x  \right) } \dd{u} = \sin{ \left( x  \right) } \dd{x}, \dd{v} = e^{x} \dd{x}, v = e^{x} \). We get 
\begin{align*}
    J &= \cos{ \left( x  \right) }e^{x} - \int \sin{ \left( x \right) } e^{x} \dd{x} \\
    J &= \cos{ \left( x  \right) }e^{x } - I
\end{align*}
Now substituting the expression for \( J \) into the expression for \( I \), we get 
\begin{align*}
    I&= \sin{ \left( x  \right) } e^{x} + \cos{ \left( x  \right) }e^{x } - I\\
    2I &= \sin{ \left( x  \right) } e^{x} + \cos{ \left( x  \right) }e^{x } \\
\end{align*}
Dividing by \( 2 \) gives us the final answer 
\[ \boxed{\int \sin (x) e^{x} \dd{x} = \frac{\sin{ \left( x  \right) } e^{x} + \cos{ \left( x  \right) }e^{x }}{2} + C} \]
\end{example}
\(  \)\\
The integration by parts theorem also works for definite integrals. 

\begin{example}
    Evaluate \( \int_{0}^{1} \tan^{-1}{ \left( x \right) } \dd{x} \).\\
    Let \( u = \tan^{-1}{ \left( x \right) }, \dd{u} = \frac{1}{1+x^{2}} \dd{x}, \dd{v} = \dd{x}. v = x  \). 
   
    \begin{align*}
        \int_{0}^{1} \tan^{-1}{ \left( x \right) } \dd{x}  &= x \tan^{-1}{ \left( x \right) } \eval_{0}^{1} - \int_{0}^{1} \frac{x }{1+ x^{2 }} \dd{x}\\
        &= 1 \tan^{-1}{ \left( 1 \right) } - 0 \tan^{-1}{ \left( 0 \right) } - \int_{0}^{1} \frac{x }{1+ x^{2 }} \dd{x}\\
        &= \frac{\pi}{4}- \int_{0}^{1} \frac{x }{1+ x^{2 }} \dd{x}
    \end{align*}
    
We can use \( u \)-substitution for our new integral with \( u = 1+ x^{2}, \dd{u} = 2x \dd{x} \) 
\begin{align*}
    \int_{0}^1 \frac{x }{1+ x^{2 }} \dd{x} &= \frac{1}{2} \int_{1}^{2} \frac{1}{u } \dd{u}\\
    &= \frac{1}{2} \ln(u) \eval_{1}^{2}\\
    &= \frac{1}{2} \left( \ln(2)- \ln(1) \right)\\
    &= \frac{1}{2} \ln(2)\\
    &= \ln \left( \sqrt{2} \right)
\end{align*}
So finally, we have 
\[ \boxed{\int_{0}^{1} \tan^{-1}{ \left( x \right) } \dd{x} = \frac{\pi }{4}- \ln \left( \sqrt{2} \right) } \]
\end{example}

\begin{lemma}
    For all \( n \ge 2 \),
    \[ \int \sin^{n}{ \left( x  \right) } \dd{x} = = \frac{n-1}{n}\int \sin^{n-2}{ \left( x \right) } \dd{x} -\frac{1}{n } \sin^{n-1}{ \left( x  \right) }\cos{ \left( x \right)} . \]
    This is known as the \vocab{reduction formula.} 
\end{lemma}
\begin{proof}
    We can rewrite \( \int \sin^{n}{ \left( x \right) } \dd{x} \) as \( \int \sin^{n-1}{ \left( x  \right) } \sin{ \left( x  \right) } \dd{x} \) so we can do 
   \begin{align*}
    u = \sin^{n-1}{ \left( x  \right) }, &\quad  \dd{u} = \left( n-1  \right) \sin^{n-2}{ \left( x  \right) } \cos{ \left( x \right) } \dd{x} \\
    v=- \cos{ \left( x \right) } , & \quad \dd{v} = \sin{ \left( x  \right) } \dd{x}
   \end{align*}
   So 
   \begin{align*}
    \int \sin^{n}{ \left( x  \right) } \dd{x} &= -\sin^{n-1}{ \left( x  \right) }\cos{ \left( x \right) } - \int - \cos{ \left( x \right) } \left( n-1  \right) \sin^{n-2}{ \left( x  \right) } \cos{ \left( x \right) } \dd{x} \\ &=  -\sin^{n-1}{ \left( x  \right) }\cos{ \left( x \right) } + \left( n-1 \right) \int \sin^{n-2}{ \left( x \right) } \cos^{2}{ \left( x \right) } \dd{x}
     \\ &=  -\sin^{n-1}{ \left( x  \right) }\cos{ \left( x \right) } + \left( n-1 \right) \int \sin^{n-2}{ \left( x \right) }  \left( 1 - \sin^{2}{ \left( x \right) } \right)\dd{x}
      \\ &=  -\sin^{n-1}{ \left( x  \right) }\cos{ \left( x \right) } + \left( n-1 \right) \int \sin^{n-2}{ \left( x \right) }  - \sin^{n}{ \left( x \right) }\dd{x}\\
       \int \sin^{n}{ \left( x  \right) } \dd{x} &= -\sin^{n-1}{ \left( x  \right) }\cos{ \left( x \right) } + \left( n-1 \right) \int \sin^{n-2}{ \left( x \right) } \dd{x} - \left( n-1 \right) \int \sin^{n}{ \left( x  \right) }\dd{x}\\
      n\int \sin^{n}{ \left( x  \right) } \dd{x} &= -\sin^{n-1}{ \left( x  \right) }\cos{ \left( x \right) } + \left( n-1 \right) \int \sin^{n-2}{ \left( x \right) } \dd{x} 
   \end{align*}
which gives us the desired formula 
\[ \boxed{\int \sin^{n}{ \left( x  \right) } \dd{x} = \frac{n-1}{n}\int \sin^{n-2}{ \left( x \right) } \dd{x} -\frac{1}{n } \sin^{n-1}{ \left( x  \right) }\cos{ \left( x \right)}  } \]
\end{proof}

\begin{exercise}
    Evaluate \( \int x e^{2x} \dd{x} \)
\end{exercise}
\begin{solution}
    Set 
    \begin{align*}
        u = x, &\quad \dd{u} = \dd{x}\\
        v = \frac{1}{2}e^{2x}, &\quad \dd{v} = e^{2x } \dd{x}
    \end{align*}
So 
\begin{align*}
    \int x e^{2x } \dd{x} &= \frac{1}{2} x e^{2x} - \int \frac{1}{2} e^{ 2x } \dd{x}\\
    &= \frac{1}{2 }x e^{2x } - \frac{1}{4} e^{2x}+ C
\end{align*}

\end{solution}

\begin{exercise}
    Evaluate \( \int \sqrt{x} \ln \left( x  \right) \dd{x} \).
\end{exercise}
\begin{solution}
    Set 
    \begin{align*}
        u = \ln \left( x  \right), &\quad \dd{u} = \frac{1}{x }\dd{x}\\
        v = \frac{2}{3} x^{\frac{3}{2}}, &\quad \dd{v} = x^{\frac{1}{2}} \dd{x}
    \end{align*}
This gives us 
\begin{align*}
    \int \sqrt{x} \ln \left( x  \right) \dd{x} &=  \frac{2}{3} x^{\frac{3}{2}} \ln(x)- \int \frac{x^{\frac{1}{2}}}{x} \dd{x}\\
    &= \frac{2}{3} x^{\frac{3}{2}} \ln(x)- \int x^{-\frac{1 }{2}} \dd{x}\\
    &\frac{2}{3} x^{\frac{3}{2}} \ln(x)- 2 x^{\frac{1}{2}}+C
\end{align*}
So 
\[ \boxed{\int \sqrt{x} \ln \left( x  \right) \dd{x} = \frac{2}{3} \sqrt{x^{3}} \ln(x) -2 \sqrt{x}+C .} \]
\end{solution}

\begin{exercise}
    Find \( \int x \cos{ \left( 4x \right) } \dd{x} \)
\end{exercise}
\begin{solution}
    Set 
    \begin{align*}
        u= x ,&\quad  \dd{u} = \dd{x}\\
        v = \frac{1}{4} \sin{ \left( 4x \right) } ,& \quad \dd{v} = \cos{ \left( 4 x \right) } \dd{x}
    \end{align*}
    \begin{align*}
        \int x \cos{ \left( 4x \right) } \dd{x} &= \frac{1}{4} x \sin{ \left( 4x \right) } - \int \frac{1}{4} \sin{ \left( 4x \right) } \dd{x} \\
        &= \frac{1}{4} x \sin{ \left( 4x \right) } + \frac{1}{16} \cos{ \left( 4x \right) }+ C 
    \end{align*}
 So 
 \[ \boxed{ \int x \cos{ \left( 4x \right) } \dd{x} = \frac{1}{4} x \sin{ \left( 4x \right) } + \frac{1}{16} \cos{ \left( 4x \right) }+ C } \]   
\end{solution}

\begin{exercise}
    Evaluate 
    \[ \int \frac{x}{1-x} \dd{x} \]
\end{exercise}
\begin{solution}
    Set 
    \begin{align*}
        u= x, & \quad \dd{u} = \dd{x} \\
        v = - \ln(1-x)  & \quad \dd{v} = \frac{1}{1-x} \dd{x}
    \end{align*}
    So 
    \begin{align*}
        \int \frac{x}{1-x} \dd{x} &= - x \ln(1-x) + \int \ln(1-x) \dd{x}
    \end{align*}
    Making a substitution into our second integral of \( t = 1-x \Rightarrow -\dd{t} = \dd{x} \) so 
    \begin{align*}
        \int \ln(1-x) \dd{x} & - \int \ln(t) \dd{t} \\
        &= - t \ln(t) +t +C\\
        &= - \left( 1-x \right) \ln(1-x) + \left( 1-x \right) +C \\
        &= x \ln(1-x) -\ln(1-x) -x + C \tag{Since $C$ is an arbitrary constant, we will absorb the $1$ here.}
     \end{align*}
    Substituting into our earlier expression, we have the final answer of 
    \[ \boxed{ \int \frac{x }{1-x } \dd{x} = - \ln(1-x) -x +C} \]
\end{solution}