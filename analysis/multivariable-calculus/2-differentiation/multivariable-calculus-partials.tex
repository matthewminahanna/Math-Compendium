\begin{dfn}
    Suppose that \( f: \mathbb{R}^{n} \to \mathbb{R} \). Then \( \pdv{f}{x_1}, \pdv{f}{x_2}, \dots, \pdv{f}{x_n} \) denote the \vocab{partial derivatives} with respect to the first, second, ..., $n$-th variables, if they exist. The $k$-th partial derivative exists if the following limit exists and is finite: 
    \[ \lim_{h \to 0} \frac{f \left( x_1, x_2, \dots, x_k+h, \dots, x_n \right) - f \left( x_1, x_2, \dots, x_n \right)}{h}. \] 
    Or equivalently, 
    \[ \lim_{h \to 0} \frac{f \left( \vb{x} + h \cdot \vb{e}_{j} \right) - f \left( \vb{x} \right)}{h} \]
\end{dfn}