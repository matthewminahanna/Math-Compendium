\documentclass[10pt,twoside=false,open=any,numbers=noenddot]{scrbook}
\usepackage[utf8]{inputenc}
\usepackage{makeidx}
\makeindex
\usepackage[utf8]{inputenc}
\usepackage[T1]{fontenc}


\usepackage{amsmath}
\usepackage{amsfonts}
\usepackage{amsthm}
\usepackage{mathtools}
\usepackage{amssymb}
\usepackage{enumitem}


\usepackage{tikz-cd,tikz}
    \usetikzlibrary{hobby, calc, intersections, decorations.markings, decorations.pathreplacing} %libraries
    \tikzset{>=latex}

\usepackage{xifthen}
\usepackage{pdfpages}
\usepackage{transparent}
\usepackage{import}
\usepackage{subfiles}
\usepackage[sfdefault,lining,book]{FiraSans}
\usepackage{FiraMono}
\usepackage{sansmathfonts}


\usepackage{faktor}
\DeclareMathSizes{10.95}{11.2}{7}{7}
%\boldmath
\everymath{\displaystyle}
\usepackage{hyperref}
\usepackage{nameref,cleveref}
\usepackage[a4paper, 
left=0.05in,
right=0.1in,
top=0.70in,
bottom=0.70in
]{geometry}
\usepackage[mathscr]{eucal}
\PassOptionsToPackage{usenames,svgnames,dvipsnames,table,xcdraw}{xcolor}
\usepackage{xcolor}

\usepackage[tight]{minitoc}
\mtcsetfont{parttoc}{chapter}{\sffamily\bfseries}
\mtcsetfont{parttoc}{section}{\footnotesize\upshape\mdseries}
\mtcsetfont{parttoc}{subsection}{\footnotesize\upshape\mdseries}
\mtcsetfont{parttoc}{subsubsection}{\footnotesize\upshape\mdseries}
%\mtcsetdepth{parttoc}{1}
\setcounter{parttocdepth}{3}
\renewcommand*{\partheadstartvskip}{\vspace*{-7em}}
\renewcommand*{\partheadendvskip}{}
%\noptcrule
\renewcommand\beforeparttoc{\noindent{}}
%\hspace{\fill}\rule{0.95\linewidth}{2pt}\hspace{\fill}
\doparttoc[n]

\usepackage{tikz}
\usepackage{tikz-cd}
\usetikzlibrary{intersections, angles, quotes, calc, positioning}
\usetikzlibrary{arrows.meta}
\usepackage{pgfplots}
\pgfplotsset{compat=1.13}
\tikzset{
    force/.style={thick, {Circle[length=2pt]}-stealth, shorten <=-1pt}
}
\makeatother
\usepackage{thmtools}
\usepackage[framemethod=TikZ]{mdframed}
\mdfsetup{skipabove=1em,skipbelow=0em}
\usepackage{wasysym}

\newenvironment{ignore}{}{}

\theoremstyle{definition}

\newenvironment{solutiont}
  {\begin{proof}[Solution]}
  {\end{proof}}

% Catppuccin Latte - Blue for theorems
\definecolor{theoremcolor}{HTML}{1e66f5}
\declaretheoremstyle[
    headfont=\bfseries\sffamily\color{theoremcolor!70!black}, bodyfont=\normalfont,
     postheadspace=\newline,
    mdframed={
        linewidth=4pt,
        rightline=false, topline=false, bottomline=false,
        linecolor=theoremcolor, backgroundcolor=theoremcolor!20,
    }
]{theoremst}

\declaretheorem[style=theoremst, name=Theorem,numberwithin=section]{theorem}
\declaretheorem[style=theoremst, name=Lemma,sibling=theorem]{lemma}
\declaretheorem[style=theoremst, name=Proposition,numberwithin=section]{proposition}
\declaretheorem[style=theoremst, name=Corollary, numbered=no]{corollary}

% Catppuccin Latte - Lavender for ideas
\definecolor{ideacolor}{HTML}{7287fd}
\declaretheoremstyle[
    headfont=\bfseries\sffamily\color{ideacolor!70!black}, bodyfont=\normalfont,
    mdframed={
        linewidth=2pt,
        rightline=false, topline=false, bottomline=false,
        linecolor=ideacolor, backgroundcolor=ideacolor!20,
    }
]{ideast}

\declaretheorem[style=ideast, name=Idea]{idea}

\declaretheoremstyle[
	numbered = no,
    headfont=\bfseries\sffamily\color{theoremcolor!70!black}, bodyfont=\normalfont,
    postheadspace=\newline,
    mdframed={
        linewidth=2pt,
        linewidth=4pt,
        skipabove = 0pt,
        rightline=false, topline=false, bottomline=false,
        linecolor=theoremcolor, backgroundcolor=theoremcolor!10,
    },
    qed = $\blacksquare$
]{proofst}

\declaretheorem[style=proofst, name=Proof]{replacementproof}
\renewenvironment{proof}[1][\proofname]{\begin{replacementproof}}{\end{replacementproof}}

% Catppuccin Latte - Red for definitions
\definecolor{definitioncolor}{HTML}{d20f39}
\declaretheoremstyle[
    headfont=\bfseries\sffamily\color{definitioncolor!70!black}, bodyfont=\normalfont,
    postheadspace=\newline,
    mdframed={
        linewidth=3pt,
        rightline=false, topline=false, bottomline=false,
        linecolor=definitioncolor, backgroundcolor=definitioncolor!20,
    }
]{definitionst}

\declaretheorem[style=definitionst, name=Definition, numberwithin=section]{dfn}

% Catppuccin Latte - Teal for examples
\definecolor{examplecolor}{HTML}{179299}
\declaretheoremstyle[
    headfont=\bfseries\sffamily\color{examplecolor!70!black}, bodyfont=\normalfont,
    postheadspace=\newline,
    mdframed={
        linewidth=3pt,
        rightline=false, topline=false, bottomline=false,
        linecolor=examplecolor, backgroundcolor=examplecolor!20,
    }
]{examplest}

\declaretheorem[style=examplest, name=Example, numberwithin=section]{example}

% Catppuccin Latte - Lavender for exercises
\definecolor{exercisecolor}{HTML}{7287fd}
\declaretheoremstyle[
    headfont=\bfseries\sffamily\color{exercisecolor!70!black}, bodyfont=\normalfont,
    postheadspace=\newline,
    mdframed={
        linewidth=3pt,
        rightline=false, topline=false, bottomline=false,
        linecolor=exercisecolor, backgroundcolor=exercisecolor!20,
    }
]{exercisest}

\declaretheorem[style=exercisest, name=Exercise, numberwithin=section]{exercise}


\declaretheoremstyle[
    numbered= no,
    headfont=\bfseries\sffamily\color{exercisecolor!70!black}, bodyfont=\normalfont,
    postheadspace=\newline,
    mdframed={
        linewidth=3pt,
        skipabove = 0pt,
        rightline=false, topline=false, bottomline=false,
        linecolor=exercisecolor, backgroundcolor=exercisecolor!10,
    },
qed = $\newmoon$
]{solutionst}

\declaretheorem[style=solutionst, name=Solution]{solution}
\renewenvironment{solutiont}[1][\solutionstname]{\vspace{-20pt}\begin{solution}}{\end{solution}}

% Catppuccin Latte - Peach for recall
\definecolor{recallcolor}{HTML}{fe640b}
\declaretheoremstyle[
    headfont=\bfseries\sffamily\color{recallcolor!70!black}, bodyfont=\normalfont,
    numbered = no,
    mdframed={
        linewidth=3pt,
        rightline=false, topline=false, bottomline=false,
        linecolor=recallcolor, backgroundcolor=recallcolor!10,
    }
]{recallst}

\declaretheorem[style=recallst, name=Recall]{recall}

% Catppuccin Latte - Maroon for vocabulary
\definecolor{vocabb}{HTML}{e64553}

\newcommand{\vocab}[1]{\textbf{\color{vocabb} #1}}

\newcommand{\comment}[1]{%
  \text{\phantom{(#1)}} \tag{#1}
}

\usepackage{physics}

\DeclareMathOperator*{\str}{\ast}

% Catppuccin Latte - Subtext0 for section colors
\definecolor{sectcolor}{HTML}{6c6f85}

\renewcommand*{\sectionformat}{\color{sectcolor}\S\firaextrabold\thesection\autodot\enskip}
\renewcommand*{\subsectionformat}{\color{sectcolor}\S\thesubsection\autodot\enskip}



\addtokomafont{chapterprefix}{}{\raggedleft}
\RedeclareSectionCommand[beforeskip=0.5em]{chapter}
\renewcommand*{\chapterformat}{%
\mbox{\scalebox{1.5}{\chapappifchapterprefix{\nobreakspace}}%
\scalebox{1.5}{\color{sectcolor}\firaheavy\thechapter\autodot}\enskip}}


\addtokomafont{partprefix}{\firaheavy}
\renewcommand*{\partformat}{}


\setkomafont{chapter}{\Huge\firaheavy}
\setkomafont{section}{\Large\firaextrabold}
\setkomafont{subsection}{\large\firasemibold}
% for smarter referencing (optional but helpful)

% Custom equation numbering format: Part.Chapter.Section.Equation
\renewcommand{\theequation}{\thesection.\arabic{equation}}

% Reset equation counter at each new section
\makeatletter
\@addtoreset{equation}{section}
\makeatother

\usepackage{subcaption}

% *** quiver ***
% A package for drawing commutative diagrams exported from https://q.uiver.app.
%
% This package is currently a wrapper around the `tikz-cd` package, importing necessary TikZ
% libraries, and defining new TikZ styles for curves of a fixed height and for shortening paths
% proportionally.
%
% Version: 1.6.0
% Authors:
% - varkor (https://github.com/varkor)
% - AndréC (https://tex.stackexchange.com/users/138900/andr%C3%A9c)
% - Andrew Stacey (https://tex.stackexchange.com/users/86/andrew-stacey)

\NeedsTeXFormat{LaTeX2e}
\ProvidesPackage{quiver}[2025/09/20 quiver]

% `tikz-cd` is necessary to draw commutative diagrams.
\RequirePackage{tikz-cd}
% `amssymb` is necessary for `\lrcorner` and `\ulcorner`.
\RequirePackage{amssymb}
% `calc` is necessary to draw curved arrows.
\usetikzlibrary{calc}
% `pathmorphing` is necessary to draw squiggly arrows.
\usetikzlibrary{decorations.pathmorphing}
% `spath3` is necessary to draw shortened edges.
\usetikzlibrary{spath3}

% A TikZ style for curved arrows of a fixed height, due to AndréC.
\tikzset{curve/.style={settings={#1},to path={(\tikztostart)
    .. controls ($(\tikztostart)!\pv{pos}!(\tikztotarget)!\pv{height}!270:(\tikztotarget)$)
    and ($(\tikztostart)!1-\pv{pos}!(\tikztotarget)!\pv{height}!270:(\tikztotarget)$)
    .. (\tikztotarget)\tikztonodes}},
    settings/.code={\tikzset{quiver/.cd,#1}
        \def\pv##1{\pgfkeysvalueof{/tikz/quiver/##1}}},
    quiver/.cd,pos/.initial=0.35,height/.initial=0}

% A TikZ style for shortening paths without the poor behaviour of `shorten <' and `shorten >'.
\tikzset{between/.style n args={2}{/tikz/execute at end to={
    \tikzset{spath/split at keep middle={current}{#1}{#2}}
}}}

% TikZ arrowhead/tail styles.
\tikzset{tail reversed/.code={\pgfsetarrowsstart{tikzcd to}}}
\tikzset{2tail/.code={\pgfsetarrowsstart{Implies[reversed]}}}
\tikzset{2tail reversed/.code={\pgfsetarrowsstart{Implies}}}
% TikZ arrow styles.
\tikzset{no body/.style={/tikz/dash pattern=on 0 off 1mm}}

\endinput

%end quiver


\usepackage{tocbasic}
\makeatletter
\renewcommand*{\l@part}[2]{%
  \ifnum \scr@tocdepth >-2\relax
    \addpenalty\@secpenalty
    \addvspace{1.0em plus\p@}%
    \begingroup
      \parindent \z@
      \rightskip \@tocrmarg
      \parfillskip -\rightskip
      \leavevmode
      \advance\leftskip\@tempdima
      \hskip -\leftskip
      \llap{\thepart\enskip \quad}% ↠ADD SPACE HERE
      #1\nobreak\hfil\nobreak\hb@xt@\@pnumwidth{\hss #2}%
      \par
    \endgroup
  \fi}
\makeatother

\usepackage{xpatch}


\usepackage{graphicx}
\usepackage[
backend=biber,
style=alphabetic,
sorting=nyt  % Changed from ynt to nyt for name-year-title sorting
]{biblatex}
\usepackage[toc,page]{appendix}
\usepackage{booktabs}
\usepackage{array}
\usepackage{epigraph}
\usepackage{float}
\usepackage{placeins}
\newcommand{\acts}{\curvearrowright}
\usepackage[thicklines]{cancel}
\newcommand\Ccancel[2][black]{\renewcommand\CancelColor{\color{#1}}\xcancel{#2}}
\usepackage{tocloft}
\renewcommand{\cftchapdotsep}{\cftdotsep}
\renewcommand\cftchapfont{\mdseries}
\renewcommand\cftchappagefont{\mdseries}
\usepackage[scale=1.1, boldweight =Bold, opstyle=sans]{fdsymbol}
\makeatletter
\renewcommand*\env@matrix[1][*\c@MaxMatrixCols c]{%
  \hskip -\arraycolsep
  \let\@ifnextchar\new@ifnextchar
  \array{#1}}
\makeatother

% Catppuccin Latte - Mauve for hyperlinks
\definecolor{tturqddark}{HTML}{8839ef}
% Catppuccin Latte - Overlay2 for TOC
\definecolor{toccolor}{HTML}{7c7f93}
\hypersetup{
    colorlinks=true,
    linkcolor=tturqddark,
    citecolor=tturqddark,
    urlcolor=tturqddark
}
\addtocontents{toc}{\protect\hypersetup{linkcolor=toccolor}}
\makeatletter
\renewcommand{\ptcfont}{\hypersetup{linkcolor=toccolor}}
\makeatother

% Catppuccin Latte - Colorful accent colors
\definecolor{DeepRed}{HTML}{d20f39}        % Latte Red
\definecolor{AmberOrange}{HTML}{fe640b}    % Latte Peach
\definecolor{GoldenYellow}{HTML}{df8e1d}   % Latte Yellow
\definecolor{Emerald}{HTML}{40a02b}        % Latte Green
\definecolor{AzureBlue}{HTML}{1e66f5}      % Latte Blue
\definecolor{RoyalPurple}{HTML}{8839ef}    % Latte Mauve

\renewcommand{\Re}{\mathfrak{Re}}
\renewcommand{\Im}{\mathfrak{Im}}
\newcommand{\colorboxed}[3][theoremcolor!10]{\fcolorbox{#2}{#1}{$\displaystyle #3$}}







\usepackage[
  automark,
  autooneside=false,  % Keep leftmark and rightmark separate in one-sided docs
  headsepline,        % Add line under header
  footsepline         % Add line above footer
]{scrlayer-scrpage}

% Clear default page styles
\clearpairofpagestyles

% Initialize current part name
\newcommand{\currentpartname}{}

% Redefine \part to save the part name
\let\oldpart\part
\renewcommand{\part}[1]{%
  \oldpart{#1}%
  \renewcommand{\currentpartname}{Part \thepart: #1}%
}

% Set up what goes into marks
% \automark[right mark]{left mark}
\automark[section]{chapter}

% Set up headers and footers (one-sided)
\lohead{\leftmark}             % Top left: chapter
\rohead{\rightmark}            % Top right: section  
\lofoot{\currentpartname}      % Bottom left: part
\rofoot{\pagemark}             % Bottom right: page number

% Apply the page style
\pagestyle{scrheadings}

\usepackage{tocloft}
\setlength{\cftpartnumwidth}{1em}
\cftsetindents{part}{0em}{3.5em} %IF NUMBERS BLEED INTO PART IN TOC, INCREASE 2ND NUMBER 

\renewcommand{\cftpartfont}{\sffamily\bfseries\large\firaextrabold}

\addbibresource{sample.bib}

% Simpler underline for authors
\renewcommand*{\mkbibnamefamily}[1]{\underline{#1}}
\renewcommand*{\mkbibnamegiven}[1]{\underline{#1}}
\renewcommand*{\mkbibnameprefix}[1]{\underline{#1}}
\renewcommand*{\mkbibnamesuffix}[1]{\underline{#1}}

% Bold book titles
\DeclareFieldFormat[book]{title}{\textbf{#1}}
\DeclareFieldFormat[book]{booktitle}{\textbf{#1}}
\DeclareFieldFormat{isbn}{\href{https://isbnsearch.org/isbn/#1}{\fbox{\strut#1}}}

\usepackage[Glenn]{fncychap}
\ChNameAsIs
\ChTitleAsIs
\definecolor{catppuccinBlue}{RGB}{70, 110, 160}      % Sapphire
\definecolor{catppuccinGrey}{RGB}{108, 112, 134}      % Overlay0

\ChTitleVar{\firaheavy\Huge\color{catppuccinBlue}}
\ChNumVar{\firaheavy\huge\color{catppuccinBlue}}
\ChNameVar{\firaheavy\huge\color{catppuccinBlue}}

\usepackage{xpatch}
\xpatchcmd\DOCH
  {\mghrulefill}{\color{catppuccinGrey}\mghrulefill}
  {}{\PatchFailed}
\xpatchcmd\DOTI
  {\mghrulefill}{\color{catppuccinGrey}\mghrulefill}
  {}{\PatchFailed}
\xpatchcmd\DOTIS
  {\mghrulefill}{\color{catppuccinGrey}\mghrulefill}
  {}{\PatchFailed}

% Add negative space to reduce gaps
\makeatletter
\let\old@makechapterhead\@makechapterhead
\let\old@makeschapterhead\@makeschapterhead
\renewcommand{\@makechapterhead}[1]{%
  \vspace*{-100pt}% Reduce space before
  \old@makechapterhead{#1}%
  \vspace*{-75pt}% Reduce space after
}
\renewcommand{\@makeschapterhead}[1]{%
  \vspace*{-100pt}% Reduce space before
  \old@makeschapterhead{#1}%
  \vspace*{-75pt}% Reduce space after
}
\makeatother

\DeclareNameAlias{sortname}{family-given}
\DeclareBibliographyDriver{book}{%
  \usebibmacro{bibindex}%
  \usebibmacro{begentry}%
  \printnames{author}%
  \newunit\newblock
  \printfield{title}%
  \newunit\newblock
  \printfield{isbn}%
  \finentry}




  
 



\begin{document}

\subfile{cover/frontcover}

\newpage

\frontmatter

\setcounter{tocdepth}{0}
\noptcrule
\chapter*{Contents}
\addcontentsline{toc}{chapter}{Contents}
\renewcommand{\contentsname}{\vspace{-3em}}
\tableofcontents

\chapter*{About}
\addcontentsline{toc}{chapter}{About}
This compendium contains my solutions to exercises and reformulations of proofs from various mathematics textbooks. It serves as a demonstration of my mathematical abilities and self-directed learning for graduate programs and potential employers.

\subsection*{Background and Purpose}
I completed my master's degree and am currently self-studying mathematics in preparation for a PhD program. This project reflects my commitment to deepening my mathematical understanding while showcasing my problem-solving abilities. My goal is straightforward: to improve at mathematics and demonstrate proficiency in the subject.

\subsection*{Scope and Content}
The topics covered range in difficulty from precalculus to graduate-level mathematics, organized roughly in the order I learned them and according to my interests. This is an ongoing, perpetual project—topics are not set in stone and will continue to evolve as I progress through new material. I include essentially every problem I work through, providing a comprehensive record of my mathematical development.

\subsection*{Methodology}
Each problem is first attempted by hand in \href{https://www.goodnotes.com}{Goodnotes}\footnote{For the love of God, Goodnotes, please stop trying to add A"I" features. Just make your app better.} for iPad before being typeset. All proofs included here have been memorized and recreated from memory rather than copied directly. When I develop an original solution or proof that works, I use that instead of reproducing the textbook's approach.

I also have a deep appreciation for mathematical exposition as an art form—writers like John Lee, John Stillwell, and Tristan Needham exemplify this craft. While my goal isn't to surpass such masters of exposition, I believe that writing substantial amounts of mathematics in \LaTeX{} can only help develop that skill.

\subsection*{For Students}
While this document is optimized as a portfolio piece rather than a pedagogical resource, students may find value in using it as a supplementary exercise collection alongside the original textbooks. If you are a student looking to learn mathematics, I suggest the following approach:
\begin{enumerate}[label=\textbf{\roman*)}]
    \item Locate the textbooks referenced in each section. The boxed ISBN numbers in the bibliography can be searched on library catalogs, book databases, or academic archives to help you find copies.
    \item Use those textbooks as your primary learning resource.
    \item Treat this document as a collection of worked exercises for practice and verification.
\end{enumerate}

\chapter*{Study Tips}
\addcontentsline{toc}{chapter}{Study Tips}
\subsection*{A Method for Memorizing Proofs}
All proofs recreated in this document have been memorized and written from memory rather than copied directly from the source material. One effective technique that I use for proof memorization is to work in pairs:
\begin{enumerate}
    \item Select two theorems, A and B.
    \item Write and annotate (fill in the {gaps in logic}\footnote{\cite{ref:rudin_pma} is good practice for this skill.} or compute specific examples) the proof for theorem A. Review if needed before you proceed to the next step. 
    \item Completely move on from theorem A. Then write and annotate the proof for theorem B.
    \item Close the book or file (or look away) and attempt to recreate the proof for A, then the proof for B.
\end{enumerate}
The key insight: proving B ``purges'' your short-term memory of proof A, and vice versa. This creates a more rigorous test of whether you've truly memorized the material rather than simply holding it in working memory.\\
\textbf{Note:} This technique works equally well for practice problems. Work through problem A, then problem B, then attempt to solve both from scratch without referring to your previous work.

\subsection*{Bloom's Taxonomy in Pure Mathematics}
Memorization is the lowest level of learning in Bloom's Taxonomy, but it's still learning. The hierarchy progresses through six levels:

\begin{enumerate}
    \item \textbf{Memorization:} Recall definitions, theorems, and formulas. For example, memorizing the definition of continuity, the statement of the Intermediate Value Theorem, or the formula for Taylor series.  
    
   \item \textbf{Understand:} Explain concepts in your own words and grasp why theorems are true. For instance, understanding why continuous functions on compact sets attain their extrema, or why the proof of a theorem requires certain hypotheses. You can implement this by omitting certain hypotheses in the statement of a theorem and attempting to prove it—you'll discover exactly why those conditions are needed.\footnote{See: Reverse mathematics.}
    
    \item \textbf{Apply:} Use theorems and techniques to solve problems. This means applying the Mean Value Theorem to prove inequalities, using the definition of convergence to show a sequence converges, or employing integration techniques to evaluate integrals.
    
    \item \textbf{Analyze:} Break down complex problems and recognize which tools are needed. This involves identifying which theorem applies to a given situation, understanding the structure of a proof, or determining why a particular approach fails and what conditions are missing.
    
    \item \textbf{Evaluate:} Assess the validity of proofs, compare different approaches, and judge whether solutions are correct or complete. For example, finding gaps in arguments, determining whether a proof generalizes to other contexts, or critiquing the elegance and efficiency of different solution methods.
    
    \item \textbf{Create:} Construct original proofs, develop new problem-solving approaches, or formulate conjectures. This is the highest level—producing novel mathematics, whether it's finding your own proof of a known theorem, solving an unseen problem, or discovering new patterns and relationships.
\end{enumerate}

Understanding should always be the goal, but sometimes memorizing key results, techniques, or proof structures provides scaffolding that deeper understanding can build upon later. Don't dismiss memorization as worthless—it's the foundation upon which higher-order thinking is built.

\subsection*{The Struggle Hierarchy}
When working on problems, resist the urge to immediately look at solutions. Productive struggle is where real learning happens. Follow this hierarchy:
\begin{enumerate}
    \item \textbf{Struggle independently.} Give yourself substantial time (20-30 minutes minimum for challenging problems or even an hour if you're in the zone) to think, try approaches, and explore dead ends.
    \item \textbf{Struggle with a peer.} Discuss approaches and share insights, but continue working toward the solution together rather than looking it up.
    \item \textbf{Struggle with a mentor.} Seek guidance from someone more experienced who can point you in the right direction without giving away the answer.
    \item \textbf{Look at the first step only.} If still stuck, reveal just the initial approach or first line of the solution, then close it and try to continue on your own.
    \item \textbf{Repeat.} If needed, look at the next step, then try again independently. Continue this cycle until you complete the problem.
\end{enumerate}
Looking at a full solution should be a last resort. When you do, study it carefully, then close it and recreate the entire solution from memory to ensure you've internalized the reasoning.


\chapter*{Acknowledgments}
\addcontentsline{toc}{chapter}{Acknowledgments}
This project benefited from open-source tools and examples created by:\\\\

\textbf{Evan Chen} — The Napkin project served as the initial model for this compendium, and portions of its code are used throughout. GitHub: \url{https://github.com/vEnhance/napkin/} \\

\textbf{Aareyan Manzoor (The TeXromancers)} — Provided permission to reuse the code used for the front cover. GitHub: \url{https://aareyanmanzoor.github.io/Texromancers.html} \\

\textbf{Gilles Castel (R.I.P.)} — His work on integrating UltiSnips into LaTeX workflows (I use HyperSnips, a VS Code port) removed many bottlenecks and sped up development. GitHub: \url{https://github.com/gillescastel}



\mainmatter


\part{Precalculus}
\label{part: Precalculus}
\vspace*{2em} The primary resource for this part is \cite{ref:axler_precalc}.  \vspace*{1em}
\parttoc
\chapter{Real Numbers and Functions}
\input{content/analysis/precalculus/real-numbers-and-functions/real-numbers-and-functions}

\chapter{Rational Functions and Conic Sections}
\section{Polynomial Functions}
\begin{dfn}\label{def:polynomial}
    A \vocab{(real) polynomial} is a function of the form:
    \[ p(x) = a_{0} + a_{1}x+ a_{2}x^{2} + \cdots + a_{n}x^{n} = \sum_{k=0}^{n}a_{k}x^{k} \] 
    where the \vocab{coefficients} \( a_{k} \in \mathbb{R} \). The \vocab{degree} of a polynomial is the largest \( n \) for which \( a_{n} \neq 0 \).
    
    A polynomial of degree: 
    \begin{enumerate}[label=\textbf{\arabic*)},start=0]
    \item is called a \vocab{constant function}. (This is literally just the function that sends every \( x \) to the same number.) 
    \item is called a \vocab{linear function}. 
    \item is called a \vocab{quadratic function}.
    \item is called a \vocab{cubic function}. 
    \item is called a \vocab{quartic function}. 
    \item is called a \vocab{quintic function}.   
    \end{enumerate}
    Those are the polynomials with special names. Any polynomial can be referred to as a \vocab{polynomial of degree \( n \)}.
\end{dfn}

\section{Rational Functions}
\begin{dfn}
    A \vocab{(real) rational function} is a function of the form: 
    \[ r(x) = \frac{p(x )}{q(x)} \] where \( p(x) \) and \( q(x) \) are \hyperref[def:polynomial]{polynomials}. Notice that all polynomials are rational functions by setting \( q(x)=1 \).
\end{dfn}

\section{Conic Sections}
\subsection{Parabolas}

\begin{lemma}
    Fix a point \( \vb{p} = \left( p_{x}, p_{y} \right) \) and a line \( y=c \), such that \( p_{y} \neq c \) (The point does not lie on the line). The locus of points equidistant from both the point and the line is the graph of a quadratic function.  Here is a \href{https://www.desmos.com/calculator/78dqsw2yji}{nice Desmos visualization}.
\end{lemma}
\begin{proof}
    \( \left( x,y \right) \) be on the locus of points described above. Setting the distances described above equal to each other, we have 
    \begin{align*}
        d \left( \left( x,y \right), \left( p_{x}, p_{y} \right) \right) &= \abs{y-c} \\
        \sqrt{ \left( x-p_{x} \right)^{2} + \left( y- p_{y} \right)^{2}} &= \abs{y-c} \\
        \left( x-p_{x} \right)^{2} + \left( y- p_{y} \right)^{2} &= \left( y-c \right)^{2} \\
        \left( x-p_{x} \right)^{2} &= \left( y-c \right)^{2} - \left( y- p_{y} \right)^{2} \\
        \left( x-p_{x} \right)^{2} &= \left[ \left( y-c \right) - \left( y-p_{y} \right) \right] \left[ \left( y-c \right) + \left( y-p_{y} \right) \right]\\
         \left( x-p_{x} \right)^{2} &= \left( p_{y}-c \right) \left[ 2y-c-p_{y} \right] \\
         \frac{1}{2 \left(  p_{y}-c  \right)} \left( x-p_{x} \right)^{2} + \frac{x+p_{y}}{2} &=y
    \end{align*}
    So 
    \[ \boxed{y=  \frac{1}{2 \left(  p_{y}-c  \right)} \left( x-p_{x} \right)^{2} + \frac{c+p_{y}}{2}} \]
    is the desired quadratic.
\end{proof}


\chapter{Exponential and Logarithmic Functions}
\begin{example}
    Suppose a drug X has a half-life of \( 45 \) minutes in the blood stream and that it is not safe to have more than \( 175 \)mg of the drug in a patient's blood. If Dr. Alice administered a \( 120 \)mg dose to her patient, how long does she need to wait before she can safely administer another \( 120 \)mg dose?\\
    Dr. Alice needs to wait until the patient's current drug level drops below \( 55 \) mg (since \( 55 + 120 = 175 \)). We need to determine the time when the patient has less than \( 55 \) mg of drug X left in their system, so we have 
\[  120 \left( \frac{1}{2} \right)^{\frac{t }{45}} <55.  \]
Dividing both sides by \( 120 \), 
   \begin{equation}
     \left( \frac{1}{2}  \right)^{ \frac{t }{45}} < \frac{55}{120} = \frac{11}{24}. \label{eqn:6/21/25/1} 
\end{equation}
Taking the natural logarithm of both sides, 
\[ \frac{t}{45} \ln \left( \frac{1}{2} \right) < \ln \left( \frac{11}{24} \right). \]
Since \( \ln \left( \frac{1}{2} \right) < 0 \), dividing both sides by this negative number flips the inequality:
\[ \frac{t}{45} > \frac{\ln \left( \frac{11}{24} \right)}{ \ln \left( \frac{1}{2} \right)}. \]
Therefore,
\[ t >   45 \cdot \frac{\ln \left( \frac{11}{24} \right)}{ \ln \left( \frac{1}{2} \right)} \approx 51 \text{ minutes}. \]
Dr. Alice must wait at least 51 minutes before she can safely administer drug X again.
\textbf{Common Mistake:} We can take a different route following from \Cref{eqn:6/21/25/1}. Namely, we might try taking \( \log_{\frac{1}{2}} \) of both sides:
\[  \frac{t}{45} \log_{\frac{1}{2}} \left( \frac{1}{2} \right) < \log_{\frac{1}{2}} \left( \frac{11}{24} \right). \]
Since \( \log_{\frac{1}{2}} \left( \frac{1}{2} \right) = 1 \), this would give us
\[ \frac{t}{45} < \log_{\frac{1}{2}} \left(  \frac{11}{24} \right), \] 
or \[ t < 45 \log_{\frac{1}{2}} \left(  \frac{11}{24} \right) \approx 51 \text{ minutes}. \]
This solution would incorrectly suggest that Dr. Alice should administer drug X \textit{before} 51 minutes have passed, which would be unsafe for the patient.
\textbf{What went wrong?} The function \( f(x) = \log_{\frac{1}{2}}(x) \) is decreasing since its base \( \frac{1}{2} < 1 \). For any decreasing function \( f \), we have 
\[ x <y \iff f(x) > f(y). \]
When we applied the decreasing function \( \log_{\frac{1}{2}} \) to both sides of the inequality, we should have flipped the inequality sign:
\[ \frac{t}{45} \log_{\frac{1}{2}} \left( \frac{1}{2} \right) > \log_{\frac{1}{2}} \left( \frac{11}{24} \right), \]
which gives us the correct answer: \( t > 51 \) minutes.
\end{example}

\begin{exercise}
   In the popular RPG \textbf{Pokémon}, there are different colorations of pocket monsters often referred to as "shiny" Pokémon. In the current games, the base odds of encountering a shiny Pokémon is \( \frac{1}{4096} \). This can be optimized to \( \frac{1}{683} \) through various in-game methods.
   \begin{enumerate}[label=\textbf{\roman*)}]
    \item How many encounters are required to have a \( 50\% \) chance of encountering a shiny at the base odds of \( \frac{1}{4096} \)? 
    \item How many encounters are required to have a \( 50\% \) chance of encountering a shiny at the optimized odds of \( \frac{1}{683} \)? 
   \end{enumerate}
\end{exercise}
\begin{solution}
    We can tackle both \textbf{(i)} and \textbf{(ii)} simultaneously. Let \( p \) be the probability of encountering a shiny Pokémon in a single encounter, so \( 1-p \) is the probability of not encountering a shiny. The probability of not encountering a shiny in \( n \) independent encounters is \( (1-p)^n \). We want this to equal \( 0.5 \):
    \[ \left( 1-p \right)^{n} = 0.5 \]
    Taking the natural logarithm of both sides:
    \[ n \ln(1-p) = \ln(0.5) \]
    \[ n = \frac{\ln(0.5)}{\ln(1-p)} \]
    
    For \textbf{(i)}, with \( p = \frac{1}{4096} \):
    \[ n = \frac{\ln(0.5)}{\ln(1-\frac{1}{4096})} \approx \frac{-0.693}{-0.000244} \approx 2839 \text{ encounters} \]
    
    For \textbf{(ii)}, with \( p = \frac{1}{683} \):
    \[ n = \frac{\ln(0.5)}{\ln(1-\frac{1}{683})} \approx \frac{-0.693}{-0.00147} \approx 473 \text{ encounters} \]

This exercise shows that the median number of encounters to find a shiny at base odds is approximately 2839, not 4096. The value \( \frac{1}{4096} \) represents the probability per encounter, not the expected number of encounters for a 50\% success rate.
\end{solution}




\chapter{Trigonometric Functions}
\section{The Unit Circle}
\begin{dfn}
    The \vocab{unit circle} is the collection of all points in \( \mathbb{R}^{2} \) that are a distance of \( 1 \) from the origin. 
\end{dfn}

\begin{exercise}
    Use the above definition to derive an explicit \( x,y \) representation of the unit circle. 
\end{exercise}
\begin{solution}
    Applying the distance formula 
    \[ \boxed{d \left( \vb{p}, \vb{q} \right) = \sqrt{ \left( p_{x} - q_{x} \right)^{2} + \left( p_{y}- q_{y} \right)^{2}} }\] by setting 
    \( \vb{p} = \left( x,y \right) \), \( \vb{q} = \left( 0,0 \right) \) and \( d \left( \vb{p}, \vb{q} \right) =1 \). 
    We have 
    \[ 1 = \sqrt{ \left( x-0 \right)^{2} + \left( y-0 \right)^{2}} \]
    or 
    \[ 1 = \sqrt{x^{2} + y^{2}} \]
    and we can square both sides to get 
    \[ \boxed{x^{2}+ y^{2}=1.} \]
\end{solution}

\begin{exercise}
    Suppose that \( \ell_{(-1,0)} \) and \( \ell_{(1,0)} \) are lines passing through \( (-1,0) \) and \( (1,0) \), respectively, such that \( \ell_{(-1,0)} \) and \( \ell_{(1,0)} \) are perpendicular and neither is horizontal. Show that the set of all intersection points of such pairs \( \ell_{(-1,0)} \) and \( \ell_{(1,0)} \) forms the unit circle.
    
    \begin{center}
        \includegraphics[width=0.5\textwidth]{figures/analysis/precalculus/perpendicularlinesformcircle.png}
      
    \end{center}
\end{exercise}
\begin{solution}
    For any \( m \neq 0 \), we can write 
    \begin{align*}
        \ell_{(-1,0)} &: y = -m \left( x+1 \right) \\
        \ell_{(1,0)} &: y = \frac{1}{m} \left( x-1 \right)
    \end{align*}
  We can multiply both sides of the first equation by \( y \) since \( y \neq 0 \) to get 
  \[ y^{2} = -my \left( x+1 \right) .\]
  Similarly, since \( m \neq 0 \), we can multiply both sides of the second equation by \( m \) to get
  \[ my = (x-1) \]
  Now we can substitute this expression for \( my \)
  \begin{align*}
    y^{2} &= -my \left( x+1 \right) \\
    y^{2} &= - \left( x-1 \right) \left( x+1 \right) \\
    y^{2} &= 1-x^{2}
  \end{align*}
  So we get 
  \[ \boxed{x^{2}+y^{2}=1} \] as desired.
\end{solution}


\chapter{Sequences, Series, and Limits}
\input{content/analysis/precalculus/sequences-series-and-limits/sequences-series-and-limits}


\part{First Year Calculus}
\label{part: First Year Calculus}
\vspace*{2em} The primary resource for this part is \cite{ref:stewart}. \vspace*{1em}
\parttoc
\chapter{Functions and Limits}
\section{Representations of Functions}
\subsection{Functions}

The primary objects of study in single-variable calculus are functions of a single real variable; i.e. functions that "eat" a real number and spit out a (not necessarily different) real number. 


\section{A Catalog of Essential Functions}
\begin{example}
    Suppose a drug X has a half-life of \( 45 \) minutes in the blood stream and that it is not safe to have more than \( 175 \)mg of the drug in your blood. If Alice took a \( 120 \)mg dose, how long does she have to wait until she can take another \( 120 \)mg dose?\\
    Alice needs to wait until her current drug level drops below \( 55 \) mg (since \( 55 + 120 = 175 \)). We need to determine the time when Alice has less than \( 55 \) mg of drug X left in her system, so we have 
\[  120 \left( \frac{1}{2} \right)^{\frac{t }{45}} <55.  \]

Dividing both sides by \( 120 \), 
   \begin{equation}
     \left( \frac{1}{2}  \right)^{ \frac{t }{45}} < \frac{55}{120} = \frac{11}{24}. \label{eqn:6/21/25/1} 
\end{equation}

Taking the natural logarithm of both sides, 
\[ \frac{t}{45} \ln \left( \frac{1}{2} \right) < \ln \left( \frac{11}{24} \right). \]

Since \( \ln \left( \frac{1}{2} \right) < 0 \), dividing both sides by this negative number flips the inequality:
\[ \frac{t}{45} > \frac{\ln \left( \frac{11}{24} \right)}{ \ln \left( \frac{1}{2} \right)}. \]

Therefore,
\[ t >   45 \cdot \frac{\ln \left( \frac{11}{24} \right)}{ \ln \left( \frac{1}{2} \right)} \approx 51 \text{ minutes}. \]

Alice must wait at least 51 minutes before she can take drug X again.

\textbf{Common Mistake:} We can take a different route following from \Cref{eqn:6/21/25/1}. Namely, we might try taking \( \log_{\frac{1}{2}} \) of both sides:
\[  \frac{t}{45} \log_{\frac{1}{2}} \left( \frac{1}{2} \right) < \log_{\frac{1}{2}} \left( \frac{11}{24} \right). \]

Since \( \log_{\frac{1}{2}} \left( \frac{1}{2} \right) = 1 \), this would give us
\[ \frac{t}{45} < \log_{\frac{1}{2}} \left(  \frac{11}{24} \right), \] 
or \[ t < 45 \log_{\frac{1}{2}} \left(  \frac{11}{24} \right) \approx 51 \text{ minutes}. \]

This solution would incorrectly suggest that Alice has to take drug X \textit{before} 51 minutes, which is absurd.

\textbf{What went wrong?} The function \( f(x) = \log_{\frac{1}{2}}(x) \) is decreasing since its base \( \frac{1}{2} < 1 \). For any decreasing function \( f \), we have 
\[ x <y \iff f(x) > f(y). \]

When we applied the decreasing function \( \log_{\frac{1}{2}} \) to both sides of the inequality, we should have flipped the inequality sign:
\[ \frac{t}{45} \log_{\frac{1}{2}} \left( \frac{1}{2} \right) > \log_{\frac{1}{2}} \left( \frac{11}{24} \right), \]
which gives us the correct answer: \( t > 51 \) minutes.
\end{example}


\section{New Functions from Old}
Now that we have compiled a list of important functions, we will use them as building blocks to produce all the functions typically considered in a first year calculus course. 

\subsection{Transformations of Functions}

\begin{dfn}
    Let \( y=f(x) \) be any function and \( c >0 \), then we can apply the following transformations to \( f(x) \).
\begin{description}[style=nextline]
 \item[Upward shift:] 
  \( y = f(x) + c \) shifts the graph \textbf{up} by \( c \) units.  
  If \( (x_0, y_0) \) is on the graph of \( y = f(x) \), then \( (x_0, y_0 + c) \) is on the graph of \( y = f(x) + c \).

  \item[Downward shift:] 
  \( y = f(x) - c \) shifts the graph \textbf{down} by \( c \) units.  
  If \( (x_0, y_0) \) is on the graph of \( y = f(x) \), then \( (x_0, y_0 - c) \) is on the graph of \( y = f(x) - c \).

  \item[Left shift:] 
  \( y = f(x + c) \) shifts the graph \textbf{left} by \( c \) units.  
  If \( (x_0, y_0) \) is on the graph of \( y = f(x) \), then \( (x_0 - c, y_0) \) is on the graph of \( y = f(x + c) \).

  \item[Right shift:] 
  \( y = f(x - c) \) shifts the graph \textbf{right} by \( c \) units.  
  If \( (x_0, y_0) \) is on the graph of \( y = f(x) \), then \( (x_0 + c, y_0) \) is on the graph of \( y = f(x - c) \).

    \item[Vertical stretch:] 
  \( y = a f(x) \) \textbf{stretches} the graph vertically by a factor of \( |a| \) if \( |a| > 1 \).  
  If \( (x_0, y_0) \) lies on the graph of \( y = f(x) \), then \( (x_0, a y_0) \) lies on the graph of \( y = a f(x) \).

  \item[Vertical compression:] 
  \( y = a f(x) \) \textbf{compresses} the graph vertically if \( 0 < |a| < 1 \).  
  If \( (x_0, y_0) \) lies on the graph of \( y = f(x) \), then \( (x_0, a y_0) \) lies on the graph of \( y = a f(x) \).

  \item[Horizontal compression:] 
  \( y = f(ax) \) \textbf{compresses} the graph horizontally by a factor of \( \frac{1}{|a|} \) when \( |a| > 1 \).  
  If \( (x_0, y_0) \) lies on the graph of \( y = f(x) \), then \( \left( \frac{x_0}{a}, y_0 \right) \) lies on the graph of \( y = f(ax) \).

  \item[Horizontal stretch:] 
  \( y = f(ax) \) \textbf{stretches} the graph horizontally by a factor of \( \frac{1}{|a|} \) when \( 0 < |a| < 1 \).  
  If \( (x_0, y_0) \) lies on the graph of \( y = f(x) \), then \( \left( \frac{x_0}{a}, y_0 \right) \) lies on the graph of \( y = f(ax) \).

  \item[Reflection across \( x \)-axis:] 
  \( y = -f(x) \) reflects the graph across the \( x \)-axis.  
  If \( (x_0, y_0) \) lies on the graph of \( y = f(x) \), then \( (x_0, -y_0) \) lies on the graph of \( y = -f(x) \).

  \item[Reflection across \( y \)-axis:] 
  \( y = f(-x) \) reflects the graph across the \( y \)-axis.  
  If \( (x_0, y_0) \) lies on the graph of \( y = f(x) \), then \( (-x_0, y_0) \) lies on the graph of \( y = f(-x) \).


\end{description}
\end{dfn}

\begin{dfn}
Because functions output real numbers, we can combine them using familiar algebraic operations. Given two functions \( f(x) \) and \( g(x) \), we define:

\begin{description}[style=nextline, leftmargin=3.2cm]

  \item[Addition:] 
  \( (f + g)(x) := f(x) + g(x) \)

  \item[Subtraction:] 
  \( (f - g)(x) := f(x) - g(x) \)

  \item[Multiplication:] 
  \( (f \cdot g)(x) := f(x) \cdot g(x) \)

  \item[Division:] 
  \( \left( \frac{f}{g} \right)(x) := \frac{f(x)}{g(x)} \quad \) (as long as \( g(x) \ne 0 \))

\end{description}
\end{dfn}

\vspace{1em}

Functions also support a new operation that is unique and fundamental to calculus: \vocab{function composition}.

\begin{dfn}[Function Composition]
Given two functions \( f(x) \) and \( g(x) \), the \textbf{composition} of \( f \) with \( g \), written as \( (f \circ g)(x) \), means:
\[
(f \circ g)(x) := f(g(x))
\]
In words, we first evaluate \( g(x) \), and then plug that result into \( f \).

\vspace{0.5em}
\textbf{Important:} Composition is generally \emph{not commutative}, meaning \( f(g(x)) \ne g(f(x)) \) in general.
\end{dfn}


\begin{exercise}
  Suppose that \( g(x) =x-1 \) and that \( f(x) = g(x^{2}-1) \). Find all \( x \) such that 
  \[ \left( f \circ g \right)(x) = \left( g \circ f  \right)(x) \]
\end{exercise}
\begin{solution}
    We can explicitly find \( f(x) \) 
    \begin{align*}
          f(x) &= g(x^{2}-1) \\
          &= \left[ x^{2}-1 \right]-1 \\
          f(x) &= x^{2} -2
    \end{align*}
    Now to find \( \left( f \circ g \right)(x) \)
    \begin{align*}
      \left( f \circ g \right)(x)  &= \left( x-1 \right)^{2} -2\\
      &= x^{2}-2x+1 -2 \\
      \left( f \circ g \right)(x) &= x^{2}-2x-1
    \end{align*}
    For \( \left( g \circ f \right)(x) \) 
    \begin{align*}
       \left( g \circ f \right)(x)  &= \left( x^{2}-2 \right)-1 
        \left( g \circ f \right)(x) &= x^{2}-3
    \end{align*}
    Finally, 
    \begin{align*}
       \left( g \circ f \right)(x)  &=  \left( f \circ g \right)(x) \\
       x^{2}-3 &= x^{2}-2x-1 \\
       -3 &=-2x -1 \\
       -2 &= -2x \\
       1 &= x
    \end{align*}
    
\end{solution}

\begin{example}
  Let \( f(x)= \dfrac{b}{x-a}+a \). Find \( (f \circ f)(x) \), that is, compute \( f(f(x)) \).
  
  \vspace{1em}
  First, recall that composition means we evaluate:
  \[
  (f \circ f)(x) = f(f(x))
  \]

  Step 1: Start with the inner function:
  \[
  f(x) = \frac{b}{x - a} + a
  \]

  Step 2: Plug this into the outer \( f \). That is, replace every \( x \) in \( f(x) \) with \( f(x) \):
  \[
  f(f(x)) = \frac{b}{\left( \frac{b}{x - a} + a \right) - a} + a
  \]

  \vspace{0.5em}
  Step 3: Simplify the expression inside the denominator:
  \[
  f(f(x)) = \frac{b}{\frac{b}{x - a}} + a
  \]

  Step 4: Invert the inner fraction:
  \[
  \frac{b}{\frac{b}{x - a}} = x - a
  \]

  So:
  \[
  f(f(x)) = (x - a) + a = x
  \]

  \vspace{0.5em}
 \( (f \circ f)(x) = \boxed{x} \)
\end{example}

In the above example, we can say \( f(x) \) is its own \vocab{inverse}. 
\begin{dfn}
Given a function \( f(x) \), another function \( g(y) \) is called the \vocab{inverse function} of \( f(x) \) if each undoes the effect of the other:
\[
g(f(x)) = x \quad \text{for all } x \text{ in the domain of } f, 
\quad \text{and} \quad
f(g(y)) = y \quad \text{for all } y \text{ in the domain of } g.
\]
We often write \( g = f^{-1} \).
\end{dfn}

A function is invertible if and only if it passes the \textbf{Horizontal Line Test}, or equivalently, if it is \textbf{strictly monotonic} (either strictly increasing or strictly decreasing) on its domain.

\begin{example}
  Let us test (without trying to find an explicit inverse) if the function 
  \[ f(x) = x^{3}+ x- 4 \]
  has an inverse. \\ 
  Let \( h \neq 0 \), if \( f(x) \) is invertible then 
  \( f(x+ h) - f(x) \) cannot be zero. 
  \begin{align*}
    f(x + h) - f(x) &= \left[ \left( x+ h \right)^{3} + \left( x +h \right) -4\right] - \left[ x^{3} +x -4 \right] \\
    &= \left[ \left( x+h  \right)^{3} - x^{3} \right] + \left[ \left( x+h \right) -x \right] + \left[ -4 - \left( -4 \right) \right] \\
    &= \left( x+h -x \right)\left( \left( x+h \right)^{2}  +x \left( x+h \right) +x^{2 }\right) +h \tag{Applying difference of cubes.} \\
    &= h \left( 3x^{2} + 2hx + h^{2} \right) +h
  \end{align*}
  It can be easily verified that when \( h \neq 0 \), the quadratic \( 3x^{2} +2hx + h^{2} \ge 0 \) for every \( x \in \mathbb{R} \) (Setting \(a = 3, b =2h, c =h^{2}\) gives you a negative discriminant.) Since \( f(x+h) - f(x) \) is the sum of positive terms (if \( h \) is positive) or the sum of negatives terms (if \( h  \) is negative), \( f(x+h)- f(x) \neq 0 \) every \( x \in \mathbb{R} \) and \( h \neq 0 \). As such, \( f^{-1} (x) \) exists. \\ 
  Now suppose we wanted to find \( f^{-1} (2) \). Again, we don't need to calculate an explicit inverse. If \( f^{-1}(2)=a \) then, by definition, \( f(a) =-2 \) So 
  \begin{align*}
    f(a) &= -2 \\
    a^{3} + a -4 &= -2 \\
    a^{3} + a =2 
  \end{align*}
We can see that \( f(1) =-2 \) so \(  f^{-1} (-2) = 1 \).
\end{example}

\begin{exercise}
  Suppose that \( f: \left( -1,1  \right) \to \mathbb{R} \) be defined by 
  \[ f(x) = \frac{x}{1-x^{2}} .\] 
  Find \( f^{-1} \).
\end{exercise}
\begin{solution}
    We apply the algorithm of exchanging \( y \) and \( x \)
    \[ x = \frac{y}{1-y^{2}} .\]
    
    

    
\end{solution}



\section{The Tangent and Velocity Problems}
\input{content/analysis/first-year-calculus/1-functions-and-limits/tangent-and-velocity-problems}

\section{Limits}
\begin{lemma}
    \( \lim_{h \to 0} \frac{e^{h}-1}{h} =1\)
\end{lemma}
\begin{proof}
    Since \( \left( 1 + \frac{h }{n} \right)^{n} < e^{h} < \left( 1 + \frac{ h }{n } \right)^{n+1} \) for every natural number \( n \) and real number \( h \), we have 
  \begin{equation}\label{eq:e-bound-ineq-pos}
\frac{\left( 1 + \tfrac{h}{n} \right)^{n}}{h} 
   < \frac{e^{h} - 1}{h} 
   < \frac{\left( 1 + \tfrac{h}{n} \right)^{n+1} - 1}{h},
   \quad \text{if } h > 0.
\end{equation}

\begin{equation}\label{eq:e-bound-ineq-neg}
\frac{\left( 1 + \tfrac{h}{n} \right)^{n}}{h} 
   > \frac{e^{h} - 1}{h} 
   > \frac{\left( 1 + \tfrac{h}{n} \right)^{n+1} - 1}{h},
   \quad \text{if } h < 0.
\end{equation}

    First looking at \( \frac{\left( 1 + \frac{h }{n} \right)^{n} }{h}  \), we have 
    \begin{align*}
       \frac{\left( 1 + \frac{h }{n} \right)^{n} }{h}  &= \frac{\displaystyle \sum_{j=0}^{n} {n \choose j} \left( \frac{h }{n} \right)^{j} -1 }{h} \\
       &= \frac{\displaystyle \sum_{j=1}^{n} {n \choose j} \left( \frac{h }{n} \right)^{j} }{h} \\
       &=  \sum_{j=1}^{n} {n \choose j} \frac{h^{j-1}}{n^{j}}\\
       &= {n \choose 1} \frac{1}{n} + \sum_{j=2}^{n} {n \choose j} \frac{h^{j-1}}{n^{j}} \\
       &= 1 +  \sum_{j=2}^{n} {n \choose j} \frac{h^{j-1}}{n^{j}}
    \end{align*}
    So taking the limit as \( h \to 0 \), we have 
    \begin{align*}
         \lim_{ h \to 0}  \frac{\left( 1 + \frac{h }{n} \right)^{n} }{h} &= \lim_{h \to 0 } \left(  1 +  \sum_{j=2}^{n} {n \choose j} \frac{h^{j-1}}{n^{j}} \right) \\
         &=1
    \end{align*}
For \( \frac{\left( 1 + \tfrac{h}{n} \right)^{n+1} - 1}{h} \), we will similarly get 
\begin{align*}
         \lim_{ h \to 0}  \frac{\left( 1 + \frac{h }{n} \right)^{n+1} -1}{h} &= \lim_{h \to 0 } \left(  1 + \sum_{j=2}^{n+1} {n+1 \choose j} \frac{h^{j-1}}{n^{j}} \right) \\
         &=1
    \end{align*}
    Applying the squeeze theorem to inequalities \eqref{eq:e-bound-ineq-pos} and \eqref{eq:e-bound-ineq-neg}, we see that
    \[ \boxed{\lim_{h \to 0} \frac{e^{h}-1}{h} =1}. \]
\end{proof}

\begin{exercise}
    Find 
    \( \lim_{x \to \infty} \left( x - x \cos{ \left( \frac{1}{\sqrt{x}} \right) } \right) \sin{ \left( \frac{1}{\sqrt{x}} \right) }\).
\end{exercise}
\begin{solution}
    Let \( t = \frac{1}{\sqrt{x}} \) then 
    \begin{align*}
     \lim_{x \to \infty} \left( x - x \cos{ \left( \frac{1}{\sqrt{x}} \right) } \right) \sin{ \left( \frac{1}{\sqrt{x}} \right) } &= \lim_{t \to 0+} \left( \frac{1}{t^{2}} - \frac{1}{t^{2}} \cos{ \left( t \right) } \right) \sin{ \left( t \right) } \\
     &= \lim_{t \to 0+} \frac{ \sin{ \left( t  \right) } \left( 1- \cos{ \left( t \right) } \right)}{t^{2}} \\
     &= \left[ \lim_{t \to 0+}\frac{ \sin{ \left( t \right) }}{t} \right] \left[ \lim_{t \to 0^{+}}  \frac{1- \cos{ \left( t \right) }}{t}\right] \\
     &= 1 \cdot 0
    \end{align*}
    So 
    \[ \boxed{\lim_{x \to \infty} \left( x - x \cos{ \left( \frac{1}{\sqrt{x}} \right) } \right) \sin{ \left( \frac{1}{\sqrt{x}} \right) } =0 \text{ .}} \]
\end{solution}

\subsection{Sequences} 

\begin{dfn}
    By a \vocab{sequence}, we mean a function 
    \[
        a: \{ 1, 2, \dots, n, \dots \} \to \mathbb{R},
    \]
    which we usually write as 
    \[
        \{ a_{1}, a_{2}, \dots, a_{n}, \dots \},
    \]
    where
    \begin{align*}
        a_{1} &= a(1), \\
        a_{2} &= a(2), \\
              &\ \ \vdots \\
        a_{n} &= a(n), \\
              &\ \ \vdots
    \end{align*}
    We say that a sequence \( \{ a_n \} \) is \vocab{convergent}, or that it \vocab{converges to the real number} \( L \), if for every \( \epsilon > 0 \) there exists a sufficiently large \( N \) such that whenever \( n > N \),
    \[
        \abs{a_{n} - L} < \epsilon.
    \]
    A sequence is \vocab{divergent}, or said to \vocab{diverge}, if no such \( L \) exists.
\end{dfn}

\begin{example}
    Suppose a sequence is given by \( a_{n} = \frac{2n-1}{n} \). Writing out the first members of this sequence, we have 
    \begin{align*}
        a_{1} &= \frac{2(1)-1}{\left( 1 \right)} = 1 \\
        a_{2} &= \frac{2(2)-1}{(2)} = \frac{3}{2} \\
        a_{3} &= \frac{2(3)-1}{(3)} = \frac{5}{3} \\
        a_{4} &= \frac{2(4)-1}{4} = \frac{7}{4} \\
        & \ \ \vdots
    \end{align*}
    Notice that the difference between each successive terms decreases. Indeed if we define \( b_{n} = a_{n+1} -a_{n}\) then 
    \begin{align*}
        b_{1} &= a_{2}- a_{1} = \frac{3}{2}-1 = \frac{1}{2} \\
        b_{2} &= a_{3} - a_{2} = \frac{5}{3} - \frac{3}{2} = \frac{1}{6} \\
        b_{3} &= a_{4} - a_{3} = \frac{7}{4} - \frac{5}{3} = \frac{1}{12} \\\
        &\ \ \vdots
    \end{align*}
    While the tendency for \( b_{n} \) to go zero does not itself guarantee convergence, it is a good sign and indeed \( a_{n} \) converges. We can write 
    \[ a_{n} = \frac{2n}{n} - \frac{1}{n} = 2 - \frac{1}{n} \text{ .}\]
    This suggests that \( a_{n} \) converges to \( 2 \). To show this, we can pick any \( \epsilon >0 \) and can pick an \( n \) sufficiently large such that \( n \epsilon > 1 \) or \( \frac{1}{n} < \epsilon \) Then 
    \begin{align*}
        \abs{2 - a_{n}} &= \abs{ 2 - \left( 2- \frac{1}{n} \right)} \\\
        &= \abs{\frac{1}{n}} \\
        &= \frac{1}{n } \\
        &< \epsilon
    \end{align*}
    We can do something similar to show that \( b_{n} \) converges to \( 0 \) since 
    \begin{align*}
        b_{n} &= a_{n+1} - a_{n} \\
        &= \frac{2 (n+1) -1}{n+1} - \frac{2n -1}{n} \\
        &= \frac{n \left( 2n+1  \right) - \left( n+1  \right)(2n-1)}{n(n+1)} \\
        &= \frac{1}{n(n+1)}
    \end{align*}
    As such the same \( n \) chosen for \( a_{n} \) is overkill since 
    \begin{align*}
        \abs{0 - b_{n}} &= \abs{-\frac{1}{n(n+1)}} \\
        &= \frac{1}{n(n+1)} \\
        &= \frac{1}{n+1} \frac{1}{n} \\
        & < \frac{1}{n^{2}}\\
        & < \epsilon^{2} \\
        &< \epsilon \tag{If $\epsilon < 1$}
    \end{align*}
    So \( b_{n} \) converges "faster" than \( a_{n} \).
\end{example}

\begin{example}
    Let 
    \[ H_{n} = 1 + \frac{1}{2} + \frac{1}{3} + \cdots + \frac{1}{n} = \sum_{k=1}^{n} \frac{1}{n}  \]
    This is called the \vocab{harmonic series}. 
    Writing out the first couple of terms,we have 
  \begin{alignat*}{2}
    H_{1} &= 1 \qquad & &= 1 \\
    H_{2} &= 1 + \frac{1}{2} & &= \frac{3}{2} \\
    H_{3} &= 1 + \frac{1}{2} + \frac{1}{3} & &= \frac{11}{6} \\
    H_{4} &=  1 + \frac{1}{2} + \frac{1}{3} + \frac{1}{4} & &= \frac{25}{12} \\
    H_{5} &= 1 + \frac{1}{2} + \frac{1}{3} + \frac{1}{4}  & &= \frac{137}{60} \\
    & \ \ \vdots & & \ \ \vdots
\end{alignat*}
    If we define \( I_{n}= H_{n+1}-H_{n} \), it is clear that \( I_{n} = \frac{1}{n+1} \), which converges to \( 0 \). But it is \emph{not} the case that \( H_{n} \) converge to any real number \( L \). To see this, 
    \[ 1 + \frac{1}{2} + \underbrace{\frac{1}{3} + \frac{1}{4}}_{\displaystyle >  \frac{1}{4} + \frac{1}{4} = \frac{1}{2}} + \underbrace{ \frac{1}{5} + \frac{1}{6} + \frac{1}{7} + \frac{1}{8}}_{\displaystyle > \frac{1}{8} + \frac{1}{8} + \frac{1}{8} + \frac{1}{8} = \frac{1}{2}} + \cdots \]
    This generalizes 
    \[ 1 + \sum_{k=0}^{n} \frac{1}{2} < H_{2^{n+1}} \]
    This shows that the harmonic series diverges (albeit very slowly) since I can make \( H_{n} \) as large as I want.
\end{example}

\begin{exercise}
    The sequence 
    \[ x_{n} = \frac{4n^{3} -n^{2} +5n}{2n^{3} + 6n^{2} -11} \]
    converges. Find its limit.
\end{exercise}
\begin{solution}
    We can multiply the top and bottom by \( \frac{1}{n^{3}} \) so 
    \begin{align*}
        \frac{4n^{3} -n^{2} +5n}{2n^{3} + 6n^{2} -11} \cdot \frac{ \displaystyle\frac{1}{n^{3}} }{ \displaystyle\frac{1}{n^{3}}} &= \frac{ \displaystyle\frac{4n^{3}}{n^{3}} -  \frac{n^{2}}{n^{3}} +  \frac{5n}{n^{3}}}{ \displaystyle\frac{2n^{3}}{n^{3}} +  \frac{6n^{2}}{n^{3}} -  \frac{11}{n^{3}}} \\
        &= \frac{\displaystyle 4 - \frac{1}{n } + \frac{5}{n^{2}}}{\displaystyle2 + \frac{6}{n} - \frac{11}{n^{3}}}
    \end{align*}
    So as \( n \to \infty \), the terms with \( n \) in the denominator get arbitrarily small so we can just ignore them so then all we have is 
    \[ \boxed{\lim_{n \to \infty}  \frac{4n^{3} -n^{2} +5n}{2n^{3} + 6n^{2} -11} = 2 \text{ .}}\]
\end{solution}

\begin{exercise}
    Find the limit of the sequence 
    \[ x_{n} = \sqrt{n + k} - \sqrt{n}  \] where \( k \) is a fixed constant. 
\end{exercise}
\begin{solution}
    We have 
    \begin{align*}
        \sqrt{n+k} - \sqrt{n} \cdot \frac{\sqrt{n+k} + \sqrt{n}}{\sqrt{n+k} + \sqrt{n}} &= \frac{n+k -n}{\sqrt{n+k} + \sqrt{n}} \\
        &= \frac{k}{ \sqrt{n+k} + \sqrt{n}}
    \end{align*}
    Since \( n \) can be made arbitrarily large and \( k \) is fixed, we have 
    \[ \boxed{\lim_{n \to \infty}\sqrt{n + k} - \sqrt{n} = 0 \text{ .}} \]
\end{solution}

\begin{exercise}
    Show that the sequence 
    \[ x_{n} = \frac{1 \cdot 3 \cdot 5 \cdots (2n-1)}{2 \cdot 4 \cdot 6 \cdots (2n)} \]
    converges.
\end{exercise}
\begin{solution}
    Notice that \( x_{n} < x_{n-1} \) for every \( n \ge 2 \) since 
    \[ x_{n} = x_{n-1} \cdot \frac{2n-1}{2n} \text{ .}\]
    Since the sequence is decreasing and bounded below by \( 0 \), it has no choice but to converge. 
\end{solution}

\begin{exercise}
    Show that the sequence 
    \[ x_{n} = \left( -1  \right)^{n} \frac{1}{n^{2}} \]
    converges.
\end{exercise}
\begin{solution}
    For any \( \epsilon > 0 \), we can choose a sufficiently large \( n \) such that 
    \[ \abs{\frac{1}{n^{2}}} < \epsilon \]
    the \( (-1)^{n} \) does not change this. 
\end{solution}



\section{Continuity}
\subsection{The Intermediate Value Theorem}

We now state the Intermediate Value Theorem.\\[6pt]

\textbf{The Intermediate Value Theorem.}
If \( f : [a,b] \to \mathbb{R} \) is continuous and \( N \) is any number between \( f(a) \) and \( f(b) \),
then there exists at least one \( c \in [a,b] \) such that \( f(c) = N \).

\begin{lemma}
    Suppose that \( f : [a,b] \to \mathbb{R} \) is continuous and that 
    \( f([a,b]) \subseteq [a,b] \)
    (that is, \( f \) maps the interval into itself). 
    Then there exists at least one \( c \in [a,b] \) such that \( f(c) = c \).
\end{lemma}
\begin{proof}
    If \( f(a)=a \) or \( f(b) =b \), we are done. So let us assume otherwise. In particular, this means that \( f(a) >a \) and \( f(b) <b \). Define a function 
    \[ g(x):= f(x)-x .\]
    \( g(x) \) is continuous as it is the sum of continuous functions. We also have that 
    \[ g(a) = f(a)-a  >0 \quad \text{and} \quad g(b) = f(b)-b <0 .\]
    By the Intermediate Value Theorem, there must be a \( c \in \left( a,b \right) \) such that \( g(c) =0 = f(c)-c \) and hence, \( f(c)=c \).
\end{proof}


\begin{exercise}
  Let 
  \[ f(x) = \sqrt{ \frac{x+1}{ \abs{2x-1}}} .\]
  \begin{enumerate}[label=\textbf{\roman*)}]
    \item Find the domain of \( f(x) \). 
    \item Determine which values of \( x \) (if any) are fixed points of \( f(x) \). That is, find all \( x \) such that \( f(x) = x \).
  \end{enumerate}
\end{exercise}
\begin{solution} $ $
  \begin{enumerate}[label=\textbf{\roman*)}]
    \item $ $\\ 
      We require that the denominator be non-zero, so 
      \[ \abs{2x-1} = 0 \iff 2x - 1 = 0. \]
      Thus \( x = \frac{1}{2} \) is not in the domain of \( f(x) \). 
      Next, we require that the expression inside the square root be non-negative. Since \( \abs{2x-1} \) is always non-negative, we just need to check where \( x + 1 \ge 0 \), that is, 
      \[ x + 1 \ge 0 \Rightarrow x \ge -1. \]
      Therefore, the domain of \( f(x) \) is 
      \[
        \left[ -1, \frac{1}{2} \right) \cup \left( \frac{1}{2}, \infty \right).
      \]
    \item $ $\\ 
      We need to solve \( f(x) = x \). Squaring both sides gives
      \[ \frac{x+1}{\abs{2x-1}} = x^{2}. \]
      Because of the absolute value, we must solve two equations:
      \[ 
        \frac{x+1}{2x-1} = x^{2} 
        \quad \text{or} \quad 
        \frac{x+1}{1-2x} = x^{2}. 
      \]
      Equivalently,
      \[ 
        2x^{3}-x^{2}-x-1 = 0 
        \quad \text{or} \quad 
        2x^{3}-x^{2}+x+1 = 0. 
      \]
  \end{enumerate}
  By the rational root theorem, the possible rational roots are \( \pm 1 \) and \( \pm \tfrac{1}{2} \). In the context of this problem, we eliminate \( x = \tfrac{1}{2} \). Testing each remaining rational root, we find that \( x = -\tfrac{1}{2} \) satisfies the second equation. However, since the range of \( f(x) \) is non-negative, \( x = -\tfrac{1}{2} \) cannot be a fixed point. 

  Nonetheless, this shows that \( 2x + 1 \) is a factor of \( 2x^{3} - x^{2} + x + 1 \). Indeed, 
  \[
    2x^{3} - x^{2} + x + 1 = (2x + 1)(x^{2} - x + 1),
  \]
  and \( x^{2} - x + 1 \) has no real roots. Therefore, that equation yields no real fixed points. 

  Now let us see whether the other equation, \( 2x^{3} - x^{2} - x - 1 = 0 \), can yield one. Plugging in \( x = 1 \) gives 
  \[ g(1) = 2 - 1 - 1 - 1 = -1 < 0, \]
  and \( x = 2 \) gives 
  \[ g(2) = 16 - 4 - 2 - 1 = 9 > 0. \]
  Hence, by the Intermediate Value Theorem, there exists \( \lambda \in (1, 2) \) such that \( g(\lambda) = 0 \), and thus \( f(\lambda) = \lambda \).
\end{solution}




\chapter{Derivatives}
\section{Rates of Change}
\begin{example}\label{example: average velocity 1}
Suppose the distance you have traveled (as a function of time) is given by \( y = t^2 \). What is your average velocity between \( t = 1 \) second and \( t = 3 \) seconds?

At time \( t = 1 \), you have traveled \( y = 1^2 = 1 \) unit, and at \( t = 3 \), you have traveled \( y = 3^2 = 9 \) units. The average velocity over this time interval is
\begin{align*}
\text{average velocity} &= \frac{\Delta \text{distance}}{\Delta \text{time}} \\
&= \frac{9 - 1}{3 - 1} \, \frac{\text{units}}{\text{second}} \\
&= \frac{8}{2} \, \frac{\text{units}}{\text{second}} = 4 \, \frac{\text{units}}{\text{second}}.
\end{align*}
Therefore, the average velocity is
\[
\boxed{\text{average velocity} = 4 \, \frac{\text{units}}{\text{second}}}.
\]

Now suppose instead we want to find the average velocity between \( t = 1 \) second and \( t = 2 \) seconds.

At time \( t = 2 \), the distance traveled is \( y = 2^2 = 4 \) units. Then the average velocity is
\begin{align*}
\text{average velocity} &= \frac{\Delta \text{distance}}{\Delta \text{time}} \\
&= \frac{4 - 1}{2 - 1} \, \frac{\text{units}}{\text{second}} \\
&= \frac{3}{1} \, \frac{\text{units}}{\text{second}} = 3 \, \frac{\text{units}}{\text{second}}.
\end{align*}
Therefore, the average velocity over this shorter interval is
\[
\boxed{\text{average velocity} = 3 \, \frac{\text{units}}{\text{second}}}.
\]
\end{example}


\begin{example}\label{example: average velocity 2}
Now suppose the distance you have traveled is given by \( y = t^3 + 1 \). If you start a timer at \( t_0 = 2 \) seconds and end the timer 3 seconds later, what is your average velocity over that period?

The timer runs from \( t = 2 \) to \( t = 5 \). We compute the total change in distance over this interval:

\[
y(2) = 2^3 + 1 = 8 + 1 = 9, \quad y(5) = 5^3 + 1 = 125 + 1 = 126.
\]

Now compute the average velocity:

\begin{align*}
\text{average velocity} &= \frac{\Delta \text{distance}}{\Delta \text{time}} \\
&= \frac{126 - 9}{5 - 2} \, \frac{\text{units}}{\text{second}} \\
&= \frac{117}{3} \, \frac{\text{units}}{\text{second}} = 39 \, \frac{\text{units}}{\text{second}}.
\end{align*}

Therefore, the average velocity is
\[
\boxed{\text{average velocity} = 39 \, \frac{\text{units}}{\text{second}}}.
\]

Now suppose instead that the timer runs for only 1 second, from \( t = 2 \) to \( t = 3 \).

We compute the change in distance over this shorter interval:

\[
y(2) = 2^3 + 1 = 9, \quad y(3) = 3^3 + 1 = 27 + 1 = 28.
\]

Then the average velocity is
\begin{align*}
\text{average velocity} &= \frac{\Delta \text{distance}}{\Delta \text{time}} \\
&= \frac{28 - 9}{3 - 2} \, \frac{\text{units}}{\text{second}} \\
&= \frac{19}{1} \, \frac{\text{units}}{\text{second}} = 19 \, \frac{\text{units}}{\text{second}}.
\end{align*}

Therefore, the average velocity over this shorter interval is
\[
\boxed{\text{average velocity} = 19 \, \frac{\text{units}}{\text{second}}}.
\]
\end{example}




Given a function \( y = f(x) \) and two points on its graph, \( (a, f(a)) \) and \( (b, f(b)) \), we have two common approaches to finding the slope of the secant line connecting them. By taking an appropriate limit, we can then obtain the slope of the tangent line at a point.

In the first approach—illustrated in \Cref{example: average velocity 1}, we treat \( a \) as fixed and let the second point \( b \) move closer to \( a \). The slope of the secant line is given by
\[
m_{\mathrm{secant}} = \frac{f(b) - f(a)}{b - a},
\]
and the slope of the tangent line is defined as the limit:
\[
m_{\mathrm{tangent}} = \lim_{b \to a} m_{\mathrm{secant}} = \lim_{b \to a} \frac{f(b) - f(a)}{b - a}.
\]

In the second approach, used in \Cref{example: average velocity 2}, we express the second point as a displacement \( h \) from \( a \), so that \( b = a + h \). The secant slope becomes
\[
m_{\mathrm{secant}} = \frac{f(a + h) - f(a)}{(a + h) - a} = \frac{f(a + h) - f(a)}{h},
\]
and again, we define the tangent slope as the limit:
\[
m_{\mathrm{tangent}} = \lim_{h \to 0} m_{\mathrm{secant}} = \lim_{h \to 0} \frac{f(a + h) - f(a)}{h}.
\]

This motivates the following crucial definition. 

\begin{dfn}
   Let \( f \) be a function. If the following limit exists at \( x = a \),
   \[
      \lim_{h \to 0} \frac{f(a + h) - f(a)}{h},
   \]
   then this limit is called \vocab{the derivative of \( f \) at \( x = a \)}, and we denote it by \( f'(a) \).  
   \par
   The \vocab{derivative function} \( f'(x) \) is defined by
   \[
      f'(x) = \lim_{h \to 0} \frac{f(x + h) - f(x)}{h},
   \]
   for all \( x \) at which this limit exists.
\end{dfn}


\section{Derivative Rules}
\subsection{Non-negative Integer Powers}

We will derive the derivative rules for functions of the form \( x^n \), where \( n = 0, 1, 2, \dots \).

The simplest case is \( y = 1 \), an example of a constant function:
\begin{align*}
f'(a) &= \lim_{h \to 0} \frac{f(a + h) - f(a)}{h} \\
&= \lim_{h \to 0} \frac{1 - 1}{h} \tag*{Since \( f(a) = f(a+h) = 1 \)} \\
&= \lim_{h \to 0} 0 \\
&= 0
\end{align*}

So,
\begin{equation}\label{eqn: Derivative of 1}
\boxed{\dv{x}\left[ 1 \right] = 0}
\end{equation}

\begin{theorem}[The Power Rule for Derivatives]
   For any \( n \in \mathbb{N} \), we have 
   \[ \dv{x}\left[ x^{n} \right] = n \cdot x^{n-1} \]
\end{theorem}
\begin{proof}
    We have taken care of the case \( n=0 \) earlier. 
    Now, if \( n \neq 0 \), consider \( f(x) = x^n \). Then:
\begin{align*}
f'(a) &= \lim_{h \to 0} \frac{f(a + h) - f(a)}{h} \\
&= \lim_{h \to 0} \frac{(a + h)^n - a^n}{h} \tag*{Substitute \( f(x) = x^n \)} \\
&= \lim_{h \to 0} \frac{\displaystyle \sum_{j = 0}^n {n \choose j} a^{n - j} h^j - a^n}{h} \tag*{Apply the binomial expansion to \( (a + h)^n \)} \\
&= \lim_{h \to 0} \frac{\displaystyle \sum_{j = 1}^n {n \choose j} a^{n - j} h^j}{h} \tag*{Cancel the \( a^n \) terms} \\
&= \lim_{h \to 0} \sum_{j = 1}^n {n \choose j} a^{n - j} h^{j - 1} \tag*{Factor out \( \frac{1}{h} \)} \\
&= \lim_{h \to 0} \left[ {n \choose 1} a^{n - 1} + \sum_{j = 2}^n {n \choose j} a^{n - j} h^{j - 1} \right] \tag*{Separate the \( j = 1 \) term} \\
&= {n \choose 1} a^{n - 1} + 0 \tag*{Higher-order terms vanish as \( h \to 0 \)} \\
&= n \cdot a^{n - 1} \tag*{Since \( {n \choose 1} = n \)}
\end{align*}

Therefore,
\begin{equation}\label{eqn: Derivative of power function}
\boxed{\dv{x}\left[ x^n \right] = n \cdot x^{n - 1}}
\end{equation}
\end{proof}




\subsection{Linearity Properties of the Derivative}

The derivative satisfies two important linearity properties:
\[
\dv{x} \left[ c \cdot f(x) \right] = c \cdot \dv{x} \left[ f(x) \right] \quad \text{and} \quad
\dv{x} \left[ f(x) + g(x) \right] = \dv{x} \left[ f(x) \right] + \dv{x} \left[ g(x) \right].
\]
For every \( c \in \mathbb{R} \) and differentiable functions \( f(x) \) and \( g(x) \).
As always in mathematics, we require proof.

\medskip



\begin{align*}
\dv{x} \left[ c \cdot f(x) \right] 
&= \lim_{h \to 0} \frac{c \cdot f(x + h) - c \cdot f(x)}{h} \tag*{Substitute into the definition of the derivative} \\
&= \lim_{h \to 0} c \cdot \frac{f(x + h) - f(x)}{h} \tag*{Factor out the constant \( c \)} \\
&= c \cdot \lim_{h \to 0} \frac{f(x + h) - f(x)}{h} \tag*{Limit of a constant times a function is the constant times the limit} \\
&= c \cdot \dv{x} \left[ f(x) \right].
\end{align*}

\medskip



\begin{align*}
\dv{x} \left[ f(x) + g(x) \right] 
&= \lim_{h \to 0} \frac{\left[ f(x + h) + g(x + h) \right] - \left[ f(x) + g(x) \right]}{h} \tag*{Apply the definition of the derivative to the sum} \\
&= \lim_{h \to 0} \frac{f(x + h) - f(x) + g(x + h) - g(x)}{h} \tag*{Group terms by function} \\
&= \lim_{h \to 0} \left( \frac{f(x + h) - f(x)}{h} + \frac{g(x + h) - g(x)}{h} \right) \tag*{Split the single fraction into two terms} \\
&= \lim_{h \to 0} \frac{f(x + h) - f(x)}{h} + \lim_{h \to 0} \frac{g(x + h) - g(x)}{h} \tag*{Limit of a sum is the sum of the limits (when both exist)} \\
&= \dv{x} \left[ f(x) \right] + \dv{x} \left[ g(x) \right].
\end{align*}

With this new rule in our tool belt, we are now capable of taking derivatives of any polynomial function. 

\subsection{Derivatives of Products and Quotients of Functions}

It is \textbf{not} true that
\[
\dv{x} \left[ f(x) \cdot g(x) \right] = \dv{x} \left[ f(x) \right] \cdot \dv{x} \left[ g(x) \right]
\quad \text{or} \quad
\dv{x} \left[ \frac{f(x)}{g(x)} \right] = \frac{\dv{x} \left[ f(x) \right]}{\dv{x} \left[ g(x) \right]}.
\]

In fact, such rules would contradict the linearity of the derivative. For example, take any constant \( c \neq 0 \). Then,
\begin{align*}
\dv{x} \left[ c \cdot f(x) \right] &= \dv{x} \left[ c \right] \cdot \dv{x} \left[ f(x) \right] \\
&= 0 \cdot \dv{x} \left[ f(x) \right] = 0,
\end{align*}
which contradicts the established rule:
\[
\dv{x} \left[ c \cdot f(x) \right] = c \cdot \dv{x} \left[ f(x) \right].
\]

Instead, the correct rules are derived as follows.

\begin{theorem}[The Product Rule for Derivatives]\label{thrm: Product Rule}
 Let \( f \) and \( g \) be differentiable functions. Then   \[ \dv{x} \left[ f(x)g(x) \right] = f'(x)g(x) + f(x)g'(x) \]
\end{theorem}
\begin{proof}
    \begin{align*}
\dv{x} \left[ f(x) \cdot g(x) \right] 
&= \lim_{h \to 0} \frac{f(x+h)g(x+h) - f(x)g(x)}{h} \tag*{Apply the definition of the derivative} \\
&= \lim_{h \to 0} \frac{f(x+h)g(x+h) - f(x)g(x+h) + f(x)g(x+h) - f(x)g(x)}{h} \tag*{Add and subtract \( f(x)g(x+h) \)} \\
&= \lim_{h \to 0} \left[
\frac{f(x+h) - f(x)}{h} \cdot g(x+h)
+ f(x) \cdot \frac{g(x+h) - g(x)}{h}
\right] \tag*{Group and factor each difference} \\
&= \dv{x} \left[ f(x) \right] \cdot g(x) + f(x) \cdot \dv{x} \left[ g(x) \right]. \tag*{Take the limit of each term separately}
\end{align*}
\end{proof}

\begin{exercise}
Prove from the product rule that 
\[ \dv{x} \left[ f(x)^n \right] = n \cdot f(x)^{n-1} \cdot f'(x). \]
\end{exercise}
\begin{solution}
We will prove this by induction. The case $n=1$ is trivial so, for practice, let us set the base case equal to $n=2$. So 
\begin{align*}
\dv{x} \left[ f(x)^2 \right] &= \dv{x} \left[ f(x) \cdot f(x) \right]\\
&= f'(x) \cdot f(x) + f(x) \cdot f'(x) \tag{By \nameref{thrm: Product Rule}}\\
&= 2 \cdot f'(x) \cdot f(x)
\end{align*}

Now for the inductive step, suppose that we have shown the result up to $n=k$. Then for the case $n=k+1$, we have 
\begin{align*}
\dv{x} \left[ f(x)^{k+1} \right] &= \dv{x} \left[ f(x) \cdot f(x)^k \right]\\
&= f'(x) \cdot f(x)^k + f(x) \cdot \left( f(x)^k \right)' \tag{By \nameref{thrm: Product Rule}}\\
&= f'(x) \cdot f(x)^k + f(x) \cdot \left( k \cdot f(x)^{k-1} \cdot f'(x) \right) \tag{Applying the inductive hypothesis}\\
&= f'(x) \cdot f(x)^k + k \cdot f(x)^k \cdot f'(x)\\
&= f'(x) \cdot f(x)^k \cdot (1 + k)\\
&= (k+1) \cdot f(x)^k \cdot f'(x)
\end{align*}

This completes the induction.
\end{solution}

\begin{exercise}
   Use the preceding exercise to extend the power rule to positive rational exponents.
\end{exercise}
\begin{solution}
    Suppose that we have \( x^{\frac{p}{q}} \), where \( p,q \in \mathbb{Z}_{>0} \). Then, by definition, 
    \[ \left(  x^{\frac{p }{q}} \right)^{q} = x^{p} .\] Now we differentiate both sides to get 
    \begin{align*}
      \dv{x} \left[ \left(  x^{\frac{p }{q}} \right)^{q} \right] &= \dv{x} \left[ x^{p} \right]\\
      q \cdot  \left( x^{\frac{p }{q}} \right)^{q-1} \cdot \dv{x} \left[   x^{\frac{p }{q}}\right] &= \dv{x} \left[ x^{p} \right] \tag{Applying the result of the previous exercise.} \\
       q \cdot  \left( x^{\frac{p }{q}} \right)^{q-1} \cdot \dv{x} \left[   x^{\frac{p }{q}}\right] &= p x^{p-1} \tag{By the Power Rule}\\
        q \cdot x^{ \frac{p \cdot \left( q-1 \right)}{q}}\cdot \dv{x} \left[   x^{\frac{p }{q}}\right] &= p x^{p-1} 
    \end{align*}
    Now we can solve for \(  \dv{x} \left[   x^{\frac{p }{q}}\right] \).
    \begin{align*}
       \dv{x} \left[   x^{\frac{p }{q}}\right] & = \frac{p }{q} \cdot x^{p-1} \cdot x^{ -\frac{p \cdot \left( q-1 \right)}{q}} \\
        \dv{x} \left[   x^{\frac{p }{q}}\right] & = \frac{p }{q} \cdot x^{ \frac{p}{q}-1} \tag{Verify this.}
    \end{align*}
    This completes the proof.
\end{solution}


\begin{exercise}
   Use the product rule to show that the power rule extends to negative rational exponents.
\end{exercise}
\begin{solution}
    Let \( r \) be a positive rational number. Then
    \[
        1 = x^{-r} \cdot x^r.
    \]
    Taking the derivative of both sides with respect to \( x \), we have
    \begin{align*}
        \dv{x}[1] &= \dv{x} \left[ x^{-r} \cdot x^r \right] \\
        0 &= \dv{x} \left[ x^{-r} \cdot x^r \right] \tag*{Derivative of a constant is zero} \\
        &= \dv{x} \left[ x^{-r} \right] \cdot x^r + x^{-r} \cdot \dv{x} \left[ x^r \right] \tag*{By the Product Rule} \\
        &= \dv{x} \left[ x^{-r} \right] \cdot x^r + r x^{-r} \cdot x^{r-1} \\
        &= \dv{x} \left[ x^{-r} \right] \cdot x^r + r x^{-1}.
    \end{align*}
    Solving for \( \dv{x} \left[ x^{-r} \right] \), we find:
    \begin{align*}
        \dv{x} \left[ x^{-r} \right] \cdot x^r &= -r x^{-1} \\
        \dv{x} \left[ x^{-r} \right] &= -r x^{-r-1}.
    \end{align*}
    Therefore the power rule
    \[
        \dv{x} \left[ x^s \right] = s x^{s - 1}
    \]
    holds for all negative rational numbers \( s = -r \).
\end{solution}




\begin{theorem}[The Quotient Rule for Derivatives]\label{them: Quotient Rule}
   Let \( f \) and \( g \) be differentiable functions, and suppose \( g(x) \neq 0 \). Then \[ \dv{x} \left[ \frac{f(x)}{g(x)} \right] = \frac{f'(x)g(x) - f(x)g'(x)}{g(x)^2} \]
\end{theorem}
\begin{proof}
   \begin{align*}
\dv{x} \left[ \frac{f(x)}{g(x)} \right] 
&= \lim_{h \to 0} \frac{\dfrac{f(x+h)}{g(x+h)} - \dfrac{f(x)}{g(x)}}{h} \\
&= \lim_{h \to 0} \frac{\displaystyle\frac{f(x+h)g(x) - f(x)g(x+h) }{g(x+h)g(x)}}{h  }  \\\\
&= \lim_{h \to 0} \frac{\displaystyle\frac{f(x+h)g(x) -f(x)g(x) - f(x)g(x+h) + f(x)g(x)}{g(x+h)g(x)}}{h  }  \tag*{Add and subtract \( f(x)g(x) \) in the numerator} \\\\
&=\lim_{h \to 0} \frac{\displaystyle\frac{f(x+h)g(x) -f(x)g(x) - f(x)g(x+h) + f(x)g(x)}{h}}{g(x+h)g(x) } \\\\
&=\lim_{h \to 0} \frac{\displaystyle\frac{f(x+h)g(x) -f(x)g(x) }{h} - \frac{f(x)g(x+h) -f(x)g(x)}{h}}{g(x+h)g(x) } \tag*{Seperating the fractions} \\\\
&= \frac{ \left[ \displaystyle\lim_{h \to 0}\frac{f(x+h)g(x) -f(x)g(x) }{h}  \right]-  \left[ \displaystyle\lim_{h \to 0} \frac{f(x)g(x+h) -f(x)g(x)}{h} \right]}{ \displaystyle\lim_{h \to 0}g(x+h)g(x) } \tag*{Moving the limit inside.} \\\\
&= \frac{ \left[ \displaystyle\lim_{h \to 0}\frac{f(x+h) -f(x) }{h}  \right] g(x)-  f(x)\left[\displaystyle \lim_{h \to 0} \frac{g(x+h) -g(x)}{h} \right]}{ \displaystyle\lim_{h \to 0}g(x+h)g(x) }  \\\\
&= \frac{\displaystyle\dv{x} \left[ f(x) \right]g(x) - f(x)  \dv{x} \left[ g(x) \right]}{g(x)^2}. \tag*{Take the limit and simplify the denominator}
\end{align*} 
\end{proof}

\begin{exercise}
    Provide an alternate proof of the quotient rule using the product rule. \\
    \textbf{Hint:} If \( h(x) = \frac{f(x )}{g(x)} \), then we may write \( h(x) g(x)=f(x) \).
\end{exercise}
\begin{solution}
    Since \( h(x) g(x) = f(x) \), we have 
    \[ \dv{x} \left[ h(x) g(x) \right] = \dv{x} \left[ f(x) \right] .\]
    Applying the product rule, we have 
    \[ h'(x)g(x) + h(x)g'(x) = f'(x).\]
    Multiply both sides by \( g(x) \) to get 
    \[ h'(x) \left( g(x)  \right)^{2} + \left[ h(x)g(x) \right] g'(x) = f'(x) g(x).\]
    So 
     \[ h'(x) \left( g(x)  \right)^{2} + \left[ f(x)\right] g'(x) = f'(x)g(x) .\]
     Now solving for \( h'(x) \), we have 
     \[ \boxed{h'(x) = \frac{f'(x) g(x) - f(x)g'(x)}{\left( g(x) \right)^{2}} } \]
\end{solution}


\subsection{Chain Rule}

We will state the chain rule here without proof, as it is fairly more involved than the proofs of the other rules. If you are curious about the proof, check out \Cref{thm: chain rule}.\\
\textbf{The Chain Rule} If \( f \) and \( g \) are functions with \( f'(x) \) and \( g'(f(x)) \) both existing, then 
\[ \boxed{\left( g \circ f\right)'(x) = g'\left( f(x) \right) \cdot f'(x)}. \]

\begin{example}
    Verify the chain rule for the function \( h(x) = \left( x^{2}+4 \right)^{3} .\)\\
    We can take the derivative of this function first without the chain rule. However, we have to expand it using the binomial theorem.
    \begin{align*}
        \left( x^{2}+4 \right)^{3} &= \sum_{j=0}^{3} {3 \choose j} \left( x^{2} \right)^{3-j} \cdot 4^{j}\\
        &= {3 \choose 0}(x^2)^3 \cdot 4^0 + {3 \choose 1}(x^2)^2 \cdot 4^1 + {3 \choose 2}(x^2)^1 \cdot 4^2 + {3 \choose 3}(x^2)^0 \cdot 4^3\\
        &= x^{6} +12 x^{4} + 48 x^{2} +64
    \end{align*}
    Taking the derivative we get 
    \[ h'(x) = 6x^{5} + 48 x^{3} + 96x . \]
    
    If we instead consider \( h(x) = g\left( f(x) \right) \), where \( g(x) = x^{3} \) and \( f(x)= x^{2}+4 \), then the chain rule tells us 
    \begin{align*}
        h'(x) &= g'(f(x)) \cdot f'(x)\\
        &= 3 \left( x^{2}+4 \right)^{2} \cdot 2x\\
        &= 6x\left( x^{2}+4 \right)^{2}
    \end{align*}
    Notice how much simpler the chain rule approach is, we didn't need to expand a cubic!
    
    Let's verify that both expressions are equivalent by expanding the chain rule result:
    \begin{align*}
        6x\left( x^{2}+4 \right)^{2} &= 6x\left( x^{4} + 8x^{2} + 16 \right)\\
        &= 6x^{5} + 48x^{3} + 96x
    \end{align*}
    Indeed, both methods give the same answer.
\end{example}


\subsection{Derivatives of Trigonometric Functions}

\begin{theorem}
    \[
    \dv{x} \left[ \sin(x) \right] = \cos(x).
    \]
\end{theorem}
\begin{proof}
    \begin{align*}
        \dv{x} \left[ \sin(x) \right]
        &= \lim_{h \to 0} \frac{\sin(x + h) - \sin(x)}{h}  \\
        &= \lim_{h \to 0} \frac{\sin(x)\cos(h) + \cos(x)\sin(h) - \sin(x)}{h} \\
        &= \lim_{h \to 0} \left[ \cos(x) \cdot \frac{\sin(h)}{h} + \sin(x) \cdot \frac{\cos(h) - 1}{h} \right] 
    \end{align*}

    Now we apply known trigonometric limits:
    \begin{align*}
        \dv{x} \left[ \sin(x) \right]
        &= \cos(x) \cdot \lim_{h \to 0} \frac{\sin(h)}{h}
           + \sin(x) \cdot \lim_{h \to 0} \frac{\cos(h) - 1}{h} \\
        &= \cos(x) \cdot 1 + \sin(x) \cdot 0 \tag*{\( \boxed{\lim_{h \to 0} \frac{\sin(h)}{h} = 1 , \lim_{h \to 0} \frac{\cos(h) - 1}{h} = 0} \)} \\
        &= \cos(x)
    \end{align*}

    Therefore,
    \[
    \dv{x} \left[ \sin(x) \right] = \cos(x).
    \]
\end{proof}

\begin{theorem}
    \[
    \dv{x} \left[ \cos{ \left( x \right) } \right] = - \sin{ \left( x \right) }.
    \]
\end{theorem}
\begin{proof}
    \begin{align*}
        \dv{x} \left[ \cos{ \left( x \right) } \right] &= \lim_{h \to 0} \frac{ \cos{ \left( x +h  \right) } - \cos{ \left( x \right) }}{h} \\
        &=  \lim_{h \to 0}\frac{ \cos{ \left( x  \right) } \cos{ \left( h  \right) } - \sin{ \left( x  \right) } \sin{ \left( h  \right) } - \cos{ \left( x  \right) }}{h} \\
        &=\lim_{h \to 0} \frac{ \cos{ \left( x  \right) } \cos{ \left( h  \right) } - \cos{ \left( x \right) }}{h} - \lim_{h \to 0} \frac{ \sin{ \left( x  \right) } \sin{ \left( h  \right) }}{h} \\
        &= \cos{ \left( x  \right) } \lim_{h \to 0} \frac{ \cos{ \left( h  \right) } -1}{h} - \sin{ \left( x \right) } \lim_{ h \to 0} \frac{ \sin{ \left( h \right) }}{h}\\
        &= \cos{ \left( x \right) } \cdot 0 - \sin{ \left( x \right) } \cdot 1
    \end{align*}
    so 
    \[ \boxed{ \dv{x} \left[ \cos{ \left( x \right) } \right] = - \sin{ \left( x \right) }.} \]
\end{proof}

\begin{theorem}
    \[ \dv{x} \left[ \tan{ \left( x \right) } \right] = \sec^{2} \left( x \right). \]
\end{theorem}
\begin{proof}
    \begin{align*}
        \dv{x} \left[ \tan{ \left( x \right) } \right] &= \lim_{h \to 0} \frac{ \tan{ \left( x +h  \right) } - \tan{ \left( x \right) }}{h}\\
        &= \lim_{h \to 0} \frac{ \displaystyle \frac{ \tan{ \left( x \right) } + \tan{ \left( h \right) }}{1 - \tan{ \left( x  \right) } \tan{ \left( h \right) }} - \tan{ \left( x \right) }}{h} \\
        &= \lim_{h \to 0} \frac{ \displaystyle \frac{ \tan{ \left( x \right) } + \tan{ \left( h \right) }}{1 - \tan{ \left( x  \right) } \tan{ \left( h \right) }} - \frac{ \tan{ \left( x  \right) } \left( 1 - \tan{ \left( x \right) } \tan{ \left( h \right) } \right)}{1- \tan{ \left( x \right) } \tan{ \left( h \right) }}}{h}\\
        &= \lim_{h \to 0} \frac{ \tan{ \left( x  \right) } + \tan{ \left( h  \right) } - \tan{ \left( x \right) } + \tan^{2}{ \left( x \right) } \tan{ \left( h \right) }}{h \left( 1 - \tan{ \left( x \right) } \tan{ \left( h \right) } \right)} \\
        &= \lim_{h \to 0} \frac{ \tan{ \left( h \right) } + \tan^{2}{ \left( x \right) } \tan{ \left( h \right) }}{h \left( 1 - \tan{ \left( x \right) } \tan{ \left( h \right) } \right)} \\
        &= \lim_{h \to 0} \frac{ \tan{ \left( h  \right) } \left( 1 + \tan^{2}{ \left( x  \right) } \right)}{h \left( 1 - \tan{ \left( x \right) } \tan{ \left( h \right) } \right)} \\
        &= \left( 1 + \tan^{2}{ \left( x \right) } \right) \cdot \left( \lim_{h \to 0} \frac{ \tan{ \left( h \right) }}{h} \right) \cdot \left( \lim_{h \to 0} \frac{1}{1- \tan{ \left( x \right) } \tan{ \left(  h \right) }} \right) \\
        &= \left( 1 + \tan^{2}{ \left( x \right) } \right) \cdot 1 \cdot 1 \\
        &= \sec^{2} \left( x \right)
    \end{align*}
   so 
   \[ \boxed{\dv{x} \left[ \tan{ \left( x \right) } \right] = \sec^{2} \left( x \right).} \] 
\end{proof}

\begin{exercise}
    Use the quotient rule to find an easier way to take the derivative of \( \tan{ \left( x \right) } \).
\end{exercise}
\begin{solution}
    We write \( \tan{ \left( x  \right) } = \frac{ \sin{ \left( x \right) }}{ \cos{ \left( x \right) }} \). Then, 
    \begin{align*}
        \dv{x} \left[ \tan{ \left( x \right) } \right] &= \dv{x } \left[ \frac{ \sin{ \left( x  \right) }}{ \cos{ \left( x \right) }} \right] \\
        &= \frac{\displaystyle \dv{x} \left[ \sin{ \left( x \right) } \right] \cos{ \left( x \right) } -  \dv{x } \left[ \cos{ \left( x  \right) } \right] \sin{ \left( x \right) }}{ \cos^{2}{ \left( x \right) }} \\
        &= \frac{ \cos^{2}{ \left( x \right) } + \sin^{2}{ \left( x \right) }}{ \cos^{2}{ \left( x \right) }}
    \end{align*}
    So 
      \[ \boxed{\dv{x} \left[ \tan{ \left( x \right) } \right] = \sec^{2} \left( x \right).} \] 
\end{solution}

\begin{exercise}
    Use the product rule and the fact that 
    \[ \tan{ \left( x \right) } \cos{ \left( x \right) } = \sin{ \left( x \right) } \] to show that \( \dv{x} \left[ \tan{ \left( x \right) } \right] = \sec^{2} \left( x \right). \)
\end{exercise}
\begin{solution}
    We have 
    \begin{align*}
        \dv{x} \left[ \tan{ \left( x  \right) } \cos{ \left( x  \right) } \right] &= \dv{x} \left[ \sin{ \left( x \right) } \right] \\
         \dv{x } \left[ \tan{ \left( x  \right) } \right] \cos{ \left( x  \right) } + \tan{ \left( x  \right) } \dv{x } \left[ \cos{ \left( x  \right) } \right]&= \dv{x } \left[ \sin{ \left( x \right) } \right] \\
         \dv{x } \left[ \tan{ \left( x \right) } \right] \cos{ \left( x \right) } - \tan{ \left( x \right) } \sin{ \left( x \right) } &= \cos{ \left( x \right) } \\
         \dv{x } \left[ \tan{ \left( x \right) } \right] - \tan^{2}{ \left( x \right) } &=1 \\
         \dv{x } \left[ \tan{ \left( x \right) } \right] &= 1 + \tan^{2}{ \left( x \right) }
    \end{align*}
  \[ \boxed{\dv{x} \left[ \tan{ \left( x \right) } \right] = \sec^{2} \left( x \right).} \]    
\end{solution}


\begin{theorem}
    \[ \dv{x} \left[ \sec(x) \right] = \sec(x)\tan(x) .\]
\end{theorem}
\begin{proof}
    \begin{align*}
        \dv{x} \left[ \sec(x) \right] &= \lim_{h \to 0} \frac{ \sec{ \left( x +h \right) } - \sec{ \left( x \right) }}{h} \\
        &= \lim_{h \to 0} \frac{\displaystyle \frac{1}{ \cos{ \left( x+h  \right) }} -  \frac{1}{ \cos{ \left( x \right) }}}{h} \\
        &= \lim_{h \to 0} \frac{1}{h} \cdot\frac{ \cos{ \left( x  \right) } - \cos{ \left( x+h \right) }}{ \cos{ \left( x  \right) } \cos{ \left( x+h \right) }} \\
        &= \lim_{h \to 0} \frac{ \cos{ \left( x  \right) } - \cos{ \left( x+h  \right) }}{h} \cdot \lim_{ h \to 0} \frac{1}{ \cos{ \left( x \right) } \cos{ \left( x+h \right) }} \\
        &= \left( - \dv{x} \left[ \cos{ \left( x \right) } \right] \right) \cdot \left( \lim_{h \to 0} \frac{1}{ \cos{ \left( x \right) } \cos{ \left( x+h \right) }} \right) \\
        &= \sin{ \left( x  \right) } \sec^{2}{ \left( x \right) }
    \end{align*}
    or 
    \[ \boxed{\dv{x} \left[ \sec(x) \right] = \sec(x)\tan(x)} \]
\end{proof}

\begin{exercise}
    Prove that \( \dv{x } \left[ \sec{ \left( x \right) } \right] = \sec{ \left( x \right) } \tan{ \left( x \right) } \) \emph{the long way}. (Don't use prior knowledge of the derivative of cosine.)
\end{exercise}
\begin{solution}
    The first part of this proof starts off in the same way as above.
     \begin{align*}
        \dv{x} \left[ \sec(x) \right] &= \lim_{h \to 0} \frac{ \sec{ \left( x +h \right) } - \sec{ \left( x \right) }}{h} \\
        &= \lim_{h \to 0} \frac{\displaystyle \frac{1}{ \cos{ \left( x+h  \right) }} -  \frac{1}{ \cos{ \left( x \right) }}}{h} \\
        &= \lim_{h \to 0} \frac{1}{h} \cdot\frac{ \cos{ \left( x  \right) } - \cos{ \left( x+h \right) }}{ \cos{ \left( x  \right) } \cos{ \left( x+h \right) }} 
    \end{align*}
Here is where the steps diverge. 
\begin{align*}
    \dv{x} \left[ \sec(x) \right] &= \lim_{h \to 0} \frac{1}{h} \cdot \frac{ \cos{ \left( x \right) } - \cos{ \left( x  \right) } \cos{ \left( h  \right) } + \sin{ \left( x \right) } \sin{ \left( h \right) }}{\cos{ \left( x  \right) } \cos{ \left( x+h \right) }} \\
    &=\left[  \lim_{h \to 0} \left( \frac{1 - \cos{ \left( h \right) }}{h}  \right) \left( \frac{ \cos{ \left( x \right) }}{ \cos{ \left( x  \right) } \cos{ \left( x+h \right) }} \right) \right] + \left[ \lim_{h \to 0} \left( \frac{ \sin{ \left( h \right) }}{h} \right) \left( \frac{ \sin{ \left( x \right) }}{ \cos{ \left( x  \right) } \cos{ \left( x+h \right) }} \right)\right] \\
    &= \left(  \lim_{h \to 0}\frac{1 - \cos{ \left( h \right) }}{h}  \right)  \left( \lim_{h \to 0} \frac{ \cos{ \left( x \right) }}{ \cos{ \left( x  \right) } \cos{ \left( x+h \right) }} \right) + \left( \lim_{h \to 0} \frac{ \sin{ \left( h \right) }}{h} \right)  \left(  \lim_{h \to 0}\frac{ \sin{ \left( x \right) }}{ \cos{ \left( x  \right) } \cos{ \left( x+h \right) }} \right)\\
    &= 0 \cdot \sec{ \left( x \right) } + 1 \cdot \sec{ \left( x \right) } \tan{ \left( x \right) }
\end{align*}
So 
 \[ \boxed{\dv{x} \left[ \sec(x) \right] = \sec(x)\tan(x)} \]
\end{solution}

\subsection{Derivatives of Logarithmic and Exponential Functions}

\begin{theorem}
    \[
    \dv{x} \left[ \log_{a}(x) \right] = \frac{1}{\ln(a)x}
    \]
\end{theorem}
\begin{proof}
    \begin{align*}
        \dv{x} \left[ \log_{a}(x) \right]
        &= \lim_{h \to 0} \frac{\log_{a}(x + h) - \log_{a}(x)}{h}  \\
        &= \lim_{h \to 0} \frac{1}{h} \log_{a} \left( \frac{x + h}{x} \right)  \\
        &= \lim_{h \to 0} \log_{a} \left( \left( 1 + \frac{h}{x} \right)^{\frac{1}{h}} \right) \\
    \end{align*}

    We now handle the right-hand limit. Let \( t = \frac{1}{h} \), so that \( h \to 0^{+} \) corresponds to \( t \to \infty \). Then:
    \begin{align*}
        \lim_{h \to 0^{+}} \log_{a} \left( \left( 1 + \frac{h}{x} \right)^{\frac{1}{h}} \right)
        &= \lim_{t \to \infty} \log_{a} \left( \left( 1 + \frac{1}{xt} \right)^{t} \right) \\
        &= \log_{a} \left( \lim_{t \to \infty} \left( 1 + \frac{\frac{1}{x}}{t} \right)^{t} \right) \\
        &= \log_{a} \left( e^{\frac{1}{x}} \right) \tag*{Definition of \( e \): \( \lim_{n \to \infty} \left(1 + \frac{a}{n} \right)^n = e^{a} \)} \\
        &= \frac{1}{x} \log_{a}(e)  \\
        &= \frac{1}{x} \cdot \frac{\ln(e)}{\ln(a)}  \\
        &= \frac{1}{\ln(a)x}
    \end{align*}

    The left-hand limit as \( h \to 0^{-} \) proceeds similarly and yields the same result. (Hint: Negatives conspire to cancel.)

    Hence, the derivative exists and is given by:
    \[
    \dv{x} \left[ \log_{a}(x) \right] = \frac{1}{\ln(a)x}.
    \]
\end{proof}

\begin{theorem}
    \[ \dv{x } \left[ a^{x} \right] =  \ln{ \left( a  \right) } \cdot a^{x}. \]
\end{theorem}
\begin{proof}
    We have 
    \begin{align*}
        \dv{x } \left[ a^{x} \right] &= \lim_{h \to 0} \frac{a^{x+h} - a^{x}}{h} \\
        &= a^{x} \lim_{h \to 0} \frac{a^{h} -1}{h} \\
        &=a^{x} \lim_{h \to 0} \frac{ e^{ \displaystyle h \ln{ \left( a \right) }} -1}{h} \\
        &=a^{x} \lim_{k \to 0} \frac{ e^{ \displaystyle k} -1}{k} \cdot \ln{ \left( a \right) }\tag*{Setting $k = h \ln{ \left( a \right) }$.} \\
        &= \ln{ \left( a  \right) } \cdot a^{x} \lim_{k \to 0} \frac{ e^{ \displaystyle k} -1}{k} \\
        &= \ln{ \left( a  \right) } \cdot a^{x } \cdot 1
    \end{align*}
    So 
    \[ \boxed{\dv{x } \left[ a^{x} \right] =  \ln{ \left( a  \right) } \cdot a^{x}.} \]
\end{proof}


\section{Implicit Differentiation}
Sometimes, we are not so lucky to be given \( y \) as an explicit function of \( x \), we are give some relation between \( x \) and \( y \). But we still want to measure how \( y \) changes with respect to a change in \( x \). Implicit differentiation allows us to do this. \\ 

If we want to find the rate of change of the expression with respect to \( x \), \( f(y) \), we can think of this as \( f(y(x)) \) and then apply the chain rule to get 
\[ \dv{x} \left[ f(y(x )) \right] = f'(y(x)) y'(x)  \text{ .}\]

\begin{example}
    Suppose that \( x^{2} + y^{2}=1 \), the unit circle. Find \( \dv{y}{x}  \). \\ 
    Applying the derivate operator to both sides 
    \begin{align*}
        \dv{x} \left[ x^{2} + y^{2} \right] &= \dv{x} \left[ 1 \right] \\
        2x + 2y \dv{y}{x} &=0 \\
        \dv{y}{x} &= - \frac{x}{y}
    \end{align*}
Let us find \( \dv{y}{x}  \) another way. \\ 
Solving for \( y \) as an explicit function of \( x \), we have two functions: 
\[ y = \pm \sqrt{1-x^{2}} = \pm \left( 1-x^{2} \right)^{\frac{1}{2}} \]
Taking the derivative, we have 
\begin{align*}
    y' &= \pm \frac{1}{2}\left( 1-x^{2} \right)^{\frac{1}{2}-1} (-2x) \\
    &= \mp \frac{x}{\sqrt{1-x^{2}}} \\
    &=- \frac{x }{y}
\end{align*}

\end{example}

\begin{example}
    Find \( \dv[2]{y}{x}  \) for \( x^{2}+y^{2}=1 \). \\ 
    Let us use the expression we found earlier that \( 2x +2y \dv{y}{x} =0  \) or \( x + y \dv{y}{x} = 0 \). We can apply the derivative operator to get 
    \begin{align*}
        \dv{x } \left[ x + y \dv{y }{x}  \right]  &= \dv{x } \left[ 0 \right] \\
        1 + \dv{y }{x}^{2} + y \dv[2]{y }{x} &= 0 \tag{Applying the product rule.} \\
        \dv[2]{y }{x}  &= \frac{-1- \left(  \dv{y }{x} \right)^{2}}{y} \\
        \dv[2]{y }{x}  &= \frac{-1 - \left( -\frac{x}{y} \right)^{2}}{y} \\
        \dv[2]{y }{x}  &= \frac{-y^{2}-x^{2}}{y^{3}}\\
         \dv[2]{y }{x}  &= - \frac{y^{2}+x^{2}}{y^{3}}
    \end{align*}
This is much easier than taking the derivative the other way 
\begin{align*}
    \dv{x}\left[y'  \right]  &= \dv{x}\left[ \mp \frac{x }{\sqrt{1-x^{2}}} \right] 
\end{align*}

\end{example}


\section{Inverse Functions}
Recall that an inverse function for \( f(x) \) is a function \( f^{-1}(x) \) with the property that 
\[ f(f^{-1}(x)) = x \quad \text{and} \quad f^{-1}(f(x)) = x \]
for all \( x \) in the appropriate domains. \\ 
This lets us derive information about an inverse function's derivative given information about the derivative of the original function. 

\begin{theorem}
    Suppose that \( f \) is an invertible function with inverse \( g \), that \( f \) is differentiable at \( g(a) \), and that \( f'(g(a)) \neq 0\). Then \( g \) is differentiable at \( a \) and
    \[ g'(a) = \frac{1}{f'(g(a))} .\]
\end{theorem}
\begin{proof} 
    We apply the chain rule to the identity \( f(g(x)) = x \).
    \begin{align*}
        \dv{x} \left[ f(g(x)) \right] &= \dv{x} \left[ x \right] \\
        f'(g(x)) \cdot g'(x) &= 1 \\
        g'(x) &= \frac{1}{f'(g(x))}
    \end{align*}
    The result is achieved by setting \( x = a \). 
\end{proof}










\chapter{Applications of the Derivative}
\section{Maximum and Minimum}
\begin{example}\label{example:cos2_plus_2sin2}
    Find the maximum and minimum of the function
    \[
        f(\theta)=\cos^{2}\theta+2\sin^{2}\theta
    \]
    over the interval \([0,2\pi]\).

    First note that
    \[
        f(\theta)=\cos^{2}\theta+2\sin^{2}\theta=(\cos^{2}\theta+\sin^{2}\theta)+\sin^{2}\theta=1+\sin^{2}\theta.
    \]
    Differentiating,
    \[
        f'(\theta)=2\sin\theta\cos\theta=\sin(2\theta).
    \]
    Setting \(f'(\theta)=0\) gives
    \[
        \sin(2\theta)=0\quad\Longrightarrow\quad 2\theta=k\pi\quad(k\in\mathbb{Z}),
    \]
    so the critical points in \([0,2\pi]\) are
    \[
        \theta=0,\ \tfrac{\pi}{2},\ \pi,\ \tfrac{3\pi}{2},\ 2\pi.
    \]

    \medskip

     We test one point in each sub-interval determined by the critical points.

    \[
    \begin{array}{c|c|c|c}
        \text{Interval} & \text{Test point } \theta & \sin(2\theta) & \text{Monotonicity of } f\\\hline
         \left( 0,\tfrac{\pi}{2} \right)& \tfrac{\pi}{4} & \sin(\tfrac{\pi}{2})=+1 & \text{increasing}\\[4pt]
        \left( \tfrac{\pi}{2},\pi \right) & \tfrac{3\pi}{4} & \sin(\tfrac{3\pi}{2})=-1 & \text{decreasing}\\[4pt]
        \left( \pi,\tfrac{3\pi}{2} \right)& \tfrac{5\pi}{4} & \sin(\tfrac{5\pi}{2})=+1 & \text{increasing}\\[4pt]
        \left( \tfrac{3\pi}{2},2\pi \right) & \tfrac{7\pi}{4} & \sin(\tfrac{7\pi}{2})=-1 & \text{decreasing}
    \end{array}
    \]

    From this sign chart: \(f\) increases on \((0,\tfrac{\pi}{2})\), decreases on \((\tfrac{\pi}{2},\pi)\), increases on \((\pi,\tfrac{3\pi}{2})\), and decreases on \((\tfrac{3\pi}{2},2\pi)\). Hence \(\theta=\tfrac{\pi}{2}\) and \(\theta=\tfrac{3\pi}{2}\) are local maxima, while \(\theta=0,\pi,2\pi\) are local minima.

    \medskip
Testing each candidate \( \theta \) value, we get:
    \[
        f(0)=1+\sin^{2}0=1,\qquad
        f\!\bigl(\tfrac{\pi}{2}\bigr)=1+\sin^{2}\!\bigl(\tfrac{\pi}{2}\bigr)=2,
    \]
    \[
        f(\pi)=1+\sin^{2}\pi=1,\qquad
        f\!\bigl(\tfrac{3\pi}{2}\bigr)=1+\sin^{2}\!\bigl(\tfrac{3\pi}{2}\bigr)=2,
    \]
    \[
        f(2\pi)=1+\sin^{2}(2\pi)=1.
    \]

    \medskip

     So, on \([0,2\pi]\),
    \[
        \boxed{\text{Maximum value } 2 \text{ attained at } \theta=\tfrac{\pi}{2},\ \tfrac{3\pi}{2},}
        \qquad
        \boxed{\text{Minimum value } 1 \text{ attained at } \theta=0,\ \pi,\ 2\pi.}
    \]
\end{example}




\chapter{Integrals}
We now begin the second question calculus was intended to solve, how do we find the area underneath the graph of a function.

\section{Approximations}
\subsection{Riemann Sums}
\subsubsection{Left and Right Riemann Sums}
The idea behind a Riemann sum is very simple: we approximate the area underneath a function using rectangles.

\textbf{Setting Up the Problem:} Suppose we want to find the area under a continuous function \( f(x) \) between \( x = a \) and \( x = b \). The first step is to divide this interval into \( n \) equal pieces.

\textbf{Width of each rectangle:} Since we're dividing the interval \([a,b]\) into \( n \) equal pieces, each piece has width
\[ \Delta x = \frac{b-a}{n} \]

\textbf{The subintervals:} This divides \([a,b]\) into the subintervals
\[ [a, a+\Delta x], \quad [a+\Delta x, a+2\Delta x], \quad [a+2\Delta x, a+3\Delta x], \quad \ldots, \quad [a+(n-1)\Delta x, b] \]

Notice that the left endpoints of these subintervals are \( a, a+\Delta x, a+2\Delta x, \ldots, a+(n-1)\Delta x \), which we can write as \( a + k\Delta x \) for \( k = 0, 1, 2, \ldots, n-1 \).

\textbf{Left Riemann Sum:} To form rectangles, we need both a width and a height. We already have the width (\( \Delta x \)). For the height, we use the function value at the \textbf{left endpoint} of each subinterval.

For the \( k \)-th rectangle (where \( k \) goes from 0 to \( n-1 \)):
\begin{itemize}
    \item \textbf{Width:} \( \Delta x \)
    \item \textbf{Height:} \( f(a + k\Delta x) \)
    \item \textbf{Area:} \( f(a + k\Delta x) \cdot \Delta x \)
\end{itemize}

Adding up all \( n \) rectangles, the total approximate area is
\[ \mathrm{Area}(f) \approx \sum_{k=0}^{n-1} f(a + k \Delta x) \cdot \Delta x \]
where \( \Delta x = \frac{b-a}{n} \).


\begin{figure}[h]
    \centering
    \includegraphics[width=0.9\textwidth]{figures/analysis/calculus/LHRS.png}
    \caption{Left Riemann Sum with $n$ subdivisions. Each rectangle has width $\Delta x = \frac{b-a}{n}$ and height $f(a + k\Delta x)$ for $k = 0, 1, \ldots, n-1$.}
    \label{fig:LHRS}
\end{figure}



\chapter{Integration Techniques}
\section{Integration by Parts}
Much like how substitution is the inverse of the chain rule, we have an inverse of the product rule. 

\begin{align*}
    \dv{x} \left[ f(x) g(x) \right] = &f'(x)g(x)+ f(x)g'(x)\\
     \dv{x} \left[ f(x) g(x) \right]-f(x)g'(x) = &f'(x)g(x)\\
     \int \dv{x} \left[ f(x) g(x) \right] \dd{x} - \int f(x)g'(x) \dd{x} =& \int f'(x)g(x) \dd{x}
\end{align*}

This gives us 

\begin{equation}
    \int f'(x)g(x) \dd{x} = f(x)g(x)-  \int f(x)g'(x) \dd{x} \label{eqn:6/5/25/1}
\end{equation}

We will often use a more compact version of \Cref{eqn:6/5/25/1} 

\begin{equation}
    \int u \dd{v} = uv - \int v \dd{u} \label{eqn:6/5/25/2}
\end{equation}

This is the \vocab{integration by parts} formula. 

\begin{example}
    Find \( \int x \sin(x) \dd{x} \). \\
    Using \Cref{eqn:6/5/25/2}, we set \( u =x \), \( \dd{v}= \sin(x) \dd{x} \) ,so \( v= - \cos(x) \) and \( \dd{u}= \dd{x} \). This gives us 
    \begin{align*}
        \int x \sin(x) \dd{x} &= -x \cos(x)- \int - \cos(x) \dd{x}\\
        &= \sin(x) -x \cos{ \left( x \right) } +C
    \end{align*}
    
\end{example}

\begin{example}
    Find \( \int \ln(x) \dd{x} \).\\
    Setting \( u = \ln(x) \) and \( \dd{v}= 1 \dd{x} \), we get \( \dd{u}= \frac{1}{x} \dd{x} \) and \( v = x \) so 
    \begin{align*}
        \int \ln(x) \dd{x} &= x \ln(x) - \int 1 \dd{x}\\
        &=  x \ln(x) - x +C
    \end{align*}
    
\end{example}

\begin{example}
    Find \( \int x^{2} e^{x} \dd{x} \).\\
    Set \(  u = x^{2} \) so \( \dd{u} = 2x \dd{x} \). This gives us \( \dd{v}= e^{x} \dd{x} \) and \( v = e^{x} \).
    \[ \int x^{2} e^{x} \dd{x} = x^{2} e^{x} - \int x e^{x} \dd{x}. \]
    \( \int x e^{x} \dd{x} \) needs to be tackled again through integration by parts. So we reassign \( u =x, \dd{u}= \dd{x}, \dd{v}= e^{x} \dd{x}, v= e^{x} \) so 
    \begin{align*}
        \int x e^{x} \dd{x}&= x e^{x} - \int e^{x} \dd{x} \\
         &= x e^{x}- e^{x} + C.
    \end{align*}
Substitution into our earlier expression, we get 
\begin{align*}
    \int x^{2} e^{x} \dd{x} &= x^{2} e^{x} - \int x e^{x} \dd{x}\\
    &=  x^{2} e^{x} - \left( x e^{x}- e^{x} \right) +C\\
    &=  x^{2} e^{x} - x e^{x}+ e^{x} +C
\end{align*}

\end{example}

\begin{example}
    Find \( \int \sin (x) e^{x} \dd{x} \).\\
    For reasons that will become clear, let us set \( I = \int \sin{ \left( x  \right) } e^{x} \dd{x} \). We also set \( u = \sin{ \left( x \right) } , \dd{u} = -\cos{ \left( x \right) } \dd{x}, \dd{v} = e^{x } \dd{x}, v = e^{x} \). This gives us 
    \begin{align*}
        I &= \sin{ \left( x  \right) } e^{x} - \int - \cos{ \left( x \right) } e^{x } \dd{x}\\
        &=\sin{ \left( x  \right) } e^{x} + \int  \cos{ \left( x \right) } e^{x } \dd{x}\\
        &= \sin{ \left( x  \right) }e^{x }+ J
    \end{align*}
where  \( J =\int  \cos{ \left( x \right) } e^{x } \dd{x} \) and similarly let \( u = \cos{ \left( x  \right) } \dd{u} = \sin{ \left( x  \right) } \dd{x}, \dd{v} = e^{x} \dd{x}, v = e^{x} \). We get 
\begin{align*}
    J &= \cos{ \left( x  \right) }e^{x} - \int \sin{ \left( x \right) } e^{x} \dd{x} \\
    J &= \cos{ \left( x  \right) }e^{x } - I
\end{align*}
Now substituting the expression for \( J \) into the expression for \( I \), we get 
\begin{align*}
    I&= \sin{ \left( x  \right) } e^{x} + \cos{ \left( x  \right) }e^{x } - I\\
    2I &= \sin{ \left( x  \right) } e^{x} + \cos{ \left( x  \right) }e^{x } \\
\end{align*}
Dividing by \( 2 \) gives us the final answer 
\[ \boxed{\int \sin (x) e^{x} \dd{x} = \frac{\sin{ \left( x  \right) } e^{x} + \cos{ \left( x  \right) }e^{x }}{2} + C} \]
\end{example}
\(  \)\\
The integration by parts theorem also works for definite integrals. 

\begin{example}
    Evaluate \( \int_{0}^{1} \tan^{-1}{ \left( x \right) } \dd{x} \).\\
    Let \( u = \tan^{-1}{ \left( x \right) }, \dd{u} = \frac{1}{1+x^{2}} \dd{x}, \dd{v} = \dd{x}. v = x  \). 
   
    \begin{align*}
        \int_{0}^{1} \tan^{-1}{ \left( x \right) } \dd{x}  &= x \tan^{-1}{ \left( x \right) } \eval_{0}^{1} - \int_{0}^{1} \frac{x }{1+ x^{2 }} \dd{x}\\
        &= 1 \tan^{-1}{ \left( 1 \right) } - 0 \tan^{-1}{ \left( 0 \right) } - \int_{0}^{1} \frac{x }{1+ x^{2 }} \dd{x}\\
        &= \frac{\pi}{4}- \int_{0}^{1} \frac{x }{1+ x^{2 }} \dd{x}
    \end{align*}
    
We can use \( u \)-substitution for our new integral with \( u = 1+ x^{2}, \dd{u} = 2x \dd{x} \) 
\begin{align*}
    \int_{0}^1 \frac{x }{1+ x^{2 }} \dd{x} &= \frac{1}{2} \int_{1}^{2} \frac{1}{u } \dd{u}\\
    &= \frac{1}{2} \ln(u) \eval_{1}^{2}\\
    &= \frac{1}{2} \left( \ln(2)- \ln(1) \right)\\
    &= \frac{1}{2} \ln(2)\\
    &= \ln \left( \sqrt{2} \right)
\end{align*}
So finally, we have 
\[ \boxed{\int_{0}^{1} \tan^{-1}{ \left( x \right) } \dd{x} = \frac{\pi }{4}- \ln \left( \sqrt{2} \right) } \]
\end{example}

\begin{lemma}
    For all \( n \ge 2 \),
    \[ \int \sin^{n}{ \left( x  \right) } \dd{x} = = \frac{n-1}{n}\int \sin^{n-2}{ \left( x \right) } \dd{x} -\frac{1}{n } \sin^{n-1}{ \left( x  \right) }\cos{ \left( x \right)} . \]
    This is known as the \vocab{reduction formula.} 
\end{lemma}
\begin{proof}
    We can rewrite \( \int \sin^{n}{ \left( x \right) } \dd{x} \) as \( \int \sin^{n-1}{ \left( x  \right) } \sin{ \left( x  \right) } \dd{x} \) so we can do 
   \begin{align*}
    u = \sin^{n-1}{ \left( x  \right) }, &\quad  \dd{u} = \left( n-1  \right) \sin^{n-2}{ \left( x  \right) } \cos{ \left( x \right) } \dd{x} \\
    v=- \cos{ \left( x \right) } , & \quad \dd{v} = \sin{ \left( x  \right) } \dd{x}
   \end{align*}
   So 
   \begin{align*}
    \int \sin^{n}{ \left( x  \right) } \dd{x} &= -\sin^{n-1}{ \left( x  \right) }\cos{ \left( x \right) } - \int - \cos{ \left( x \right) } \left( n-1  \right) \sin^{n-2}{ \left( x  \right) } \cos{ \left( x \right) } \dd{x} \\ &=  -\sin^{n-1}{ \left( x  \right) }\cos{ \left( x \right) } + \left( n-1 \right) \int \sin^{n-2}{ \left( x \right) } \cos^{2}{ \left( x \right) } \dd{x}
     \\ &=  -\sin^{n-1}{ \left( x  \right) }\cos{ \left( x \right) } + \left( n-1 \right) \int \sin^{n-2}{ \left( x \right) }  \left( 1 - \sin^{2}{ \left( x \right) } \right)\dd{x}
      \\ &=  -\sin^{n-1}{ \left( x  \right) }\cos{ \left( x \right) } + \left( n-1 \right) \int \sin^{n-2}{ \left( x \right) }  - \sin^{n}{ \left( x \right) }\dd{x}\\
       \int \sin^{n}{ \left( x  \right) } \dd{x} &= -\sin^{n-1}{ \left( x  \right) }\cos{ \left( x \right) } + \left( n-1 \right) \int \sin^{n-2}{ \left( x \right) } \dd{x} - \left( n-1 \right) \int \sin^{n}{ \left( x  \right) }\dd{x}\\
      n\int \sin^{n}{ \left( x  \right) } \dd{x} &= -\sin^{n-1}{ \left( x  \right) }\cos{ \left( x \right) } + \left( n-1 \right) \int \sin^{n-2}{ \left( x \right) } \dd{x} 
   \end{align*}
which gives us the desired formula 
\[ \boxed{\int \sin^{n}{ \left( x  \right) } \dd{x} = \frac{n-1}{n}\int \sin^{n-2}{ \left( x \right) } \dd{x} -\frac{1}{n } \sin^{n-1}{ \left( x  \right) }\cos{ \left( x \right)}  } \]
\end{proof}

\begin{exercise}
    Evaluate \( \int x e^{2x} \dd{x} \)
\end{exercise}
\begin{solution}
    Set 
    \begin{align*}
        u = x, &\quad \dd{u} = \dd{x}\\
        v = \frac{1}{2}e^{2x}, &\quad \dd{v} = e^{2x } \dd{x}
    \end{align*}
So 
\begin{align*}
    \int x e^{2x } \dd{x} &= \frac{1}{2} x e^{2x} - \int \frac{1}{2} e^{ 2x } \dd{x}\\
    &= \frac{1}{2 }x e^{2x } - \frac{1}{4} e^{2x}+ C
\end{align*}

\end{solution}

\begin{exercise}
    Evaluate \( \int \sqrt{x} \ln \left( x  \right) \dd{x} \).
\end{exercise}
\begin{solution}
    Set 
    \begin{align*}
        u = \ln \left( x  \right), &\quad \dd{u} = \frac{1}{x }\dd{x}\\
        v = \frac{2}{3} x^{\frac{3}{2}}, &\quad \dd{v} = x^{\frac{1}{2}} \dd{x}
    \end{align*}
This gives us 
\begin{align*}
    \int \sqrt{x} \ln \left( x  \right) \dd{x} &=  \frac{2}{3} x^{\frac{3}{2}} \ln(x)- \int \frac{x^{\frac{1}{2}}}{x} \dd{x}\\
    &= \frac{2}{3} x^{\frac{3}{2}} \ln(x)- \int x^{-\frac{1 }{2}} \dd{x}\\
    &\frac{2}{3} x^{\frac{3}{2}} \ln(x)- 2 x^{\frac{1}{2}}+C
\end{align*}
So 
\[ \boxed{\int \sqrt{x} \ln \left( x  \right) \dd{x} = \frac{2}{3} \sqrt{x^{3}} \ln(x) -2 \sqrt{x}+C .} \]
\end{solution}

\begin{exercise}
    Find \( \int x \cos{ \left( 4x \right) } \dd{x} \)
\end{exercise}
\begin{solution}
    Set 
    \begin{align*}
        u= x ,&\quad  \dd{u} = \dd{x}\\
        v = \frac{1}{4} \sin{ \left( 4x \right) } ,& \quad \dd{v} = \cos{ \left( 4 x \right) } \dd{x}
    \end{align*}
    \begin{align*}
        \int x \cos{ \left( 4x \right) } \dd{x} &= \frac{1}{4} x \sin{ \left( 4x \right) } - \int \frac{1}{4} \sin{ \left( 4x \right) } \dd{x} \\
        &= \frac{1}{4} x \sin{ \left( 4x \right) } + \frac{1}{16} \cos{ \left( 4x \right) }+ C 
    \end{align*}
 So 
 \[ \boxed{ \int x \cos{ \left( 4x \right) } \dd{x} = \frac{1}{4} x \sin{ \left( 4x \right) } + \frac{1}{16} \cos{ \left( 4x \right) }+ C } \]   
\end{solution}

\begin{exercise}
    Evaluate 
    \[ \int \frac{x}{1-x} \dd{x} \]
\end{exercise}
\begin{solution}
    Set 
    \begin{align*}
        u= x, & \quad \dd{u} = \dd{x} \\
        v = - \ln(1-x)  & \quad \dd{v} = \frac{1}{1-x} \dd{x}
    \end{align*}
    So 
    \begin{align*}
        \int \frac{x}{1-x} \dd{x} &= - x \ln(1-x) + \int \ln(1-x) \dd{x}
    \end{align*}
    Making a substitution into our second integral of \( t = 1-x \Rightarrow -\dd{t} = \dd{x} \) so 
    \begin{align*}
        \int \ln(1-x) \dd{x} & - \int \ln(t) \dd{t} \\
        &= - t \ln(t) +t +C\\
        &= - \left( 1-x \right) \ln(1-x) + \left( 1-x \right) +C \\
        &= x \ln(1-x) -\ln(1-x) -x + C \tag{Since $C$ is an arbitrary constant, we will absorb the $1$ here.}
     \end{align*}
    Substituting into our earlier expression, we have the final answer of 
    \[ \boxed{ \int \frac{x }{1-x } \dd{x} = - \ln(1-x) -x +C} \]
\end{solution}


\section{Weierstrass Substitution}
Although this substitution technique is not typically introduced in standard calculus courses, it is remarkably elegant and powerful for integrating rational functions of trigonometric expressions.

The main idea is to re-parameterize the unit circle using the half-angle tangent. We begin with the substitution 
\begin{equation}
    u = \tan\left( \frac{x}{2} \right). \label{eqn:6/21/25/2}
\end{equation}

From this substitution, we can find the differential:
\begin{align*}
    \dd{u} &= \dd{\left( \tan\left( \frac{x}{2} \right) \right)} \\
    &= \frac{1}{2} \sec^2\left( \frac{x}{2} \right) \dd{x}\\
    &= \frac{1}{2} \left( 1+ \tan^2\left( \frac{x}{2} \right) \right) \dd{x}\\
    &= \frac{1}{2} \left( 1+u^2 \right) \dd{x}.
\end{align*}

Solving for $\dd{x}$:
\begin{equation}
    \boxed{ \dd{x} = \frac{2}{1 +u^2} \dd{u}}. \label{eqn:6/21/25/3}
\end{equation}

Now we derive the substitution formulas for the main trigonometric functions.


Using the double angle formula for tangent:
\begin{align*}
    \tan(x) &= \tan\left( 2 \cdot \frac{x}{2} \right)\\
    &= \frac{2\tan\left( \frac{x}{2} \right)}{1 - \tan^2\left( \frac{x}{2} \right)}\\
    &= \frac{2u}{1-u^2}.
\end{align*}

Therefore:
\begin{equation}
    \boxed{ \tan(x) = \frac{2u}{1-u^2} }. \label{eqn:6/21/25/4}
\end{equation}


To derive the substitutions for sine and cosine, we interpret $u = \tan\left( \frac{x}{2} \right)$ geometrically. Consider a right triangle with angle $\frac{x}{2}$, where the opposite side has length $u$ and the adjacent side has length $1$. By the Pythagorean theorem, the hypotenuse has length $\sqrt{1+u^2}$.

This gives us:
\[ \sin\left( \frac{x}{2} \right) = \frac{u}{\sqrt{1+u^2}}, \quad \cos\left( \frac{x}{2} \right) = \frac{1}{\sqrt{1+u^2}}. \]

Using the double angle formula for sine:
\begin{align*}
    \sin(x) &= \sin\left( 2 \cdot \frac{x}{2} \right)\\
     &= 2 \sin\left( \frac{x}{2} \right) \cos\left( \frac{x}{2} \right)\\
     &= 2 \cdot \frac{u}{\sqrt{1+u^2}} \cdot \frac{1}{\sqrt{1+u^2}}\\
     &= \frac{2u}{1+u^2}.
\end{align*}

Therefore:
\begin{equation}
    \boxed{ \sin(x) = \frac{2u}{1+u^2} }. \label{eqn:6/21/25/5}
\end{equation}

Using the double angle formula for cosine:
\begin{align*}
    \cos(x) &= \cos\left( 2 \cdot \frac{x}{2} \right)\\
    &= \cos^2\left( \frac{x}{2} \right) - \sin^2\left( \frac{x}{2} \right)\\
    &= \frac{1}{1+u^2} - \frac{u^2}{1+u^2}\\
    &= \frac{1-u^2}{1+u^2}.
\end{align*}

Therefore:
\begin{equation}
    \boxed{ \cos(x) = \frac{1-u^2}{1+u^2} }. \label{eqn:6/21/25/6}
\end{equation}


The Weierstrass substitution $u = \tan\left( \frac{x}{2} \right)$ transforms any trigonometric rational function into an algebraic rational function, which can then be integrated using partial fractions or other algebraic techniques.

\begin{example}
    Find \( \int \sec(x) \, dx \).
    
    We have 
    \[ \int \frac{1}{\cos(x)} \, dx. \]
    
    Applying \Cref{eqn:6/21/25/3,eqn:6/21/25/6}, we have  
    \begin{align*}
        \int \sec(x) \, dx &= \int \frac{1+u^{2}}{1-u^{2}} \cdot \frac{2}{1 +u^2} \, du \\
        &= \int \frac{2}{1-u^{2}} \, du\\
        &= \int \left(\frac{1}{1+u} + \frac{1}{1-u}\right) \, du\\
        &= \ln|1+u| - \ln|1-u| + C \\
        &= \ln\left|\frac{1+u}{1-u}\right| + C
    \end{align*}
    
    It is tempting to substitute \( u = \tan\left(\frac{x}{2}\right) \) and it would still be correct, but we can arrive at the standard answer with some patience: 
    \begin{align*}
         \ln\left|\frac{1+u}{1-u}\right| &= \ln\left|\frac{1+u}{1-u} \cdot \frac{1+u}{1+u}\right|\\
         &= \ln\left|\frac{(1 +u)^{2}}{1-u^{2}}\right|\\
         &= \ln\left|\frac{1+2u+u^{2}}{1-u^{2}}\right|\\
         &= \ln\left|\frac{1+u^{2}}{1-u^{2}} + \frac{2u}{1-u^{2}}\right|\\
         &= \ln \abs{\sec(x) + \tan(x)}
    \end{align*}
So 
\[ \boxed{ \int \sec \left( x  \right) \dd{x} = \ln \abs{\sec \left( x  \right) + \tan{ \left( x  \right) }} +C.} \]
\end{example}




\chapter{Applications of the Integral}
\section{Volumes}
Integrals are not only useful for computing areas; they also allow us to calculate volumes. Suppose that for each value of \( x \), we are given a cross-sectional area \( A(x) \). By approximating the solid as a stack of thin slices of thickness \( \dd{x} \), the total volume is obtained in the limit as
\[
    V = \int_{a}^{b} A(x) \, \dd{x}.
\]



\section{Volumes of Solids of Revolution and Cylindrical Shells}
Suppose we wish to calculate the volume of a solid obtained by rotating the graph of a function \( f(x) \geq 0 \) about the \( x \)-axis for \( x \) between \( a \) and \( b \). We think of slicing the solid perpendicular to the \( x \)-axis. At position \( x \), the cross-section is a circle of radius \( f(x) \), so its area is 
\[
A(x) = \pi \big(f(x)\big)^{2}.
\]
The volume is then obtained by integrating these cross-sectional areas:
\[
\boxed{V = \int_a^b A(x)\,\dd{x} 
= \pi \int_a^b \big(f(x)\big)^{2} \,\dd{x}.}
\]

Now suppose we want to find the volume of the solid formed by rotating the region bounded by two functions $f(x)$ and $g(x)$ about the $x$-axis. Suppose that $f(x)$ and $g(x)$ intersect at $x=a$ and $x=b$, and that $f(x) \geq g(x)$ on the interval $[a,b]$.

When we take a vertical cross-section at any point $x$ in $[a,b]$ and rotate it about the $x$-axis, we obtain a \textbf{washer} (a disk with a hole in the middle). The outer radius of this washer is $f(x)$, and the inner radius is $g(x)$.

The area of a washer is the area of the outer circle minus the area of the inner circle:
\[\text{Area of washer} = \pi R^2 - \pi r^2 = \pi(R^2 - r^2)\]

In our case, this gives us:
\[\text{Area of cross-section} = \pi[f(x)]^2 - \pi[g(x)]^2 = \pi\left([f(x)]^2 - [g(x)]^2\right)\]

To find the total volume, we integrate these cross-sectional areas from $x = a$ to $x = b$:
\[ \boxed{V = \int_a^b \pi\left([f(x)]^2 - [g(x)]^2\right) \, \dd{x} = \pi \int_a^b \left([f(x)]^2 - [g(x)]^2\right) \, \dd{x}}\]

\begin{exercise}
    Use the washer method to find the volume of the region above the curve \( y=x^{3} \) and below \( y=1 \) between \( x=0 \) and \( x=1 \) about the \( x \)-axis.
\end{exercise}
\begin{solution}
    We have 
    \begin{align*}
        V &= \pi \int_{0}^{1} 1^{2}- \left( x^{3 } \right)^{2 } \dd{x}\\
        &= \pi \int_{0}^{1} 1 - x^{6} \dd{x}\\
        &= \pi \left( x - \frac{x^{7}}{7} \right)\eval_{0}^{1}
    \end{align*}
    \[ \boxed{V = \frac{6 \pi}{7}} \]
\end{solution}

\begin{exercise}
    Calculate the same volume as above using cylindrical shells.
\end{exercise}
\begin{solution}
    We calculate the inverse function of \( y=x^{3} \) to get \( y= x^{\frac{1}{3}}. \) Then using the cylindrical shells formula, we get 
    \begin{align*}
        V &= 2\pi \int_{0}^{1} x \cdot x^{\frac{1}{3}} \dd{x}\\
        &= 2 \pi \int_{0}^{1} x^{\frac{4}{3}} \dd{x}\\
        &= 2 \pi \cdot \frac{3}{7} \cdot x^{\frac{7}{3}} \eval_{0}^{1}
    \end{align*}
    \[ \boxed{V = \frac{6 \pi}{7}} \]
\end{solution}

\begin{exercise}
    Now rotate the same region but about the \( y \)-axis using cylindrical shells.
\end{exercise}
\begin{solution}
   Our "height" is \( 1-x^{3} \). So we have 
   \begin{align*}
    V &= 2 \pi \int_{0}^{1} x \left( 1-x^{3} \right) \dd{x} \\
    &= 2 \pi \int_{0}^{1} x -x^{4} \dd{x} \\
    &= 2 \pi \left( \frac{x^{2}}{2}- \frac{x^{5}}{5} \eval_{0}^{1} \right)
   \end{align*}
    \[ \boxed{V = \frac{3 \pi}{5}} \]
\end{solution}

\begin{exercise}
    Calculate the same volume as above but using the washer method.
\end{exercise}
\begin{solution}
    \begin{align*}
        V &= \pi \int_{0}^{1} \left( x^{\frac{1}{3}} \right)^{2} \dd{x} \\
        &= \pi \int_{0}^{1} x^{\frac{2}{3}} \dd{x} \\
        &= \pi \cdot \frac{3}{5} \cdot x ^{\frac{5}{3}} \eval_{0}^{1}
    \end{align*}
    \[ \boxed{V = \frac{3 \pi}{5}} \]
\end{solution}

\begin{exercise}
    Let \( R \) be the region above the \( x \)-axis, below the graph of \( y= 1- \frac{1}{x} \) and to the right of \( x=3 \). Find the volume of \( R \) rotated about the \( x \)-axis by the disk method.
\end{exercise}
\begin{solution}
    We have 
    \begin{align*}
        V &= \pi \int_{1}^{3} \left( 1 - \frac{1}{x} \right)^{2} \dd{x}\\
        &= \pi \int_{1}^{3} 1- \frac{2}{x}+ \frac{1}{x^{2}} \dd{x} \\
        &= \pi \left( x - 2 \ln(x) - \frac{1}{x} \right) \eval_{1}^{3}
    \end{align*}
    \[ \boxed{V = \left( \frac{8}{3} - \ln(9) \right) \pi} \]
\end{solution}

\begin{exercise}
    Find the same volume as above but with cylindrical shells. 
\end{exercise}
\begin{solution}
    With shells, our height is \( 3 - \frac{1}{1-y} \) so our integral is 
    \begin{align*}
        V &= 2 \pi \int_{0}^{\frac{2}{3}} y \left( 3- \frac{1}{1-y} \right) \dd{y} \\
        &= 2 \pi \int_{0}^{\frac{2}{3}} 3y - \frac{y}{1-y} \dd{y} \\
        &= 2 \pi \left( \frac{3}{2}y^{2} + \ln(1-y) +y \right)\eval_{0}^{\frac{2}{3}} \\
        &= 2 \pi \left( \frac{2}{3} + \ln \left( \frac{1}{3} \right) + \frac{2}{3} \right)
    \end{align*}
    It can be easily verified that 
    \[ \boxed{V = \left( \frac{8}{3} - \ln(9) \right) \pi} \]
\end{solution}







\chapter{Parametric and Polar Equations}
\input{content/analysis/first-year-calculus/10-parametric-and-polar/first-year-calculus-10-parametric-and-polar}


\chapter{Sequences, Series, and Power Series}
\section{Convergence Tests}
\subsection{The Integral Test and the Divergence Test}
\subsection{Direct Comparison Test}

Suppose we are given two positive sequences, \( a_{n} \) and \( b_{n} \) with the property that if \( a_{n} \le  b_{n}\) for all \( n \ge N \) for some \( N \in \mathbb{N} \). If \({\displaystyle \sum_{k =1}^{\infty} b_{n}} \) converges, then \( {\displaystyle \sum_{k =1}^{n} a_{n}} \) also converges. \\ 

Similarly, if we are given two positive sequences, \( a_{n} \) and \( b_{n} \) with the property that if  \( a_{n} \; {\color{red} \ge}\;  b_{n} \) for all \( n \ge N \) for some \( N \in \mathbb{N} \). If \({\displaystyle \sum_{k =1}^{\infty} b_{n}} \) diverges, then \( {\displaystyle \sum_{k =1}^{n} a_{n}} \) also diverges.

\subsection{The Limit Comparison Test} 

Suppose we are given two positive sequences, \( a_{n} \) and \( b_{n} \) such that 
\[ \lim_{n \to \infty} \frac{a_{n }}{b_{n}} = L < \infty .\]
Then \( {\displaystyle \sum_{k =1}^{\infty} a_{n}} \) and \( {\displaystyle \sum_{k =1}^{\infty} b_{n}} \) both converge or both diverge.

\subsection{Alternating Series and Absolute Convergence} 

\begin{dfn}
    A series is \vocab{alternating} if every term is of the form \( (-1)^{n}a_{n} \) or \( (-1)^{{n+1}}a_{n} \).
\end{dfn}

Suppose that \( A= \sum_{k =1}^{\infty} \left( -1  \right)^{k}a_{k} \) is an alternating series. If 
\begin{enumerate}[label=\textbf{\roman*)}]
    \item \( a_{n+1} \le a_{n} \) for all \( n \ge N \), \( N \in \mathbb{N} \)
    \item \( \lim_{n \to \infty} a_{n} = 0 \)
\end{enumerate}
Then the series converges.

\begin{dfn}
    A series \( {\displaystyle \sum_{k =1}^{\infty} a_{k}} \) is \vocab{absolutely convergent} or \vocab{converges absolutely} if 
    \[ \sum_{k =1}^{\infty} \abs{a_{k}} \text{ converges.} \]
\end{dfn}

\begin{lemma}
    If a series converges absolutely, then it converges.
\end{lemma}
\begin{proof}
    Suppose that the series \( \sum_{k=1}^{\infty} a_{k} \) converges absolutely. 
    Then the series \( \sum_{k=1}^{\infty} \lvert a_{k} \rvert \) converges. 
    Since \( \sum_{k=1}^{\infty} -\lvert a_{k} \rvert = - \sum_{k=1}^{\infty} \lvert a_{k} \rvert \), it also converges.
    
    For each \( n \in \mathbb{N} \), define the partial sums
    \[
        S_{n} = \sum_{k=1}^{n} a_{k}
        \quad\text{and}\quad
        T_{n} = \sum_{k=1}^{n} \lvert a_{k} \rvert .
    \]
    Since \( -\lvert a_{k} \rvert \le a_{k} \le \lvert a_{k} \rvert \) for all \( k \),
    we have
    \[
        -T_{n} \le S_{n} \le T_{n} \quad \text{for all } n.
    \]
    As \( T_{n} \to \sum_{k=1}^{\infty} \lvert a_{k} \rvert \),
    it follows that \( -T_{n} \to -\sum_{k=1}^{\infty} \lvert a_{k} \rvert \).
    Hence \( S_{n} \) is squeezed between two convergent sequences and therefore converges.
    Thus \( \sum_{k=1}^{\infty} a_{k} \) converges.
\end{proof}



\begin{dfn}
    If a series \( {\displaystyle \sum_{k =1}^{\infty} a_{n}} \) converges but \( \displaystyle \sum_{k =1}^{\infty} \abs{a_{n}} \) does not, we say that it is \vocab{conditionally convergent} or \vocab{converges conditionally}.
\end{dfn}

\subsection{The Root and Ratio Tests}
 \subsubsection{The Ratio Test}
\( \bullet \) If 
\[ \lim_{n \to \infty} \abs{\frac{a_{n+1 }}{a_{n}}} <1 \]
then the series \( \displaystyle \sum_{k =1}^{\infty} a_{n} \) converges absolutely. \\ 

\( \bullet \) If 
\[ \lim_{n \to \infty} \abs{\frac{a_{n+1 }}{a_{n}}} >1 \]
then the series \( \displaystyle \sum_{k =1}^{\infty} a_{n} \) diverges.

\( \bullet \) If 
\[ \lim_{n \to \infty} \abs{\frac{a_{n+1 }}{a_{n}}} =1 \]
then the ratio test provides no information about the series \( \displaystyle \sum_{k =1}^{\infty} a_{n} \) .

\begin{exercise}
    Determine if 
    \[ \sum_{k =1}^{\infty} \frac{n}{5^{n}} \] converges or diverges. 
\end{exercise}
\begin{solution}
       Let \( a_{n} \) denote the \( n \)-th term in the given series. Then applying the ratio test, we have 
       \begin{align*}
        \frac{\abs{a_{n+1}}}{a_{n}} &=   \frac{{\displaystyle \frac{n+1}{5^{n+1}}}}{{\displaystyle \frac{n}{5^{n}}}} \\ 
        &= \frac{n+1}{5^{n+1}} \cdot \frac{5^{n}}{n} \\
        &= \frac{n+1}{5^{n} \cdot 5}  \cdot \frac{5^{n}}{n} \\
        &=\frac{n+1}{\Ccancel[DeepRed]{5^{n}}  \cdot 5}  \cdot \frac{\Ccancel[DeepRed]{5^{n}}}{n}\\
        &= \frac{1}{5} \cdot \frac{n+1}{n}
       \end{align*}
     Now taking the limit, we have 
     \[ \lim_{n \to \infty} \frac{1}{5} \frac{n+1}{n} = \frac{1}{5} <1 \] 
     so 
     \[ \boxed{ \sum_{k =1}^{\infty} \frac{n}{5^{n}} \text{ converges absolutely.}} \]
\end{solution}

\begin{exercise}
     Determine if 
    \[ \sum_{k =1}^{\infty} \left( -1 \right)^{n-1} \frac{3^{n}}{2^{n}n^{3}} \] converges or diverges. 
\end{exercise}
\begin{solution}
    Let \( a_{n} \) denote the \( n \)-th term in the given series. Then applying the ratio test, we have 
\begin{align*}
    \frac{\abs{a_{n+1}}}{\abs{a_{n}}} &=  \frac{{\displaystyle \frac{3^{n+1}}{2^{n+1} (n+1)^{3}}}}{{\displaystyle \frac{3^{n}}{2^{n}n^{3}}}} \\
    &= \frac{3^{n+1}}{2^{n+1} (n+1)^{3}} \cdot \frac{2^{n} n^{3}}{3^{n}} \\
    &= \frac{3 \cdot 3^{n}}{2 \cdot 2^{n} \left( n+1 \right)^{3}} \cdot \frac{2^{n} n^{3}}{3^{n}}\\
    &=\frac{3 \cdot \Ccancel[DeepRed]{3^{n}} }{2 \cdot  \Ccancel[AmberOrange]{2^{n}}  \left( n+1 \right)^{3}} \cdot \frac{\Ccancel[AmberOrange]{2^{n}}  n^{3}}{\Ccancel[DeepRed]{3^{n}}} \\
    &= \frac{3}{2} \frac{n^{3}}{\left( n+1 \right)^{3}}
\end{align*}
 Now taking the limit, we have 
     \[ \lim_{n \to \infty}\frac{3}{2} \cdot \frac{n^{3}}{\left( n+1 \right)^{3}} = \frac{3}{2} >1 \] 
     so 
     \[ \boxed{\sum_{k =1}^{\infty} \left( -1 \right)^{n-1} \frac{3^{n}}{2^{n}n^{3}} \text{ diverges.}} \]
\end{solution}



\begin{exercise}
    Determine if the following series converges or diverges
    \[ \sum_{n =1}^{\infty} \frac{ \prod_{k=1}^{n } 2k}{n!} \text{ .}\]
\end{exercise}
\begin{solution}
    Let \( a_{n} \) denote the \( n \)-th term in the given series. Then applying the ratio test, we have 
    \begin{align*}
        \frac{\abs{a_{n+1}}}{\abs{a_{n}}} &=   \frac{{\displaystyle  \frac{\prod_{k =1}^{n+1} 2k}{\left( n+1  \right)!}} }{ {\displaystyle \frac{\prod_{k =1}^{n} 2k}{n!}}} \\
        &= \frac{\prod_{k =1}^{n+1} 2k}{\left( n+1  \right)!} \cdot \frac{n!}{\prod_{k =1}^{n} 2k} \\
        &= \frac{\left( \prod_{k =1}^{n} 2k  \right) \cdot 2(n+1)}{n! \left( n+1 \right)} \cdot \frac{n!}{\prod_{k =1}^{n} 2k} \\
        &= \frac{\Ccancel[DeepRed]{\left( \prod_{k =1}^{n} 2k  \right) }  \cdot 2(n+1) }{\Ccancel[AmberOrange]{n!} \cdot \left( n+1 \right) } \cdot \frac{\Ccancel[AmberOrange]{n!}}{\Ccancel[DeepRed]{ \prod_{k =1}^{n} 2k  }} \\
        &= \frac{2 (n+ 1 )}{n+1} \\
        &=2
    \end{align*}
    Taking the limit 
    \[ \lim_{n \to \infty} 2 =2 >1\]
    so 
    \[ \boxed{ \sum_{n =1}^{\infty} \frac{ \prod_{k=1}^{n } 2k}{n!} \text{ diverges .}} \]
\end{solution}

\subsubsection{The Root Test}

\( \bullet \) If 
\[ \lim_{n \to \infty} \sqrt[n]{\abs{a_{n}}} <1 ,\] then the series \( \sum_{k =1}^{\infty} a_{n} \) is absolutely convergent. 

\( \bullet \) If \[ \lim_{n \to \infty} \sqrt[n]{\abs{a_{n}}} >1 ,\] then the series \( \sum_{k =1}^{\infty} a_{n} \) is divergent. 

\( \bullet \) If \[ \lim_{n \to \infty} \sqrt[n]{\abs{a_{n}}} =1 ,\] then no information about the series \( \sum_{k =1}^{\infty} a_{n} \) can be extracted from the root test.




\part{First Year Linear Algebra}
\label{part: First Year Linear}
\vspace*{2em} The primary resources for this part are \cite{ref:strang_laaa} and \cite{ref:strang_itla}. \vspace*{1em}
\parttoc
\chapter*{Why Linear Algebra?}
It is with very little exaggeration that I say that linear algebra is the singular most important area of mathematics. Our intuition for this subject rests largely on how we as humans naturally think about space, transformation, and relationships. Without prompt, we naturally organize data in arrays, think about directions and dimensions, and have well-developed intuition around addition and scaling, which are the fundamental operations of linear algebra. \\

What makes linear algebra so powerful is its remarkable ability to be both a highly effective practical tool and an interesting theoretical subject to study for its own sake. Systems of linear equations appear everywhere, from balancing chemical reactions to modeling physical phenomena to training neural networks. All this while the theoretical side is one of the best ways to introduce new mathematicians to proofs and its mastery and applications often illuminate other areas of math, like our understanding of symmetry (through representation theory) or spaces (through homology).  \\

Linear algebra bridges the concrete and the abstract, and through crossing this bridge many rewards await. A matrix can simultaneously represent a database of information, a system of equations, a geometric transformation, or an abstract linear operator. Much like life, a matrix is what you make it. This versatility explains linear algebra's ubiquity across quantitative fields: computer graphics, quantum mechanics, machine learning, optimization, statistics, differential equations, and far beyond. Your ability to understand applications will provide manifold examples to test your theoretical knowledge, while your mastery of theory helps you make deeper deductions in applications. Each side enriches the other.\\

This book will honor the canonical way of introducing linear algebra, recognizing that this pedagogical structure exists for good reason. The first-year course is mostly calculation-based and is meant for you to get comfortable with performing concrete computations. This computational fluency builds the intuition to appreciate the abstract theory that follows. The second-year course is more abstract, introducing vector spaces axiomatically, exploring the structure of linear transformations, and proving the fundamental theorems that explain why the calculations work. Students who rush to abstraction without computational experience often struggle to develop geometric intuition, while those who master only calculations miss the unifying principles that make linear algebra so powerful. By working through both stages thoughtfully, you'll develop both the technical facility to solve problems and the conceptual framework to understand the deeper structure of the subject.

\chapter{Vectors}
\section{Vectors and Linear Combinations}
The core object of study in linear algebra is the \emph{vector}.

\begin{dfn}
    By a \vocab{vector} (in this part), often denoted by \( \vb{v} \) or \( \va{v} \), we mean an ordered list of \( n \) numbers:
    \[
        \vb{v} = \left< v_1, v_2, \dots, v_n \right>.
    \]
    The numbers \( v_1, v_2, \dots, v_n \) are called the \vocab{components} or \vocab{entries} of the vector. 
    
    We will restrict our attention to vectors whose components come from either the real numbers \( \mathbb{R} \) or the complex numbers \( \mathbb{C} \). That is, we write \( \vb{v} \in \mathbb{R}^n \) or \( \vb{v} \in \mathbb{C}^n \) depending on context.
    
    Two basic operations on vectors are:
    \begin{itemize}
        \item \textbf{Vector addition:} Given \( \vb{v} = \left< v_1, \dots, v_n \right> \) and \( \vb{w} = \left< w_1, \dots, w_n \right> \), their sum is
        \[
            \vb{v} + \vb{w} = \left< v_1 + w_1, \dots, v_n + w_n \right>.
        \]
        
        \item \textbf{Scalar multiplication:} Given a scalar \( c \in \mathbb{R} \) or \( \mathbb{C} \) and a vector \( \vb{v} = \left< v_1, \dots, v_n \right> \), their product is
        \[
            c \vb{v} = \left< c v_1, \dots, c v_n \right>.
        \]
    \end{itemize}
Throughout linear algebra, we will often represent vectors with \( n \) components as an \vocab{\( n \times 1 \) matrix}, also called a column vector. That is,
\[
    \vb{v} = \left< v_1, \dots, v_n \right> = \begin{bmatrix}
        v_1 \\
        \vdots \\
        v_n
    \end{bmatrix}.
\]
\end{dfn}

This definition carries some immediate consequences on which the rest of linear algebra is built.

For any vector \( \vb{v} = \left< v_1, \dots, v_n \right> \), we can write:
\[
\vb{v} = \begin{bmatrix}
    v_1\\
    v_2\\
    \vdots\\
    v_n
\end{bmatrix} = 
v_1 \begin{bmatrix}
    1\\
    0\\
    \vdots\\
    0
\end{bmatrix} +
v_2 \begin{bmatrix}
    0\\
    1\\
    \vdots\\
    0
\end{bmatrix} +
\cdots +
v_n \begin{bmatrix}
    0\\
    0\\
    \vdots\\
    1
\end{bmatrix}
= \sum_{j=1}^n v_j \vb{e}_j,
\]
where \( \vb{e}_j \) is the vector with a \( 1 \) in the \( j \)th component and \( 0 \) elsewhere.

\begin{example}
    Consider the two-dimensional vector \( {\color[HTML]{7200fa} \vb{v} = \left< 3, 4 \right>} \). We can decompose \( {\color[HTML]{7200fa} \vb{v}} \) as
    \[
        {\color[HTML]{7200fa} \vb{v}} = 3\, {\color[HTML]{ff0000} \vb{e}_1} + 4\, {\color[HTML]{0000ff} \vb{e}_2}
    \]
    or equivalently,
    \[
        {\color[HTML]{7200fa} \left< 3, 4 \right>} = 3\, {\color[HTML]{ff0000} \left< 1, 0 \right>} + 4\, {\color[HTML]{0000ff} \left< 0, 1 \right>}.
    \]

    Geometrically, this means:
    \begin{itemize}
        \item Take 3 copies of the {\color[HTML]{ff0000} unit vector in the \( x \)-direction} and place them tip-to-tail.
        \item Then take 4 copies of the {\color[HTML]{0000ff} unit vector in the \( y \)-direction} and continue placing them tip-to-tail.
        \item The resulting vector \( {\color[HTML]{7200fa} \left< 3, 4 \right>} \) is the diagonal of the resulting "L" shape — the vector sum.
    \end{itemize}

    \begin{center}
        \includegraphics[width=0.5\textwidth]{figures/algebra/linearalgebra/vector34.png}
    \end{center}
\end{example}

This example highlights a subtle but important point about vectors: when we write the vector \( \left< 3,4 \right> \), we have implicitly chosen a basis. In this case, we have chosen the unit vector in the \( x \)-direction to be \( \left< 1,0 \right> \), and the unit vector in the \( y \)-direction to be \( \left< 0,1 \right> \). But this was a \textbf{choice} we made, nature did not hand us this grid.

We could just as well have chosen a different pair of linearly independent vectors and called those our new \( \left< 1,0 \right> \) and \( \left< 0,1 \right> \). This change of basis would then alter the meaning of \( \left< 3,4 \right> \), since that vector is \textbf{defined} as “3 copies of \( \left< 1,0 \right> \)” plus “4 copies of \( \left< 0,1 \right> \).” The numbers 3 and 4 are coordinates relative to a basis, not intrinsic properties of the vector itself. 

\begin{center}
    \includegraphics[width=0.5\textwidth]{figures/algebra/linearalgebra/another34.png}
\end{center}

This figure shows an equally valid way to define \( {\color[HTML]{ff0000} \left< 1,0 \right>} \) and \( {\color[HTML]{0000ff} \left< 0,1 \right>} \), leading to a vector \( {\color[HTML]{7200fa} \left< 3,4 \right>} \) that is distinct from the one in our earlier example.

\vspace{1em}

The key idea is this: mathematics, and any rigorous quantitative discipline, should not depend on how we choose to measure things. If we're describing something objective and external to us, then the way we draw our grid should not affect the truth of what we're describing.

In linear algebra, we will develop tools to \emph{compare grids}. That is, if you and your colleague make observations using different measurement systems (different bases), we need a way to transform your measurements into theirs, and vice versa, without losing the underlying geometric or physical meaning.

\begin{dfn}
    The \vocab{length} or \vocab{norm} of a vector \( \vb{v} = \left< v_{1}, v_{2 }, \dots, v_{n} \right> \) in \( \mathbb{R}^{n} \) is given by 
    \[ \abs{\vb{v}} = \norm{ \vb{v}} = \sqrt{\sum_{j=1}^{n}  \left( v_{j} \right)^{2}} = \sqrt{ \left( v_{1} \right)^{2} + \left( v_{2} \right)^{2} + \cdots + \left( v_{n} \right)^{2}} \]
\end{dfn}

\begin{example}\label{ex:sphere-closest-furthest-vector}
       What are the points on the sphere \( x^{2}+ y^{2}+ z^{2}=4 \) that are closest and furthest from the point \( (x,y,z) = (3,1, -1) \)? \\
       
       For the point closest to \( (3,1,-1) \), we want a vector that points in the same direction as \( \vb{v} = \left< 3,1,-1 \right>\) but has length \( 2 \) (the radius of the sphere). To achieve this, we simply scale the components of \( \vb{v}\) by \( \frac{2}{\norm{\vb{v}}} = \frac{2}{\sqrt{(3)^{2}+ (1)^{2} + (-1)^{2}}} = \frac{2}{\sqrt{11}} = \frac{2 \sqrt{11}}{11}\). So we have the point 
       \[ P_{1} =  \frac{2 \sqrt{11}}{11} \left( 3,1,-1 \right) = \boxed{\left( \frac{6\sqrt{11}}{11}, \frac{2\sqrt{11}}{11}, -\frac{2\sqrt{11}}{11} \right)}  \]
       
       And to find the point furthest from \( (3,1,-1) \), we want a vector pointing in the opposite direction, so we simply reverse the sign:
       \[ P_{2} = -\frac{2 \sqrt{11}}{11} \left( 3,1,-1 \right) = \boxed{\left( -\frac{6\sqrt{11}}{11}, -\frac{2\sqrt{11}}{11}, \frac{2\sqrt{11}}{11} \right)} \]
\end{example}



\section{Dot Products and Cross Products}
\subsection{Dot Products}
\begin{dfn}
An \( n \)-dimensional \vocab{row vector} is a \( 1 \times n \) matrix:
\[
    \vb{\upsilon} = \begin{bmatrix}
        \upsilon_1 & \cdots & \upsilon_n
    \end{bmatrix}.
\]
\end{dfn}

While row and column vectors contain the same components, they behave differently in matrix operations. For now, we will set aside these distinctions and treat them informally.

\begin{dfn}
Given two real vectors 
\[
    \vb{v}=  \begin{bmatrix}
        v_1 \\
        \vdots \\
        v_n
    \end{bmatrix}, \quad
    \vb{w} =  \begin{bmatrix}
        w_1 \\
        \vdots \\
        w_n
    \end{bmatrix},
\]
we define the \vocab{dot product} of \( \vb{v} \) and \( \vb{w} \), denoted \( \vb{v} \cdot \vb{w} \), as
\[
    \vb{v} \cdot \vb{w} = \sum_{j=1}^{n} v_j w_j.
\]
For those familiar with matrix multiplication, the dot product can also be viewed as the product of the row vector corresponding to \( \vb{v} \) and the column vector \( \vb{w} \). \\
We will sometimes denote the dot product of \( \vb{v} \) and \( \vb{w} \) as \( \left< \vb{v}, \vb{w} \right> \).
\end{dfn}

Before we investigate the geometric properties of the dot product, we first verify a key algebraic property.

\begin{theorem}
The dot product is linear in each argument.
\end{theorem}
\begin{proof}
We begin by noting that the dot product is symmetric in its arguments: \( \vb{v} \cdot \vb{w} = \vb{w} \cdot \vb{v} \). This means it suffices to show linearity in the first argument. Let \( \vb{u}, \vb{v}, \vb{w} \in \mathbb{R}^n \) and \( \lambda \in \mathbb{R} \). Then
\[
    \vb{u} + \lambda \vb{v} = \begin{bmatrix}
        u_1 + \lambda v_1 \\
        \vdots \\
        u_n + \lambda v_n
    \end{bmatrix}.
\]
So,
\begin{align*}
    (\vb{u} + \lambda \vb{v}) \cdot \vb{w} 
    &= \sum_{j=1}^n (u_j + \lambda v_j) w_j \\
    &= \sum_{j=1}^n u_j w_j + \lambda \sum_{j=1}^n v_j w_j \\
    &= \vb{u} \cdot \vb{w} + \lambda\, \vb{v} \cdot \vb{w}.
\end{align*}
\end{proof}

Note that we can express the length \( \abs{\vb{v}} \) of a vector \( \vb{v} \) in terms of the dot product:
\[
    \abs{\vb{v}}^2 = \vb{v} \cdot \vb{v}.
\]

Armed with this identity, we can now give a geometric interpretation of what the dot product measures. Consider the following figure:

\begin{center}
    \includegraphics[width=0.25\textwidth]{figures/algebra/linearalgebra/dotproduct1.png}
    \label{fig:dot-product-interpretation}
\end{center}

Applying the Law of Cosines to the triangle formed by the vectors, we obtain:
\[
    \abs{{\color[HTML]{2f21a1} \vb{v} - \vb{w}}}^2 
    = \abs{{\color[HTML]{3a8939} \vb{v}}}^2 
    + \abs{{\color[HTML]{a1212e} \vb{w}}}^2 
    - 2 \abs{{\color[HTML]{3a8939} \vb{v}}} \abs{{\color[HTML]{a1212e} \vb{w}}} \cos(\theta).
\]

We now rewrite both sides using the dot product:
\begin{align*}
    (\vb{v} - \vb{w}) \cdot (\vb{v} - \vb{w}) 
    &= \vb{v} \cdot \vb{v} + \vb{w} \cdot \vb{w} - 2\, \vb{v} \cdot \vb{w} \tag*{by bilinearity and symmetry} \\
    &= \vb{v} \cdot \vb{v} + \vb{w} \cdot \vb{w}  - 2 \abs{\vb{v}} \abs{\vb{w}} \cos(\theta).
\end{align*}

Comparing both expressions, we conclude
\[
    \vb{v} \cdot \vb{w} = \abs{\vb{v}} \abs{\vb{w}} \cos(\theta).
\]

This final expression tells us that the dot product measures how much the vector \( {\color[HTML]{3a8939} \vb{v}} \) points in the direction of \( {\color[HTML]{a1212e} \vb{w}} \), and vice versa. Here is the first consequence of this fact. 

\begin{theorem}[The dot product detects orthogonality]\label{thm: dot product detects orthogonality}
    Two nonzero vectors \( \vb{v} \) and \( \vb{w} \) are perpendicular if and only if \( \vb{v} \cdot \vb{w} = 0 \).
\end{theorem}
\begin{proof}
    \( \Rightarrow \) Suppose \( \vb{v} \) and \( \vb{w} \) are perpendicular. Then the angle \( \theta \) between them satisfies \( \cos(\theta) = 0 \). By the geometric definition of the dot product,
    \[
        \vb{v} \cdot \vb{w} = \abs{\vb{v}} \abs{\vb{w}} \cos(\theta) = \abs{\vb{v}} \abs{\vb{w}} \cdot 0 = 0.
    \]
    \( \Leftarrow \) Conversely, suppose \( \vb{v} \cdot \vb{w} = 0 \). Then again using the geometric definition,
    \[
        \vb{v} \cdot \vb{w} = \abs{\vb{v}} \abs{\vb{w}} \cos(\theta) = 0.
    \]
    Since \( \vb{v} \) and \( \vb{w} \) are nonzero, \( \abs{\vb{v}} \neq 0 \) and \( \abs{\vb{w}} \neq 0 \). Therefore, it must be that \( \cos(\theta) = 0 \), which implies that \( \theta = \frac{\pi}{2} \). In other words, the vectors are perpendicular.
\end{proof}


\begin{theorem}[The dot product detects colinearity or parallelism]
    Two nonzero vectors \( \vb{v} \) and \( \vb{w} \) are colinear (parallel or antiparallel) if and only if
    \[
        \vb{v} \cdot \vb{w} = \lvert \vb{v} \rvert \lvert \vb{w} \rvert
        \quad\text{or}\quad
        \vb{v} \cdot \vb{w} = -\lvert \vb{v} \rvert \lvert \vb{w} \rvert.
    \]
\end{theorem}
\begin{proof}
    \(\Rightarrow\)  
    Suppose \( \vb{v} \) and \( \vb{w} \) are colinear. Then the angle \(\theta\) between them is either \(0\) (parallel, same direction) or \(\pi\) (parallel, opposite direction).  
    By the dot product formula,
    \[
        \vb{v} \cdot \vb{w} = \lvert\vb{v}\rvert \lvert\vb{w}\rvert \cos\theta,
    \]
    so if \(\theta = 0\), \(\cos\theta = 1\) and \(\vb{v} \cdot \vb{w} = \lvert\vb{v}\rvert \lvert\vb{w}\rvert\).  
    If \(\theta = \pi\), \(\cos\theta = -1\) and \(\vb{v} \cdot \vb{w} = -\lvert\vb{v}\rvert \lvert\vb{w}\rvert\).

    \(\Leftarrow\) 
    Conversely, suppose
    \[
        \vb{v} \cdot \vb{w} = \pm \lvert\vb{v}\rvert \lvert\vb{w}\rvert.
    \]
    Then by the dot product formula,
    \[
        \cos\theta = \pm 1,
    \]
    which means \(\theta = 0\) or \(\theta = \pi\) radians.  
    In either case, \( \vb{v} \) and \( \vb{w} \) are colinear.
\end{proof}

\begin{exercise}
    Let \( \vb{x} = \left< 1,1 \right> \). Find all vectors \( \vb{y} \in \mathbb{R}^{2} \) such that \( \vb{x} \cdot \vb{y} =0 \) and \( \norm{\vb{x}} = \norm{\vb{y}} \).
\end{exercise}
\begin{solution}
    We require that \( \vb{y} \) is perpendicular to \( \vb{x} \). So \( \vb{y} = \left< \lambda, - \lambda \right> \) for some \( \lambda \in \mathbb{R}. \) Since \( \norm{\vb{x}} = \norm{\vb{y}} \), we have 
    \begin{align*}
        \sqrt{2} &= \sqrt{ \left( \lambda  \right)^{2} + \left( -\lambda \right)^{2}} \\
        2 &= 2 \left( \lambda \right)^{2}\\
        1&= \lambda^{2}
    \end{align*}
    This gives that \( \lambda =\pm 1 \) so \( \boxed{ \vb{y} = \left< 1,-1 \right>} \) and \( \boxed{\vb{y} = \left< -1,1 \right>} \; \).
\end{solution}


Another useful application of the dot product is that it helps us define lines, planes, hyperplanes etc. 


Consider a line \( a x+by =c \). We know that this line is parallel to \( ax+by=0 \), we will focus on this form for the time being. We can express \( ax+by=0 \) in terms of the dot product. Namely,
\[ \begin{bmatrix}
    a & b\\
\end{bmatrix} \cdot \begin{bmatrix}
    x \\
    y\\
\end{bmatrix} =0 .\]

By \Cref{thm: dot product detects orthogonality}, this means that the line \( ax+by=0 \) consists of all vectors \( \left< x,y \right> \) that are perpendicular to \( \left< a,b \right> \) and that adding \( c \) to the right hand side simply shifts this line without change its slope/direction.

This is a preferable interpretation because it generalizes quite easily to higher dimensions. In particular, if we have the equation 

\[ \sum_{j=1}^{n} a_{j} x_{j} =0  \]

Then the solution consists of all vectors \( \left< x_{1}, x_{2}, \dots, x_{n} \right> \) that are perpendicular to \( \left< a_{1}, a_{2}, \dots, a_{n} \right> \). Adding a constant to the right hand side simply shifts this hyperplane in \( \mathbb{R}^{n} \), again, without changing slope or direction. 

\subsubsection{Vector Projections}
Another very useful application of the dot product is the calculation of how much one vector points in the direction of the other.

We denote the \vocab{vector projection of \( \vb{w} \) onto \( \vb{v} \)} as \( \mathrm{proj}_{\vb{v}}\vb{w} .\) Now since \( \vb{w} \) projects onto \( \vb{v} \), \( \mathrm{proj}_{\vb{v}}\vb{w} \) must point in the direction of \( \vb{v} \) or more specifically, \( \mathrm{proj}_{\vb{v}}\vb{w} \) must point in the direction of \( \frac{\vb{v}}{\abs{\vb{v}}} \), the unit vector in the direction of \( \vb{v} \). So 
\[ \mathrm{proj}_{\vb{v}}\vb{w} =\abs{\mathrm{proj}_{\vb{v}}\vb{w}} \frac{\vb{v}}{\abs{\vb{v}}} .\] So all we need to do is find \( \abs{\mathrm{proj}_{\vb{v}}\vb{w}}\). Luckily our friend, the dot product, can help. If we consider the right triangle with \( \vb{w} \) as the hypotenuse and \( \mathrm{proj}_{\vb{v}}\vb{w} \) as one of the legs, then 
\[ \cos{ \left( \theta \right) }= \frac{\abs{\mathrm{proj}_{\vb{v}}\vb{w}}}{\abs{\vb{w}}} \]
Now we can multiply both sides by \( \abs{\vb{w}} \) to get 
\[ \abs{\vb{w}} \cos{ \left( \theta  \right) } = \abs{\mathrm{proj}_{\vb{v}}\vb{w}} .\] While it might seem like we are done, I did promise you that the dot product will show up. In that spirit, let us multiply and divide the left hand side by \( \abs{\vb{v}} \) to get 
\[ \frac{\abs{\vb{v}} \abs{\vb{w}} \cos{ \left( \theta \right) }}{\abs{\vb{v}}}  = \abs{\mathrm{proj}_{\vb{v}}\vb{w}}\] so 
\[ \boxed{\abs{\mathrm{proj}_{\vb{v}}\vb{w}} = \frac{\vb{v} \cdot \vb{w}}{\abs{\vb{v}}}} \]
Substituting this into our earlier expression, we get 
\[ \boxed{\mathrm{proj}_{\vb{v}} \vb{w} = \frac{\vb{v} \cdot \vb{w}}{\vb{v} \cdot \vb{v}} \vb{v}} \]

\subsection{Square Matrices and the Determinant}

\begin{dfn}
    By an \vocab{\( m \times n \) matrix}, we mean an \( m \)-by-\( n \) rectangular array of numbers:
    \[
        A = \begin{bmatrix}
            a_{11} & a_{12} & \dots & a_{1n} \\
            a_{21} & a_{22} & \dots & a_{2n} \\
            \vdots & \vdots & \ddots & \vdots \\
            a_{m1} & a_{m2} & \dots & a_{mn}
        \end{bmatrix}.
    \]
    The entry \(a_{ij}\) denotes the element in the \(i\)th row and \(j\)th column.  
    Unless otherwise stated, we assume the entries \(a_{ij}\) are real numbers.  
    If \(m = n\), we say \(A\) is a \vocab{square matrix}.  
\end{dfn}

Recall that a \(1 \times n\) matrix is a \emph{row vector} and an \(m \times 1\) matrix is called a \emph{column vector}. \\

If we are given two vectors \(  {\color[HTML]{FF0000} \vb{v} = \left< v_{1},v_{2} \right>}\) and \(  {\color[HTML]{0000FF} \vb{w} = \left< w_{1},w_{2} \right>}\), we may arrange them as \textbf{columns of a matrix \( A \)}.
\[ A = \begin{bmatrix}
    {\color[HTML]{FF0000} v_{1}} & {\color[HTML]{0000FF} w_{1}}\\
     {\color[HTML]{FF0000} v_{2}} &  {\color[HTML]{0000FF} w_{2}}\\
\end{bmatrix}. \]
Furthermore there is a special operation called the \textbf{determinant} of \( A \) that returns the signed area spanned by the vectors \( {\color[HTML]{FF0000} \vb{v}} \) and \(  {\color[HTML]{0000FF} \vb{w}} \). We will denote this as \( {\color[HTML]{7E00FF} \det \left( A \right)} \).
\begin{center}
    \includegraphics[width=0.25\textwidth]{figures/algebra/linearalgebra/determinant.png}
    \label{fig:cross-product picture}
\end{center}
One of the reasons the dot product is so attractive is that it can be computed using components only. The determinant can be similarly calculated from components only. Here is how.\\ 
Recall that the {\color[HTML]{7E00FF}area of a parallelogram} is {\color[HTML]{FF0000}base} times {\color[HTML]{3B00FF} height}. We can find the height by taking the length of the component of {\color[HTML]{0000FF} \( \vb{w} \)} which does not point in direction of {\color[HTML]{FF0000} \( \vb{v} \)}. This is simply {\color[HTML]{3B00FF} \( \abs{\vb{w}} \abs{\sin{ \left( \theta \right) }} \)}, where \( \theta \) is the angle formed between {\color[HTML]{FF0000}\( \vb{v} \)} and {\color[HTML]{0000FF} \( \vb{w} \)}. To summarize, 
\begin{align*}
    \text{{\color[HTML]{7E00FF}area of a parallelogram}} &= \text{{\color[HTML]{FF0000}base} } \text{ times }\text{ {\color[HTML]{3B00FF} height}} \\
     {\color[HTML]{7E00FF} \abs{\det \left( A \right)}} &= {\color[HTML]{FF0000} \abs{\vb{v}}}  {\color[HTML]{3B00FF} \abs{\vb{w}} \abs{\sin{ \left( \theta \right) }}}
\end{align*}
\textbf{Note:} Here we are assuming $ \theta \in \left[ 0, \frac{\pi}{2} \right]$ for visualization's sake. Of course, $\theta \in \left[ 0, 2 \pi \right)$ would explain why the area is \emph{signed}. 
\begin{center}
    \includegraphics[width=0.25\textwidth]{figures/algebra/linearalgebra/areaofparallelogram.png}
    \label{fig:cross-product picture}
\end{center}
Now, we can use a useful identity \( \sin{ \left( \theta \right) } = \cos{ \left( \frac{\pi}{2} - \theta \right) } .\) Substituting this in, we have 
\[ {\color[HTML]{7E00FF} \abs{\det \left( A \right)}} = {\color[HTML]{FF0000} \abs{\vb{v}}} {\color[HTML]{3B00FF} \abs{\vb{w}} \cos{ \left( \frac{\pi }{2} -\theta \right) }} \]
This is looking like a dot product. To get us over the finish line, we will define a new vector {\color[HTML]{3B00FF}\( \vb{w}' \)} with the same length as {\color[HTML]{0000FF} \( \vb{w} \)}, forms an angle of \( \varphi= \frac{\pi}{2}- \theta \) with {\color[HTML]{FF0000} \( \vb{v} \)}, and forms a \( \frac{\pi}{2} \) angle with {\color[HTML]{0000FF} \( \vb{w} \)}. Therefore 
\begin{align*}
    {\color[HTML]{7E00FF} \det \left( A \right)} &=  {\color[HTML]{FF0000} \abs{\vb{v}}} \ {\color[HTML]{3B00FF} \abs{\vb{w}'} \cos{ \left( \varphi\right) }} \\
    &= {\color[HTML]{FF0000} \vb{v}} \cdot {\color[HTML]{3B00FF} \vb{w}'}
\end{align*}
\begin{center}
    \includegraphics[width=0.25\textwidth]{figures/algebra/linearalgebra/dettodot.png}
    \label{fig:dettodot}
\end{center}
By rotating \( {\color[HTML]{0000FF} \vb{w}} \) counterclockwise by 90°, we obtain \( {\color[HTML]{3B00FF} \vb{w}' = \left< w_{2},-w_{1} \right>} \). This choice ensures that \( \vb{w}' \) is perpendicular to \( \vb{w} \), has the same magnitude, and preserves the correct sign for the determinant based on the orientation of the vectors. So 
\begin{align*}
    {\color[HTML]{7E00FF} \det \left( A \right)} &= {\color[HTML]{FF0000} \vb{v}} \cdot {\color[HTML]{3B00FF} \vb{w}'} \\
    &=  {\color[HTML]{FF0000} \left< v_{1}, v_{2} \right>} \cdot {\color[HTML]{3B00FF} \left< w_{2}, - w_{1} \right>} \\
    &= {\color[HTML]{FF0000} v_{1}}{\color[HTML]{3B00FF} w_{2}} -{\color[HTML]{FF0000}  v_{2}}{\color[HTML]{3B00FF} w_{1}} 
\end{align*}
\subsection{The Cross-Product}

\begin{dfn}
    The \vocab{cross product} is a function \( C: \mathbb{R}^{2} \times \mathbb{R}^{3} \to \mathbb{R}^{3} \) that takes two vectors \( \vb{x} = \left< x_{1}, x_{2 }, x_{3} \right> \) and \( \vb{y} = \left< y_{1 }, y_{2 }, y_{3} \right> \) and produces the vector:
    \begin{align*}
         \vb{x} \times \vb{y} &= \det \begin{bmatrix}
        \hat{\vb{i}} & \hat{\vb{j}} & \hat{\vb{k}}\\
        x_{1} & x_{2} & x_{3}\\
        y_{1} & y_{2} & y_{3}\\
    \end{bmatrix}\\ 
    &= \det \begin{bmatrix}
        x_{2 } & x_{3}\\
         y_{2} & y_{3}\\
    \end{bmatrix}  \hat{\vb{i}} - \det \begin{bmatrix}
        x_{1 } & x_{3}\\
         y_{1} & y_{3}\\
    \end{bmatrix}  \hat{\vb{j}} + \det \begin{bmatrix}
        x_{1 } & x_{2}\\
         y_{1} & y_{2}\\
    \end{bmatrix}  \hat{\vb{k}} \\
    &= \left( x_{2}y_{3} - x_{3}y_{2} \right) \hat{\vb{i}} + \left( x_{3}y_{1}-x_{1}y_{3} \right) \hat{\vb{j}} + \left( x_{1}y_{2}- x_{2}y_{1} \right) \hat{\vb{k}} \\
    &= \left< x_{2}y_{3} - x_{3}y_{2} , \;  x_{3}y_{1}-x_{1}y_{3}, \; x_{1}y_{2}- x_{2}y_{1}  \right>.
    \end{align*}
    It is trivial to check that \( \vb{x} \times  \vb{y} = - (\vb{y} \times \vb{x}). \)
\end{dfn}

\begin{lemma}
    The vector \( \vb{x} \times  \vb{y} \) is perpendicular to both \( \vb{x} \) and \( \vb{y} \).
\end{lemma}
\begin{proof}
    
\end{proof}
\begin{exercise}
    Suppose that \( \vb{A} = \left< -1, 0,1  \right> \) and \( \vb{B} = \left< 1, -2, 2 \right> \). Find 
    \begin{enumerate}[label=\textbf{\roman*)}]
        \item A vector perpendicular to both \( \vb{A} \) and \( \vb{B} \) whose \( y \)-component is \( 6 \). 
        \item A vector perpendicular to both \( \vb{A} \) and \( \vb{B} \) whose length is \( 6 \). 
    \end{enumerate}
\end{exercise}
\begin{solution} For both parts of the problem, we need to compute \( \vb{A} \times \vb{B} \). 
    \begin{align*}
        \vb{A} \times \vb{B} &= \mathrm{det} \begin{bmatrix}
            \hat{\vb{i}} & \hat{\vb{j}} & \hat{\vb{k}}\\
            -1 & 0 & 1\\
            1 & -2 & 2\\
        \end{bmatrix} \\
        &= \mathrm{det} \begin{bmatrix}
            0  & 1 \\
            -2  & 2 \\
        \end{bmatrix} \hat{\vb{i}} - \mathrm{det} \begin{bmatrix}
            -1  & 1 \\
            1  & 2 \\
        \end{bmatrix} \hat{\vb{j}} + \mathrm{det} \begin{bmatrix}
            -1  & 0 \\
            1 & -2 \\
        \end{bmatrix} \hat{\vb{k}} \\
        &= 2 \hat{\vb{i}} + \hat{\vb{j}} + 2 \hat{\vb{k}}\\
        &= \left< 2,3,2 \right>
    \end{align*}
    We can verify that we have the correct vector since \( \left( \vb{A} \times \vb{B} \right) \cdot \vb{A} = 0\) and e \( \left( \vb{A} \times \vb{B} \right) \cdot \vb{B} = 0\).
    \begin{enumerate}[label=\textbf{\roman*)}]
        \item $ $\\ 
        To find a vector perpendicular to both \( \vb{A} \) and \( \vb{B} \) whose \( y \)-component is \( 6 \), we just need to take our \( \vb{A} \times \vb{B} \) vector and scale it so that its y-component is \( 6 \). This gives us \[ \boxed{\vb{v} = \left< 4,6,4 \right>} .\]
        \item $ $\\ 
        To find a vector perpendicular to both \( \vb{A} \) and \( \vb{B} \) whose length is \( 6 \), we first need to calculate \( \abs{\vb{A} \times \vb{B}} \), which is 
        \[  \abs{\vb{A} \times \vb{B}}= \sqrt{2^{2} + 3^{2} + 2^{2}} = \sqrt{17} \]
        Now, we just need to scale \(  \vb{A} \times \vb{B} \) by the quantity \( \frac{6 \sqrt{17}}{17} \) (since \( \frac{\sqrt{17 }}{17} \) normalizes \( \vb{A} \times \vb{B}   \) and multiplying by \( 6 \) will scale this new unit vector ) so we have 
        \[ \boxed{\vb{w} = \left<  \frac{12 \sqrt{17}}{17},\frac{18 \sqrt{17}}{17}, \frac{12 \sqrt{17}}{17}  \right> \text{ .}} \]
    \end{enumerate}    
\end{solution}
Recalling our discussion of how the dot product 
\[ a \cdot \left( \vb{x} - p \right)=0 \]
determines a line in \( \mathbb{R}^{n} \), we now have the tools to find a plane in \( \mathbb{R}^{3} \) given \( 3 \) points on it. 
\begin{example}
    Find the equation of a plane that contains the points. \( A = \left( 4,1,3 \right) \), \( B = (1,5,4) \), and \( C = (-3,2,6) \). \\ 
    We first find \( \va{CA} \) and \( \va{CB} \). 
    \[ \va{CA} = \left< 4 +3, 1-2, 2-6  \right> = \boxed{\left< 7, -1, -4  \right>} \quad \text{and} \quad \va{CB} = \left< 1+3, 5-2 ,4-6 \right> = \boxed{\left< 4,3,-2 \right>} .\]
    To find the normal, we have \
    \begin{align*}
        \hat{\vb{n}} &= \va{CA} \times \va{CB} \\
        &= \mathrm{det} \begin{bmatrix}
            \hat{\vb{i}} & \hat{\vb{j}}& \hat{\vb{k}}\\
            7 & -1 & -4\\
            4 & 3 & -2\\
        \end{bmatrix}\\
        &= \mathrm{det} \begin{bmatrix}
            -1 & -4\\
            3 & -2\\
        \end{bmatrix} \hat{\vb{i}} - \mathrm{det} \begin{bmatrix}
            7 & -4\\
            4 & -2 \\
        \end{bmatrix}\hat{\vb{j}} + \mathrm{det}\begin{bmatrix}
            7 & -1\\
            4 & 3\\
        \end{bmatrix} \hat{\vb{k}} \\
        &= \left< 14, -2, 25 \right>
    \end{align*}
    So the equation of our plane is 
    \begin{align*}
        14(x-1) - 2 \left( y-5 \right) + 25(z-4) &=0 \\
        14x -2y + 25z &= 104
    \end{align*}
    We can check that \( A \) and \( C \) also lie on the plane.
\end{example}

Using the cross and dot products, we can find the volume of a parallelepiped to be 
\[ \text{volume of a parallelepiped} = \abs{ \left( A \times B  \right) \cdot C} \]
where \( A, B \), and \( C \) are vectors that make up its side. 



\chapter{Solving Linear Equations}
\section{Vectors and Linear Equations}
\subsection{Introduction to the Row and Column Picture: Two Equations, Two Unknowns}


    Suppose that we are given the following system of equations
    \begin{align}
2x + 3y &= 12 \label{6/4/25/1} \\
x - y &= 1 \label{6/4/25/2}
\end{align}
We wish to find values for \( x \) and \( y \) that simultaneously solve \cref{6/4/25/1} and \cref{6/4/25/2}. \\
We can first view this system by the rows; that is, we wish to find the point of intersection of the lines \(  2x + 3y = 13 \) and \( x - y = 1 \), which is shown in \Cref{fig:FirstLinearSystem}. This picture is quite familiar.\\
The novel idea is to now consider the column picture. We can combine the above system into a single vector equation 
\[ x\begin{pmatrix}2\\1\end{pmatrix} +y \begin{pmatrix}
3\\
-1
\end{pmatrix}= \begin{pmatrix}
12\\
1
\end{pmatrix} .\]
Now, we wish to find the correct scalars \( x \) and \( y \) that makes this equation true, this is highlighted in \Cref{fig:FirstScalars}. \\



\begin{figure}[ht]
  \centering

  \begin{subfigure}[b]{0.48\textwidth}
    \centering
    \fbox{\includegraphics[width=\textwidth]{figures/algebra/firstlinearsystem}}
    \caption{The lines \(2x+3y=12\) (blue) and \(x-y=1\) (green) intersect at \((3,2)\).}
    \label{fig:FirstLinearSystem}
  \end{subfigure}
  \hfill
  \begin{subfigure}[b]{0.48\textwidth}
    \centering
    \fbox{\includegraphics[width=\textwidth]{figures/algebra/firstscalars}}
    \caption{\(c_1 = x = 3\) copies of \(u = \begin{pmatrix}2\\1\end{pmatrix}\) plus \(c_2 = y = 2\) copies of \(v = \begin{pmatrix}3\\-1\end{pmatrix}\) yields \(w = \begin{pmatrix}12\\1\end{pmatrix}\).}
    \label{fig:FirstScalars}
  \end{subfigure}

  \caption{Geometric and algebraic views of solving a linear system.}
  \label{fig:GeometricAndAlgebgraicViews}
\end{figure}

Now to solve this equation, we observe that we can add \( 3 \) copies of \Cref{6/4/25/2} to \Cref{6/4/25/1} to get 
\begin{equation}
	2x+3y+ 3(x-y)= 12 + 3(1) \Rightarrow 5x =15 \label{6/4/25/3}
\end{equation}

From \Cref{6/4/25/3}, we see that \( x=3 \) and then substitution into either \Cref{6/4/25/1} or \Cref{6/4/25/2} (If you are new to this, substitute into both to verify consistency) and solving for \( y \), we get that \( y=2 \), which is consistent with both of our pictures.\\
The standard way to write this equation in linear algebra is collect all of our left coefficients in a \vocab{coefficient matrix} \( A \) where 
\[ A = \begin{pmatrix}
2 & 3\\
1 & -1
\end{pmatrix} .\] Then we collect our variables 
\[ \vb{x}= \begin{pmatrix}
x \\
y
\end{pmatrix} \] and finally our right hand side 
\[ \vb{b} = \begin{pmatrix}
12\\
1
\end{pmatrix} .\]
Combining everything, we have 
\[ A \vb{x}= \vb{b} \quad \text{or} \quad  \begin{pmatrix}
2 & 3\\
1 & -1
\end{pmatrix} \begin{pmatrix}
x \\
y
\end{pmatrix} = \begin{pmatrix}
12\\
1
\end{pmatrix} . \]

\subsection{Three Equations, Three Unknowns}


\begin{example}
  Consider the following system of three equations with three unknowns:
\[
\begin{aligned}
  3x+y-z &= 2 \quad\quad&(1)\\
  4x+2y +3z &=23 \quad\quad&(2)\\
  x-3y+2z &=19 \quad\quad&(3)
\end{aligned}
\]

Using the row/plane picture, the normals to the three planes are
\[
\mathbf{n}_1=(3,1,-1),\qquad \mathbf{n}_2=(4,2,3),\qquad \mathbf{n}_3=(1,-3,2).
\]
None of these is a scalar multiple of another so no two planes are parallel. This means that the first two planes intersect along a line and that line will intersect the third plane at a point. That point will be the solution to our system.

Now solve the system by eliminating \(y\).  One convenient approach is to form combinations that cancel the \(y\)-terms.

First, add (1), (2), and (3):
\[
(3x+y-z)+(4x+2y+3z)+(x-3y+2z)=8x+0y+4z=44,
\]
so
\[
8x+4z=44 \quad\Longrightarrow\quad 2x+z=11. \tag{A}
\]

Next, take \(-1\) times (1), plus \(2\) times (2), plus \(1\) times (3):
\[
-1\cdot(3x+y-z)+2\cdot(4x+2y+3z)+1\cdot(x-3y+2z)
\]
Compute coefficients:
\[
x:\ -3+8+1=6,\qquad y:\ -1+4-3=0,\qquad z:\ 1+6+2=9,
\]
RHS: \(-2+46+19=63\). Thus
\[
6x+9z=63 \quad\Longrightarrow\quad 2x+3z=21. \tag{B}
\]

Now solve the \(2\times2\) system (A) and (B):
\[
\begin{cases}
2x+z=11\\[4pt]
2x+3z=21
\end{cases}
\]
Subtract (A) from (B) to eliminate \(x\):
\[
(2x+3z)-(2x+z)=2z=21-11=10 \quad\Longrightarrow\quad z=5.
\]
Substitute \(z=5\) into (A):
\[
2x+5=11 \quad\Longrightarrow\quad 2x=6 \quad\Longrightarrow\quad x=3.
\]
Finally substitute \(x=3,z=5\) into equation (1) to find \(y\):
\[
3(3)+y-(5)=2 \quad\Longrightarrow\quad 9+y-5=2 \quad\Longrightarrow\quad y=-2.
\]
So the solution is
\[
\boxed{(x,y,z)=(3,-2,5)}.
\]

Writing the system in matrix form \(A\mathbf{x}=\mathbf{b}\):
\[
A=\begin{bmatrix}
3 & 1 & -1\\[4pt]
4 & 2 & 3\\[4pt]
1 & -3 & 2
\end{bmatrix},\qquad
\mathbf{x}=\begin{bmatrix}x\\y\\z\end{bmatrix},\qquad
\mathbf{b}=\begin{bmatrix}2\\23\\19\end{bmatrix}.
\]
\end{example}


\section{Elimination}
We want a systematic way of solving systems of linear equations. Gaussian elimination provides this. 
We first note that the system
\begin{align*}
  3x+y-z &= 2 \\
  4x+2y +3z &=23 \\
  x-3y+2z &=19 
\end{align*}
can be written more compactly as an \vocab{augmented matrix}.
\[ \begin{bmatrix}[*2cr@{\quad}|@{\quad}>{\color{red}}r]
  3 & 1 & -1  &  2 \\
  4 & 2 & 3 & 23\\
  1 & -3 & 2 & 19
\end{bmatrix} \] 
Here each equation gets assigned a row. We will dedicate this section to motivating the row operations needed for Gaussian elimination. \\
Let us take, for example the first two equations
\begin{align*}
  3x+y-z &= 2 \\
  4x+2y +3z &=23 \\
\end{align*}
We can take arbitrary linear combinations of these two equations and \emph{replace one equation}. For example, we can take three copies of the second equation and subtract from it four copies of the first. This gives us 
\[ 3 \left( 4x+2y +3z =23 \right) - 4 \left(  3x+y-z = 2 \right) \Rightarrow \left( 12x + 6y + 9z =69 \right) - \left( 12x +4y -4z =8 \right) \Rightarrow \boxed{0x +2y +13z = 61} \]
Notice that because we are taking linear combinations of these two equations, the solution we found earlier of \( \left( x,y,z \right) = \left( 3,-2,5 \right) \) is a solution to \[ 0x +2y +13z = 61. \] 
We can verify this: \( 2(-2) + 13(5) = -4 + 65 = 61 \). As such the system 
\begin{align*}
  3x+y-z &= 2 \\
  0x +2y +13z &= 61 \\
  x-3y+2z &=19 
\end{align*}
still has the same solution as above. So we can write our augmented system as
\[ \begin{bmatrix}[*2cr@{\quad}|@{\quad}>{\color{red}}r]
  3 & 1 & -1  &  2 \\
  0 & 2 & 13 & 61\\
  1 & -3 & 2 & 19
\end{bmatrix} \]
Finally note that row exchange is trivially acceptable since our solution shouldn't depend on the order in which we write out equations.\\
Notice that if we have an augmented matrix of the form
\[ \begin{bmatrix}[*2cr@{\quad}|@{\quad}>{\color{red}}r]
  a & b & c  &  s  \\
  0 & d &e  & t \\
  0 & 0 & f & u
\end{bmatrix} \]
This corresponds to the system of equations
\begin{align*}
  ax+by+ cz &= s \\
  dy +ez &=t  \\
  fz &=u
\end{align*} 
So then we can simply solve for \( z \) in the third equation, then back substitute our solution for \( z \) into the second equation to solve for \( y \), and finally solve for \( x \) by substituting the solutions for \( y \) and \( z \). \emph{This is the goal for Gaussian elimination}. We wish to convert an arbitrary linear system into an \vocab{upper triangular matrix.}

\begin{example}
  Let us try Gaussian elimination with a system we are very familiar with 
  \begin{align*}
  3x+y-z &= 2 \\
  4x+2y +3z &=23 \\
  x-3y+2z &=19 
\end{align*}
Converting to an augmented matrix, we get
\[ \begin{bmatrix}[*2cr@{\quad}|@{\quad}>{\color{red}}r]
  3 & 1 & -1  &  2 \\
  4 & 2 & 3 & 23\\
  1 & -3 & 2 & 19
\end{bmatrix} \] 

First, we eliminate the $x$-term in the second row by replacing $R_2$ with $(3R_2 - 4R_1)$:
\[\begin{bmatrix}[*2cr@{\quad}|@{\quad}>{\color{red}}r]
  3 & 1 & -1  &  2 \\
  4 & 2 & 3 & 23\\
  1 & -3 & 2 & 19
\end{bmatrix}\quad  \underrightarrow{(3R_{2} - 4R_{1}) \to R_{2}} \quad \begin{bmatrix}[*2cr@{\quad}|@{\quad}>{\color{red}}r]
  3 & 1 & -1  &  2 \\
  0 & 2 & 13 & 61\\
  1 & -3 & 2 & 19
\end{bmatrix} \]

Next, we eliminate the $x$-term in the third row by replacing $R_3$ with $(3R_3 - R_1)$:
\[  \begin{bmatrix}[*2cr@{\quad}|@{\quad}>{\color{red}}r]
  3 & 1 & -1  &  2 \\
  0 & 2 & 13 & 61\\
  1 & -3 & 2 & 19
\end{bmatrix} \quad  \underrightarrow{(3R_{3} - R_{1}) \to R_{3}} \quad  \begin{bmatrix}[*2cr@{\quad}|@{\quad}>{\color{red}}r]
  3 & 1 & -1  &  2 \\
  0 & 2 & 13 & 61\\
  0 & -10 &7 & 55
\end{bmatrix} \]

Finally, we eliminate the $y$-term in the third row by replacing $R_3$ with $(R_3 + 5R_2)$:
\[ \begin{bmatrix}[*2cr@{\quad}|@{\quad}>{\color{red}}r]
  3 & 1 & -1  &  2 \\
  0 & 2 & 13 & 61\\
  0 & -10 &7 & 55
\end{bmatrix}  \quad  \underrightarrow{(R_{3} + 5R_{2}) \to R_{3}} \quad \begin{bmatrix}[*2cr@{\quad}|@{\quad}>{\color{red}}r]
  3 & 1 & -1  &  2 \\
  0 & 2 & 13 & 61\\
  0 & 0 & 72 & 360
\end{bmatrix}\]

We now have the upper triangular system 
\begin{align*}
  3x +y -z &= 2\\
  2y + 13z &= 61\\
  72z &= 360
\end{align*}

Now we use back substitution. From the third equation, we see that $z = \frac{360}{72} = 5$, so $\boxed{z=5}$. 

From the second equation:
\begin{align*}
  2y + 13(5) &= 61\\
  2y + 65 &= 61\\
  2y &= -4\\
  y &= -2
\end{align*}
so $\boxed{y=-2}$. 

Finally, from the first equation:
\begin{align*}
  3x + (-2) - 5 &= 2\\
  3x - 7 &= 2\\
  3x &= 9\\
  x &= 3
\end{align*}
so $\boxed{x=3}$.

Therefore, our solution is $(x,y,z) = (3,-2,5)$.
\end{example}

\begin{example}
  Use Gaussian elimination to solve the system 
  \begin{align*}
    3w +2x +11y + 5z &=25 \\
    -2 w +7x -8y +z &=13\\
    12w-3x+9y + 4z &=-21 \\
    -6w +x -4y -3z &= 15
  \end{align*}
  We have the following augmented matrix 
  \[ \begin{bmatrix}[*3rr@{\quad}|@{\quad}>{\color{black}}r]
  3 & 2& 11  &  5 & 25\\
  -2 & 7 & -8 & 1 & 13\\
  12 & -3 & 9 & 4 & -21 \\
  -6 & 1 & -4 & -3 & 15
\end{bmatrix} \]
For the first column, we have 
  \[ \begin{bmatrix}[*3rr@{\quad}|@{\quad}>{\color{black}}r]
  3 & 2& 11  &  5 & 25\\
  -2 & 7 & -8 & 1 & 13\\
  12 & -3 & 9 & 4 & -21 \\
  -6 & 1 & -4 & -3 & 15
\end{bmatrix} \quad  \underrightarrow{(3R_{2} + 2R_{1}) \to R_{2}} \quad \begin{bmatrix}[*3rr@{\quad}|@{\quad}>{\color{black}}r]
  3 & 2& 11  &  5 & 25\\
  0 & 25 & -2 & 13 & 89\\
  12 & -3 & 9 & 4 & -21 \\
  -6 & 1 & -4 & -3 & 15
\end{bmatrix}\]
\[ \begin{bmatrix}[*3rr@{\quad}|@{\quad}>{\color{black}}r]
  3 & 2& 11  &  5 & 25\\
  0 & 25 & -2 & 13 & 89\\
  12 & -3 & 9 & 4 & -21 \\
  -6 & 1 & -4 & -3 & 15
\end{bmatrix}  \quad  \underrightarrow{(R_{3} -4 R_{1}) \to R_{3}} \quad \begin{bmatrix}[*3rr@{\quad}|@{\quad}>{\color{black}}r]
  3 & 2& 11  &  5 & 25\\
  0 & 25 & -2 & 13 & 89\\
  0 & -11 & -35 & -16 & -121 \\
  -6 & 1 & -4 & -3 & 15
\end{bmatrix} \]
\[ \begin{bmatrix}[*3rr@{\quad}|@{\quad}>{\color{black}}r]
  3 & 2& 11  &  5 & 25\\
  0 & 25 & -2 & 13 & 89\\
  0 & -11 & -35 & -16 & -121 \\
  -6 & 1 & -4 & -3 & 15
\end{bmatrix}\quad  \underrightarrow{(R_{4} +2 R_{1}) \to R_{3}} \quad  \begin{bmatrix}[*3rr@{\quad}|@{\quad}>{\color{black}}r]
  3 & 2& 11  &  5 & 25\\
  0 & 25 & -2 & 13 & 89\\
  0 & -11 & -35 & -16 & -121 \\
  0 & 5 & 18 & 7 & 65
\end{bmatrix} \]
For the second column, we have 
\[  \begin{bmatrix}[*3rr@{\quad}|@{\quad}>{\color{black}}r]
  3 & 2& 11  &  5 & 25\\
  0 & 25 & -2 & 13 & 89\\
  0 & -11 & -35 & -16 & -121 \\
  0 & 5 & 18 & 7 & 65
\end{bmatrix} \quad  \underrightarrow{(25R_{3} +11 R_{2}) \to R_{3}} \quad  \begin{bmatrix}[*3rr@{\quad}|@{\quad}>{\color{black}}r]
  3 & 2& 11  &  5 & 25\\
  0 & 25 & -2 & 13 & 89\\
  0 & 0 & -897 & -257 & -2046 \\
  0 & 5 & 18 & 7 & 65
\end{bmatrix} \]
\[ \begin{bmatrix}[*3rr@{\quad}|@{\quad}>{\color{black}}r]
  3 & 2& 11  &  5 & 25\\
  0 & 25 & -2 & 13 & 89\\
  0 & 0 & -897 & -257 & -2046 \\
  0 & 5 & 18 & 7 & 65
\end{bmatrix}  \quad  \underrightarrow{(5R_{4} - R_{2}) \to R_{4}} \quad \begin{bmatrix}[*3rr@{\quad}|@{\quad}>{\color{black}}r]
  3 & 2& 11  &  5 & 25\\
  0 & 25 & -2 & 13 & 89\\
  0 & 0 & -897 & -257 & -2046 \\
  0 & 0 & 92 & 22 & 236
\end{bmatrix} \]
\[ \begin{bmatrix}[*3rr@{\quad}|@{\quad}>{\color{black}}r]
  3 & 2& 11  &  5 & 25\\
  0 & 25 & -2 & 13 & 89\\
  0 & 0 & -897 & -257 & -2046 \\
  0 & 0 & 92 & 22 & 236
\end{bmatrix} \quad  \underrightarrow{ \left( \frac{1}{2}R_{4} \right) \to R_{4}} \quad \begin{bmatrix}[*3rr@{\quad}|@{\quad}>{\color{black}}r]
  3 & 2& 11  &  5 & 25\\
  0 & 25 & -2 & 13 & 89\\
  0 & 0 & -897 & -257 & -2046 \\
  0 & 0 & 46 & 11 & 118
\end{bmatrix}\]
Finally for the third column, we have 
\[  \begin{bmatrix}[*3rr@{\quad}|@{\quad}>{\color{black}}r]
  3 & 2& 11  &  5 & 25\\
  0 & 25 & -2 & 13 & 89\\
  0 & 0 & -897 & -257 & -2046 \\
  0 & 0 & 46 & 11 & 118
\end{bmatrix}\quad  \underrightarrow{ \left( 897R_{4} + 46 R_{3} \right) \to R_{4}} \quad \begin{bmatrix}[*3rr@{\quad}|@{\quad}>{\color{black}}r]
  3 & 2& 11  &  5 & 25\\
  0 & 25 & -2 & 13 & 89\\
  0 & 0 & -897 & -257 & -2046 \\
  0 & 0 & 0 & -1955 & 11730
\end{bmatrix} \]
We now have the upper triangular system 
\begin{align*}
  3w + 2x + 11y + 5z &= 25\\
  25x - 2y + 13z &= 89\\
  -897y - 257z &= -2046\\
  -1955z &= 11730
\end{align*}

Now we use back substitution. From the fourth equation, we see that $z = \frac{11730}{-1955} = -6$, so $\boxed{z=-6}$. 

From the third equation:
\begin{align*}
  -897y - 257(-6) &= -2046\\
  -897y + 1542 &= -2046\\
  -897y &= -3588\\
  y &= 4
\end{align*}
so $\boxed{y=4}$. 

From the second equation:
\begin{align*}
  25x - 2(4) + 13(-6) &= 89\\
  25x - 8 - 78 &= 89\\
  25x - 86 &= 89\\
  25x &= 175\\
  x &= 7
\end{align*}
so $\boxed{x=7}$. 

Finally, from the first equation:
\begin{align*}
  3w + 2(7) + 11(4) + 5(-6) &= 25\\
  3w + 14 + 44 - 30 &= 25\\
  3w + 28 &= 25\\
  3w &= -3\\
  w &= -1
\end{align*}
so $\boxed{w=-1}$.

Therefore, our solution is $(w,x,y,z) = (-1,7,4,-6)$.
\end{example}

\subsection{Elimination Matrices}

Much like how in \( \mathbb{R}^{3} \) every vector can be written as a linear combination of the standard basis vectors
\( \vb{e}_{1}, \vb{e}_{2}, \vb{e}_{3} \),
we can describe the steps of Gaussian elimination using linear combinations of elimination matrices.

\begin{dfn}
By an \vocab{elimination matrix}, denoted \( \vb{E}_{ji} \) with \( j > i \), we mean a matrix that is equal to the identity except for a scalar \( m \) in the \( (j,i) \) entry.
\end{dfn}


\section{LU Factorization}
This section relies on the key observation that the steps behind elimination are linear. As such, we can record the elimination steps in a matrix. 

\begin{example}
    Suppose that we want to transform the matrix 
    \( A = \begin{bmatrix}
        2 & 1\\
        6 & 8\\
    \end{bmatrix} \) into an upper triangular matrix \( U \). To do this, we need to subtract \( 3 \) copies of row 1 from row 2. We record this operation in a matrix \( E_{21} = \begin{bmatrix}
        1 & 0\\
        -3 & 1\\
    \end{bmatrix} \) The ones on the diagonal indicate that we are not scaling any rows and the \( -3 \) in the \( (i,j) = (2,1) \) entry records the operation of subtracting \( 3 \) copies of row 1 from row 2. Sure enough, 
    \[ \begin{bmatrix}
        1 & 0\\
        -3 & 1\\
    \end{bmatrix} \begin{bmatrix}
        2 & 1\\
        6 & 8\\
    \end{bmatrix} = \begin{bmatrix}
        (1 \cdot 2) + (0 \cdot 6) & (1 \cdot 1) + (0 \cdot 8)\\
        (-3 \cdot 2) + (1 \cdot 6) & (-3 \cdot 1) + (1 \cdot 8)\\
    \end{bmatrix} =\begin{bmatrix}
        2 & 1\\
        0 & 5\\
    \end{bmatrix} \]
  Moreover, since we know what \( E_{21} \) represents, we can calculate \( E^{-1}_{21} \) in our heads. Namely, to undo \( R_{2} - 3R_{1} \to R_{2} \), we simply apply the operation \( R_{2} + 3 R_{1} \to R_{2} \). So 
  \[ E^{-1}_{21} = \begin{bmatrix}
        1 & 0\\
        3 & 1\\
    \end{bmatrix}. \]
    It easy to check that \( E^{-1}_{21} E_{21} = I \). Moreover, 
    \begin{align*}
        E_{21} A &= U \\
        E^{-1}_{21} E_{21} A &= E^{-1}_{21} U \\
        A &= E^{-1}_{21} U
    \end{align*}
    This is the goal. Letting \( E^{-1}_{21} = L \), we have factored \( A \) into a lower triangular matrix times an upper triangular matrix. This is an example of\vocab{LU factorization}. 
\end{example}





\chapter{Subspaces}
\section{The Nullspace}
\begin{dfn}
    Let \( A \) be an \( m \times n \) matrix. The \vocab{nullspace} of \( A \), denoted by \( \mathrm{null}\left( A \right) \) is the set of vectors in \(\vb{v} \in \mathbb{R}^{n} \) such that \( A\vb{v} =\vb{0} \in \mathbb{R}^{m} .\)
\end{dfn}

\begin{example}
    What is the nullspace of the matrix 
    \[ A = \begin{bmatrix*}[r]
        3  & 1 & -7\\
        1 & 0 & -2\\
        2 & 1 & -5\\
    \end{bmatrix*} .\]
    Gaussian elimination comes to the rescue!
    \begin{align*}
         \begin{bmatrix}
        3  & 1 & -7\\
        1 & 0 & -2\\
        2 & 1 & -5\\
    \end{bmatrix} \quad  &\underrightarrow{(3R_{2} - R_{1}) \to R_{2}} \quad  \begin{bmatrix*}[r]
        3  & 1 & -7\\
        0 & -1 & 1\\
        2 & 1 & -5\\
    \end{bmatrix*} \\
    \begin{bmatrix*}[r]
        3  & 1 & -7\\
        0 & -1 & 1\\
        2 & 1 & -5\\
    \end{bmatrix*}  \quad  &\underrightarrow{(3R_{3} - 2R_{1}) \to R_{3}} \quad \begin{bmatrix*}[r]
        3  & 1 & -7\\
        0 & -1 & 1\\
        0 & 1 & -1\\
    \end{bmatrix*}\\
    \begin{bmatrix*}[r]
        3  & 1 & -7\\ 
        0 & -1 & 1\\
        0 & 1 & -1\\
    \end{bmatrix*} \quad  &\underrightarrow{(R_{3} + R_{2}) \to R_{3}}  \quad  \begin{bmatrix*}[r]
        3  & 1 & -7\\
        0 & -1 & 1\\
        0 & 0 & 0\\
    \end{bmatrix*}
    \end{align*}
    Reading off the equations from this reduced matrix, we have the system 
    \begin{align*}
        3x_{1} + x_{2} - 7x_{3} &=0\\
        -x_{2} + x_{3} &= 0
    \end{align*}
    From the second equation, \( x_{2} = x_{3} \). Since \( x_{3} \) is a free variable, we can set it to any value. Let's choose \( x_{3} = 1 \) for convenience, so \( x_{2} = 1 \) as well. Then 
    \begin{align*}
        3x_{1} +1 -7 &=0 \\
        3x_{1} &= 6\\
        x_{1} &= 2
    \end{align*}
    So 
    \[ \boxed{\mathrm{null} \left( A \right) = \mathrm{span} \left\{\;  \begin{bmatrix}
        2\\
        1\\
        1\\
    \end{bmatrix}\;  \right\} \text{ .}} \]
\end{example}

\begin{exercise}
    Let \( A: \mathbb{R}^{4} \to \mathbb{R}^{3} \) be represented by the matrix: 
    \[ A= \begin{bmatrix}
        1 & 2 & 3 & 5\\
        2 & 4 & 8 & 12\\
        3 & 6 & 7 & 13\\
    \end{bmatrix} \]

\begin{enumerate}[label=\textbf{\roman*)}]
\item Convert the augmented matrix \( \left[ A \mid \va{b} \right] \) to an upper triangular system \( \left[ U \mid \va{c} \right] \). 
\item Convert \( \left[ U \mid \va{c} \right] \) to reduced row echelon form \( \left[ R \mid \va{d} \right] \).
\item What is \( \mathrm{col} \left( A \right) \)?
\item What is \( \mathrm{null} \left( A \right) \)?
\item Given a vector \( \va{b} \in \mathrm{col}(A) \), what is the general form of all solutions to \( A\va{x} = \va{b} \)?
\item Find a particular and then a general solution to the \( A \va{x}= \left< 0,6,-6 \right> \).
\end{enumerate}
\end{exercise}
\begin{solution} $ $
    \begin{enumerate}[label=\textbf{\roman*)}]
        \item We have 
        \begin{align*}
            \begin{bmatrix}[*3cr@{\quad}|@{\quad}>{\color{black}}r]
            1 & 2 & 3  &  5 &b_{1}\\
            2 & 4 & 8 & 12 & b_{2}\\
            3 & 6 & 7 & 13 &b_{3}
            \end{bmatrix} \quad  &\underrightarrow{(R_{2} - 2R_{1}) \to R_{2}} \quad  \begin{bmatrix}[*3cr@{\quad}|@{\quad}>{\color{black}}c]
            1 & 2 & 3  &  5 &b_{1}\\
            0 & 0 & 2 & 2 & b_{2}- 2b_{1}\\
            3 & 6 & 7 & 13 &b_{3}
            \end{bmatrix} \\
            \begin{bmatrix}[*3cr@{\quad}|@{\quad}>{\color{black}}c]
            1 & 2 & 3  &  5 &b_{1}\\
            0 & 0 & 2 & 2 & b_{2}- 2b_{1}\\
            3 & 6 & 7 & 13 &b_{3}
            \end{bmatrix}  \quad  &\underrightarrow{(R_{3} - 2R_{3}) \to R_{3}} \quad  \begin{bmatrix}[*3cr@{\quad}|@{\quad}>{\color{black}}c]
            1 & 2 & 3  &  5 &b_{1}\\
            0 & 0 & 2 & 2 & b_{2}- 2b_{1}\\
            0 & 0 & -2 & -2 &b_{3} - 3b_{1}
            \end{bmatrix} \\
             \begin{bmatrix}[*3cr@{\quad}|@{\quad}>{\color{black}}c]
            1 & 2 & 3  &  5 &b_{1}\\
            0 & 0 & 2 & 2 & b_{2}- 2b_{1}\\
            0 & 0 & -2 & -2 &b_{3} - 3b_{1}
            \end{bmatrix} \quad  &\underrightarrow{(R_{2} + R_{3}) \to R_{3}} \quad   \begin{bmatrix}[*3cr@{\quad}|@{\quad}>{\color{black}}c]
            1 & 2 & 3  &  5 &b_{1}\\
            0 & 0 & 2 & 2 & b_{2}- 2b_{1}\\
            0 & 0 & 0 & 0 &b_{2} +b_{3} - 5b_{1}
            \end{bmatrix} 
        \end{align*}
        So our upper triangular system becomes: 
        \begin{equation}\label{2025-10-11 11:02:53}
            \left[ U \mid \va{c} \right] = \begin{bmatrix}[*3cr@{\quad}|@{\quad}>{\color{black}}c]
            1 & 2 & 3  &  5 &b_{1}\\
            0 & 0 & 2 & 2 & b_{2}- 2b_{1}\\
            0 & 0 & 0 & 0 &b_{2} +b_{3} - 5b_{1}
            \end{bmatrix} 
        \end{equation}
        \item Working from our upper-triangular system, we have 
        \begin{align*}
            \begin{bmatrix}[*3cr@{\quad}|@{\quad}>{\color{black}}c]
            1 & 2 & 3  &  5 &b_{1}\\
            0 & 0 & 2 & 2 & b_{2}- 2b_{1}\\
            0 & 0 & 0 & 0 &b_{2} +b_{3} - 5b_{1}
            \end{bmatrix}  \quad  &\underrightarrow{\left( \frac{1}{2}R_{2} \right) \to R_{2}} \quad  \begin{bmatrix}[*3cr@{\quad}|@{\quad}>{\color{black}}c]
            1 & 2 & 3  &  5 &b_{1}\\
            0 & 0 & 1 & 1 &  \frac{1}{2}b_{2}- b_{1}\\
            0 & 0 & 0 & 0 &b_{2} +b_{3} - 5b_{1}
            \end{bmatrix}  \\
             \begin{bmatrix}[*3cr@{\quad}|@{\quad}>{\color{black}}c]
            1 & 2 & 3  &  5 &b_{1}\\
            0 & 0 & 1 & 1 &  \frac{1}{2}b_{2}- b_{1}\\
            0 & 0 & 0 & 0 &b_{2} +b_{3} - 5b_{1}
            \end{bmatrix} \quad  &\underrightarrow{\left( R_{1} - 3 R_{2} \right) \to R_{1}} \quad   \begin{bmatrix}[*3cr@{\quad}|@{\quad}>{\color{black}}c]
            1 & 2 & 0  &  2 &-3b_{1} - \frac{3}{2} b_{2}\\
            0 & 0 & 1 & 1 &  \frac{1}{2}b_{2}- b_{1}\\
            0 & 0 & 0 & 0 &b_{2} +b_{3} - 5b_{1}
            \end{bmatrix}  
        \end{align*}
        So the reduced row echelon form is: 
        \begin{equation}\label{2025-10-11 11:15:17}
            \left[ R \mid \va{d} \right] = \begin{bmatrix}[*3cr@{\quad}|@{\quad}>{\color{black}}c]
            1 & 2 & 0  &  2 &-3b_{1} - \frac{3}{2} b_{2}\\
            0 & 0 & 1 & 1 &  \frac{1}{2}b_{2}- b_{1}\\
            0 & 0 & 0 & 0 &b_{2} +b_{3} - 5b_{1}
            \end{bmatrix}  
        \end{equation}
        \item From \cref{2025-10-11 11:02:53} or \cref{2025-10-11 11:15:17}, we see that our system has a consistent solution only when 
        \begin{equation}\label{2025-10-11 11:34:08}
             -5b_{1}+b_{2}+b_{3}=0 
        \end{equation}
        which is the equation of a plane in \( \mathbb{R}^{3} \). \\ 
        We also see that our pivot columns from our original matrix are 
        \[ \left\{ \; \begin{bmatrix}
            1 \\
            2\\
            3\\
        \end{bmatrix}, \; \begin{bmatrix}
            3\\
            8\\
            7 \\
        \end{bmatrix} \;  \right\} \]
        So 
        \[ \mathrm{span} \left\{ \; \begin{bmatrix}
            1 \\
            2\\
            3\\
        \end{bmatrix}, \; \begin{bmatrix}
            3\\
            8\\
            7 \\
        \end{bmatrix} \;  \right\}  \]
        should also be a representation of the column space. To verify this, take any vector \( \va{v} \in \mathrm{span} \left\{ \left< 1,2,3 \right>, \left< 3,8,7 \right> \right\} \). Then for some scalars \( s,t \in \mathbb{R} \), we have \( \va{v} = s \left< 1,2,3 \right> +t \left< 3,8,7 \right>\) or \( \va{v} = \left< s +3t, 2s+8t, 3s+7t \right> \). Substituting the components of \( \va{v} \) into \cref{2025-10-11 11:34:08}, we have 
        \begin{align*}
            -5 v_{1}+ v_{2}+v_{3} &= -5 \left( s +3t \right) + \left( 2s+8t \right) + \left( 3s+7t \right) \\
            &= \left( -5s +2s +3s \right)  + \left( -15t +8t +7t \right) \\
            &= 0
        \end{align*}
        Since the dimensions of both descriptions of the column space are equal and we have shown that one is contained in the other, it follows that both descriptions are equivalent. 
        \item From  \cref{2025-10-11 11:02:53}, we want to solve 
        \[  \begin{bmatrix}[*3cr@{\quad}|@{\quad}>{\color{black}}c]
            1 & 2 & 3  &  5 &0\\
            0 & 0 & 2 & 2 & 0\\
            0 & 0 & 0 & 0 &0
            \end{bmatrix}  \]
            So 
            \begin{align*}
                x_{1} + 2x_{2} + 3x_{3} + 5x_{4} &=0\\
                x_{3} + x_{4} &=0
            \end{align*}
            If we set \( \boxed{x_{3}=1} \), we have \( \boxed{x_{4} =-1} \) through the second equation. Then our first equation becomes 
            \[ x_{1} + 2x_{2} =2 \]
            Setting \( \boxed{x_{2} =1} \) implies \( \boxed{x_{1}= 0} \) so the first vector in our nullspace is \( \boxed{ \left< 0,1,1,-1 \right>} \). Similarly setting \( \boxed{x_{1} -2} \) implies that \( \boxed{x_{2}=0} \) so the next vector in our nullspace is \( \boxed{\left< 2,0,1,-1 \right>} \). So 
            \[ \mathrm{null} \left( A \right) = \mathrm{span} \left\{ \; \begin{bmatrix}
                0\\
                1\\
                1\\
                -1\\
            \end{bmatrix}, \; \begin{bmatrix}
                2\\
                0\\
                1\\
                -1\\
            \end{bmatrix}  \; \right\} \]
            To show that this is indeed the nullspace of \( A \) pick any \( \va{v} \in \mathrm{null} \left( A \right) \). So \( \va{v} = x \left< 0,1,1, -1, \right> + y \left< 2, 0,1 -1 \right> \) for some \( x, y \in \mathbb{R} \). So \( \va{v} = \left< 2y, x, x+y, -x-y \right> \). Then 
            \begin{align*}
                A \va{v} &= \begin{bmatrix}
        1 & 2 & 3 & 5\\
        2 & 4 & 8 & 12\\
        3 & 6 & 7 & 13\\
    \end{bmatrix} \begin{bmatrix}
        2y \\
        x \\
        x + y \\
        -x-y\\
    \end{bmatrix} \\
    &= \begin{bmatrix}
        2y + 2x + 3 (x+y) + 5 (-x-y)\\
        4y +4x 8 \left( x+y \right) + 12 \left( -x-y \right)\\
        6y + 6x + 7 \left( x+y \right) + 13 \left( -x-y \right)\\
    \end{bmatrix}\\
    &= \begin{bmatrix}
        0\\
        0\\
        0\\
    \end{bmatrix}
            \end{align*}
    \item From \textbf{(iii)}, we know that output space is the span of the pivot columns and from \textbf{(iv)}, we have a description of the null space. So if \( \va{b} = s \left< 1,2,3 \right> + t \left< 3,8,7 \right> \). Then the \( \va{x} \) that solves \( A \va{x} = \va{b} \) is precisely of the form 
    \[ \va{x} = \begin{bmatrix}
        s + 2v \\
        u\\
        t + u +v\\
        -u-v\\
    \end{bmatrix} \quad \text{ for all } u, v \in \mathbb{R}. \]
    \item Clearly the vector \( \left< 0,6,-6 \right> \) is a member of the output space since it solves \( b_{2} + b_{3} - 5b_{1} =0 \). 
    \end{enumerate}
    
\end{solution}






\chapter{The Einstein Summation Convention}

Although this topic is normally outside the scope of a first-year course in linear algebra, this is as good a time as any to begin planting the seeds for understanding tensors. This chapter may be skipped on a first reading.

\section{Repeated Indices in Sums}
We begin with the familiar definition of the dot product:
\[
    \sum_{j=1}^{n} a_{j}x_{j} := a_{1}x_{1} + a_{2}x_{2} + \cdots + a_{n}x_{n}.
\]

However, if the limits of the sum and the range of indices are clear from context, then writing out the summation symbol provides little additional information. Thus, in many areas of mathematics and physics, we adopt the following convention:

\[
    a_{j}x_{j} := \sum_{j=1}^{n} a_{j}x_{j}.
\]

This is known as the \vocab{Einstein summation convention}: whenever an index variable appears exactly twice in a single term, it is understood to be summed over.

% Optional continuation:
This not only saves notation, but also prepares us to work with more complex indexed expressions, especially when dealing with tensors in differential geometry, general relativity, or continuum mechanics.
\begin{dfn} 
    In an indexed expression, an index is called a \vocab{dummy index} (or \vocab{summation index}) if it appears exactly twice in a single term and is implicitly summed over by the Einstein summation convention. 

    An index is called a \vocab{free index} if it is not summed over and appears exactly once in each term of an equation. Free indices determine the components of the resulting expression and must match on both sides of an equation.
\end{dfn}

Some immediate consequences of the above definition are:
\begin{enumerate}[label=\textbf{\roman*)}]
    \item \( a_{ij}x_{j} \neq a_{kj}x_{j} \) because \( i \neq k \)
    \item No index can appear three or more times in an expression.
\end{enumerate}

\begin{example}
    Suppose \( n = 4 \). Consider the expressions \( a_{ii}x_{k} \) and \( a_{ij}x_{j} \). By the Einstein summation convention, we interpret:
    \[
        a_{ii}x_{k} = a_{11}x_{k} + a_{22}x_{k} + a_{33}x_{k} + a_{44}x_{k},
    \]
    since the repeated index \( i \) is implicitly summed over, while \( k \) remains a free index.

    Similarly,
    \[
        a_{ij}x_{j} = a_{i1}x_{1} + a_{i2}x_{2} + a_{i3}x_{3} + a_{i4}x_{4},
    \]
    where the index \( j \) is summed over and \( i \) is free.
\end{example}

\begin{example}
    If \( n=3 \), write down the equations represented by \( y_{i}=a_{ij}x_j \).\\
    Since \( j \) is the summation index, we have 
    \[ y_{i}=a_{i1}x_{1}+a_{i2}x_{2}+a_{i3}x_{3}\]
    which in turn become the equations
    \begin{align*}
        y_{1}&=a_{11}x_{1}+a_{12}x_{2}+a_{13}x_{3}\\
        y_{2}&=a_{21}x_{1}+a_{22}x_{2}+a_{23}x_{3}\\
        y_{3}&=a_{31}x_{1}+a_{32}x_{2}+a_{33}x_{3}
    \end{align*}
    
\end{example}

\subsection{Double Sums}

Suppose we wish to substitute \( y_i = a_{ij}x_j \) into the expression \( Q = b_{ij}y_i x_j \). A careless substitution yields:
\[
Q = b_{ij}a_{ij}x_j x_j,
\]
which is problematic: the index \( j \) appears \textbf{four} times on the right-hand side, violating the rules of summation convention. Recall that a dummy index should appear exactly twice—once per term being summed over. 

To remedy this, we take advantage of the fact that dummy indices are arbitrary labels. We can safely rename one pair of repeated \( j \) indices to a new label, say \( k \). Here's the general procedure:

\begin{enumerate}[label=\textbf{\roman*)}]
    \item Identify the overused dummy index. In this case, \( j \) appears too many times:
    \[
    y_i = a_{ij}x_j, \qquad Q = b_{ij}y_i x_j.
    \]
    
    \item Rename one pair of the repeated \( j \)'s using a new index (e.g., \( k \)):
    \[
    y_i = a_{ik}x_k, \qquad Q = b_{ij}y_i x_j.
    \]
    
    \item Now the substitution is well-formed:
    \[
    Q = b_{ij}a_{ik}x_k x_j.
    \]
\end{enumerate}

\begin{example}
    If \( n = 3 \), write out explicitly the equation given by 
    \[
    Q = b_{ij}a_{ik}x_k x_j.
    \]
    
    First, sum over the \( k \) index:
    \[
    Q = b_{ij}(a_{i1}x_1 + a_{i2}x_2 + a_{i3}x_3)x_j
    = b_{ij}a_{i1}x_1 x_j + b_{ij}a_{i2}x_2 x_j + b_{ij}a_{i3}x_3 x_j.
    \]
    Now sum over the \( j \) index:
    \begin{align*}
    Q &= b_{i1}a_{i1}x_1 x_1 + b_{i2}a_{i1}x_1 x_2 + b_{i3}a_{i1}x_1 x_3 \\
      &\quad + b_{i1}a_{i2}x_2 x_1 + b_{i2}a_{i2}x_2 x_2 + b_{i3}a_{i2}x_2 x_3 \\
      &\quad + b_{i1}a_{i3}x_3 x_1 + b_{i2}a_{i3}x_3 x_2 + b_{i3}a_{i3}x_3 x_3.
    \end{align*}
    Finally, expand over \( i \):
    \begin{align*}
    Q &= b_{11}a_{11}x_1 x_1 + b_{12}a_{11}x_1 x_2 + b_{13}a_{11}x_1 x_3 
      + b_{11}a_{12}x_2 x_1 + b_{12}a_{12}x_2 x_2 + b_{13}a_{12}x_2 x_3 
      + b_{11}a_{13}x_3 x_1 + b_{12}a_{13}x_3 x_2 + b_{13}a_{13}x_3 x_3 \\
      &\quad + b_{21}a_{21}x_1 x_1 + b_{22}a_{21}x_1 x_2 + b_{23}a_{21}x_1 x_3 
      + b_{21}a_{22}x_2 x_1 + b_{22}a_{22}x_2 x_2 + b_{23}a_{22}x_2 x_3 
      + b_{21}a_{23}x_3 x_1 + b_{22}a_{23}x_3 x_2 + b_{23}a_{23}x_3 x_3 \\
      &\quad + b_{31}a_{31}x_1 x_1 + b_{32}a_{31}x_1 x_2 + b_{33}a_{31}x_1 x_3 
      + b_{31}a_{32}x_2 x_1 + b_{32}a_{32}x_2 x_2 + b_{33}a_{32}x_2 x_3 
      + b_{31}a_{33}x_3 x_1 + b_{32}a_{33}x_3 x_2 + b_{33}a_{33}x_3 x_3.
    \end{align*}
    This expansion shows how rapidly the number of terms grows. The compact form
    \[
    Q = b_{ij}a_{ik}x_k x_j
    \]
    is far more efficient.
\end{example}

\begin{example}
Suppose \( y_i = a_{ik}x_k \), and we wish to compute
\[
Q = g_{ij}y_i y_j.
\]

If we substitute directly, we get:
\[
Q = g_{ij}(a_{ik}x_k)(a_{jk}x_k),
\]
which is invalid since the index \( k \) appears four times.

We fix this by renaming one of the repeated \( k \)'s:
\[
Q = g_{ij}(a_{ik}x_k)(a_{jm}x_m),
\]
which now respects the summation convention.
\end{example}

\subsection{Kronecker Delta and Algebraic Rules}

\begin{dfn}
The \vocab{Kronecker delta} is the tensor defined by
\[
\delta_{ij} = \delta^i_j = \delta^{ij} = 
\begin{cases}
    1 & \text{if } i = j, \\
    0 & \text{if } i \neq j.
\end{cases}
\]
\end{dfn}


\part{Multivariable Calculus}
\label{part: Multivariable Calculus}
\parttoc
\chapter{Functions of Several Variables}
In single-variable calculus, we studied functions that take a single real input and produce a single real output. Our geometric intuition was built around this setting: the derivative represented the slope of a curve in the plane, and the integral represented the area under that curve.
In multivariable calculus, we now consider functions of several real inputs and possibly several real outputs. In this broader setting, our earlier visual intuition of slope and area begins to break down. As such, we need to reinterpret the notions of derivative and integral to make sense in higher dimensions.

The good news is that while single-variable functions only map the real line to itself, offering limited geometric complexity, multivariable functions map between higher-dimensional spaces such as the plane or 3-dimensional space. These richer domains and codomains allow us to explore more interesting geometric behavior. We can now ask: what kinds of geometric information do these functions encode? What does “change” or “accumulation” mean in higher-dimensional settings?


\section{\( \mathbb{R}^{3} \)}
\subsection{Lines in \( \mathbb{R}^{3} \)}

If we are given a point \( P = \left( p_{1}, p_{2}, p_{3} \right) \) and a direction \( \va{v} = \left< v_{1}, v_{2}, v_{3} \right> \), then we can form a line that contains \( P \) traveling in the direction of \( \vb{v} \) by 
\[ L(t) = P + \va{v}t = \left( p_{1} + v_{1}t, p_{2} + v_{2}t, p_{3} + v_{3}t \right) \]
or 
\[ L = \begin{cases}
    &x = p_{1} + v_{1}t \\
    &y = p_{2} + v_{2}t \\
    &z = p_{3} + v_{3}t
\end{cases} \]
If we solve for \( t \), we get the following equality 
\[ \frac{x-p_{1}}{v_{1}} = \frac{y-p_{2}}{v_{2}} = \frac{z-p_{3}}{v_{3}} \]
If any component of \( \va{v} \) is \( 0 \), then we just omit its mention in the above equation since \( L \) would not change in that respective direction.

\begin{exercise}
    Suppose that \( L \) is the line that contains the points \( P = (-3, -1, 2) \) and \( Q = (5,8,4) \). Where does \( L \) pierce the \( xy \)-plane?
\end{exercise}
\begin{solution}
    First, we need to find the direction vector \( \va{PQ} \). We have 
    \begin{align*}
        \va{PQ} &= \left< 5+3,8+1,4-2 \right> \\
        &= \left< 8,9,2 \right>
    \end{align*}
    Using the symmetric equations for a line, we have to find the \( (x,y,0) \) on \( L \). Solving for \( z \), we have 
    \[ \frac{0-2}{2}=-1 \]
    so 
    \[ \frac{x+3}{8} =-1 \Rightarrow x= -11 \]
    and 
    \[ \frac{y+1}{9}=-1 \Rightarrow y = -10 \]
    To check that we have the correct answer, let use \( Q \) in the symmetric equations 
    \[ \frac{0-4}{2} = -2 \]
    so 
    \[ \frac{x-5}{8} =-2 \Rightarrow x = -11 \]
    and 
    \[ \frac{y-8}{9}= -2 \Rightarrow y = -10 \]
    so \( L \) pierces the \( xy \)-plane at 
    \[ \boxed{\left( -11,-10, 0 \right)} \]
\end{solution}



\chapter{Differentiation}
\section{Partial Derivatives}
\begin{dfn}
    Suppose that \( f: \mathbb{R}^{n} \to \mathbb{R} \). Then \( \pdv{f}{x_1}, \pdv{f}{x_2}, \dots, \pdv{f}{x_n} \) denote the \vocab{partial derivatives} with respect to the first, second, ..., $n$-th variables, if they exist. The $k$-th partial derivative exists if the following limit exists and is finite: 
    \[ \lim_{h \to 0} \frac{f \left( x_1, x_2, \dots, x_k+h, \dots, x_n \right) - f \left( x_1, x_2, \dots, x_n \right)}{h}. \] 
    Or equivalently, 
    \[ \lim_{h \to 0} \frac{f \left( \vb{x} + h \cdot \vb{e}_{j} \right) - f \left( \vb{x} \right)}{h} \]
\end{dfn}

\section{Maximum and Minimum Values}
\input{content/analysis/multivariable-calculus/2-differentiation/max-and-min-values}

\section{Lagrange Multipliers}
\begin{example}
 Find the extreme values of the function \( f(x,y) = x^{2}+2y^{2} \) over the circle \( x^{2 } + y^{2 }=1 \). \\

     We apply the method of Lagrange multipliers the functions \( f(x,y) = x^{2}+2y^{2}  \), \( g(x,y)= x^{2} +y^{2} \)to get 
    \[ 2x = \lambda \cdot 2x , \quad 4y = \lambda \cdot 2y .\]
    Now if \( \lambda =1 \) to resolve the equation \( 2x = \lambda \cdot 2x \). Then the second equation must have \( y=0 \). Applying this to the constraint equation gives us the points \( \boxed{(x,y)=(1,0)} \) and \(  \boxed{(x,y)=(-1,0)} \) as our first two candidates for critical points. Conversely, if \( \lambda =2 \) to resolve the second equation, then \( x=0 \). As such, our second pair of candidates are  \( \boxed{(x,y)=(0,1)} \) and \(  \boxed{(x,y)=(0,-1)} \). Substituting our candidates into \( f(x,y) \), we get 
    \begin{align*}
        f(-1,0) &= 1 \\
        f(1,0) &= 1 \\
        f(0,-1) &= 2 \\
        f(0,1) &= 2 
    \end{align*}
    So over the unit-circle, \( f \) attains its maximum at \( \boxed{(x,y)=(0,1)} \) and \(  \boxed{(x,y)=(0,-1)} \) and its minimum at \( \boxed{(x,y)=(1,0)} \) and \(  \boxed{(x,y)=(-1,0)} \). \\
    Now parametrize \( f(x,y) \) by \( \theta \) with \( x = \cos{ \left( \theta \right) } \) and \( y = \sin{ \left( \theta \right) } \) and compare this problem with \Cref{example:cos2_plus_2sin2}

\end{example}

\begin{exercise}
    Suppose that you are give \( 12 \) square feet of cardboard and asked to make a box without a lid. What is the maximum volume of such a box? What are the dimensions?
\end{exercise}
\begin{solution}
    Let \( x \) and \( y \) denote the dimensions of the base and \( z \) denote the height of the box. Then we wish to solve the optimization problem 
    \begin{align*}
        \text{maximize: }\quad  &f(x,y,z) = xyz \\
         \text{subject to: } \quad &g(x,y,z) = xy + 2xz + 2yz - 12 = 0\\
         &x,y,z >0
    \end{align*}
    
    Applying the method of Lagrange multipliers, we need $\largetriangledown f = \lambda \largetriangledown g$, which gives us the system 
    \begin{equation}\label{exc:cardboard-lagrange-1}
        yz = \lambda \left( y+ 2z \right)
    \end{equation}
    \begin{equation}\label{exc:cardboard-lagrange-2}
        xz = \lambda \left( x+ 2z \right)
    \end{equation}
    \begin{equation}\label{exc:cardboard-lagrange-3}
        xy = \lambda \left( 2x +2y \right)
    \end{equation}
    
    Since we're looking for a maximum in the interior of the domain where $x,y,z > 0$, and since $\largetriangledown g \neq 0$ at any feasible point, we know that $\lambda \neq 0$.
    
    Multiplying \Cref{exc:cardboard-lagrange-1} by \( x \) and \Cref{exc:cardboard-lagrange-2} by \( y \), we get
    \[ xyz = \lambda \left( xy + 2xz  \right) \quad \text{and} \quad xyz = \lambda \left( xy + 2yz \right) \]
    
    Since both expressions equal $xyz$, we have:
    \begin{align*}
        \lambda \left( xy + 2xz  \right) &= \lambda \left( xy + 2yz \right) \\
        xy + 2xz &= xy + 2yz \tag{since $\lambda \neq 0$} \\
        2xz &= 2yz \\
        x &= y \tag{since $z > 0$}
    \end{align*}
    
    Therefore, $\boxed{x=y}$. 
    
    Substituting this result into \Cref{exc:cardboard-lagrange-3}, we have 
    \[ x^{2} = \lambda \left( 2x + 2x \right) = 4\lambda x \]
    Since $x > 0$, we can divide by $x$ to get $\boxed{ x = 4 \lambda}$.
    
    Substituting $x = y = 4\lambda$ into \Cref{exc:cardboard-lagrange-2}, we have
    \[ (4\lambda) z = \lambda \left( 4 \lambda + 2z \right) \]
    \[ 4\lambda z = \lambda(4\lambda + 2z) \]
    \begin{align*}
        4z &= 4\lambda + 2z \tag{since $\lambda \neq 0$} \\
        2z &= 4\lambda \\
    \end{align*}
    \[ \boxed{z = 2 \lambda} \]
    
    Now that we have expressions for \( x,y,z \) in terms of \( \lambda \), we can solve for \( \lambda \) using the constraint equation:
    \begin{align*}
        12 &= xy + 2xz + 2 yz \\
        &= \left( 4 \lambda \right) \left(  4 \lambda \right) + 2 \left( 4 \lambda \right) \left( 2 \lambda \right) + 2 \left( 4 \lambda \right) \left( 2 \lambda \right) \\
        &= 16\lambda^2 + 16\lambda^2 + 16\lambda^2 \\
        &= 48\lambda^2
    \end{align*}
    
    Therefore, $48\lambda^2 = 12$, which gives us $\lambda^2 = \frac{1}{4}$. Since we need $\lambda > 0$ (as the constraint gradient and objective gradient point in the same direction at the maximum), we have $\lambda = \frac{1}{2}$.
    
    Substituting this into our expressions for \( x,y,z \), we get:
    \[ \boxed{x = 4 \cdot \frac{1}{2} = 2} \]
    \[ \boxed{y = 4 \cdot \frac{1}{2} = 2} \]  
    \[ \boxed{z = 2 \cdot \frac{1}{2} = 1} \]
    
    This gives us a maximum volume of $\boxed{ \text{Volume } = xyz = 2 \cdot 2 \cdot 1 = 4 \text{ cubic feet}}$.
    
\end{solution}

\begin{exercise}
    What are the points on the sphere \( x^{2}+ y^{2}+ z^{2}=4 \) that are closest and furthest from the point \( (x,y,z) = (3,1, -1) \)?
\end{exercise}
\begin{solution}
    The distance from a point \( (x,y,z) \) to the point \( (3,1,-1) \) is given by 
    \[ d(x,y,z) = \sqrt{\left( x-3  \right)^{2} + \left( y-1  \right)^{2} + \left( z +1  \right)^{2}}. \]
    However, we can make our lives easier by choosing to optimize \( f \left( x,y,z  \right) = \left( d (x,y,z ) \right)^{2} \) since extremizing the distance is equivalent to extremizing the squared distance. So our problem is: 
    \begin{align*}
        \text{extremize: } \quad &f(x,y,z) =\left( x-3  \right)^{2} +\left( y-1  \right)^{2} + \left( z +1  \right)^{2} \\
        \text{subject to: } \quad &g (x,y,z) = x^{2} + y^{2} + z^{2} -4 =0
    \end{align*}
    
    Through Lagrange multipliers, we have \( \largetriangledown f = \lambda \largetriangledown g \) which becomes the system:
    \begin{align*}
        2(x - 3) &= 2\lambda x \\
        2(y - 1) &= 2\lambda y \\
        2(z + 1) &= 2\lambda z
    \end{align*}
    
    Dividing by 2 and rearranging each equation:
    \begin{align*}
        x - 3 &= \lambda x \implies x(1 - \lambda) = 3 \\
        y - 1 &= \lambda y \implies y(1 - \lambda) = 1 \\
        z + 1 &= \lambda z \implies z(1 - \lambda) = -1
    \end{align*}
    
 If \( \lambda =1 \), then the above three equations are absurd, so we can assume \( \lambda \neq 1 \). Therefore:
    \[ \boxed{x = \frac{3}{1- \lambda}, \quad y = \frac{1}{1- \lambda}, \quad z = - \frac{1}{1- \lambda}} \]
    
    We can now substitute these expressions into the constraint:
    \begin{align*}
        4 &= x^{2} +y^{2} + z^{2} \\
        4 &= \left( \frac{3}{1 - \lambda } \right)^{2} + \left( \frac{1 }{1 - \lambda } \right)^{2} + \left( - \frac{1}{1- \lambda} \right)^{2} \\
        4 &= \frac{9 + 1 + 1}{(1 - \lambda)^2} \\
        4(1 - \lambda)^2 &= 11 \\
        (1 - \lambda)^2 &= \frac{11}{4} \\
        1 - \lambda &= \pm \frac{\sqrt{11}}{2}
    \end{align*}
    
    Therefore: $\lambda = 1 \pm \frac{\sqrt{11}}{2}$, giving us $\lambda_1 = 1 - \frac{\sqrt{11}}{2}$ and $\lambda_2 = 1 + \frac{\sqrt{11}}{2}$.
    
    For $\lambda_1 = 1 - \frac{\sqrt{11}}{2}$:
    \begin{align*}
        1 - \lambda_1 &= \frac{\sqrt{11}}{2} \\
        x &= \frac{3}{\sqrt{11}/2} = \frac{6}{\sqrt{11}} = \frac{6\sqrt{11}}{11} \\
        y &= \frac{1}{\sqrt{11}/2} = \frac{2}{\sqrt{11}} = \frac{2\sqrt{11}}{11} \\
        z &= \frac{-1}{\sqrt{11}/2} = \frac{-2}{\sqrt{11}} = -\frac{2\sqrt{11}}{11}
    \end{align*}
    
    This gives us the point $P_1 = \left( \frac{6\sqrt{11}}{11}, \frac{2\sqrt{11}}{11}, -\frac{2\sqrt{11}}{11} \right)$.
    
    For $\lambda_2 = 1 + \frac{\sqrt{11}}{2}$:
    \begin{align*}
        1 - \lambda_2 &= -\frac{\sqrt{11}}{2} \\
        x &= \frac{3}{-\sqrt{11}/2} = -\frac{6}{\sqrt{11}} = -\frac{6\sqrt{11}}{11} \\
        y &= \frac{1}{-\sqrt{11}/2} = -\frac{2}{\sqrt{11}} = -\frac{2\sqrt{11}}{11} \\
        z &= \frac{-1}{-\sqrt{11}/2} = \frac{2}{\sqrt{11}} = \frac{2\sqrt{11}}{11}
    \end{align*}
    
    This gives us the point $P_2 = \left( -\frac{6\sqrt{11}}{11}, -\frac{2\sqrt{11}}{11}, \frac{2\sqrt{11}}{11} \right)$.
    
    To determine which point is closest and which is furthest, we compute the squared distances:
    
    For $P_1$: 
    \begin{align*}
        f(P_1) &= \left(\frac{6\sqrt{11}}{11} - 3\right)^2 + \left(\frac{2\sqrt{11}}{11} - 1\right)^2 + \left(-\frac{2\sqrt{11}}{11} + 1\right)^2 \\
        &= \frac{1}{121}\left[(6\sqrt{11} - 33)^2 + (2\sqrt{11} - 11)^2 + (11 - 2\sqrt{11})^2\right] \\
        &= \frac{1}{121} \cdot 4(11 - 6\sqrt{11} + 121 - 22\sqrt{11} + 121) \\
        &= 4 - 2\sqrt{11}
    \end{align*}
    
    For $P_2$:
    \begin{align*}
        f(P_2) &= \left(-\frac{6\sqrt{11}}{11} - 3\right)^2 + \left(-\frac{2\sqrt{11}}{11} - 1\right)^2 + \left(\frac{2\sqrt{11}}{11} + 1\right)^2 \\
        &= 4 + 2\sqrt{11}
    \end{align*}
    
    Since $\sqrt{11} > 0$, we have $f(P_1) < f(P_2)$.
    
    Therefore:
    \begin{itemize}
        \item The \textbf{closest} point is $\boxed{\left( \frac{6\sqrt{11}}{11}, \frac{2\sqrt{11}}{11}, -\frac{2\sqrt{11}}{11} \right)}$
        \item The \textbf{furthest} point is $\boxed{\left( -\frac{6\sqrt{11}}{11}, -\frac{2\sqrt{11}}{11}, \frac{2\sqrt{11}}{11} \right)}$
    \end{itemize}
    
\emph{Note:} Geometrically, these points lie on the line passing through the center of the sphere $(0,0,0)$ and the external point $(3,1,-1)$, which explains why they represent the closest and furthest points on the sphere. (Compare this approach to the one shown in \Cref{ex:sphere-closest-furthest-vector}.)
\end{solution}

\begin{exercise}
   Let \( S =\left\{ \va{v} \in \mathbb{R}^{2}\  \middle|\  \va{v} \cdot \left< 1,2 \right>  =5\right\} \). What is the shortest vector in \( S \)?
\end{exercise}
\begin{solution}$ $
    \begin{align*}
        \text{minimize: } &f(x,y) = x^{2}+ y^{2} \\
        \text{subject to: } &g(x,y) =x + 2y -5 = 0
    \end{align*}
    This gives 
    \begin{align*}
        2x &= \lambda  \Rightarrow x = \frac{\lambda}{2}\\
        2y &= 2 \lambda \Rightarrow y = \lambda
    \end{align*}
    Substituting this into our constraint, we get 
    \[ \frac{\lambda}{2} + 2 \lambda = 5 \Rightarrow \lambda = 2 \]
    Therefore \( \boxed{ \left< v,y  \right> = \left< 1,2 \right>} \) is the vector in \( S \) of minimal length.
\end{solution}

\begin{exercise}
    Let \( \va{v} \in \mathbb{R}^{n} \) and \( c \in \mathbb{R} \) be fixed and non-zero. Define \( S = \left\{ \va{x} \in \mathbb{R}^{n} \ \middle| \ \va{x} \cdot \va{v} = c \right\} \) find the vector in \( S \) with the shortest length.
\end{exercise}
\begin{solution} Suppose that \( \va{x} = \left< x_{1}, x_{2}, \dots, x_{n} \right> \) and \( \va{v} = \left< v_{1}, v_{2}, \dots, v_{n} \right> \). Then
    \begin{align*}
        \text{minimize: } &f\left( \va{x} \right) = \sum_{k =1}^{n} \left( x_{k} \right)^{2 } \\
        \text{subject to: } &g(\va{x}) = \left( \sum_{k=1}^{n} v_{k} x_{k} \right) - c = 0
    \end{align*}
    So for any \( 1 \le k \le n \), we have 
    \begin{align*}
        \pdv{f }{x_{k}} &= \lambda \pdv{g }{x_{k}} \\
        2 x_{k} &= \lambda v_{k} \\
        x_{k} &= \frac{\lambda v_{k}}{2}
    \end{align*}
    Substituting this into our constraint, we have 
    \begin{align*}
        \sum_{k =1}^{n} v_{k} \left(  \frac{\lambda v_{k}}{2}\right) &=c \\
        \frac{\lambda}{2} \sum_{k =1}^{n} \left( v_{k} \right)^{2} & = c \\
        \lambda \frac{\norm{\va{v}}^{2}}{2} &= c \\
        \lambda &= \frac{2c}{\norm{\va{v}}^{2}}
    \end{align*}
    So 
    \[ x_{k} = \frac{1}{2}\frac{2c}{\norm{\va{v}}^{2}} v_{k} \Rightarrow \frac{c}{\norm{\va{v}}^{2}} v_{k} \]
    So the minimal element of \( S \) is \[ \boxed{\frac{c}{\norm{\va{v}}^{2}} \va{v}} \]
\end{solution}




\chapter{Multiple Integration}
\section{Change of Variables}
\subsection{Two Variable Case}


\begin{dfn}
    Let \( T: \mathbb{R}^{2} \to \mathbb{R}^{2} \) be a transformation defined by
    \[
        x = \vb{x}(u, v), \quad y = \vb{y}(u, v),
    \]
    where both \( \vb{x} \) and \( \vb{y} \) are of class \( \mathcal{C}^{1} \) with respect to the variables \( u \) and \( v \). \\
     The \vocab{Jacobian determinant} of \( T \), denoted by \( \abs{\pdv{(x, y)}{(u, v)}} \), is the absolute value of the determinant (orientation is irrelevant for our purposes) of the derivative matrix \( \mathbf{D}[T(u, v)] \). That is,
    \[
       \abs{ \pdv{(x, y)}{(u, v)}} =
        \left|
        \begin{array}{cc}
        \pdv{x}{u} & \pdv{x}{v} \\[1ex]
        \pdv{y}{u} & \pdv{y}{v}
        \end{array}
        \right|.
    \]
\end{dfn}

\begin{example}
    Suppose \( P \) is the parallelogram bounded by \( y= 2x \), \( y=2x-2 \), \( y=x \), and \( y=x+1 \). Evaluate \( \int \int_{P} xy \dd{x} \dd{y} \).\\
    The easiest type of surface to integrate over is a rectangle. With that in mind, notice that 
    \[ y-2x=0 \quad y-2x=-2 \]
    and 
    \[ y-x=0 \quad y-x =1 \]
    lend themselves to a nice substitution of 
    \[ u = y-2x \quad v =y-x .\]
    Now to find \( x \) and \( y \) in terms of \( u \) and \( v \), notice that 
    \[ v-u = (y-x)-(y-2x) \]
    or 
    \[ \boxed{x=v-u}. \]
    Back-substitution gives us 
   \[ v=y-x \Rightarrow v=y- (v-u) \Rightarrow \boxed{y= 2v-u} .\]
   By construction, our limits of integration are 
   \[ -2 \le u \le 0 ,\quad  0 \le v \le 1. \]
   The Jacobian determinant then becomes 
   \begin{align*}
    \abs{ \pdv{(x, y)}{(u, v)}} &=  \left|
        \begin{array}{cc}
        \pdv{x}{u} & \pdv{x}{v} \\[1ex]
        \pdv{y}{u} & \pdv{y}{v}
        \end{array}
        \right|\\
        &=\left|
        \begin{array}{cc}
        \pdv{u}\left( v-u \right)  & \pdv{v} \left( v-u \right) \\[1ex]
        \pdv{u} \left(2v-u \right) & \pdv{v} \left( 2v-u \right)
        \end{array}
        \right|\\
        &=\left|
        \begin{array}{cc}
        -1  & 1 \\[1ex]
       -1&2
        \end{array}
        \right|
   \end{align*}
   So 
   \[ \boxed{ \abs{ \pdv{(x, y)}{(u, v)}} =1} \]
   With everything in place, we are now ready to evaluate 
   \begin{align*}
    \int \int_{P} xy \dd{x} \dd{y} &= \int \int _{P^{*}} \vb{x} \left( u,v \right) \vb{y} \left( u,v \right)  \abs{ \pdv{(x, y)}{(u, v)}} \dd{u} \dd{v}\\
    &= \int_{0}^{1} \int_{-2}^{0} \left( v-u \right) \left( 2v-u \right) \dd{u} \dd{v}\\
    &=\int_{0}^{1} \int_{-2}^{0} 2v^{2}-3vu+u^{2} \dd{u} \dd{v}\\
    &=\int_{0}^{1}\left( 2v^{2}u- \frac{3}{2}v u^{2} + \frac{u^{3}}{3} \eval_{-2}^{0} \right) \dd{v}\\
     &=\int_{0}^{1} 4v^{2}+6v+\frac{8}{3} \dd{v}\\
     &= \frac{4}{3} v^{3} + 3v^{2} + \frac{8}{3}v \eval_{0}^{1}\\
     &= 7
   \end{align*}
So 
\[ \boxed{\int \int_{P} xy \dd{x} \dd{y} =7}. \]   
\end{example}

\subsection{Three Variable Case}

\begin{dfn}
    Let \( T: \mathbb{R}^{3} \to \mathbb{R}^{3} \) be a transformation defined by 
    \[ x = \vb{x} \left( u,v,w \right), \quad y = \vb{y} \left( u,v,w \right) , \quad z = \vb{z} \left( u,v,w \right)  \]
    where \( \vb{x}, \vb{y} \), and \( \vb{z} \) are of class \( \mathcal{C}^{1} \) with respect to \( u,v \) and \( w \).
    The Jacobian determinant of \( T \) is denoted by \( \abs{\pdv{(x,y,z)}{(u,v,w)}} \) and is similarly defined to be 
    \[ \abs{\pdv{(x,y,z)}{(u,v,w)}} =  \left|
        \begin{array}{ccc}
        \pdv{x}{u} & \pdv{x}{v} & \pdv{x}{w} \\[1ex]
        \pdv{y}{u} & \pdv{y}{v}& \pdv{y}{w} \\[1ex]
        \pdv{z}{u} & \pdv{z}{v}& \pdv{z}{w} 
        \end{array}
        \right|.\]
\end{dfn}

\begin{example}
    Calculate the Jacobian determinant for cylindrical coordinates. \\
    We have 
    \[ x = r \cos{ \left( \theta \right) }, \quad  y = r \sin{ \left( \theta \right) }, \quad z=z \]
    So 
    \begin{align*}
         \abs{\pdv{(x,y,z)}{(r,\theta,z)}} &=  \left|
        \begin{array}{ccc}
        \pdv{x}{r} & \pdv{x}{\theta} & \pdv{x}{z} \\[1ex]
        \pdv{y}{r} & \pdv{y}{\theta}& \pdv{y}{z} \\[1ex]
        \pdv{z}{r} & \pdv{z}{\theta}& \pdv{z}{z} 
        \end{array}
        \right|\\
        &=  \left|
        \begin{array}{ccc}
        \pdv{r} \left( r \cos{ \left( \theta \right) } \right) & \pdv{\theta} \left( r \cos{ \left( \theta \right) } \right)& \pdv{z} \left( r \cos{ \left( \theta \right) } \right)\\[1ex]
        \pdv{r} \left( r \sin{ \left( \theta \right) } \right) & \pdv{\theta}  \left( r \sin{ \left( \theta \right) } \right)& \pdv{z} \left( r \sin{ \left( \theta \right) } \right) \\[1ex]
        \pdv{r} \left( z \right) & \pdv{\theta} \left( z \right)& \pdv{z} \left( z \right)
        \end{array}
        \right|\\
        &= \left|
        \begin{array}{ccc}
        \cos{ \left( \theta \right) }& -r \sin{ \left( \theta \right) } & 0 \\[1ex]
       \sin{ \left(  \theta \right) } & r \cos{ \left( \theta \right) }& 0\\[1ex]
       0& 0& 1
        \end{array}
        \right|\\
        &= r
    \end{align*}
    This gives us
    \[ \boxed{ \abs{\pdv{(x,y,z)}{(r,\theta,z)}} =r} \]
\end{example}

\begin{example}
    Calculate the Jacobian determinant for spherical coordinates. \\
    We have 
    \[ x = \rho \sin{ \left( \varphi \right) } \cos{ \left( \theta \right) }, \quad  y= \rho \sin{ \left( \varphi \right) } \sin{ \left( \theta \right) }, \quad z = \rho \cos{ \left( \varphi \right) }.\]
So 
\begin{align*}
    \abs{\pdv{(x,y,z)}{(\rho, \varphi ,\theta)}} &=  \left|
        \begin{array}{ccc}
        \pdv{x}{\rho} & \pdv{x}{\varphi} & \pdv{x}{\theta} \\[1ex]
        \pdv{y}{\rho} & \pdv{y}{\varphi}& \pdv{y}{\theta} \\[1ex]
        \pdv{z}{\rho} & \pdv{z}{\varphi}& \pdv{z}{\theta} 
        \end{array}
        \right|\\\\
        &= \left|
        \begin{array}{ccc}
        \pdv{\rho} \left(  \rho \sin{ \left( \varphi \right) } \cos{ \left( \theta \right) } \right) & \pdv{\varphi}\left(  \rho \sin{ \left( \varphi \right) } \cos{ \left( \theta \right) } \right) & \pdv{\theta} \left(  \rho \sin{ \left( \varphi \right) } \cos{ \left( \theta \right) } \right)\\[1ex]
        \pdv{\rho} \left( \rho \sin{ \left( \varphi \right) } \sin{ \left( \theta \right) } \right) & \pdv{\varphi} \left( \rho \sin{ \left( \varphi \right) } \sin{ \left( \theta \right) } \right)& \pdv{\theta} \left( \rho \sin{ \left( \varphi \right) } \sin{ \left( \theta \right) } \right) \\[1ex]
        \pdv{\rho}\left(  \rho \cos{ \left( \varphi \right) } \right) & \pdv{\varphi} \left(  \rho \cos{ \left( \varphi \right) } \right)& \pdv{\theta} \left(  \rho \cos{ \left( \varphi \right) } \right)
        \end{array}
        \right| \\\\
        &= \left|
        \begin{array}{ccc}
            \sin{ \left( \varphi \right) } \cos{ \left( \theta \right) } & \rho \cos{ \left( \varphi \right) } \cos{ \left( \theta \right) } & - \rho \sin{ \left( \varphi \right) } \sin{ \left( \theta \right) } \\[1ex]
            \sin{ \left( \varphi \right) } \sin{ \left( \theta \right) } & \rho \cos{ \left( \varphi \right) } \sin{ \left(  \theta \right) } & \rho \sin{ \left( \varphi \right) } \cos{ \left( \theta \right) }\\[1ex]
            \cos{ \left( \varphi \right) } & - \rho \sin{ \left( \varphi \right) } & 0
        \end{array}
        \right|
\end{align*}
Unlike the previous example, we elect to explicitly show the determinant calculation here. 
\begin{align*}
    \left|
        \begin{array}{ccc}
            \sin{ \left( \varphi \right) } \cos{ \left( \theta \right) } & \rho \cos{ \left( \varphi \right) } \cos{ \left( \theta \right) } & - \rho \sin{ \left( \varphi \right) } \sin{ \left( \theta \right) } \\[1ex]
            \sin{ \left( \varphi \right) } \sin{ \left( \theta \right) } & \rho \cos{ \left( \varphi \right) } \sin{ \left(  \theta \right) } & \rho \sin{ \left( \varphi \right) } \cos{ \left( \theta \right) }\\[1ex]
            \cos{ \left( \varphi \right) } & - \rho \sin{ \left( \varphi \right) } & 0
        \end{array}
        \right| &= \sin{ \left( \varphi \right) } \cos{ \left( \theta \right) } \left|
        \begin{array}{cc}
     \rho \cos{ \left( \varphi \right) } \sin{ \left(  \theta \right) } & \rho \sin{ \left( \varphi \right) } \cos{ \left( \theta \right) }\\[1ex]
        - \rho \sin{ \left( \varphi \right) } & 0
        \end{array}
        \right|\\
        &- \rho \cos{ \left( \varphi \right) } \cos{ \left( \theta \right) } \left|
        \begin{array}{cc}
    \sin{ \left( \varphi \right) } \sin{ \left( \theta \right) } & \rho \sin{ \left( \varphi \right) } \cos{ \left( \theta \right) }\\[1ex]
        \cos{ \left( \varphi \right) } & 0
        \end{array}
        \right|\\
        &- \rho \sin{ \left( \varphi \right) } \sin{ \left( \theta \right) } \left|
        \begin{array}{cc}
    \sin{ \left( \varphi \right) } \sin{ \left( \theta \right) } & \rho \cos{ \left( \varphi \right) } \sin{ \left(  \theta \right) }\\[1ex]
        \cos{ \left( \varphi \right) } & - \rho \sin{ \left( \varphi \right) }
        \end{array}
        \right|\\
        &= \rho^{2} \sin^{3}{ \left( \varphi \right) } \cos^{2}{ \left( \theta \right) } + \rho^{2} \cos^{2}{ \left( \varphi \right) } \sin{ \left( \varphi \right) } \cos^{2}{ \left( \theta \right) }+ \rho^{2} \sin{ \left(  \varphi \right) }  \sin^{2}{ \left( \theta \right) }\\
        &= \rho^{2} \sin{ \left(  \varphi \right) } \left( \sin^{2}{ \left( \varphi \right) } \cos^{2}{ \left( \theta \right) } + \cos^{2}{ \left( \varphi \right) } \cos^{2}{ \left( \theta \right) } + \sin^{2}{ \left( \theta \right) }\right)\\
        &=\rho^{2} \sin{ \left(  \varphi \right) } \left( \left(  \sin^{2}{ \left( \varphi \right) }  + \cos^{2}{ \left( \varphi \right) } \right)  \cos^{2}{ \left( \theta \right) } + \sin^{2}{ \left( \theta \right) }\right) \\
        &= \rho^{2} \sin{ \left(  \varphi \right) } \left( \cos^{2}{ \left( \theta \right) } + \sin^{2}{ \left( \theta \right) }\right) 
\end{align*}
So 
\[ \boxed{\abs{\pdv{(x,y,z)}{(\rho, \varphi ,\theta)}}  = \rho^{2} \sin{ \left( \varphi \right) }} .\]
\end{example}

\begin{example}
    Evaluate 
    \[ \int \int \int_{W} \exp \left( x^{2}+y^{2}+ z^{2} \right)^{\frac{3}{2}} \dd{x} \dd{y} \dd{z} \]
    where \( W \) is unit ball in \( \mathbb{R}^{3} \).\\
    This question is \emph{begging} to be integrated in spherical coordinates and we will oblige it. \\
    Set \( x^{2} + y^{2}+z^{2} = \rho^{2} \) and the limits of integration become 
    \[ 0 \le \rho \le 1, \quad 0 \le \varphi \le \pi, \quad 0 \le \theta \le 2 \pi . \]
    This gives us 
    \begin{align*}
        \int \int \int_{W} \exp \left( x^{2}+y^{2}+ z^{2} \right)^{\frac{3}{2}} \dd{x} \dd{y} \dd{z}  &= \int_{0}^{1} \int_{0}^{\pi} \int_{0}^{2 \pi} \exp \left( \rho^{2} \right)^{\frac{3}{2}} \abs{\pdv{(x,y,z)}{(\rho, \varphi ,\theta)}} \dd{\theta} \dd{\varphi}  \dd{\rho} \\
        &= \int_{0}^{1} \int_{0}^{\pi} \int_{0}^{2 \pi} \exp \left( \rho^{3} \right) \rho^{2} \sin{ \left( \varphi \right) } \dd{\theta} \dd{\varphi} \dd{\rho} \\
        &= \int_{0}^{1} \exp \left( \rho^{3} \right) \rho^{2} \dd{\rho} \cdot \int_{0}^{\pi} \sin{ \left( \varphi \right) } \dd{\varphi} \cdot \int_{0}^{2 \pi} \dd{\theta}\\
        &= \frac{1}{3} \left(  e-1 \right) \cdot 2 \cdot 2 \pi
    \end{align*}
    So 
    \[ \boxed{\int \int \int_{W} \exp \left( x^{2}+y^{2}+ z^{2} \right)^{\frac{3}{2}} \dd{x} \dd{y} \dd{z} = \frac{4 \pi \left( e-1 \right)}{3}} \]
\end{example}






\chapter{Vector Analysis}
\input{content/analysis/multivariable-calculus/4-vector-analysis/multivariable-calculus-4-vector-analysis}


\chapter{Integrals Over Curves and Surfaces}
\section{Line Integrals}

\begin{example}\label{example: line integral of x4xy over triangle}
    Evaluate the line integral of \( \vb{F} \left< x,y \right>= \left( x^{4} , xy\right) \) along the triangular path with vertices \( \left( 0,0 \right) , \left( 0,1 \right)\) and \( \left( 1,0 \right) \). \\
    We have 3 paths to integrate over 
\begin{align*}
    \gamma_{1}(t) &= \left< t, 0 \right> & t \in &\left[ 0,1 \right] \\
    \gamma_{2}(t) &= \left< 1-t,\,t \right> & t \in & \left[ 0,1 \right] \\
    \gamma_{3}(t) &= \left< 0,\,1-t \right> & t \in & \left[ 0,1 \right]
\end{align*}
So we have 
\begin{align*}
    I_{1} &= \int_{0}^{1} \vb{F} \left( \gamma_{1} \left( t \right) \right) \cdot \gamma_{1}'(t) \dd{t}\\
    &= \int_{0}^{1} \left< t^{4}, \left( t \right) \left( 0 \right) \right> \cdot \left< 1,0 \right> \dd{t} \\
    &= \int_{0}^{1} t^{4} \dd{t}\\
    &= \frac{1}{5}t^{5} \eval_{0}^{1}
\end{align*}
So 
\[ \boxed{I_{1} = \frac{1}{5}}. \]
Next we have 
\begin{align*}
    I_{2} &= \int_{0}^{1} \vb{F} \left( \gamma_{2} \left( t \right) \right) \cdot \gamma_{2}'(t) \dd{t} \\
    &= \int_{0}^{1} \left< \left( 1-t \right)^{4}, t-t^{2}\right> \cdot \left< -1,1 \right> \dd{t}\\
    &= \int_{0}^{1} - \left( 1-t\right)^{4} +t-t^{2}\dd{t} \\
    &= \left( \frac{1}{5} \left( 1-t \right)^{5} +\frac{1}{2}t^{2}- \frac{1}{3}t^{3} \right) \eval_{0}^{1}
\end{align*}
So 
\[ \boxed{I_{2} = - \frac{1}{30}} .\]
Finally, 
\begin{align*}
    I_{3} &=  \int_{0}^{1} \vb{F} \left( \gamma_{3} \left( t \right) \right) \cdot \gamma_{3}'(t) \dd{t}  \\
    &= \int_{0}^{1} \left< 0,0 \right> \cdot \left< 0,-1 \right> \dd{t} \\
    &= \int_{0}^{1} 0 \dd{t}
\end{align*}
So 
\[ \boxed{I_{3}=0}. \]
Finally, we have 
\begin{align*}
    I &= I_{1}+I_{2}+I_{3}\\
    &= \frac{1}{5} - \frac{1}{30} + 0
\end{align*}
So 
\[ \boxed{I = \frac{1}{6}} .\]
An easier method of performing this calculation is showcased in \Cref{example: Green's theorem integral of x4xy over triangle}
\end{example}





\chapter{Vector Analysis Integration Theorems}
\section{Green's Theorem}
\begin{theorem}[Green's theorem]\label{theorem: Green's theorem}
    Let \( D \) be a simply connected region in \( \mathbb{R}^{2} \) whose boundary \( \partial D \) is a piecewise smooth, simple closed curve oriented counterclockwise. Let \( P \) and \( Q \) be continuously differentiable functions on an open set containing \( D \cup \partial D \). Then 
    \[ \oint_{\partial D} P \, dx + Q \, dy = \iint_{D} \left(\frac{\partial Q}{\partial x} - \frac{\partial P}{\partial y}\right) \, dA . \]
\end{theorem}
\begin{proof}
    
\end{proof}

\begin{example}\label{example: Green's theorem integral of x4xy over triangle}
    Suppose that \( D \) is the triangular region with vertices given by \( (0,0), (1,0) \) and \( (0,1) \). Apply \nameref{theorem: Green's theorem} to \( P = x^{4} \) and \( Q = xy \).\\ 
    We have 
    \begin{align*}
        \oint_{\partial D} P \, dx + Q \, dy &= \iint_{D}  \left(\frac{\partial Q}{\partial x} - \frac{\partial P}{\partial y}\right) \, dA  \\
        &= \int_{0}^{1} \int_{0}^{1-x}  \pdv{x} \left( xy \right) - \pdv{y} \left( x^{4} \right) \dd{y} \dd{x} \\
        &= \int_{0}^{1} \int_{0}^{1-x} y \dd{y} \dd{x}\\
        &= \frac{1}{2} \int_{0}^{1} y^{2} \eval_{0}^{1-x} \dd{x} \\
        &= \frac{1}{2} \int_{0}^{1} \left( 1-x \right)^{2} \dd{x} \\
        &= -\frac{1}{6} \left( 1-x \right)^{3} \eval_{0}^{1}
    \end{align*}
    So 
    \[ \boxed{ \oint_{\partial D} P \, dx + Q \, dy = \frac{1}{6}}.   \]
    Notice that his method is easier than explicitly computing the line integral as shown in \Cref{example: line integral of x4xy over triangle}.
\end{example}


If we know only the boundary of a region, it is natural to ask whether we can compute the area enclosed. Green's theorem provides a direct method.

Recall that the area of a region \(D\) is
\[
\text{Area}(D) = \iint_{D} 1 \, dA.
\]
If we choose functions \(P(x,y), Q(x,y)\) such that
\[
\pdv{Q}{x} - \pdv{P}{y} = 1,
\]
then Green's theorem tells us
\[
\text{Area}(D) = \oint_{\partial D} P\,dx + Q\,dy.
\]

For instance, taking \(P=0, Q=x\) yields
\[
\text{Area}(D) = \oint_{\partial D} x \, dy,
\]
while taking \(P=-y, Q=0\) gives
\[
\text{Area}(D) = -\oint_{\partial D} y \, dx.
\]
or taking \( P = - \frac{1}{2} y \), \( Q = \frac{1}{2} x \) yields
\[ \text{Area} \left( D  \right) = \frac{1}{2} \oint_{\partial D} x \dd{y} - y \dd{x} \]

\begin{example}
    Calculate the area of an ellipse given by the equation 
    \[
        \frac{x^{2}}{a^{2}} + \frac{y^{2}}{b^{2}} = 1.
    \]
    Let \(E\) denote the region bounded by the ellipse. 
    Choose \(P = -\tfrac{1}{2}y\), \(Q = \tfrac{1}{2}x\), so that 
    \[
        \text{Area}(E) 
        = \oint_{\partial E} P\,dx + Q\,dy
        = \frac{1}{2}\oint_{\partial E} x\,dy - y\,dx.
    \]
    The ellipse can be parametrized by 
    \[
        x = a\cos\theta, 
        \quad 
        y = b\sin\theta,
        \quad \theta \in [0,2\pi].
    \]
    Substituting, we obtain
    \[
        \text{Area}(E) 
        = \frac{1}{2} \int_{0}^{2\pi} 
        \Big( a\cos\theta\, d(b\sin\theta) 
        - b\sin\theta\, d(a\cos\theta)\Big).
    \]
    Simplifying,
    \[
        \text{Area}(E) 
        = \frac{1}{2}\int_{0}^{2\pi} 
        ab(\cos^{2}\theta + \sin^{2}\theta)\,d\theta
        = \frac{1}{2}\int_{0}^{2\pi} ab\,d\theta
        = \pi ab.
    \]
    Thus the area of the ellipse is 
    \[
        \boxed{ab \pi}.
    \]
\end{example}


\section{Stokes' Theorem}
\begin{theorem}[Stokes' Theorem]
    Let \( S \subset \mathbb{R}^3 \) be an oriented, piecewise smooth surface with positively oriented, piecewise smooth boundary curve \( \partial S \). 
    Suppose \(\vb{F} \in \mathcal{C}^{1}(U; \mathbb{R}^3)\), where \(U \subset \mathbb{R}^3\) is open and contains \(S\). 
    Then
    \[
     \iint_{S} \mathrm{curl}( \vb{F}) \cdot \dd{\vb{S}} =  \iint_{S} ( \largetriangledown \times \vb{F}) \cdot \dd{\vb{S}}
        = \oint_{\partial S} \vb{F} \cdot \dd{\vb{s}}.
    \]
\end{theorem}
\begin{proof}
    
\end{proof}


\section{Divergence Theorem}
\begin{theorem}[Divergence Theorem]
    Suppose that \( R \) is a simple solid region with \( \partial R \) as the boundary with outward orientation. Suppose that \(\vb{F} \in \mathcal{C}^{1}(U; \mathbb{R}^3)\), where \(U \subset \mathbb{R}^3\) is open and contains \(R\). Then 
    \[  \iiint_{R} \mathrm{div}( \vb{F}) \dd{V} =\iiint_{R} \largetriangledown \cdot \vb{F} \dd{V} = \iint_{ \partial R} \vb{F} \cdot \dd{\vb{A}} \]
\end{theorem}
\begin{proof}
    
\end{proof}




\part{Ordinary Differential Equations}
\label{part: Ordinary Differential Equations}
\parttoc
\input{content/analysis/ordinary-differential-equations/ordinary-differential-equations-1-introduction}

\chapter{First Order Differential Equations}


We will deal with differential equations of the form 
 
\begin{equation}
    \dv{y}{t} = f(t,y) \label{6/4/25/7}
\end{equation}

\section{Linear Equations, Integrating Factors}
\begin{dfn}
    If \( f \) in \Cref{6/4/25/7} is of the form \( f(t,y) = p(t)y + g(t) \), we call \Cref{6/4/25/7} a \vocab{first-order linear equation}.
\end{dfn}

Observe that this is equivalent to the form
\begin{equation}
    P(t) \dv{y }{t } + Q(t)y = G(t), \quad P(t) \neq 0, \label{6/4/25/8}
\end{equation}
which can always be rewritten in the form of \Cref{6/4/25/7} by dividing both sides by \( P(t) \). 


\begin{exercise}
    Solve the differential equation 
    \begin{equation}
        \left( \sin \left( t^{2} \right) + 2 \right) \dv{y }{t} + \left( 2t \cos \left( t^{2} \right) \right)y = 6t^{2}. \label{6/4/25/9}
    \end{equation}
\end{exercise}

\begin{solution}
    We observe that the left-hand side is the derivative of a product:
    \[
        \dv{t} \left[ \left( \sin \left( t^{2} \right) + 2 \right) y \right] = 6t^{2}.
    \]
    Integrating both sides, we obtain
    \[
        \int \dv{t} \left[ \left( \sin \left( t^{2} \right) + 2 \right) y \right] \dd{t} = \int 6t^{2} \dd{t},
    \]
    which gives
    \[
        \left( \sin \left( t^{2} \right) + 2 \right) y = 2t^{3} + C.
    \]
    Solving for \( y \), we find the general solution:
    \[
        \boxed{y = \frac{2t^{3} + C}{\sin \left( t^{2} \right) + 2}}.
    \]
\end{solution}

We can generalize the above procedure. Suppose we are given a differential equation of the form
\[
    P(t) \dv{y}{t} + P'(t)y = G(t), \quad P(t) \neq 0.
\]
Then the left-hand side is the derivative of \( P(t)y \), and we may write
\[
    \dv{t} \left[ P(t) y \right] = G(t),
\]
so that
\[
    y = \frac{1}{P(t)} \int G(t) \dd{t}.
\]


Now consider the general first-order linear equation
\[
    \dv{y}{t} + p(t)y = g(t).
\]
Unless \( p(t) = 0 \), we cannot directly apply the previous trick. Instead, we multiply through by a nonzero function \( \mu(t) \), chosen so that the left-hand side becomes the derivative of a product:
\[
    \mu(t) \dv{y}{t} + \mu(t)p(t)y = \mu(t) g(t),
\]
or equivalently,
\[
    \dv{t} \left[ \mu(t)y \right] = \mu(t)g(t).
\]

It is straightforward to verify that setting
\[
    \mu(t) = e^{\int p(t) \dd{t}}
\]
achieves this goal. We call \( \mu(t) \) the \vocab{integrating factor}. In particular, the linear equation becomes
\begin{align*}
    \dv{t} \left[  e^{\int p(t) \dd{t}} y \right] &=  e^{\int p(t) \dd{t}} g(t) \\
    \int \dv{t} \left[  e^{\int p(t) \dd{t}} y \right] \dd{t} &=  \int e^{\int p(t) \dd{t}} g(t) \dd{t}\\
    e^{ \int p(t) \dd{t}}y &=\int e^{\int p(t) \dd{t}} g(t) \dd{t}\\
    y &= \frac{1}{e^{ \int p(t) \dd{t}}} \int e^{\int p(t) \dd{t}} g(t) \dd{t}
\end{align*}


\begin{example}
    If we are given the equation
    \[ \dv{y}{t}+ \frac{1}{2}y = \frac{1}{2}e^{\frac{t }{3}} .\]
Here \( p(t)= \frac{1}{2} \) so our integrating factor becomes 
    \[ \mu(t)= e^{\int \frac{1}{2} \dd{t}}= e^{\frac{t}{2} +c}= Ce^{\frac{t}{2}} \]
    We do not need maximum generality for out integrating factor so we can just pick \( C=1 .\)
    Multiplying through with our integrating factor, we get 
    \[ e^{\frac{t}{2}} \dv{y }{t} + \frac{1}{2}e^{\frac{t}{2}}y = \frac{1}{2}e^{\frac{t}{3}}e^{\frac{t}{2}} \]
    or 
    \[ \dv{t} \left[ e^{\frac{t }{2}}y \right] = \frac{1}{2} e^{\frac{5t}{6}}.\]
    Integrating both sides, we have 
    \[ e^{\frac{t }{2}}y= \frac{3}{5}e^{\frac{5t }{6}} +C\] or
    \[ \boxed{y = \frac{3}{5} e^{\frac{t }{3}} + C e^{\frac{-t }{2}}} \]
    Now suppose that we want to find the particular solution that goes through the point \( (0,1) \), we would have 
    \[ 1= \frac{3}{5} e^{\frac{0 }{3}} + C e^{\frac{-0 }{2}} \]
    or 
    \[ 1 = \frac{3}{5}+C \]
    so \( C= \frac{2}{5} \) and the particular solution is 
    \[ \boxed{y = \frac{3}{5} e^{\frac{t }{3}} + \frac{2}{5} e^{\frac{-t }{2}}} \]
\end{example}

\begin{example}
    Solve the initial value problem.
    \begin{align*}
        t \dv{y}{t}+ 2y &= 4t^{2}\\
        y(1)&=2
    \end{align*}
Dividing every term by \( t \), we have 
\[ \dv{y}{t} + \frac{2}{t}y =4t \]
Here \( p(t)= \frac{2}{t} \) so \( \mu(t)= e^{ \int \frac{2}{t } \dd{t}} \) or \( \mu (t) = t^{2} \). So multiplying through by \( \mu(t) \), we have 
\[ t^{2} \dv{y}{t} + 2t = 4t^{3} \]
or 
\[ \dv{t} \left[ t^{2}y \right] = 4t^{3} \]
Integrating both sides, we have 
\[ t^{2}y = t^{4}+C .\]
Applying the initial condition, we have 
\[ 1^{2} \cdot 2=1^{4}+C \Rightarrow C=1 \]
So 
\[ \boxed{y= t^{2} + \frac{1}{t^{2}}, \quad t>0} \]
\end{example}






\part{A Second Course in Linear Algebra}
\label{Proof Based Linear Algebra}
\vspace*{2em} The primary resource for this part is \cite{ref:axler_linalg}. \vspace*{-1em}
\parttoc
\chapter{Vector Spaces}
\section{Definition and Examples of Vector Spaces}
\begin{dfn}\label{def:vector-space}
    A \vocab{vector space} over a field \( \mathbb{K} \) is a set \( \mathbf{V} \) with elements called \vocab{vectors} such that all of the following criteria are held:
\begin{description}
  \item[\textbf{Closure under addition:}] 
  There is a binary operation called \vocab{vector addition} where 
  \begin{align*}
    + : \; &\mathbf{V} \times \mathbf{V} \to \mathbf{V} \\
    &(\vb{u}, \vb{v}) \mapsto \vb{u} + \vb{v},
  \end{align*}
    \item[\textbf{Closure under scalar multiplication:}] There is an operation called \vocab{scalar multiplication} where
    \begin{align*}
     \cdot : \; & \mathbb{K} \times \mathbf{V} \to \mathbf{V} \\
    &(\lambda, \vb{v}) \mapsto \lambda \vb{v},
  \end{align*}
  \item[\textbf{Commutativity:}] \( \vb{u} + \vb{v} = \vb{v} + \vb{u} \) for all \( \vb{u}, \vb{v} \in \mathbf{V} \). 
  \item[\textbf{Associativity:}] \( \left( \vb{u} + \vb{v} \right) + \vb{w} = \vb{u} + \left( \vb{v} + \vb{w} \right) \) and \( \alpha \left( \beta \vb{v}   \right)  = \left( \alpha \beta \right) \vb{v} \) for all \( \vb{u}, \vb{v},\vb{w} \in \mathbf{V} \) and \( \alpha, \beta \in \mathbb{K} \).
  \item[\textbf{Distributive Properties:}] \( \alpha \left( \vb{u} + \vb{v} \right) = \alpha \vb{u} + \alpha \vb{v} \) and \( \left( \alpha + \beta \right)\vb{v} = \alpha \vb{v} + \beta \vb{v} \) for all \( \vb{u}, \vb{v} \in \mathbf{V} \) and \( \alpha, \beta \in \mathbb{K} \).
  \item[\textbf{Existence of an Additive Identity:}] There exists an element  \( \vb{0} \in \mathbf{V} \) called the \vocab{additive identity} such that for all \( \vb{v} \in \mathbf{V} \), we have \( \vb{v} + \vb{0} = \vb{v} \).
  \item[\textbf{Existence of an Additive Inverse:}] For every \( \vb{v} \in \mathbf{V} \), there exists a \( \vb{w} \in \mathbf{V} \) called the \vocab{additive inverse} such that \( \vb{v} + \vb{w} = \vb{0} \).
  \item[\textbf{Multiplicative Identity:}] For \( 1 \in \mathbb{K} \), we have that \( 1 \vb{v} = \vb{v} \) for all \( \vb{v} \in \mathbf{V} \).
\end{description}
We will usually pick \( \mathbb{K} \) to be \( \mathbb{R} \) or \( \mathbb{C} \). If \( \mathbb{K} = \mathbb{R} \), we will sometimes call \( \mathbf{V} \) a \vocab{real vector space}. If \( \mathbb{K} = \mathbb{C} \), we will sometimes call \( \mathbf{V} \) a \vocab{comples vecotr space}. 

\end{dfn}



\begin{example}
  Let \( \mathbb{K} \) be a field and let \( \mathbb{K}^{n} \) be the set of all \( n \)-tuples whose entries come from \( \mathbb{K} \). We define 
  \[ \left( a_{1}, a_{2}, \dots, a_{n} \right) + \left( b_{1}, b_{2}, \dots , b_{n} \right) = \left( a_{1} +b_{1}, a_{2} +b_{2}, \dots, a_{n} +b_{n} \right) \] and 
  \[ \lambda \left( a_{1}, a_{2}, \dots, a_{n} \right) = \left( \lambda a_{1}, \lambda a_{2}, \dots , \lambda a_{n} \right) \] for all \( \lambda , a_{1}, a_{2}, \dots, a_{n},b_{1}, b_{2}, \dots , b_{n} \in \mathbb{K} \). \\ 
  Closure under addition and scalar multiplication is immediate from inheritance from the field \( \mathbb{K} \). \\ 
  For \textbf{commutativity} , we have if \( \vb{a} =  \left( a_{1}, a_{2}, \dots, a_{n} \right) \) and \( \vb{b} =  \left( b_{1}, b_{2}, \dots , b_{n} \right) \) then 
  \begin{align*}
    \vb{a} + \vb{b} &= \left( a_{1}, a_{2}, \dots, a_{n} \right) + \left( b_{1}, b_{2}, \dots , b_{n} \right) \\
    &= \left( a_{1} +b_{1}, a_{2} +b_{2}, \dots, a_{n} +b_{n} \right) \\
    &=  \left( b_{1} +a_{1}, b_{2} +a_{2}, \dots, b_{n} +a_{n} \right) \tag{Since $\mathbb{K}$ is a field.}\\
    &= \left( b_{1}, b_{2}, \dots , b_{n} \right) + \left( a_{1}, a_{2}, \dots, a_{n} \right) \\
    &= \vb{b} + \vb{a}
  \end{align*}
  For \textbf{associativity}, let \( \vb{c} = \left( c_{1}, c_{2}. \dots, c_{n} \right) \) so 
  \begin{align*}
    \left[ \vb{a} + \vb{b} \right] + \vb{c} &= \left[ \left( a_{1}, a_{2}, \dots, a_{n} \right) + \left( b_{1}, b_{2}, \dots , b_{n} \right) \right] + \left( c_{1}, c_{2}. \dots, c_{n} \right) \\
    &= \left( a_{1} +b_{1}, a_{2} +b_{2}, \dots, a_{n} +b_{n} \right) + \left( c_{1}, c_{2}. \dots, c_{n} \right) \\
    &= \left( \left[ a_{1} +b_{1} \right]  + c_{1},  \left[ a_{2} +b_{2} \right]  + c_{2}, \dots, \left[ a_{n} +b_{n} \right]  + c_{n} \right) \\
    &= \left( a_{1} + \left[ b_{1} + c_{1} \right] , a_{2} + \left[ b_{2} + c_{2} \right], \dots, a_{n} + \left[ b_{n} + c_{n} \right]  \right) \tag{Since $\mathbb{K}$ is a field.}  \\
    &= \left( a_{1}, a_{2}, \dots, a_{n} \right) +\left( b_{1} + c_{1}, b_{2} + c_{2}, \dots, b_{n} + c_{n} \right) \\
    &=  \left( a_{1}, a_{2}, \dots, a_{n} \right) + \left[  \left( b_{1}, b_{2}, \dots , b_{n} \right) +  \left( c_{1}, c_{2}. \dots, c_{n} \right)\right] \\
    &=\vb{a} + \left[ \vb{b} + \vb{x} \right]
  \end{align*}
  For the second part of associativity, pick \( \lambda, \mu \in \mathbb{K} \). Then 
  \begin{align*}
    \lambda \left[ \mu \vb{a} \right] &= \lambda \left( \mu a _{1}, \mu a_{2}, \dots, \mu a_{n} \right) \\
    &= \left( \lambda \left[ \mu a_{1} \right], \lambda \left[ \mu a_{2} \right], \dots, \lambda \left[ \mu a_{n} \right] \right) \\
    &= \left( \left[ \lambda \mu \right] a_{1},  \left[ \lambda \mu \right] a_{2}, \dots , \left[ \lambda \mu \right] a_{n} \right) \tag{Since $\mathbb{K}$ is a field.} \\
    &= \left[ \lambda \mu \right] \left( a_{1}, a_{2}, \dots, a_{n} \right) \\ 
    &= \left[ \lambda \mu \right] \vb{a}
  \end{align*}
  For the \textbf{distributive properties}, we have 
  \begin{align*}
    \lambda \left[ \vb{a} + \vb{b} \right] &= \lambda \left( a_{1} +b_{1}, a_{2} +b_{2}, \dots, a_{n} +b_{n} \right) \\
    &= \left( \lambda \left[ a_{1} + b_{1} \right],  \lambda \left[ a_{2} + b_{2} \right], \dots  \lambda \left[ a_{n} + b_{n} \right] \right) \\
    &= \left( \lambda a_{1} + \lambda b_{1}, \lambda a_{2} + \lambda b_{2} , \dots , \lambda a_{n} + \lambda b_{n}  \right) \tag{Since $\mathbb{K}$ is a field.} \\
    &= \left( \lambda a_{1}, \lambda a_{2}, \dots, \lambda a_{n} \right) +\left( \lambda b_{1}, \lambda b_{2}, \dots, \lambda b_{n} \right) \\
    &= \lambda \left( a_{1}, a_{2}, \dots, a_{n} \right) + \lambda\left( b_{1}, b_{2}, \dots , b_{n} \right) \\
    &= \lambda \vb{a} + \lambda \vb{b}
  \end{align*}
For the second distributive property, we have 
\begin{align*}
  \left[ \lambda + \mu \right] \vb{a} &= \left[ \lambda + \mu \right] \left( a_{1}, a_{2}, \dots, a_{n} \right) \\
  &= \left(\left[ \lambda + \mu \right] a_{1},\left[ \lambda + \mu \right]a_{2}, \dots, \left[ \lambda + \mu \right]a_{n} \right) \\
  &= \left( \lambda a_{1} + \mu a_{1}, \lambda a_{2} + \mu a_{2}, \dots, \lambda a_{n} + \mu a_{n} \right) \tag{Since $\mathbb{K}$ is a field.} \\ 
  &= \left( \lambda a_{1}, \lambda a_{2}, \dots, \lambda a_{n} \right) + \left( \mu a_{1}, \mu a_{2}, \dots, \mu a_{n} \right) \\
  &= \lambda \left( a_{1}, a_{2}, \dots, a_{n} \right) + \mu \left( a_{1}, a_{2}, \dots, a_{n} \right) \\
  &= \lambda \vb{a} + \mu \vb{a}
\end{align*}
For the \textbf{additive identity}, let \( \vb{0} = \left( 0, 0, \dots, 0 \right) \) so for any \( \vb{a} \), we have 
\begin{align*}
  \vb{a} + \vb{0} &= \left( a_{1}, a_{2}, \dots, a_{n} \right)  +  \vb{0} = \left( 0, 0, \dots, 0 \right) \\
  &=  \left( a_{1} + 0, a_{2} + 0, \dots, a_{n} +0 \right) \\
  &=  \left( a_{1}, a_{2}, \dots, a_{n} \right) \tag{Since $\mathbb{K}$ is a field.} \\ 
  &= \vb{a}
\end{align*}
For the \textbf{additive inverse}, pick for any \( \vb{a} = \left( a_{1}, a_{2}, \dots, a_{n} \right) \) the vector \( - \vb{a} = \left(- a_{1}, -a_{2}, \dots, -a_{n} \right) \). So 
\begin{align*}
  \vb{a} + \left[ - \vb{a} \right] &= \left( a_{1}, a_{2}, \dots, a_{n} \right) + \left(- a_{1}, -a_{2}, \dots, -a_{n} \right) \\
  &= \left( a_{1} - a_{1},  a_{2} - a_{2}, \dots,  a_{n} - a_{n} \right) \\
  &= \left( 0, 0, \dots, 0 \right) \tag{Since $\mathbb{K}$ is a field.} \\
  &= \vb{0}
\end{align*}
For the \textbf{multiplicative identity} , we have 
\begin{align*}
  1 \vb{a} &= 1 \left( a_{1}, a_{2}, \dots, a_{n} \right)  \\
  &= \left(1 a_{1}, 1a_{2}, \dots, 1a_{n} \right) \\
  &= \left( a_{1}, a_{2}, \dots, a_{n} \right) \tag{Since $\mathbb{K}$ is a field.}  \\
  &= \vb{a}
\end{align*}

\end{example}


\begin{lemma}
  In a vector space \( \mathbf{V} \), the additive identity is unique.
\end{lemma}
\begin{proof}
    Suppose that \( \vb{0} \) and \( \vb{0}' \) are each additive identities for a vector space \( \mathbf{V} \). Then 
    \begin{align*}
      \vb{0} &= \vb{0} + \vb{0}' \tag{Since $\vb{0}'$ is an additive identity. }\\
      &= \vb{0}' \tag{Since $\vb{0}$ is an additive identity. }
    \end{align*}
    
\end{proof}

\begin{lemma}
For each \( \vb{v} \in \mathbf{V} \), its inverse is unique.
\end{lemma}
\begin{proof}
    Suppose that \( \vb{w} \) and \( \vb{w}' \) are each inverses to \( \vb{v} \). Then 
    \begin{align*}
      \vb{w} &= \vb{w} + \vb{0}\\
      &= \vb{w} + \left( \vb{v} + \vb{w}' \right) \\
      &= \left( \vb{w} + \vb{v} \right) + \vb{w}'\\
      &= \vb{0} + \vb{w}' \\
      &= \vb{w}'
    \end{align*}
    
\end{proof}


\begin{exercise}
  Suppose that \( \mathbf{V} \) is a real vector space. We define the \vocab{complexification} of \( \mathbf{V} \), denoted as \( \mathbf{V}_{\mathbb{C}} \), as \( \mathbf{V} \times  \mathbf{V} \) where we write \( \left( \vb{v}, \vb{w} \right) \) as \( \vb{v} + i \vb{w} \). We define addition on \( \mathbf{V}_{\mathbb{C}} \) as 
  \[ \left( \vb{v}_{1} + i \vb{v}_{2} \right) + \left( \vb{w}_{1} + i \vb{w}_{2} \right) = \left( \vb{v}_{1} + \vb{w}_{1} \right) + i \left( \vb{v}_{2} + \vb{w}_{2} \right) \]
  and scalar multiplication by 
  \[ \left( \alpha + i \beta \right) \left( \vb{v} + i \vb{w} \right) = \left( \alpha \vb{v} - \beta \vb{w} \right) + i \left( \beta \vb{v} + \alpha \vb{w} \right) \]
  for \( \alpha, \beta \in \mathbb{R} \). \\
  Show that \( \mathbf{V}_{\mathbb{C}}\) is a complex space. 
\end{exercise}
\begin{solution}
    
\end{solution}



\chapter{Span, Basis and Dimension}
\section{Span and Linear Independence}
\begin{dfn}
    Let \( \mathbf{V} \) be a vector space over a field \( \mathbb{K} \).  
    We say that \( \vb{v} \in \mathbf{V} \) is a \vocab{linear combination} of  
    \( \left\{ \vb{v}_{1}, \vb{v}_{2}, \dots, \vb{v}_{n} \right\} \subset \mathbf{V} \)  
    if there exist scalars \( \alpha_{1}, \alpha_{2}, \dots, \alpha_{n} \in \mathbb{K} \) such that  
    \[
        \vb{v} = \sum_{k=1}^{n} \alpha_{k} \vb{v}_{k}
        \quad = \quad
        \alpha_{1} \vb{v}_{1} + \alpha_{2} \vb{v}_{2} + \cdots + \alpha_{n} \vb{v}_{n}.
    \]
    The finiteness of the sum is built into the definition: we restrict to finite sums to avoid issues of convergence and to keep linear algebra maximally general and widely applicable.
\end{dfn}

\begin{dfn}
    Given \( X = \left\{ \vb{x}_{1}, \vb{x}_{2}, \dots, \vb{x}_{n} \right\} \subset \mathbf{V} \), we define
    \[
        \mathrm{span}(X)
        = \mathrm{span} \left( \vb{x}_{1}, \vb{x}_{2}, \dots, \vb{x}_{n} \right)
        = \left\{ \sum_{k=1}^{n} \alpha_{k} \vb{x}_{k} \ \middle| \ \alpha_{k} \in \mathbb{K} \right\}.
    \]
    That is, \( \mathrm{span}(X) \) is the set of all linear combinations of the vectors in \( X \).
\end{dfn}

\begin{exercise}
    Find a list of four vectors in \( \mathbb{K}^{3} \) whose span equals the set 
    \[P= \left\{ \left< x,y,z  \right> \in \mathbb{K}^{3} \ \middle | \ x+y+z =0 \right\} \]
\end{exercise}
\begin{solution}
    This is a two dimensional subspace of \( \mathbb{K}^{3} \) so we need to find two vectors \( \vb{u} \) and \( \vb{v} \) that span \( P \) and two additional redundancy vectors. We just set 
    \[ \vb{u} = \begin{bmatrix}
        1\\
        -1\\
        0\\
    \end{bmatrix}, \quad \vb{v} = \begin{bmatrix}
        1\\
        0\\
        -1\\
    \end{bmatrix} \]
    and just take any vector \( \vb{a} = s_{1} \vb{u} + t_{1}\vb{v} \) and \( \vb{b} = s_{2} \vb{u} + t_{2} \vb{v} \) where \( s_{1}, s_{2}, t_{1}, t_{2} \in \mathbb{K} - \left\{ 0 \right\} \)
\end{solution}

\begin{exercise}
    Does \( \mathrm{span} \left( \vb{v}_{1}, \vb{v}_{2}, \vb{v}_{3}, \vb{v}_{4} \right) = \mathrm{span} \left( \vb{v}_{1} - \vb{v}_{2}, \vb{v}_{2}- \vb{v}_{3}, \vb{v}_{3}- \vb{v}_{4}, \vb{v}_{4} \right) \)?
\end{exercise}
\begin{solution}
    Suppose that \( \vb{v} \in \mathrm{span} \left( \vb{v}_{1}, \vb{v}_{2}, \vb{v}_{3}, \vb{v}_{4} \right) \). Then \( \vb{v}= \alpha_{1} \vb{v}_{1} + \alpha_{2} \vb{v}_{2} + \alpha_{3} \vb{v}_{3} + \alpha_{4} \vb{v}_{4} \) for some \( \alpha_{1}, \alpha_{2}, \alpha_{3}, \alpha_{4} \in \mathbb{K} \). We want to find scalars \( \beta_{1} , \beta_{2}, \beta_{3}, \beta_{4} \in \mathbb{K}\) such that \( \vb{v} = \beta_{1} \left( \vb{v}_{1} - \vb{v}_{2} \right) + \beta_{2} \left( \vb{v}_{2} - \vb{v}_{3} \right) + \beta_{3} \left( \vb{v}_{3} - \vb{v}_{4} \right) + \beta_{4} \vb{v}_{4}\). Setting 
    \[ \beta_{k} = \sum_{j \le k} \alpha_{j} \]
    achieves this. So \( \mathrm{span} \left( \vb{v}_{1}, \vb{v}_{2}, \vb{v}_{3}, \vb{v}_{4} \right) \subseteq \mathrm{span} \left( \vb{v}_{1} - \vb{v}_{2}, \vb{v}_{2}- \vb{v}_{3}, \vb{v}_{3}- \vb{v}_{4}, \vb{v}_{4} \right)  \). \\ 
    Now suppose that \( \vb{v} = \beta_{1} \left( \vb{v}_{1} - \vb{v}_{2} \right) + \beta_{2} \left( \vb{v}_{2} - \vb{v}_{3} \right) + \beta_{3} \left( \vb{v}_{3} - \vb{v}_{4} \right) + \beta_{4} \vb{v}_{4} \) for arbitrary \( \beta_{1} , \beta_{2}, \beta_{3}, \beta_{4} \in \mathbb{K}\). Then 
    \[ \vb{v} = \beta_{1} \vb{v}_{1} + \left( \beta_{2} - \beta_{1}\right) \vb{v}_{2} + \left( \beta_{3} - \beta_{2} \right) \vb{v}_{3} + \left( \beta_{4} - \beta_{3} \right)\vb{v}_{4}\]
    so \( \mathrm{span} \left( \vb{v}_{1} - \vb{v}_{2}, \vb{v}_{2}- \vb{v}_{3}, \vb{v}_{3}- \vb{v}_{4}, \vb{v}_{4} \right) \subseteq  \mathrm{span} \left( \vb{v}_{1}, \vb{v}_{2}, \vb{v}_{3}, \vb{v}_{4} \right) \).
\end{solution}


\chapter{Linear Maps}
\section{Linear Maps}
\begin{dfn}
    Suppose that \( \mathbf{V} \) and \( \mathbf{W} \) are vector spaces over the same field \( \mathbb{K} \). The \vocab{linear map} is a map \( T: \mathbf{V} \to \mathbf{W} \) such that 
    \begin{enumerate}[label=\textbf{\roman*)}]
        \item \( T \left( \vb{v}_{1} + \vb{v}_{2} \right) = T \left(  \vb{v}_{1} \right) + T \left( \vb{v}_{2} \right) \) for all \( \vb{v}_{1}, \vb{v}_{2} \in \mathbf{V} \). 
        \item \( T \left( \lambda \vb{v} \right) = \lambda T \left( \vb{v} \right) \) for all \( \lambda \in \mathbb{K} \) and \( \vb{v} \in \mathbf{V} \).
    \end{enumerate}
    The set of linear maps from \( \mathbf{V} \) to \( \mathbf{W} \) is denoted by \( \mathcal{L} \left( \mathbf{V}, \mathbf{W} \right) \). In the case that the codomain is the same as the domain, (\( T: \mathbf{V} \to \mathbf{V} \)), we sometimes call such a linear map a \vocab{linear operator} and we denote the set of linear operators on a vector space \( \mathbf{V} \) as \( \mathcal{L} \left( \mathbf{V} \right) \).
\end{dfn}

\begin{lemma}\label{thm:linear-maps-0-to-0}
    Let \( \mathbf{V} \) and \( \mathbf{W} \) be vector spaces and \( T \in \mathcal{L} \left( \mathbf{V}, \mathbf{W} \right) \). \( T \) maps the additive identity of \( \mathbf{V} \) to the additive identity of \( \mathbf{W} \); that is, 
    \[ T \left( \vb{0}_{\mathbf{V}} \right) = \vb{0}_{\mathbf{W}}\]
\end{lemma}
\begin{proof}
    \begin{align*}
        T \left( \vb{0}_{\mathbf{V}} \right) &= T \left(\vb{0}_{\mathbf{V}} + \vb{0}_{\mathbf{V}}  \right) \\
        &= T \left( \vb{0}_{\mathbf{V}} \right) + T \left( \vb{0}_{\mathbf{V}} \right)
    \end{align*}
    So \( T \left( \vb{0}_{\mathbf{V}} \right) \) is the additive identity for \( \mathbf{W} \).
\end{proof}


\begin{lemma}[An easy linear map test]\label{thm:easy-linear-map-test}
    A map \( T \) between vector spaces \( \mathbf{V} \) and \( \mathbf{W} \) is a linear map if and only if 
    \( T \left( \vb{v}_{1} + \lambda \vb{v}_{2} \right) = T \left( \vb{v}_{1} \right)+ \lambda T \left( \vb{v}_{2} \right) \) for all \( \vb{v}_{1}, \vb{v}_{2} \in \mathbf{V} \) and \( \lambda \in \mathbb{K} \).
\end{lemma}
\begin{proof}
\( \left( \Rightarrow \right) \) Suppose that \( T \in \mathcal{L} \left( \mathbf{V}, \mathbf{W} \right) \). Then for any \( \vb{v}_{1}, \vb{v}_{2} \in \mathbf{V} \) and \( \lambda \in \mathbb{K} \)
    \begin{align*}
        T \left( \vb{v}_{1} + \lambda \vb{v}_{2} \right) &= T \left( \vb{v}_{1} \right) + T \left( \lambda \vb{v}_{2} \right) \\
        &= T \left( \vb{v}_{1} \right) + \lambda T \left(  \vb{v}_{2} \right)
    \end{align*}
    \( \left(  \Leftarrow \right) \) Now suppose \( T \left( \vb{v}_{1} + \lambda \vb{v}_{2} \right) = T \left( \vb{v}_{1} \right)+ \lambda T \left( \vb{v}_{2} \right) \) for all \( \vb{v}_{1}, \vb{v}_{2} \in \mathbf{V} \) and \( \lambda \in \mathbb{K} \). If we set \( \lambda =1 \), then 
    \begin{align*}
        T \left(  \vb{v}_{1} + \vb{v}_{2} \right) &= T \left( \vb{v}_{1} + 1 \cdot \vb{v}_{2} \right)\\
         &= T \left( \vb{v}_{1} \right) + 1 T \left( \vb{v}_{2} \right)\\ 
        &= T \left( \vb{v}_{1} \right) + T \left( \vb{v}_{2} \right)
    \end{align*}
    And if we set \( \vb{v}_{1} = \vb{0}_{\mathbf{V}} \), 
    \begin{align*}
        T \left( \lambda \vb{v}\right) &= T \left(  \vb{0}_{\mathbf{V}}+ \lambda \vb{v}\right) \\
        &= T \left( \vb{0}_{\mathbf{V}} \right) + \lambda T \left( \vb{v} \right) \\
        &= \vb{0}_{\mathbf{W}} +  \lambda T \left( \vb{v} \right)  \\
        &=  \lambda T \left( \vb{v} \right) 
    \end{align*}
    Therefore T satisfies both additivity and scalar multiplication, so \( T \in \mathcal{L} \left( \mathbf{V}, \mathbf{W} \right) \).
\end{proof}

\begin{example}\label{eg:zero-map-is-linear}
    The zero map 
    \begin{align*}
        Z&: \mathbf{V} \to \mathbf{W}\\
        Z&: \vb{v} \mapsto \vb{0}_{\mathbf{W}}
    \end{align*}
    is linear. To show this, pick any \( \vb{v}_{1}, \vb{v}_{2} \in \mathbf{V} \) and \(  \lambda \in \mathbb{K} \), then 
    \begin{align*}
        Z \left( \vb{v}_{1} + \lambda \vb{v}_{2} \right) &= \vb{0}_{\mathbf{W}}\\
        &= \vb{0}_{\mathbf{W}} + \lambda \vb{0}_{\mathbf{W}} \\
        &= Z \left( \vb{v}_{1} \right) + \lambda Z \left( \vb{v}_{2} \right)
    \end{align*}
\end{example}



\begin{exercise}
    Most textbooks use \( T \left( \lambda_{1} \vb{v}_{1} + \lambda_{2} \vb{v}_{2}\right) = \lambda_{1} T \left( \vb{v}_{1} \right) + \lambda_{2} T \left( \vb{v}_{2} \right)\) as the linear map test. Show directly that this test is equivalent to the test in \cref{thm:easy-linear-map-test}. That is to say, if \( T: \mathbf{V} \to \mathbf{W} \) is \emph{any} map, show that \(  T \left( \lambda_{1} \vb{v}_{1} + \lambda_{2} \vb{v}_{2}\right) = \lambda_{1} T \left( \vb{v}_{1} \right) + \lambda_{2} T \left( \vb{v}_{2} \right)\) if and only if \( T \left( \vb{v}_{1} + \lambda \vb{v}_{2} \right) = T(\vb{v}_{1}) + \lambda T \left( \vb{v}_{2} \right) \).
\end{exercise}
\begin{solution}
    \( \Rightarrow \) Suppose that \( T \left( \lambda_{1} \vb{v}_{1} + \lambda_{2} \vb{v}_{2}\right) = \lambda_{1} T \left( \vb{v}_{1} \right) + \lambda_{2} T \left( \vb{v}_{2} \right)\) for all \( \lambda_{1}, \lambda_{2} \in \mathbb{K} \) and \( \vb{v}_{1}, \vb{v}_{2} \in \mathbf{V} \). Set \( \lambda_{1} =1 \) and \( \lambda_{2} = \lambda \). We have
     \[ T \left( \vb{v}_{1} + \lambda \vb{v}_{2}\right) = T \left( \vb{v}_{1}  \right) + \lambda T \left(  \vb{v}_{2} \right).\]\\ 
    \( \Leftarrow \) Now suppose that \( T \left( \vb{v}_{1} + \lambda \vb{v}_{2}\right) = T \left( \vb{v}_{1}  \right) + \lambda T \left(  \vb{v}_{2} \right) \) for all \( \lambda \in \mathbb{K} \) and \( \vb{v}_{1}, \vb{v}_{2} \in \mathbf{V} \). We want to show that \( T \left( \lambda_{1} \vb{x}_{1} + \lambda_{2} \vb{x}_{2} \right) = \lambda_{1} T \left( \vb{x}_{1} \right) + \lambda_{2} T \left( \vb{x}_{2} \right) \) for all \( \lambda_{1}, \lambda_{2} \in \mathbb{K} \) and \( \vb{x}_{1}, \vb{x}_{2} \in \mathbf{V} \). \\
    
    The result is obvious if \( \lambda_{1} = 1 \) (by setting \( \lambda = \lambda_{2} \)), so assume \( \lambda_{1} \neq 1 \). For any \( \lambda_{1}, \lambda_{2} \in \mathbb{K} \) with \( \lambda_{1} \neq 1 \) and \( \vb{x}_{1}, \vb{x}_{2} \in \mathbf{V} \), pick 
    \[ \boxed{\vb{v}_{1} = \vb{x}_{1}} \; ,  \;  \boxed{\lambda = \lambda_{1} -1} \; , \;  \text{and} \; \boxed{ \vb{v}_{2} = \vb{x}_{1} + \frac{\lambda_{2}}{\lambda_{1}-1} \vb{x}_{2}}  .\] 
    On one hand, 
    \begin{align*}
        T \left( \vb{v}_{1} + \lambda \vb{v}_{2} \right) &= T \left( \vb{x}_{1} + \left( \lambda_{1} -1 \right)  \left( \vb{x}_{1} + \frac{\lambda_{2}}{\lambda_{1}-1} \vb{x}_{2} \right)\right)\\
        &= T \left(  \vb{x}_{1} + \left( \lambda_{1}-1 \right)  \vb{x}_{1} + \frac{\lambda_{2} \left( \lambda_{1}-1 \right)}{\lambda_{1}-1} \vb{x}_{2} \right) \\
        &= T \left( \lambda_{1} \vb{x}_{1} + \lambda_{2} \vb{x}_{2} \right)
    \end{align*}
    On the other hand, 
    \begin{align*}
         T \left( \vb{v}_{1} + \lambda \vb{v}_{2} \right) &= T \left( \vb{v}_{1} \right) + \lambda T \left( \vb{v}_{2} \right) \\
         &= T \left( \vb{x}_{1} \right) + \left( \lambda_{1} -1 \right) \left[ T \left( \vb{x}_{1} + \frac{\lambda_{2}}{\lambda_{1}-1} \vb{x}_{2} \right) \right] \\
         &= T \left( \vb{x}_{1} \right) + \left( \lambda_{1} -1 \right) \left[ T \left( \vb{x}_{1} \right) + \frac{\lambda_{2}}{\lambda_{1}-1} T \left( \vb{x}_{2} \right) \right] \\
         &= \lambda_{1} T \left( \vb{x}_{1} \right) + \lambda_{2} T \left( \vb{x}_{2} \right)
    \end{align*}
    This shows that \(T \left( \lambda_{1} \vb{x}_{1} + \lambda_{2} \vb{x}_{2} \right) = \lambda_{1} T \left( \vb{x}_{1} \right) + \lambda_{2} T \left( \vb{x}_{2} \right)\).
\end{solution}


\begin{theorem}[The set of linear maps is itself a vector space.]\label{thm:linear-maps-are-a-vector-space}
    Let \( \mathbf{V} \) and \( \mathbf{W} \) be vector spaces over \( \mathbb{K} \). The set \( \mathcal{L} \left( \mathbf{V}, \mathbf{W} \right) \) is itself a vector space where we define 
    \begin{enumerate}[label=\textbf{\roman*)}]
        \item \( \left( S +T \right) \left( \vb{v} \right): = S \left( \vb{v} \right) + T \left( \vb{v} \right) \) for all \( S,T \in   \mathcal{L} \left( \mathbf{V}, \mathbf{W} \right)\) and \( \vb{v} \in \mathbf{V} \)
        \item \( \left( \lambda T  \right)(\vb{v}) := \lambda \left( T \left( \vb{v} \right) \right) \) for all \( \lambda \in \mathbb{K} \), \( T \in   \mathcal{L} \left( \mathbf{V}, \mathbf{W} \right) \), and \( \vb{v} \in \mathbf{V} \).
    \end{enumerate}
\end{theorem}
\begin{proof}
    We need to verify that the conditions laid out in \cref{def:vector-space} hold.\\ 
    First, to show that if \( S,T \in   \mathcal{L} \left( \mathbf{V}, \mathbf{W} \right)\) then \( S+T \in   \mathcal{L} \left( \mathbf{V}, \mathbf{W} \right)\), pick \( \lambda \in \mathbb{K} \) and \( \vb{v}_{1}, \vb{v}_{2} \in \mathbf{V} \). Then 
    \begin{align*}
        \left( S+T \right) \left( \vb{v}_{1} + \lambda \vb{v}_{2} \right) &= S \left( \vb{v}_{1} + \lambda \vb{v}_{2} \right) + T \left( \vb{v}_{1} + \lambda \vb{v}_{2} \right) \\
        &= S \left( \vb{v}_{1} \right) + \lambda S \left( \vb{v}_{2} \right) + T \left( \vb{v}_{1} \right) + \lambda T \left( \vb{v}_{2} \right) \\
        &= S \left( \vb{v}_{1} \right) + T \left( \vb{v}_{1} \right) + \lambda \left( S \left( \vb{v}_{2} \right) + T \left( \vb{v}_{2} \right) \right) \\
        &= \left( S+T \right)\left( \vb{v}_{1} \right) + \lambda  \left( S+T \right)\left( \vb{v}_{2} \right)
    \end{align*}
    Similarly to show that \( \lambda T \in \mathcal{L} \left( \mathbf{V}, \mathbf{W} \right) \), for any \( \mu \in \mathbb{K} \) and \( \vb{v}_{1}, \vb{v}_{2} \in \mathbf{V} \), we have 
    \begin{align*}
        \left( \lambda T \right) \left( \vb{v}_{1} + \mu \vb{v}_{2} \right) &= \lambda \left( T  \left( \vb{v}_{1} + \mu \vb{v}_{2} \right) \right) \\
        &= \lambda \left( T \left( \vb{v}_{1} \right) \right) + \lambda \mu T \left( \vb{v}_{2} \right) \\
        &=\left( \lambda T \right) \left( \vb{v}_{1} \right) + \mu \left( \lambda T \right) \left( \vb{v}_{2} \right)
    \end{align*}
    Thus, \( \mathcal{L} \left( \mathbf{V}, \mathbf{W} \right) \) is closed under addition and scalar multiplication. \\ 
    For commutativity, we have 
    \begin{align*}
        \left( S + T \right) \left( \vb{v} \right) &= S \left( \vb{v} \right) + T \left( \vb{v} \right) \\
        &= T \left( \vb{v} \right) + S \left( \vb{v} \right) \tag{Since $\mathbf{W}$ is a vector space.} \\
        &= \left( T + S\right) \left( \vb{v} \right)
    \end{align*}
   Associativity and the distributive properties will also rely on inheritance of those properties from \( \mathbf{W} \). If \( S,T, U \in \mathcal{L} \left( \mathbf{V}, \mathbf{W} \right) \) and \( \lambda \in \mathbb{K} \)
   \begin{align*}
        \left( \left( S +T \right) + U \right) \left( \vb{v} \right) &= \left( S+T \right)\left( \vb{v} \right) + U \left( \vb{v} \right) \\ 
        &= \left( S \left( \vb{v} \right) + T \left( \vb{v} \right) \right) + U \left( \vb{v} \right) \\ 
        &= S \left( \vb{v} \right) +\left(  T \left( \vb{v} \right) + U \left( \vb{v} \right)\right) \\
        &= S \left( \vb{v} \right) + \left( T+U \right) \left( \vb{v} \right) \\
        &= \left( S + \left( T +U \right) \right) \left( \vb{v} \right)
   \end{align*}
   \begin{align*}
    \left( \lambda \left( S +T \right) \right) \left( \vb{v} \right) &= \lambda \left( \left( S+T  \right)\left( \vb{v} \right) \right) \\ 
    &= \lambda \left( S \left( \vb{v} \right) + T \left( \vb{v} \right)\right) \\
    &= \lambda \left( S \left( \vb{v} \right) \right) + \lambda \left( T \left( \vb{v} \right) \right) \\
    &= \left( \lambda S \right) \left( \vb{v} \right) +  \left( \lambda T \right) \left( \vb{v} \right) \\
    &= \left( \lambda S + \lambda T \right) \left( \vb{v} \right)
   \end{align*}
   
   Similarly, if \( \alpha, \beta \in \mathbb{K} \) and \( T \in \mathcal{L} \left( \mathbf{V}, \mathbf{W} \right) \) then 
   \begin{align*}
    \left( \alpha \left( \beta T \right) \right) \left( \vb{v} \right) &= \alpha \left( \left( \beta T \right) \left( \vb{v} \right) \right) \\
    &= \alpha \left( \beta \left( T \left( \vb{v} \right) \right) \right) \\
    &= \left( \alpha \beta \right)\left( T \left( \vb{v} \right) \right) \\
    &= \left( \left( \alpha \beta \right)T \right) \left( \vb{v} \right)
   \end{align*}
   
   \begin{align*}
    \left( \left( \alpha + \beta \right)T \right)\left( \vb{v} \right) &= \left( \alpha + \beta \right) \left( T \left( \vb{v} \right) \right) \\
    &= \alpha \left( T \left( \vb{v} \right) \right) + \beta \left( T \left( \vb{v} \right) \right) \\
    &= \left( \alpha T \right) \left( \vb{v} \right) + \left(  \beta T \right) \left(  \vb{v} \right)\\ 
    &= \left(  \alpha T + \beta T \right) \left( \vb{v} \right)
   \end{align*}
   So the associative and distributive properties hold. \\ 
   For the additive identity, let us use \cref{eg:zero-map-is-linear}. For any \( T \in \mathcal{L} \left( \mathbf{V}, \mathbf{W} \right) \) and \( \vb{v} \in \mathbf{V} \), 
   \begin{align*}
    \left( T+Z \right) \left( \vb{v} \right) &= T \left( \vb{v} \right) + Z \left( \vb{v} \right) \\
    &= T \left( \vb{v} \right) + \vb{0}_{\mathbf{W}} \\
    &= T \left( \vb{v} \right)
   \end{align*}
   So the zero map is the additive identity. We will use \( 0 \)  or \( 0_{\mathcal{L} \left( \mathbf{V}, \mathbf{W} \right)} \) (only when context is absolutely required) in place of \( Z \). \\ 
   For additive inverses, if \( T \in \mathcal{L} \left( \mathbf{V}, \mathbf{W} \right) \), define \( -T \) by \( \left( -T \right)\left( \vb{v} \right) := -T \left( \vb{v} \right) \) for all \( \vb{v} \in \mathbf{V} \). First, we verify that \( -T \in \mathcal{L} \left( \mathbf{V}, \mathbf{W} \right) \). For any \( \lambda \in \mathbb{K} \) and \( \vb{v}_{1}, \vb{v}_{2} \in \mathbf{V} \),
   \begin{align*}
       \left( -T \right)\left( \vb{v}_{1} + \lambda \vb{v}_{2} \right) &= -T \left( \vb{v}_{1} + \lambda \vb{v}_{2} \right) \\
       &= -\left( T \left( \vb{v}_{1} \right) + \lambda T \left( \vb{v}_{2} \right) \right) \\
       &= -T \left( \vb{v}_{1} \right) - \lambda T \left( \vb{v}_{2} \right) \\
       &= \left( -T \right)\left( \vb{v}_{1} \right) + \lambda \left( -T \right)\left( \vb{v}_{2} \right)
   \end{align*}
   Thus \( -T \in \mathcal{L} \left( \mathbf{V}, \mathbf{W} \right) \). Now, for any \( \vb{v} \in \mathbf{V} \),
   \begin{align*}
       \left( T + \left( -T \right) \right)\left( \vb{v} \right) &= T \left( \vb{v} \right) + \left( -T \right)\left( \vb{v} \right) \\
       &= T \left( \vb{v} \right) + \left( -T \left( \vb{v} \right) \right) \\
       &= \vb{0}_{\mathbf{W}} \\
       &= Z \left( \vb{v} \right)
   \end{align*}
   So \( -T \) is the additive inverse of \( T \). \\
   Finally, for the multiplicative identity, if \( T \in \mathcal{L} \left( \mathbf{V}, \mathbf{W} \right) \) and \( \vb{v} \in \mathbf{V} \),
   \begin{align*}
       \left( 1 T \right)\left( \vb{v} \right) &= 1 \left( T \left( \vb{v} \right) \right) \\
       &= T \left( \vb{v} \right)
   \end{align*}
   Therefore, all the vector space axioms are satisfied, and \( \mathcal{L} \left( \mathbf{V}, \mathbf{W} \right) \) is a vector space over \( \mathbb{K} \).
\end{proof}


\section{Duals}
\begin{dfn}
    Let \( \mathbf{V} \) be a vector space over a field \( \mathbb{K} \). The \vocab{dual space} of \( \mathbf{V} \) is simply \( \mathcal{L} \left( \mathbf{V} , \mathbb{K}\right) \). We often denote the dual space as \( \mathbf{V}^{\str} \). Elements of the dual space are called \vocab{dual vectors}, \vocab{covectors}, or \vocab{linear functionals}.   
\end{dfn}


\chapter{Eigenvalues and Eigenvectors}
\section{Invariant Subspaces}
\begin{dfn}
    Let \( T \in \mathcal{L} \left( \mathbf{V} \right) \). A subspace \( \mathbf{U} \subset \mathbf{V} \) is \vocab{invariant under \( T \)}  or is an \vocab{invariant subspace of \( T \)} if \( T \vb{u} \in \mathbf{U} \) for all \( \vb{u} \in \mathbf{U} \).
\end{dfn}

\begin{exercise}
    Suppose that \( T \in \mathcal{L} \left( \mathbf{V} \right) \) and \( \mathbf{U} \) is a subspace of \( \mathbf{V} \). Prove that if \( \mathbf{U} \subseteq \mathrm{null} \left( T \right) \), then \( \mathbf{U} \) is an invariant subspace of \( T \).
\end{exercise}
\begin{solution}
    Since \( \mathbf{U} \subseteq \mathrm{null}(T) \), \( T \vb{u} = \vb{0} \) for all \( \vb{u} \in \mathbf{U} \). Since \( \mathbf{U} \) is a subspace, \( \vb{0} \in \mathbf{U} \) and hence \( T \vb{u} \in \mathbf{U} \) for all \( \vb{u} \in \mathbf{U} \). Therefore \( \mathbf{U} \) is invariant under \( T \).
\end{solution}


\begin{exercise}
    Suppose that \( T \in \mathcal{L} \left( \mathbf{V} \right) \) and \( \mathbf{U} \) is a subspace of \( \mathbf{V} \). Prove that if \( \mathrm{range} \left( T \right) \subseteq \mathbf{U} \), then \( \mathbf{U} \) is an invariant subspace of \( T \).
\end{exercise}
\begin{solution}
     Let \( \vb{u} \in \mathbf{U} \). Then \( T\vb{u} \in \mathrm{range}(T) \) by definition of the range. Since \( \mathrm{range}(T) \subseteq \mathbf{U} \), we have \( T\vb{u} \in \mathbf{U} \). Therefore \( \mathbf{U} \) is invariant under \( T \).
\end{solution}

\begin{exercise}
    Suppose that \( T \in \mathcal{L} \left( \mathbf{V} \right) \) and \( \mathbf{V}_{1}, \dots , \mathbf{V}_{n} \) are invariant under \( T \). Show that \( \mathbf{V}_{1} + \cdots + \mathbf{V}_{n} \) is invariant under \( T \).
\end{exercise}
\begin{solution}
    Pick any \( \vb{v} \in \mathbf{V}_{1} + \cdots + \mathbf{V}_{n} \). Then \( \vb{v}= \sum_{k=1}^{n} \vb{v}_{k}  \) for \( \vb{v}_{k} \in \mathbf{V}_{k} \). Then 
    \begin{align*}
        T \left( \vb{v} \right) &= T \left(\sum_{k=1}^{n} \vb{v}_{k} \right) \\
        &= \sum_{k=1}^{n} T (\vb{v}_{k})
    \end{align*}
    Since each \( \mathbf{V}_{k} \) is invariant under \( T \), \( T (\vb{v}_{k}) \in \mathbf{V}_{k} \) and hence \( \sum_{k=1}^{n} T (\vb{v}_{k}) \in \mathbf{V}_{1} + \cdots + \mathbf{V}_{n}\) so \( \mathbf{V}_{1} + \cdots + \mathbf{V}_{n} \) is invariant under \( T \).
\end{solution}

\begin{exercise}
    Suppose that \( T \in \mathcal{L} \left( \mathbf{V} \right) \). Prove that the intersection of every collection of invariant subspaces of \( T \) is invariant under \( T \).
\end{exercise}
\begin{solution}
    Let \( \left\{ \mathbf{V}_{\alpha} \right\}_{\alpha \in A} \) be the collection of all invariant subspaces of \( T \). Let \( B \subseteq A \) be any subcollection, and consider \( \bigcap_{\beta \in B} \mathbf{V}_{\beta} \). 
    Pick any \( \vb{v} \in \bigcap_{\beta \in B} \mathbf{V}_{\beta} \). Then \( \vb{v} \in \mathbf{V}_{\beta} \) for all \( \beta \in B \). Since each \( \mathbf{V}_{\beta} \) is invariant under \( T \), we have \( T\vb{v} \in \mathbf{V}_{\beta} \) for all \( \beta \in B \). Hence \( T\vb{v} \in \bigcap_{\beta \in B} \mathbf{V}_{\beta} \).
\end{solution}

\begin{exercise}
    Prove or provide a counterexample: If \( \mathbf{V} \) is finite-dimensional and \( \mathbf{U} \) is a subspace of \( \mathbf{V} \) that is invariant under every operator on \( \mathbf{V} \), then \( \mathbf{U} = \left\{ \vb{0} \right\} \) or \( \mathbf{U} = \mathbf{V} \).
\end{exercise}
\begin{solution}
    Suppose that \( \mathbf{U} \) is a proper non-trivial subspace of \( \mathbf{V} \). Hence, there exists \( \vb{u} \in \mathbf{U} \) and \( \vb{v} \in \mathbf{V} - \mathbf{U} \), each non-zero. Since \( \vb{u} \neq \vb{0} \), we can extend \( \left\{ \vb{u} \right\} \) to a basis \( \left\{ \vb{u}, \vb{u}_{1}, \vb{u}_{2}, \dots, \vb{u}_{n} \right\} \) for \( \mathbf{V} \). Define 
    \[ T  \left( \vb{u} \right) = \vb{v} \quad T \left( \vb{u}_{k} \right) =0  \quad  \text{for } k =1,2, \dots, n\]
    and extend linearly. By construction, \( \mathbf{U} \) is not invariant under \( T \) and hence the claim is true.
\end{solution}


\begin{dfn}
    Suppose \( T \in \mathcal{L} \left(  \mathbf{V} \right) \). We say that \( \vb{v} \in \mathbf{V} \) is an \vocab{eigenvector} of \( T \) if there exists a \( \lambda \in \mathbb{K} \), called the \vocab{eigenvalue}, such that \( T \vb{v} = \lambda \vb{v} \) and \( \vb{v} \neq \vb{0} \).
\end{dfn}

\begin{exercise}
    Suppose that \( P \in \mathcal{L} \left( \mathbf{V} \right) \) has the property that \( P^{2} = P \). If \( \lambda \) has an eigenvalue, then \( \lambda \) must be \( 0 \) or \( 1 \).
\end{exercise}
\begin{solution}
    Suppose \( \vb{v} \) is an eigenvector of \( P \) with eigenvalue \( \lambda \). Then, on one hand,
    \[ P^{2} \left( \vb{v} \right) = P \left( \vb{v} \right) = \lambda \vb{v}\]
    On the other hand,
    \[ P^{2}\left( \vb{v} \right) = P \left( P \left( \vb{v} \right) \right) = P \left( \lambda \vb{v} \right) = \lambda P \left( \vb{v} \right) = \lambda^{2} \vb{v}\]
    So 
    \[ \lambda \vb{v} = \lambda^{2} \vb{v}. \]
    Therefore \( \lambda =0 \) or \( \lambda =1 \).
\end{solution}


\chapter{Inner Product Spaces}
\section{Inner Products and Norms}
\begin{dfn}\label{def:inner-product}
    An \vocab{inner product} on a vector space \( \mathbf{V} \) is a map 
    \begin{align*}
        \cdot : & \mathbf{V} \times \mathbf{V} \to \mathbb{K} \\
        & \left( \vb{u},\vb{v} \right) \mapsto \left< \vb{u},\vb{v} \right>
    \end{align*}
    such that all of the following criteria are held:
    \begin{description}
        \item[\textbf{Non-negative:}] \( \left< \vb{v},\vb{v} \right> \ge 0 \) for all \( \vb{v} \in \mathbf{V} \). 
        \item[\textbf{Definiteness:}] \( \left< \vb{v},\vb{v} \right> = 0 \) if and only if \( \vb{v} = \vb{0} \).
        \item[\textbf{Linearity in the first argument:}] \( \left< \vb{u} + \lambda \vb{v}, \vb{w} \right>  = \left<  \vb{u}, \vb{w} \right> + \lambda \left<  \vb{v},\vb{w} \right>\) for all \( \vb{u}, \vb{v}, \vb{w} \in \mathbf{V} \) and \( \lambda \in \mathbb{K} \).
        \item[\textbf{Conjugate symmetry:}] \( \left< \vb{u}, \vb{v} \right> = \overline{\left< \vb{v},\vb{u} \right>} \) for all \( \vb{u},\vb{v} \in \mathbf{V} \). 
    \end{description}
    If \( \mathbf{V} \) has an inner product defined, we call \( \mathbf{V} \) an \vocab{inner product space}.
\end{dfn}

\textbf{Note.} Some older texts refer to inner product spaces as \vocab{pre-Hilbert spaces}. The prefix “pre” will make sense once you study functional analysis.


\begin{lemma}
    An inner product on a vector space \( \mathbf{V} \) is conjugate linear in the second component. That is: 
    \[ \left< \vb{u}, \vb{v} + \lambda \vb{w} \right>  = \left< \vb{u},\vb{v} \right> + \overline{\lambda} \left< \vb{u},\vb{w} \right>.\]
\end{lemma}
\begin{proof}
    We have 
    \begin{align*}
        \left< \vb{u}, \vb{v} + \lambda \vb{w} \right> &= \overline{ \left< \vb{v} + \lambda \vb{w}, \vb{u} \right>} \\
        &= \overline{ \left< \vb{v}, \vb{u} \right> + \lambda \left< \vb{w}, \vb{u} \right>} \\
        &= \overline{ \left<  \vb{v}, \vb{u} \right>} + \overline{\lambda} \overline{ \left< \vb{w}, \vb{u} \right>} \\
        &= \overline{ \overline{\left<  \vb{u}, \vb{v} \right> }} + \overline{ \lambda} \overline{ \overline{ \left< \vb{u}, \vb{w} \right>}} \\
        &= \left< \vb{u},\vb{v} \right> + \overline{\lambda} \left< \vb{u},\vb{w} \right>
    \end{align*}
    
\end{proof}


\begin{exercise}\label{exc:expand-ip-diff}
    Let \( \vb{v}_{1}, \vb{v}_{2}, \vb{w}_{1}, \vb{w}_{2} \) be vectors in an inner product space \( \mathbf{V} \). Show that 
    \[ \left< \vb{v}_{1}, \vb{w}_{1} \right> - \left< \vb{v}_{2}, \vb{w}_{2} \right> = \left< \vb{v}_{1}-\vb{v}_{2}, \vb{w}_{1} -\vb{w}_{2} \right> + \left< \vb{v}_{2} , \vb{w}_{1} -\vb{w}_{2} \right> + \left<  \vb{v}_{1}-\vb{v}_{2}, \vb{w}_{2} \right>\]
\end{exercise}
\begin{solution}
    Just calculate the right hand side. 
    \begin{align*}
         \left< \vb{v}_{1}-\vb{v}_{2}, \vb{w}_{1} -\vb{w}_{2} \right> + \left< \vb{v}_{2} , \vb{w}_{1} -\vb{w}_{2} \right> + \left<  \vb{v}_{1}-\vb{v}_{2}, \vb{w}_{2} \right> &= \left< \vb{v}_{1}, \vb{w}_{1} - \vb{w}_{2} \right> - \left< \vb{v}_{2}, \vb{w}_{1} - \vb{w}_{2} \right> + \left< \vb{v}_{2} , \vb{w}_{1} -\vb{w}_{2} \right> + \left<  \vb{v}_{1}-\vb{v}_{2}, \vb{w}_{2} \right>\\ 
         &= \left<  \vb{v}_{1}, \vb{w}_{1} - \vb{w}_{2} \right> - \Ccancel[DeepRed]{\left< \vb{v}_{2}, \vb{w}_{1} - \vb{w}_{2} \right>}  + \Ccancel[DeepRed]{\left< \vb{v}_{2}, \vb{w}_{1} - \vb{w}_{2} \right>} + \left<  \vb{v}_{1}-\vb{v}_{2}, \vb{w}_{2} \right> \\
         &= \left<  \vb{v}_{1}, \vb{w}_{1} - \vb{w}_{2} \right> + \left<  \vb{v}_{1}-\vb{v}_{2}, \vb{w}_{2} \right> \\
         &= \left< \vb{v}_{1}, \vb{w}_{1} \right> - \left< \vb{v}_{1}, \vb{w}_{2} \right> + \left< \vb{v}_{1}, \vb{w}_{2} \right> - \left< \vb{v}_{2}, \vb{w}_{2} \right> \\
         &=  \left< \vb{v}_{1}, \vb{w}_{1} \right> - \Ccancel[DeepRed]{ \left< \vb{v}_{1}, \vb{w}_{2} \right>} + \Ccancel[DeepRed]{ \left< \vb{v}_{1}, \vb{w}_{2} \right>} - \left< \vb{v}_{2}, \vb{w}_{2} \right> \\
         &= \left< \vb{v}_{1}, \vb{w}_{1} \right> - \left< \vb{v}_{2}, \vb{w}_{2} \right>
    \end{align*}
    
\end{solution}


\begin{exercise}
    For a complex inner product space \( \mathbf{V} \), show that 
    \[ \Im \left( \left< \vb{v},\vb{w} \right> \right) = \Re \left( \left< \vb{v}, i \vb{w} \right> \right) \quad  \text{for all } \vb{v}, \vb{w} \in \mathbf{V}.\]
\end{exercise}
\begin{solution}
    Suppose that \( \left< \vb{v}, \vb{w} \right> = a+bi \). Then, by conjugate linearity in the second component, we have 
    \begin{align*}
        \left< \vb{v}, i \vb{w} \right>&= \overline{i} \left< \vb{v},\vb{w} \right> \\
        &= -i \left< \vb{v},\vb{w} \right> \\
        &= -i \left( a+bi \right) \\
        &= b - ai
    \end{align*}
    It is clear that \( \Im (a+bi) = \Re (b-ai) \).
\end{solution}

\begin{dfn}
    Suppose that \( \mathbf{V} \) is an inner-product space. If \( \vb{v}, \vb{w} \in \mathbf{V}\) have the property that \( \left< \vb{v},\vb{w} \right> =0\), we say that \( \vb{v} \) and \( \vb{w} \) are \vocab{orthogonal} and sometimes write \( \vb{v} \perp \vb{w} \). Note that the zero vector is orthogonal to every vector in \( \mathbf{V} \).
\end{dfn}

\begin{example}
    Suppose that \( \vb{x} \perp \vb{y} \) and \( \vb{y} \perp \vb{z} \). Is it necessarily true that \( \vb{x} \perp \vb{z} \)?\\
    No. For a counterexample, take \( \vb{x} = \vb{z} \neq \vb{0} \). Then \( \left< \vb{x}, \vb{y} \right> = 0 \) and \( \left< \vb{y}, \vb{z} \right> = 0 \) are satisfied, but \( \left< \vb{x}, \vb{z} \right> \neq 0 \).\\
    Alternatively, take \( \vb{y} = \vb{0} \) and choose any \( \vb{x} \) and \( \vb{z} \) such that \( \left< \vb{x}, \vb{z} \right> \neq 0 \).
\end{example}



\begin{lemma}
    Let \( S \) be a subset of an inner product space \( \mathbf{V} \). The set 
    \[ S^{\perp} = \left\{ \vb{v} \in \mathbf{V} \ \middle| \  \left< \vb{v},\vb{s} \right> = 0 \text{ for all } \vb{s} \in S \right\} \]
    is a subspace of \( \mathbf{V} \).
\end{lemma}
\begin{proof}
    Clearly \( \vb{0} \in S^{\perp} \). If \( \vb{v}, \vb{w} \in S^{\perp} \) and \( \lambda \in \mathbb{K}\), then for any \( \vb{s} \in S \) we have
    \begin{align*}
        \left< \vb{v} + \lambda \vb{w}, \vb{s} \right> &= \left<  \vb{v},\vb{s} \right> + \lambda \left< \vb{w}, \vb{s} \right> \\ 
        &= 0 + \lambda \cdot 0 \\
        &= 0
    \end{align*}
    So \( S^{\perp} \) is a subspace of \( \mathbf{V} \). We call \( S^{\perp} \) the \vocab{orthogonal complement} of \( S \). In speech, \( S^{\perp} \) is often read as “S perp.”
\end{proof}

\begin{exercise}
    Show that 
    \[ S^{\perp} = \left( \mathrm{span}(S) \right)^{\perp} \]
\end{exercise}
\begin{solution}
    Clearly \( \left( \mathrm{span}(S) \right)^{\perp} \subseteq S^{\perp}  \) since \( S \subseteq \mathrm{span}(S) \). Now pick \( \vb{v} \in S^{\perp } \) and pick any \( \vb{s} \in \mathrm{span}(S) \). Therefore \( \vb{s} = \sum_{k=1}^{n} \lambda_{k} \vb{s}_{k} \) for \( \vb{s}_{k} \in S \). So 
    \begin{align*}
        \left< \vb{v},\vb{s} \right> &= \left< \vb{v}, \sum_{k=1}^{n} \lambda_{k} \vb{s}_{k} \right> \\
        &= \sum_{k =1}^{n} \overline{\lambda_{k}} \left< \vb{v},\vb{s}_{k} \right> \\
        &= \sum_{k =1}^{n} \overline{\lambda_{k}} \cdot 0 \tag{Since $\vb{v} \in S^{\perp}$} \\
        &= 0
    \end{align*}
    So \( \vb{v} \in \left( \mathrm{span}(S) \right)^{\perp} \)
\end{solution}



\subsection{Norms}

\begin{dfn}\label{def:norm}
    A \vocab{norm} is a map \( \norm{ \cdot }: \mathbf{V} \to \left[ 0, \infty \right) \), where we write \( \vb{v} \mapsto \norm{\vb{v}} \), such that: 
    \begin{enumerate}[label=\textbf{\roman*)}]
        \item \( \norm{\vb{v}} = 0 \) if and only if \( \vb{v} = \vb{0} \). 
        \item \( \norm{ \lambda \vb{v}} = \abs{\lambda} \;  \norm{\vb{v}} \) for all \( \lambda \in \mathbb{K} \) and \( \vb{v} \in \mathbf{V} \). 
        \item \( \norm{\vb{v} + \vb{w}} \le \norm{\vb{v}}+ \norm{\vb{w}}\) for all \( \vb{v} , \vb{w} \in \mathbf{V} \).
    \end{enumerate}
    If \( \mathbf{V} \) has a norm defined on it, we call the pair \( \left( \mathbf{V}, \norm{ \cdot} \right) \) a \vocab{normed vector space}. 
\end{dfn}


\begin{theorem}
    If \( \mathbf{V} \) is an inner-product space, then it is also a normed vector space whose \vocab{induced norm} is defined to be:
    \[ \norm{\vb{v}} := \sqrt{\left< \vb{v},\vb{v} \right>} .\]
\end{theorem}
\begin{proof}
 We need to show that \( \norm{\cdot} = \sqrt{\left< \cdot , \cdot \right>} \) satisfies the properties of a norm. \\[6pt]
Suppose \( \norm{\vb{v}} = 0 \). Then, by definition,
\[
    \sqrt{\left< \vb{v}, \vb{v} \right>} = 0
    \quad \Rightarrow \quad
    \left< \vb{v}, \vb{v} \right> = 0
    \quad \Rightarrow \quad
    \vb{v} = \vb{0}.
\]
Thus, \( \norm{\vb{v}} = 0 \) if and only if \( \vb{v} = \vb{0} \). (The reverse direction is obvious.) \\ 
Next, for any \( \lambda \in \mathbb{K} \) and \( \vb{v} \in \mathbf{V} \), we have 
\begin{align*}
    \norm{\lambda \vb{v}} &= \sqrt{ \left< \lambda \vb{v}, \lambda \vb{v} \right>} \\
    &= \sqrt{ \lambda \overline{\lambda} \left< \vb{v},\vb{v} \right>} \\
    &= \sqrt{ \abs{\lambda}^{2} \left<  \vb{v},\vb{v} \right>} \\
    &= \sqrt{\abs{\lambda}^{2}} \sqrt{ \left<  \vb{v},\vb{v} \right>} \\
    &= \abs{\lambda} \sqrt{\left< \vb{v},\vb{v} \right>} \\
    &= \abs{\lambda} \; \norm{\vb{v}}
\end{align*}
Thus, \( \norm{\lambda \vb{v}} = \abs{\lambda} \;  \norm{\vb{v}} \) for all \( \lambda \in \mathbb{K} \) and \( \vb{v} \in \mathbf{V} \). \\ 
Finally, 
\begin{align*}
    \norm{\vb{v} + \vb{w}} &= \sqrt{ \left<  \vb{v} + \vb{w}, \vb{v} + \vb{w} \right>} \\ 
    &= \sqrt{ \left<  \vb{v},\vb{v} \right> + \left< \vb{v}, \vb{w} \right> + \left< \vb{w},\vb{v} \right> + \left< \vb{w},\vb{w} \right>} \\
    &=  \sqrt{ \left<  \vb{v},\vb{v} \right> + \left< \vb{v}, \vb{w} \right> + \overline{\left< \vb{v},\vb{w} \right>} + \left< \vb{w},\vb{w} \right>} \\
    &= \sqrt{ \left<  \vb{v},\vb{v} \right> + 2 \Re \left[ \left< \vb{v},\vb{w} \right> \right]+ \left< \vb{w},\vb{w} \right>} \\
    & \le \sqrt{ \left< \vb{v},\vb{v} \right> + 2 \abs{\left< \vb{v},\vb{w} \right>} + \left< \vb{w},\vb{w} \right>} \\
    & \le \sqrt{ \left< \vb{v},\vb{v} \right> + 2 \left< \vb{v},\vb{v} \right>^{\frac{1}{2}}\left< \vb{w},\vb{w} \right>^{\frac{1}{2}} + \left< \vb{w},\vb{w} \right>} \tag{By the \hyperref[thrm:Complex Cauchy-Schwarz]{Cauchy–Schwarz inequality}} \\
    &= \sqrt{\left( \left< \vb{v},\vb{v} \right>^{\frac{1}{2}} + \left< \vb{w},\vb{w} \right>^{\frac{1}{2}}\right)^{2}}\\
    &= \sqrt{\left<  \vb{v},\vb{v} \right>} + \sqrt{\left< \vb{w},\vb{w} \right>}
\end{align*}
This shows that \( \norm{\vb{v} + \vb{w}} \le \norm{\vb{v}} + \norm{\vb{w}} \). This completes the proof.
\end{proof}


\begin{exercise}
    Show that \( \left< \vb{v},\vb{w} \right> = 0 \) if and only if 
    \[ \norm{ \vb{v} + \lambda \vb{w}} = \norm{\vb{v} - \lambda \vb{w}} \quad \text{ for all } \lambda \in \mathbb{K}. \]
\end{exercise}
\begin{solution}
    \( \left( \Rightarrow \right) \) Suppose that \( \left< \vb{v},\vb{w} \right>  = 0\). It suffices to show that \(  \norm{ \vb{v} + \lambda \vb{w}}^{2} = \norm{\vb{v} - \lambda \vb{w}}^{2}  \) for all \( \lambda \in \mathbb{K} \). We calculate
    \begin{align*}
        \norm{\vb{v} + \lambda \vb{w}}^{2} &= \left< \vb{v}+ \lambda \vb{w}, \vb{v}+ \lambda \vb{w} \right> \\
        &= \left< \vb{v}, \vb{v} + \lambda \vb{w} \right> +  \lambda \left< \vb{w},\vb{v} + \lambda \vb{w} \right> \\
        &= \left< \vb{v},\vb{v} \right> + \overline{\lambda} \left< \vb{v},\vb{w} \right> + \lambda \left<  \vb{w},\vb{v} \right> + \abs{\lambda}^{2} \left< \vb{w},\vb{w} \right>
    \end{align*}
    So 
    \begin{equation}\label{eqn:v-plus-lamba-w}
        \boxed{\norm{\vb{v} + \lambda \vb{w}}^{2} = \norm{\vb{v}}^{2} + \overline{\lambda} \left< \vb{v},\vb{w} \right> + \lambda \overline{ \left< \vb{v},\vb{w} \right>} + \abs{\lambda}^{2} \norm{\vb{w}}^{2}}
    \end{equation} 
    Similarly, calculating \( \norm{\vb{v} - \lambda \vb{w}}^{2} \), we have 
    \begin{align*}
        \norm{\vb{v} - \lambda \vb{w}}^{2} &= \left< \vb{v} - \lambda \vb{w}, \vb{v}- \lambda \vb{w} \right> \\
        &= \left< \vb{v}, \vb{v} - \lambda \vb{w}\right> - \lambda \left< \vb{w}, \vb{v} - \lambda \vb{w} \right> \\
        &= \left< \vb{v}, \vb{v} \right> - \overline{\lambda} \left< \vb{v}, \vb{w} \right> - \lambda \left< \vb{w}, \vb{v} \right> + \abs{\lambda}^{2} \left< \vb{w}, \vb{w} \right>
    \end{align*}
    
    \begin{equation}\label{eqn:v-minus-lamba-w}
        \boxed{\norm{\vb{v} - \lambda \vb{w}}^{2} = \norm{\vb{v}}^{2} - \overline{\lambda} \left< \vb{v},\vb{w} \right> - \lambda \overline{ \left< \vb{v},\vb{w} \right>} + \abs{\lambda}^{2} \norm{\vb{w}}^{2}}
    \end{equation} 
    Subtracting \Cref{eqn:v-minus-lamba-w} from \Cref{eqn:v-plus-lamba-w}, we have 
    \begin{align*}
        \norm{\vb{v} + \lambda \vb{w}}^{2}- \norm{\vb{v} - \lambda \vb{w}}^{2} &= 2 \overline{\lambda} \left< \vb{v}, \vb{w} \right> + 2\lambda \overline{ \left< \vb{v}, \vb{w} \right>} \\
        &= 0 \tag{since $\left< \vb{v},\vb{w} \right>  = 0$}
    \end{align*}
    Therefore \( \norm{\vb{v} + \lambda \vb{w}}^{2} = \norm{\vb{v} - \lambda \vb{w}}^{2} \), which implies \(  \norm{\vb{v} + \lambda \vb{w}} = \norm{\vb{v} - \lambda \vb{w}}  \) for all \(  \lambda \in \mathbb{K} \).
    
    \medskip
    
    \noindent \( \left( \Leftarrow \right) \) Now suppose that \(  \norm{\vb{v} + \lambda \vb{w}} = \norm{\vb{v} - \lambda \vb{w}}  \) for all \( \lambda \in \mathbb{K} \). Then \( \norm{\vb{v} + \lambda \vb{w}}^{2} = \norm{\vb{v} - \lambda \vb{w}}^{2} \) for all \( \lambda \in \mathbb{K} \). 
    
    Using equations \Cref{eqn:v-plus-lamba-w} and \Cref{eqn:v-minus-lamba-w}, we have
    \begin{align*}
        0 &= \norm{\vb{v} + \lambda \vb{w}}^{2}- \norm{\vb{v} - \lambda \vb{w}}^{2} \\
        &= 2 \overline{\lambda} \left< \vb{v}, \vb{w} \right> + 2\lambda \overline{\left<  \vb{v}, \vb{w} \right>}
    \end{align*}
    for all \( \lambda \in \mathbb{K} \). 
    
    \textbf{Case 1:} If \( \mathbb{K} = \mathbb{R} \), then \( \overline{\lambda} = \lambda \) and \( \overline{\left< \vb{v}, \vb{w} \right>} = \left< \vb{v}, \vb{w} \right> \). Setting \( \lambda = 1 \), we get
    \[ 0 = 4\left< \vb{v}, \vb{w} \right>, \]
    so \( \left< \vb{v}, \vb{w} \right> = 0 \).
    
    \textbf{Case 2:} If \( \mathbb{K} = \mathbb{C} \), the equation \( 2\overline{\lambda}\left< \vb{v}, \vb{w} \right> + 2\lambda\overline{\left< \vb{v}, \vb{w} \right>} = 0 \) must hold for all \( \lambda \in \mathbb{C} \). 
    
    Setting \( \lambda = 1 \): 
    \[ 2\left< \vb{v}, \vb{w} \right> + 2\overline{\left< \vb{v}, \vb{w} \right>} = 0 \implies \Re\left(\left< \vb{v}, \vb{w} \right>\right) = 0. \]
    
    Setting \( \lambda = i \): 
    \[ 2(-i)\left< \vb{v}, \vb{w} \right> + 2(i)\overline{\left< \vb{v}, \vb{w} \right>} = 0 \implies -i\left< \vb{v}, \vb{w} \right> + i\overline{\left< \vb{v}, \vb{w} \right>} = 0 \]
    \[ \implies \overline{\left< \vb{v}, \vb{w} \right>} = \left< \vb{v}, \vb{w} \right> \implies \Im\left(\left< \vb{v}, \vb{w} \right>\right) = 0. \]
    
    Since both the real and imaginary parts of \( \left< \vb{v}, \vb{w} \right> \) are zero, we conclude \( \left< \vb{v}, \vb{w} \right> = 0 \).
\end{solution}


\begin{lemma}\label{2025-10-06 17:46:35}
    Suppose that \( \mathbf{V} \) is a real inner product space and that 
    \[ \norm{\vb{v} + \vb{w}}^{2}= \norm{\vb{v}}^{2} + \norm{\vb{w}}^{2} .\]
    Then \( \left< \vb{v},\vb{w} \right> = 0 \).
\end{lemma}
\begin{proof}
    If \(  \norm{\vb{v} + \vb{w}}^{2}= \norm{\vb{v}}^{2} + \norm{\vb{w}}^{2}  \) then \(  \norm{\vb{v} + \vb{w}}^{2}- \norm{\vb{v}}^{2} - \norm{\vb{w}}^{2} =0  \). So 
    \begin{align*}
        0 &= \left< \vb{v} + \vb{w}, \vb{v}+ \vb{w}\right> - \left< \vb{v},\vb{v} \right> - \left< \vb{w},\vb{w} \right> \\
        &=\left< \vb{v},\vb{v} \right> + 2 \left< \vb{v},\vb{w} \right> + \left<  \vb{w},\vb{w} \right>- \left<  \vb{v},\vb{v} \right> - \left<  \vb{w}, \vb{w} \right> \\
        0 &= 2 \left< \vb{v},\vb{w} \right>
    \end{align*}
    This proves the result. 
\end{proof}


\begin{exercise}
    Prove \cref{2025-10-06 17:46:35} using the contrapositive.
\end{exercise}
\begin{solution}
    Suppose that \( \left< \vb{v},\vb{w} \right> = \lambda \) for \( \lambda \neq 0 \). Then, by using a similar argument as above, we can show that
    \[ \left< \vb{v} + \vb{w}, \vb{v}+ \vb{w}\right> - \left< \vb{v},\vb{v} \right> - \left< \vb{w},\vb{w} \right> = 2\lambda \]
\end{solution}


\begin{exercise}
    Show that the result laid out in \cref{2025-10-06 17:46:35} is not necessarily true in a complex vector space.
\end{exercise}
\begin{solution}
  Take \( \mathbb{C} \) to be the vector space over itself and let \( \vb{v}= i \) and \( \vb{w} =1 \). 
  We have 
  \begin{align*}
    \norm{\vb{v}+ \vb{w}}^{2} - \norm{\vb{v}}^{2}- \norm{\vb{w}}^{2} &= \norm{1+i}^{2} - \norm{1}^{2} - \norm{i}^{2}\\
    &= 2 -1 -1 \\
    &=0
  \end{align*}
  But \[ \left< \vb{v},\vb{w} \right> = \left< i,1 \right> = i.\]
\end{solution}


\begin{exercise}\label{exc:equality-for-norm-squared}
    Suppose that \( \mathbf{V} \) is an inner product space over \( \mathbb{C} \). For every \( \vb{v},\vb{w} \in \mathbf{V} \). Show that 
    \[ \norm{\vb{v}   + \vb{w}}^{2} = \norm{\vb{v}}^{2} + \norm{\vb{w}}^{2} + 2 \Re \left( \left< \vb{v},\vb{w} \right> \right) .\]
\end{exercise}
\begin{solution}
    We can take \Cref{eqn:v-plus-lamba-w} and let \( \lambda =1 \) to get 
    \begin{align*}
        \norm{\vb{v} + \vb{w}}^{2} &= \norm{\vb{v}}^{2} + \norm{\vb{w}}^{2} + \left< \vb{v},\vb{w} \right> + \overline{\left< \vb{v},\vb{w} \right>} \\
        &= \norm{\vb{v}}^{2} + \norm{\vb{w}}^{2} + 2 \Re \left( \left< \vb{v},\vb{w} \right> \right)
    \end{align*}
\end{solution}

\begin{exercise}
  Suppose that \( \vb{u}, \vb{v} \in \mathbf{V} \) a complex inner-product space, such that \( \norm{\vb{u}}^{2} + \norm{\vb{v}}^{2} = \norm{\vb{u} + \vb{v}}^{2} \). Is it necessarily true that \( \left< \vb{u}, \vb{v} \right> = 0 \)? If not, provide a counter-example
\end{exercise}
\begin{solution}
    \Cref{exc:equality-for-norm-squared} suggests that it is not true. Let \( \mathbf{V} = \mathbb{C} \) and let \( \vb{u} = 1+i \) and \( \vb{v} = 1-i \). On one hand, we have 
    \begin{align*}
        \norm{\vb{u} + \vb{v}}^{2} -\norm{\vb{u}}^{2} - \norm{\vb{v}}^{2} &= \norm{ 1+ i + 1-i}^{2} - \norm{ 1+i}^{2} - \norm{1-i}^{2} \\
        &= 4 - 2 -2 \\
        &= 0
    \end{align*}
    So  \( \norm{\vb{u}}^{2} + \norm{\vb{v}}^{2} = \norm{\vb{u} + \vb{v}}^{2} \) But on the other hand, 
    \begin{align*}
        \left< \vb{u}, \vb{v} \right> &= \left( 1+i \right) \cdot \overline{1-i} \\
        &= (1+i) \cdot (1+i) \\
        &= 2i
    \end{align*}
    So the claim is false.
\end{solution}


\begin{exercise}
    Let \( \vb{u}, \vb{v}, \vb{w} \) be vectors in an inner-product space \( \mathbf{V} \). Show that \( \norm{\vb{u} - \vb{w}} = \norm{\vb{u} -\vb{v}} + \norm{\vb{v} - \vb{w}} \) if and only if there exists a real number \( t \in \left[ 0,1 \right] \) such that \( \vb{v} = t \vb{u} + \left( 1-t  \right)\vb{w} \).
\end{exercise}
\begin{solution}
    \( (\Leftarrow) \) We will start with the reverse direction because it's easier. 
    Suppose there exists a \( t \in [0,1] \) such that 
    \[
        \vb{v} = t \vb{u} + (1-t)\vb{w}.
    \]
    If \( t = 0 \) or \( t = 1 \), then \( \vb{v} = \vb{w} \) or \( \vb{v} = \vb{u} \) respectively, and the result is trivial. 
    So assume \( t \in (0,1) \). Then 
    \begin{align*}
        \norm{\vb{u} - \vb{v}} + \norm{\vb{v} - \vb{w}} 
        &= \norm{\vb{u} - \left( t \vb{u} + (1-t)\vb{w} \right)} 
        + \norm{t \vb{u} + (1-t)\vb{w} - \vb{w}} \\
        &= \norm{(1-t)(\vb{u}-\vb{w})} + \norm{t(\vb{u}-\vb{w})} \\
        &= (1-t)\norm{\vb{u}-\vb{w}} + t\norm{\vb{u}-\vb{w}}
        \quad\text{\footnotesize (since $t\in(0,1)$, it can be taken out of the norm without absolute value)}\\
        &= \norm{\vb{u}-\vb{w}},
    \end{align*}
    which is what we wanted. \\[6pt]

    \( (\Rightarrow) \) Suppose that  
    \[
        \norm{\vb{u} - \vb{w}} = \norm{\vb{u} - \vb{v}} + \norm{\vb{v} - \vb{w}}.
    \]
    Set \( \vb{x} = \vb{u} - \vb{v} \) and \( \vb{y} = \vb{v} - \vb{w} \). Then we know that 
    \[
        \norm{\vb{x} + \vb{y}} \le \norm{\vb{x}} + \norm{\vb{y}},
    \]
    with equality if and only if \( \vb{x} = \lambda \vb{y} \) for some \( \lambda \ge 0 \).  
    Since we are assuming equality, it follows that 
    \[
        \vb{u} - \vb{v} = \lambda(\vb{v} - \vb{w}),
    \]
    With some algebraic manipulation, we can see that 
    \[ \vb{v} = \frac{1}{1+ \lambda} \vb{u}  + \frac{\lambda }{1+ \lambda} \vb{w},\]
    we can set \( t = \frac{1}{1+\lambda} \) so \( (1-t) = \frac{\lambda}{ 1 + \lambda} \) and we are done.
\end{solution}



\begin{theorem}[The Polarization Identity]\label{thm:polarization-identiy}
    If \( \mathbf{V} \) is a complex inner product space, then 
    \[\left< \vb{u}, \vb{v} \right> = \frac{1}{4} \left( \norm{\vb{u} + \vb{v}}^{2} - \norm{\vb{u} - \vb{v}}^{2} + i \norm{\vb{u} + i \vb{v}}^{2} - i \norm{\vb{u} - i \vb{v}}^{2} \right) \text{ .}\]
\end{theorem}
\begin{proof}
    We just calculate the right-hand side. 
\begin{align*}
    \norm{\vb{u} + \vb{v}}^{2} - \norm{\vb{u} - \vb{v}}^{2} + i \norm{\vb{u} + i \vb{v}}^{2} - i \norm{\vb{u} - i \vb{v}}^{2}  &=  \left< \vb{u} + \vb{v}, \vb{u} + \vb{v} \right> - \left< \vb{u}- \vb{v}, \vb{u}- \vb{v} \right> + i \left< \vb{u} + i\vb{v}, \vb{u} + i\vb{v} \right> - i \left<  \vb{u} - i\vb{v}, \vb{u} - i \vb{v} \right> \\ \\
    &= \left< \vb{u}, \vb{u} + \vb{v} \right> + \left< \vb{v}, \vb{u} + \vb{v} \right> - \left< \vb{u}, \vb{u} - \vb{v} \right> + \left< \vb{v}, \vb{u} - \vb{v} \right> \\
    &+ i \left< \vb{u}, \vb{u} + i\vb{v} \right> + i\left< \vb{v}, \vb{u} + i\vb{v} \right> -i \left< \vb{u}, \vb{u}- i\vb{v} \right> - i\left< \vb{v}, \vb{u} - i\vb{v} \right> \\ \\
    &= \left<  \vb{u}, \vb{u} \right> + \left<  \vb{u}, \vb{v} \right> + \left< \vb{v}, \vb{u} \right> + \left<  \vb{v},\vb{v} \right> - \left< \vb{u}, \vb{u} \right> + \left< \vb{u}, \vb{v} \right> + \left< \vb{v}, \vb{u} \right> - \left< \vb{v},\vb{v} \right> \\
    &+ i \left< \vb{u}, \vb{u} \right> + \left< \vb{u}, \vb{v} \right> + i\left< \vb{v}, \vb{u} \right> + i^2 \left< \vb{v},\vb{v} \right> - i \left< \vb{u}, \vb{u} \right> + \left< \vb{u},\vb{v} \right> + i\left< \vb{v}, \vb{u} \right> - i^2 \left< \vb{v},\vb{v} \right> \\ \\
    &= \Ccancel[DeepRed]{ \left<  \vb{u}, \vb{u} \right> } + \left<  \vb{u}, \vb{v} \right> + \Ccancel[AzureBlue]{\left< \vb{v}, \vb{u} \right> } + \Ccancel[AmberOrange]{\left<  \vb{v},\vb{v} \right>}   -\Ccancel[DeepRed]{ \left<  \vb{u}, \vb{u} \right> } + \left< \vb{u}, \vb{v} \right> + \Ccancel[RoyalPurple]{\left< \vb{v}, \vb{u} \right>} -\Ccancel[AmberOrange]{\left<  \vb{v},\vb{v} \right>} \\
    &+ \Ccancel[GoldenYellow]{ i  \left< \vb{u}, \vb{u} \right> } + \left< \vb{u}, \vb{v} \right> + \Ccancel[AzureBlue]{i\left< \vb{v}, \vb{u} \right>} +  \Ccancel[Emerald]{ i^2 \left< \vb{v},\vb{v} \right>}  - \Ccancel[GoldenYellow]{ i  \left< \vb{u}, \vb{u} \right> }  + \left< \vb{u},\vb{v} \right> + \Ccancel[RoyalPurple]{i\left< \vb{v}, \vb{u} \right>} - \Ccancel[Emerald]{ i^2 \left< \vb{v},\vb{v} \right>} \\
    &= 4 \left<  \vb{u}, \vb{v} \right>
\end{align*}
    which is what we wanted to show. 
\end{proof}

 \textbf{Note:} An easy way to write (or remember) the polarization identity is \[ \boxed{\left< \vb{u}, \vb{v} \right> = \frac{1}{4} \left( \sum_{k=0}^{3}  i^{k} \norm{\vb{u} + i^{k} \vb{v}}^{2}\right)   \text{ .}} \] 

\begin{lemma}
    Suppose that \( \mathbf{V} \) and \( \mathbf{W} \) are complex inner product spaces and \( T \in \mathcal{L} \left( \mathbf{V}, \mathbf{W} \right) \). Then \( T \) preserves the inner product if and only if it preserves the induced norm. That is,
    \[ \left< \vb{x}, \vb{y} \right>_{\mathbf{V}} = \left< T \left( \vb{x} \right), T \left( \vb{y} \right)\right>_{\mathbf{W}} \text{ for all } \vb{x}, \vb{y} \in \mathbf{V}\]
    if and only if
    \[ \norm{\vb{x}}_{\mathbf{V}} = \norm{T \left( \vb{x} \right)}_{\mathbf{W}} \text{ for all } \vb{x} \in \mathbf{V}.\]
\end{lemma}
\begin{proof}
     \( \left( \Rightarrow \right) \) Suppose that \( T \) preserves the inner product. Then 
     \begin{align*}
        \norm{\vb{x}}_{\mathbf{V}} &= \sqrt{ \left< \vb{x}, \vb{x} \right>_{\mathbf{V}}} \\
        &= \sqrt{\left< T \left( \vb{x} \right), T \left( \vb{x} \right)\right>_{\mathbf{W}}} \\
        &= \norm{T \left( \vb{x} \right)}_{\mathbf{W}}
     \end{align*}
     So \( T \) also preserves norms. \\ 
     \( \left( \Leftarrow \right) \) Conversely if \( T \) preserves norms, then by the \hyperref[thm:polarization-identiy]{poloarization identity} 
     \begin{align*}
        \left< \vb{x}, \vb{y} \right>_{\mathbf{V}} &= \frac{1}{4} \left( \sum_{k=0}^{3} i^{k} \norm{\vb{x} + i^{k} \vb{y}}_{\mathbf{V}}^{2} \right) \\
        &= \frac{1}{4} \left( \sum_{k=0}^{3} i^{k} \norm{T \left( \vb{x} + i^{k} \vb{y} \right)}_{\mathbf{W}}^{2} \right) \\
        &=  \frac{1}{4} \left( \sum_{k=0}^{3} i^{k} \norm{T(\vb{x}) + i^{k} T(\vb{y})}_{\mathbf{W}}^{2} \right) \\
        &= \left< \vb{x}, \vb{y} \right>_{\mathbf{W}}
     \end{align*}
     So \( T \) also preserves the inner product.
 \end{proof}
 

\begin{exercise}\label{exc:re-and-im-polarization}
    Suppose that \( \mathbf{V} \) is a normed complex vector space and define the map \( p: \mathbf{V} \times \mathbf{V} \to \mathbb{C} \) given by 
    \[ p \left( \vb{u}, \vb{v} \right) = \frac{1}{4} \left( \sum_{k=0}^{3}  i^{k} \norm{\vb{u} + i^{k} \vb{v}}^{2}\right) \text{ .}\] Show that
    \[ \Im \left[ p \left(  \vb{u}, \vb{v} \right)  \right]= \Re \left[ p ( \vb{u}, i \vb{v}) \right]\]
\end{exercise}
\begin{solution}
    For the left-hand side, we have 
    \begin{align*}
        \Im \left[ p \left( \vb{u},\vb{v} \right) \right] &= \Im \left[ \frac{1}{4} \left( \norm{\vb{u} + \vb{v}}^{2} - \norm{\vb{u} - \vb{v}}^{2} + i \norm{\vb{u} + i \vb{v}}^{2} - i \norm{\vb{u} - i \vb{v}}^{2} \right)  \right] \\
        &= \frac{1}{4} \left( \norm{\vb{u} + i \vb{v}}^{2} + \norm{\vb{u} - i\vb{v}}^{2} \right)
    \end{align*}
For the right-hand side, we have 
    \begin{align*}
        \Re \left[ p \left( \vb{u}, i\vb{v} \right) \right] &= \Re \left[  \frac{1}{4} \left( \norm{\vb{u} + i\vb{v}}^{2} - \norm{\vb{u} - i\vb{v}}^{2} + i \norm{\vb{u} - \vb{v}}^{2} - i \norm{\vb{u} + \vb{v}}^{2} \right)\right]\\
         &= \frac{1}{4} \left( \norm{\vb{u} + i \vb{v}}^{2} + \norm{\vb{u} - i\vb{v}}^{2} \right)
    \end{align*}
    
\end{solution}

\begin{exercise}\label{exc:polarization-zero}
    Let the conditions of \cref{exc:re-and-im-polarization} hold. Then 
    \[ p \left( \vb{0}, \vb{v} \right) = p \left( \vb{v},\vb{0} \right) = 0 \quad \text{for all } \vb{v} \in \mathbf{V}. \]
\end{exercise}
\begin{solution}
    We have 
    \begin{align*}
        p \left( \vb{0},\vb{v} \right) &= \frac{1}{4}\left( \norm{\vb{0} + \vb{v}}^{2}  -\norm{\vb{0} - \vb{v}}^{2} + i \norm{\vb{0} + i\vb{v}}^{2} - i \norm{\vb{0} -i \vb{v}}^{2}\right) \\
        &= \frac{1}{4} \left( \norm{\vb{v}}^{2} - \norm{- \vb{v}}^{2} + i \norm{i\vb{v}}^{2} - i \norm{-i \vb{v}}^{2} \right) \\
        &= \frac{1}{4} \left(  \norm{\vb{v}}^{2} - \norm{\vb{v}}^{2} + i\norm{\vb{v}}^{2} - i \norm{\vb{v}}^{2} \right)\\
        &= 0
    \end{align*}
    Showing \( p \left( \vb{v}, \vb{0} \right) =0 \) is similar. 
\end{solution}



\begin{lemma}[The Parallelogram Law]
     Suppose that \( \mathbf{V} \) is an inner product space. Then for all \( \vb{u},\vb{v} \in \mathbf{V} \), we have 
    \[ \norm{\vb{u} + \vb{v}}^{2} + \norm{\vb{u} -\vb{v}}^{2} = 2 \left( \norm{\vb{u}}^{2} + \norm{\vb{v}}^{2} \right) \text{ .}\]
\end{lemma}
\begin{proof}
    This is a straight forward calculation. 
    \begin{align*}
        \norm{\vb{u} + \vb{v}}^{2} + \norm{\vb{u} -\vb{v}}^{2} &= \left< \vb{u} + \vb{v}, \vb{u} + \vb{v} \right> + \left< \vb{u} -\vb{v}, \vb{u} -\vb{v} \right> \\
        &= \left< \vb{u}, \vb{u} + \vb{v} \right> + \left<  \vb{v},\vb{u} + \vb{v} \right> + \left< \vb{u}, \vb{u}- \vb{v} \right> - \left< \vb{v}, \vb{u}- \vb{v} \right> \\
        &= \left< \vb{u}, \vb{u} \right> + \left< \vb{u},\vb{v} \right> + \left< \vb{v}, \vb{u} \right> + \left<  \vb{v},\vb{v} \right> + \left<  \vb{u}, \vb{u} \right> - \left< \vb{u}, \vb{v} \right> - \left<  \vb{v}, \vb{u} \right> + \left<  \vb{v},\vb{v} \right>\\
        &= \left< \vb{u}, \vb{u} \right> + \Ccancel[DeepRed]{ \left< \vb{u},\vb{v} \right>} + \Ccancel[AmberOrange]{ \left< \vb{v}, \vb{u} \right>}  + \left<  \vb{v},\vb{v} \right> + \left<  \vb{u}, \vb{u} \right> - \Ccancel[DeepRed]{ \left< \vb{u},\vb{v} \right>} - \Ccancel[AmberOrange]{ \left< \vb{v}, \vb{u} \right>} + \left<  \vb{v},\vb{v} \right> \\
        &= 2 \left< \vb{u} ,\vb{u} \right> + 2 \left<  \vb{v},\vb{v} \right>
    \end{align*}
   So 
   \[ \boxed{ \norm{\vb{u} + \vb{v}}^{2} + \norm{\vb{u} -\vb{v}}^{2} = 2 \left( \norm{\vb{u}}^{2} + \norm{\vb{v}}^{2} \right) \text{ .}} \] 
\end{proof}


\begin{theorem}
    If \( \mathbf{V} \) is a normed vector space with norm \( \norm{ \cdot} \) which satisfies the parallelogram law for every \( \vb{u}, \vb{v} \in \mathbf{V} \), then there exists an inner product \( \left< \cdot, \cdot \right> \) whose induced norm is exactly \( \norm{ \cdot} \).
\end{theorem}
\begin{proof}
We will use the \hyperref[thm:polarization-identiy]{polarization identity} to define our candidate inner product. 
\[ \left< \vb{u}, \vb{v} \right> = \frac{1}{4} \left( \sum_{k =0}^{3} i^{k} \norm{\vb{u} + i^{k} \vb{v}}^{2} \right) \text{ .}\]
We need to show that the conditions laid out in \cref{def:inner-product} are satisfied. \\ 
\begin{description}
    \item[\textbf{Conjugate symmetry:}] $ $\\ We need to show that \( \Re{\left( \left< \vb{u},\vb{v} \right> \right)} =\Re{\left( \left< \vb{v},\vb{u} \right> \right)}  \) and  \( \Im{\left( \left< \vb{u},\vb{v} \right> \right)} = -\Im{\left( \left< \vb{v},\vb{u} \right> \right)}  \). In that regard, we have 
    \begin{align*}
        \Re{\left( \left< \vb{u},\vb{v} \right> \right)} &= \Re \left[  \frac{1}{4} \left(\norm{\vb{u} + \vb{v}}^{2} - \norm{\vb{u} - \vb{v}}^{2} + i \norm{\vb{u} + i \vb{v}}^{2} - i \norm{\vb{u} - i \vb{v}}^{2}\right) \right] \\
        &= \frac{1}{4} \left(   \norm{\vb{u} + \vb{v}}^{2} - \norm{\vb{u} - \vb{v}}^{2}  \right) \\
        &=\frac{1}{4} \left(   \norm{\vb{v} + \vb{u}}^{2} - \norm{-1 \left( \vb{v}- \vb{u} \right)}^{2}  \right) \\
        &= \frac{1}{4} \left(   \norm{\vb{v} + \vb{u}}^{2} - \abs{-1}^{2}\; \norm{\vb{v}- \vb{u}}^{2}\right)\\
         &= \frac{1}{4} \left(   \norm{\vb{v} + \vb{u}}^{2} -  \norm{\vb{v}- \vb{u}}^{2}\right)\\
         &= \Re \left[ \frac{1}{4}  \left(   \norm{\vb{v} + \vb{u}}^{2} -  \norm{\vb{v}- \vb{u}}^{2}+ i \norm{\vb{v} + i \vb{u}}^{2} - i \norm{\vb{v}- i\vb{u}}^{2} \right) \right] \\
         &= \Re \left( \left< \vb{v},\vb{u} \right> \right) 
    \end{align*}
Similarly, we have 
\begin{align*}
    \Im \left( \left<  \vb{u}, \vb{v} \right> \right) 
    &= \Im \left[  \frac{1}{4} \left( \norm{\vb{u} + \vb{v}}^{2} - \norm{\vb{u} - \vb{v}}^{2} + i \norm{\vb{u} + i \vb{v}}^{2} - i \norm{\vb{u} - i \vb{v}}^{2} \right)\right] \\
    &= \frac{1}{4} \Im \left( i \norm{\vb{u} + i\vb{v}}^{2} - i \norm{\vb{u} - i \vb{v}}^{2} \right) \\
    &= \frac{1}{4} \Re \left( \norm{\vb{u} + i\vb{v}}^{2} - \norm{\vb{u} - i \vb{v}}^{2} \right) \\
    &= -\frac{1}{4} \Re \left( \norm{\vb{v} + i\vb{u}}^{2} - \norm{\vb{v} - i\vb{u}}^{2} \right) \\
    &= -\Im \left( \left< \vb{v}, \vb{u} \right> \right).
\end{align*}
    So this shows that \( \left< \vb{u},\vb{v} \right> = \overline{\left< \vb{v}, \vb{u} \right>} \). This incidentally shows that \( \left< \vb{v},\vb{v} \right> \in \mathbb{R} \) for all \( \vb{v} \in \mathbf{V} \).\
    \item[\textbf{Non-negative and Definite:}] $ $ \\Suppose that \( \vb{v} \in \mathbf{V} \). Then 
    \begin{align*}
        \left< \vb{v},\vb{v} \right> &= \frac{1}{4} \left( \norm{\vb{v} + \vb{v}}^{2} + \norm{\vb{v} - \vb{v}}^{2} + i \norm{\vb{v} + i\vb{v}} ^{2} - i \norm{\vb{v} - i \vb{v}} ^{2}\right) \\
        &= \frac{1}{4} \left( \norm{2 \vb{v}}^{2} + i \norm{\left( 1+i  \right) \vb{v}}^{2} - i \norm{\left( 1-i \right) \vb{v}}^{2} \right) \\
        &= \frac{1}{4} \left( 4 \norm{\vb{v}}^{2} + 2i \norm{\vb{v}}^{2} - 2i \norm{ \vb{v}}^{2} \right) \\
        &= \norm{\vb{v}}^{2} 
    \end{align*}
  This tells us :
  \begin{enumerate}[label=\textbf{\roman*)}]
    \item \( \left< \vb{v},\vb{v} \right> \ge 0 \) since \( \norm{\vb{v}}^{2} \ge 0 \).
    \item \( \left< \vb{v},\vb{v} \right> =0\) if and only if \( \vb{v} = \vb{0} \) since \( \norm{\vb{v}}^{2} = 0 \) if and only if \( \vb{v}= \vb{0} \).
    \item That once we show that \( \left< \cdot, \cdot \right> \) is an inner product that \( \norm{ \cdot} \) is the induced norm.
  \end{enumerate}
  \item[\textbf{Additivity:}]$ $\\ We want to show $\langle \vb{u}+ \vb{v}, \vb{w} \rangle = \langle \vb{u}, \vb{w} \rangle + \langle \vb{v}, \vb{w} \rangle$ for all $\vb{u}, \vb{v}, \vb{w} \in \mathbf{V}$. The parallelogram law doesn't immediately suggest how to prove this. However, we observe that $\vb{u}$ and $\vb{v}$ play symmetric roles in the expression $\langle \vb{u}+ \vb{v}, \vb{w} \rangle$. As such, any identity we derive should treat them symmetrically. This guides us to apply the parallelogram law to expressions involving $\vb{u} + \vb{w}$ and $\vb{v} + \vb{w}$, which we'll then combine to obtain additivity.\\ 
  With that in mind, we apply the parallelogram law to obtain the following two equations 
  \begin{equation}\label{eqn:plus-w-over-two}
    \norm{\left( \vb{u} + \frac{\vb{w}}{2} \right)+ \left( \vb{v} + \frac{\vb{w}}{2} \right)}^{2} + \norm{\left( \vb{u} + \frac{\vb{w}}{2} \right)- \left( \vb{v} + \frac{\vb{w}}{2} \right)}^{2} = 2 \norm{\vb{u} + \frac{\vb{w}}{2}}^2 + 2 \norm{\vb{v} + \frac{\vb{w}}{2}}^2
  \end{equation}
    \begin{equation}\label{eqn:minus-w-over-two}
    \norm{\left( \vb{u} - \frac{\vb{w}}{2} \right)+ \left( \vb{v} - \frac{\vb{w}}{2} \right)}^{2} + \norm{\left( \vb{u} + \frac{\vb{w}}{2} \right)- \left( \vb{v} + \frac{\vb{w}}{2} \right)}^{2} = 2 \norm{\vb{u} - \frac{\vb{w}}{2}}^2 + 2 \norm{\vb{v} - \frac{\vb{w}}{2}}^2
  \end{equation}
  Subtracting \cref{eqn:minus-w-over-two} from \cref{eqn:plus-w-over-two}, we have 
  \begin{align*}
     \norm{\left( \vb{u} + \frac{\vb{w}}{2} \right)+ \left( \vb{v} + \frac{\vb{w}}{2} \right)}^{2} -  \norm{\left( \vb{u} - \frac{\vb{w}}{2} \right)+ \left( \vb{v} - \frac{\vb{w}}{2} \right)}^{2} &= 2 \left( \norm{\vb{u} + \frac{\vb{w}}{2}}^2 -  \norm{\vb{u} - \frac{\vb{w}}{2}}^2 \right) + 2 \left( \norm{\vb{v} + \frac{\vb{w}}{2}}^2 -  \norm{\vb{v} - \frac{\vb{w}}{2}}^2 \right) \\
     \norm{\vb{u} + \vb{v} + \vb{w}}^{2} - \norm{\vb{u} + \vb{v} -\vb{w}}^{2} &=  2 \left( \norm{\vb{u} + \frac{\vb{w}}{2}}^2 -  \norm{\vb{u} - \frac{\vb{w}}{2}}^2 \right) + 2 \left( \norm{\vb{v} + \frac{\vb{w}}{2}}^2 -  \norm{\vb{v} - \frac{\vb{w}}{2}}^2 \right) 
  \end{align*}
  \begin{equation}\label{eqn:real-part-of-inner}
      \Re \left[ \left< \vb{u}+ \vb{v}, \vb{w} \right> \right] = 2 \Re \left[ \left< \vb{u}, \frac{\vb{w}}{2} \right> \right] + 2 \Re \left[ \left< \vb{v},\frac{\vb{w}}{2} \right> \right]
  \end{equation}
  Applying the result of \cref{exc:polarization-zero} to \cref{eqn:real-part-of-inner} by setting \( \vb{u} = \vb{0} \), we have 
  \begin{align*}
    \Re \left[ \left< \vb{0}+ \vb{v}, \vb{w} \right> \right] &= 2 \Re \left[ \left< \vb{0}, \frac{\vb{w}}{2} \right> \right] + 2 \Re \left[ \left<  \vb{v},\frac{\vb{w}}{2} \right> \right] 
  \end{align*}
  Similarly setting \( \vb{v} = \vb{0} \), we have the two equations
  \begin{equation}\label{2025-10-05 10:14:44}
    \colorboxed{DeepRed}{ \Re \left[ \left< \vb{v}, \vb{w} \right> \right] = 2 \Re \left[ \left< \vb{v}, \frac{\vb{w}}{2} \right> \right] } \quad \text{and} \quad  \colorboxed{AmberOrange}{ \Re \left[ \left< \vb{v}, \vb{w} \right> \right] = 2 \Re \left[ \left< \vb{v}, \frac{\vb{w}}{2} \right> \right] }
  \end{equation}
  Applying \cref{2025-10-05 10:14:44} to \cref{eqn:real-part-of-inner}, we have 
  \begin{align*}
     \Re \left[ \left< \vb{u}+ \vb{v}, \vb{w} \right> \right] &=  \colorboxed{DeepRed}{ 2 \Re \left[ \left< \vb{v}, \frac{\vb{w}}{2} \right> \right] } +  \colorboxed{AmberOrange}{ 2 \Re \left[ \left< \vb{v}, \frac{\vb{w}}{2} \right> \right] }\\
     &=     \colorboxed{DeepRed}{ \Re \left[ \left< \vb{v}, \vb{w} \right> \right]  }  +\colorboxed{AmberOrange}{ \Re \left[ \left< \vb{v}, \vb{w} \right> \right]}
  \end{align*}
  This shows that \( \Re \left[\left< \vb{u} + \vb{v}, \vb{w} \right> \right] \) is additive. Now we need to show that \( \Im \left[  \left< \vb{u} + \vb{v}, \vb{w} \right> \right] \) is additive. 
  \begin{align*}
    \Im \left[  \left< \vb{u} + \vb{v}, \vb{w} \right> \right] &= \Re \left[\left< \vb{u} + \vb{v}, i\vb{w} \right> \right] \tag*{By \cref{exc:re-and-im-polarization}} \\
    &= \Re \left[ \left< \vb{u}, i \vb{w} \right> \right] +  \Re \left[ \left< \vb{v}, i \vb{w} \right> \right] \\
    &=   \Im \left[  \left< \vb{u}, \vb{w} \right> \right]  +   \Im \left[  \left<  \vb{v}, \vb{w} \right> \right] 
  \end{align*}
  Finally putting it all together 
  \begin{align*}
    \left< \vb{u} + \vb{v}, \vb{w} \right> &=  \Re \left[  \left< \vb{u} + \vb{v}, \vb{w} \right> \right] +  \Im \left[  \left< \vb{u} + \vb{v}, \vb{w} \right> \right] i \\
    &= \left(  \Re \left[  \left< \vb{u} , \vb{w} \right> \right] + \Re \left[  \left< \vb{v}, \vb{w} \right> \right]\right) + \left(  \Im \left[  \left< \vb{u}, \vb{w} \right> \right] +  \Im \left[  \left<  \vb{v}, \vb{w} \right> \right] \right)i \\
    &= \left(  \Re \left[  \left< \vb{u} , \vb{w} \right> \right]  +  \Im \left[  \left< \vb{u} , \vb{w} \right> \right] i\right) +  \left(  \Re \left[  \left< \vb{v} , \vb{w} \right> \right]  +  \Im \left[  \left< \vb{v} , \vb{w} \right> \right] i\right)\\
    &= \left< \vb{u}, \vb{w} \right> + \left<  \vb{v},\vb{w} \right>
  \end{align*}
  With that, we have shown additivity. 
\item[\textbf{Homogeneity:}] $ $\\ We want to show that for each $\lambda \in \mathbb{C}$ and $\vb{u}, \vb{v} \in \mathbf{V}$, we have 
\[ \langle \lambda \vb{u}, \vb{v} \rangle = \lambda \langle \vb{u}, \vb{v} \rangle. \]
The proof proceeds by gradually extending the class of scalars for which homogeneity holds. We first establish the result for $\lambda \in \mathbb{N}$, then extend successively to $\lambda \in \mathbb{Z}$, then $\lambda \in \mathbb{Q}$, then $\lambda \in \mathbb{R}$, and finally to $\lambda \in \mathbb{C}$.
    \begin{description}
    \item[\( \bullet \) For \( \lambda \in \mathbb{N} \):]$ $\\
    We will prove this by induction while \( \lambda =1 \) is a perfectly acceptable base case. However, it is trivial and won't provide us with any insight on how to tackle the inductive step. So let us set the base case \(  \lambda =2 \). \\ 
    \textbf{Base case:} For \( \lambda =2 \), we have 
        \begin{align*}
            \left< 2 \vb{u}, \vb{v} \right> &= \left< \vb{u} + \vb{u}, \vb{v} \right> \\
            &= \left< \vb{u}, \vb{v} \right> + \left< \vb{u}, \vb{v} \right> \tag*{Since we have established additivity.} \\
            &= 2 \left< \vb{u}, \vb{v} \right>
        \end{align*}
    \textbf{Inductive step:}  Suppose that \( \left< \lambda \vb{u}, \vb{v} \right> = \lambda \left< \vb{u}, \vb{v} \right>\). Then 
        \begin{align*}
            \left< (\lambda+1) \vb{u}, \vb{v} \right> &= \left< \lambda \vb{u} + \vb{u}, \vb{v} \right> \\
            &=\left<  \lambda \vb{u} , \vb{v} \right> + \left< \vb{u}, \vb{v} \right> \\
            &= \lambda \left<  \vb{u}, \vb{v} \right> + \left<  \vb{u}, \vb{v} \right> \tag*{By the inductive hypothesis} \\
            &= \left( \lambda+1 \right) \left< \vb{u},\vb{v} \right>
        \end{align*}
        So \( \langle \lambda \vb{u}, \vb{v} \rangle = \lambda \langle \vb{u}, \vb{v} \rangle \) for every \( \lambda \in \mathbb{N} \).
    \item[$\bullet$ For \( \lambda \in \mathbb{Z} \):] $ $\\ 
    We just need to show the result for negative integers here. In other words, if \( \lambda > 0 \), then we need to show 
    \[ \left<  - \lambda \vb{u}, \vb{v} \right> = - \lambda \left< \vb{u}, \vb{v} \right> .\]
    \begin{align*}
        0 &= \left< 0 \vb{u}, \vb{v} \right>\\
        &=\left< \left( \lambda -\lambda \right) \vb{u},\vb{v} \right>\\
        &= \left< \lambda \vb{u} + \left( -\lambda \right)\vb{u}, \vb{v} \right> \\
        &= \left< \lambda \vb{u}, \vb{v} \right> +  \left< - \lambda \vb{u}, \vb{v} \right> \\
       0 &= \lambda \left<  \vb{u}, \vb{v} \right> + \left< - \lambda \vb{u}, \vb{v} \right>
    \end{align*}
    This shows that \(  \left<  - \lambda \vb{u}, \vb{v} \right> = - \lambda \left< \vb{u}, \vb{v} \right> \).
    \item[$\bullet$ For \( \lambda \in \mathbb{Q} \):] $ $\\ 
    Suppose that \( \lambda = \frac{p }{q} \) for \( p, q \in \mathbb{Z} \) and \(  q> 0 \). Then 
    \begin{align*}
        p\left< \vb{u}, \vb{v}  \right> &= \left< p \vb{u}, \vb{v} \right> \\
        &= \left< q \left( \frac{p}{q} \vb{u} \right) , \vb{v} \right> \\
        p \left<  \vb{u}, \vb{v} \right> &= q \left< \frac{p }{q} \vb{u},\vb{v} \right>
    \end{align*}
    Dividing both sides by \( q \) yields the desired result. 
    \item[$\bullet$ For \( \lambda \in \mathbb{R} \): ] This is the trickiest one to show. First, we note that as $|\alpha| \to 0$, we have $\|\alpha \vb{u}\| \to 0$. Indeed, for any $\epsilon > 0$, if $|\alpha| < \frac{\epsilon}{\|\vb{u}\|}$, then $\|\alpha \vb{u}\| = |\alpha| \|\vb{u}\| < \epsilon$. (The case $\vb{u} = \vb{0}$ is trivial.) In particular, as $|\alpha| \to 0$, we have $\|\alpha \vb{u} + \beta \vb{v}\| \to |\beta|\|\vb{v}\|$. To see this, note that by the reverse triangle inequality:
\begin{align*}
\big| \;  \|\alpha \vb{u} + \beta \vb{v}\| - |\beta|\|\vb{v}\| \; \big| 
&=\big| \; \|\alpha \vb{u} + \beta \vb{v}\| - \|\beta \vb{v}\| \; \big| \\
& \le \norm{\alpha \vb{u} + \beta \vb{v} - \beta \vb{v}} \\
&\leq \|\alpha \vb{u}\| \\
&= |\alpha| \|\vb{u}\| \to 0.
\end{align*}
With that in mind, we can now prove that for \( \lambda \in \mathbb{R} \), we have \( \left< \lambda \vb{u}, \vb{v} \right> = \lambda \left<  \vb{u}, \vb{v} \right> \). First pick \( \mu \in \mathbb{Q} \) such that \( \abs{\lambda - \mu} \to 0 \). Then 
    \begin{align*}
        \abs{\left< \lambda \vb{u}, \vb{v} \right> - \lambda \left<  \vb{u}, \vb{v} \right>} &\le \abs{\left< \lambda \vb{u}, \vb{v} \right> - \left< \mu \vb{u}, \vb{v} \right>} + \abs{ \left< \mu \vb{u}, \vb{v} \right> - \mu \left<  \vb{u}, \vb{v} \right> } + \abs{\mu \left<  \vb{u}, \vb{v} \right> - \lambda \left<  \vb{u}, \vb{v} \right> } \\
        & = \abs{ \left< \left( \lambda - \mu \right) \vb{u}, \vb{v}\right>} + 0 + \abs{\left( \mu -\lambda \right) \left< \vb{u}, \vb{v} \right>}\\
        &= \abs{ \left< \left( \lambda - \mu \right) \vb{u}, \vb{v}\right>} \tag{Since $\abs{\left( \mu -\lambda \right) \left< \vb{u}, \vb{v} \right>} \to 0$.}
    \end{align*}
    Now we just need to show that \( \abs{ \left< \left( \lambda - \mu \right) \vb{u}, \vb{v}\right>} \to 0 \) as \( \abs{\lambda - \mu} \to 0 \). 
    \begin{align*}
        \abs{ \left< \left( \lambda - \mu \right) \vb{u}, \vb{v}\right>} &= \frac{1}{4} \abs{\norm{\left( \lambda - \mu \right) \vb{u} + \vb{v}}^{2} - \norm{\left( \lambda - \mu \right) \vb{u} - \vb{v}}^{2} + i \norm{\left( \lambda - \mu \right) \vb{u} + i \vb{v}}^{2} - i \norm{\left( \lambda - \mu \right) \vb{u} - i\vb{v}}^{2}} \\
        & \to \frac{1}{4}\abs{\norm{\vb{v}}^{2} - \norm{\vb{v}}^{2} + i\norm{\vb{v}}^{2} - i \norm{\vb{v}}^{2}} \tag{By our earlier discussion.}\\
        &= 0
    \end{align*}
    \item[$\bullet$ For \( \lambda \in \mathbb{C} \):] $ $\\ 
    It is sufficient to show that \( \left< i \vb{u}, \vb{v} \right> = i \left< \vb{u}, \vb{v} \right> \), since we have demonstrated \( \mathbb{R} \)-linearity. In other words, if \( \left< \vb{u}, \vb{v} \right> = a+bi \), we need to show that \( \left< i \vb{u}, \vb{v} \right> = -b + ai\) or that \( \Re \left[ \left< \vb{u}, \vb{v} \right> \right] = \Im \left[ \left< i \vb{u}, \vb{v} \right> \right]\) and \( \Im \left[ \vb{u},\vb{v} \right] = - \Re \left[ i \vb{u}, \vb{v} \right] \). 
    So 
    \begin{align*}
        \Re \left[ \left< i \vb{u}, \vb{v} \right> \right] &= \Re \left[ \left< \vb{v}, i \vb{u} \right> \right] \\
        &= \Im \left[ \left< \vb{v}, \vb{u} \right>  \right]\\
        &= - \Im \left[ \left< \vb{u}, \vb{v} \right> \right]
    \end{align*}
    and 
    \begin{align*}
        \Im \left[ \left< i \vb{u},\vb{v} \right> \right] &= \Re \left[ \left<  i \vb{u}, i \vb{v} \right> \right] \\
        &= \Re \left[  \left< \vb{u}, \vb{v} \right> \right] \tag{Use the definition of $\left< \vb{u},\vb{v} \right>$}
    \end{align*}
    This shows that \( \left< i \vb{u} , \vb{v} \right> = i \left<  \vb{u}, \vb{v} \right> \). Now set \( \lambda = \alpha + \beta i \) and we can apply \( \mathbb{R} \)-linearity 
    \begin{align*}
        \left< \lambda \vb{u}, \vb{v} \right> &= \left< \left( \alpha + \beta i \right) \vb{u}, \vb{v}\right> \\
        &= \left< \alpha \vb{u} + \beta i \vb{u}, \vb{v} \right> \\
        &= \left< \alpha \vb{u}, \vb{v} \right> + \left<  \beta i \vb{u}, \vb{v} \right> \\
        &= \alpha \left< \vb{u}, \vb{v} \right> + \beta \left<  i \vb{u}, \vb{v} \right>\\ 
        &= \alpha \left< \vb{u}, \vb{v} \right> + \beta i\left<   \vb{u}, \vb{v} \right> \\
        &= \left( \alpha + \beta i \right) \left< \vb{u}, \vb{v} \right> \\
        &= \lambda \left< \vb{u}, \vb{v} \right>
    \end{align*}
    
    \end{description}
\end{description}
With that, the proof is (finally) complete!
\end{proof}


  

\section{Orthonormal Bases}
\begin{example}[FINISH LATER]
    Use the Gram-Schmidt process to orthonormalize the basis \( \left\{ 1,x, x^{2}, \dots, x^{k}, \dots \right\} \) for the inner product space \( \mathcal{P} \left( \left[ a,b \right] \right) \), where 
    \[ \left< f,g \right> = \int_{a}^{b} f(x)g(x) \dd{x} .\]
    We set \(\boxed{ f_{0}(x) =1} \). Then for \( f_{1}(x) \), we have 
    \begin{align*}
        f_{1}(x) &= x - \frac{\left< x, f_{0}(x) \right>}{\left< f_{0}(x), f_{0}(x) \right>} f_{0}(x) \\
        &= x - \frac{{\int_{a}^{b} x \dd{x}}}{{ \int_{a}^{b} 1^{2} \dd{x}}} \\
        &= x - \frac{1}{2}(b-a)
    \end{align*}
So \(\boxed{ f_{1}(x) =x - \frac{1}{2}(b-a)}\). \\
For \( f_{2}(x) \), we have 
    \begin{align*}
        f_{2} \left( x \right) &= x^{2} - \frac{\left< x^{2}, f_{0}(x) \right>}{ \left< f_{0}(x), f_{0}(x) \right>} f_{0}(x) - \frac{\left< x^{2}, f_{1}(x)\right>}{ \left< f_{1}(x),f_{1}(x) \right>} f_{1}(x) \\
        &= x^{2} - \frac{ \int_{a}^{b} x^{2} \dd{x}}{ \int_{a}^{b} 1^{2} \dd{x}} - \frac{ \int_{a}^{b} x^{2} \left( x - \frac{1}{2}(b-a) \right) \dd{x}}{ \int_{a}^{b} \left( x - \frac{1}{2}(b-a) \right)^{2}\dd{x}} \left( x - \frac{1}{2}(b-a) \right)
    \end{align*}
    


\end{example}
  


\chapter{Symplectic Vector Spaces}
While symplectic vector spaces are not usually taught in a first- or second-year linear algebra course, I have elected to include them in a chapter immediately following the chapter on inner products due to the parallels that many of the theorems and proofs share between these two spaces. However, symplectic vector spaces differ from inner product spaces in interesting ways. For example, a finite-dimensional symplectic vector space necessarily has even dimension. This chapter may be skipped without affecting comprehension of subsequent material, but students interested in differential geometry, classical mechanics, or quantum mechanics will benefit from this early exposure.

\begin{dfn}
    A \vocab{linear symplectic form} on a vector space \( \mathbf{V} \) is a map \( \omega: \mathbf{V} \times \mathbf{V} \to \mathbb{K} \) such that: 
    \begin{enumerate}[label=\textbf{\roman*)}]
        \item \( \omega \) is bilinear on \( \mathbf{V} \) 
        \item \( \omega \) is skew-symmetric, that is, \( \omega \left( \vb{v},\vb{w} \right) = -\omega \left( \vb{w},\vb{v} \right) \). 
        \item \( \omega \) is non-degenerate: if \( \omega \left( \vb{v},\vb{w} \right) =0\) for all \( \vb{w} \in \mathbf{V} \), then \( \vb{v}= \vb{0} \).
    \end{enumerate}
\end{dfn}

\begin{theorem}[All symplectic vector spaces have even dimension.]
    Let \( \left( \mathbf{V}, \omega \right) \) be a symplectic vector space. Then:
    \begin{enumerate}[label=\textbf{\roman*)}]
        \item\( \dim \left( \mathbf{V} \right) \) is even
        \item \( \mathbf{V} \) admits a \vocab{Darboux basis} (or \vocab{symplectic basis}) \( B = \left\{ \vb{e}_{1}, \vb{f}_{1}, \vb{e}_{2}, \vb{f}_{2}, \dots , \vb{e}_{n}, \vb{f}_{n} \right\} \) satisfying:
              \begin{enumerate}[label=\textbf{\alph*)}]
                  \item \( \omega \left( \vb{e}_{j}, \vb{f}_{k} \right) = \delta_{jk} \)
                  \item \( \omega \left(\vb{e}_{j}, \vb{e}_{k} \right) = \omega \left( \vb{f}_{j}, \vb{f}_{k} \right) =0 \)
              \end{enumerate}
    \end{enumerate}
\end{theorem}
\begin{proof}
    
\end{proof}


\part{Real Analysis}
\vspace*{2em} The primary resource for the earlier chapters of this this part is \cite{ref:rudin_pma}. Some exercises were also taken from: \cite{ref:lee_topological}. The chapters pertaining to measure theory and onward use \cite{ref:axler_measure} as the primary source.\vspace*{-1em}
\label{Real Analysis}
\parttoc
\chapter{The Real and Complex Number Systems}

\section{Incompleteness of the Rational Numbers and the Least Upper Bound Property}
\begin{theorem}
    Suppose that \( n \in \mathbb{Z}^{+} \) and that \( \sqrt{n} \not \in \mathbb{Z} \), then \( \sqrt{n} \not \in \mathbb{Q} \).
\end{theorem}
\begin{proof}
    Let the given be as stated and suppose, for the sake of contradiction, that \( \sqrt{n} = \frac{p }{q} \) for some \( p,q \in \mathbb{Z} \) and \( q >0 \). Now since \( \sqrt{n} \not \in \mathbb{Z} \), there must exist some \( k \in \mathbb{Z} \) for which
    \[ 0 <  \sqrt{n} -k <1\]
    holds. Now consider the set 
    \[ S = \left\{ \left( \sqrt{n}-k \right)^{j}: j=1,2,3 \dots  \right\} .\]
We will note two properties of the elements of \( S \) in which the reader should verify by induction. 
\begin{enumerate}[label=\textbf{\roman*)}]
    \item If we set \( s_{j}= \left( \sqrt{n}-k \right)^{j} \), then 
    \[ s_{1} > s_{2} > \cdots > s_{j-1}>s_{j}>s_{j+1}> \cdots \]
    \item All elements \( s_{j} \) of \( S \) are of the form \( a_{j}+b_{j} \cdot \sqrt{n} \) for some \( a_{j},b_{j} \in \mathbb{Z} \).
\end{enumerate}
Applying the hypothesis for contradiction to the second observation, we get that all elements of \( S \) are of the form 
\[ a_{j}+b_{j} \cdot \sqrt{n} = a_{j}+b_{j} \cdot \frac{p }{q} = \frac{a_{j } \cdot q + b_{j} \cdot p}{q} . \]
Since \( a_{j}, b_{j}, p, q \in \mathbb{Z}\) and if we consider the set \( Q  = \left\{ \frac{r }{q}: r=1,2, \dots  \right\} \), we have that \( S \subseteq Q \). However, the first observation about elements of \( S \) ensures that this cannot be the case. (Why?) The contradiction establishes the result.
\end{proof}

\begin{lemma}
    Suppose that \( n \in \mathbb{Z}^{+} \) and \(  \sqrt{n} \not \in \mathbb{Q}\). Consider the sets 
    \[ A = \left\{ x \in \mathbb{Q}^{+}: x^{2} < n \right\}  \quad B = \left\{ x \in \mathbb{Q}^{+}: x^{2} > n \right\}.\]
    Then \( A \) has no maximal element and \( B \) has no minimal element.
\end{lemma}
\begin{proof}
   Let the given be as stated, and pick \( p \in A \cup B \). Define an element \( q \) given by 
    \[ q = p - \frac{p^{2}-n}{p+n} .\]
We simplify \( q \) as follows:
\begin{align*}
    q &= \frac{p \cdot \left( p+n \right)}{p+n} - \frac{p^{2}-n}{p+n}\\
    &= \frac{p^{2}+n \cdot p - p^{2} +n}{p+n}\\
    &= \frac{n \cdot p +n}{p+n}
\end{align*}
Now, we compute \( q^{2}-n \).
\begin{align*}
    q^{2}-n &= \left( \frac{n \cdot p +n}{p+n} \right)^{2} -n \\
    &= \frac{\left( n \cdot p +n \right) ^{2}}{ \left( p+n \right)^{2}}- \frac{n \cdot \left( p+n \right)^{2} }{ \left( p+n \right)^{2}}\\
    &= \frac{n^{2} \cdot p^{2} + 2 n^{2} \cdot p +n^{2} - n \cdot p^{2} - 2n^{2} \cdot p -n^{3}}{\left( p+n \right)^{2}}\\
    &= \frac{n^{2}\cdot p^{2} -n \cdot p^{2} - \left( n^{3} -n^{2} \right)}{\left( p+n \right)^{2}}\\
    &= \frac{n \cdot p^{2} \cdot \left( n-1 \right) - n^{2} \cdot \left( n-1 \right)}{\left( p+n \right)^{2}}\\
    q^{2}-n &= \left( n^{2}-n \right) \cdot \frac{  p^{2} - n}{\left( p+n \right)^{2}}
\end{align*}
Now if \( p \in A \), our last equation shows that \( q \in A \) and the first equation shows that \( q >p \). If \(  p \in B \), our last equation shows that \( q \in B\) and the first equation shows that \( q <p \).
\end{proof}


\begin{dfn}
    Suppose that \( S \) is an ordered set. We say that \( A \subseteq S \) is \vocab{bounded above} if there exists some \( x \in S \) such that for all \( a \in A \), we have that \( a \le x \). We call \( x \) an \vocab{upper bound} for \( A \).\\
    The definition for a set that is \vocab{bounded below} and an element that is a \vocab{lower bound} is similar.
\end{dfn}

\begin{dfn}
    Suppose that \( S \) is an ordered set and \( A \subset S \) is bounded above. If there is some upper bound \( \alpha \in S \) with the property that if \( \gamma < \alpha \) then \( \gamma \) is not an upper bound for \( A \), we refer to \( \alpha \) as the \vocab{least upper bound} or the \vocab{supremum} of \( A \) and we write 
    \[ \alpha = \sup \left( A \right) .\]
    Similarly, if \( A \) is bounded below and there is some lower bound \( \beta \) such that if \( \gamma > \beta \) then \( \gamma \) is not a lower bound of \( A \), we will call \( \beta \) the \vocab{greatest lower bound} or the \vocab{infimum} of \( A \) and we write 
    \[ \beta = \inf{\left( A \right)} .\]
    If \textbf{every} subset of \( S \) that has an upper bound also has a least upper bound, we say that \( S \) has the \vocab{least upper bound property}. If every subset of \( S \) that has a lower bound also has greatest lower bound, we say that \( S \) has the \vocab{greatest lower bound property}.
\end{dfn}

\begin{lemma}
    Every set with the least upper bound property has the greatest lower bound property.
\end{lemma}
\begin{proof}
    Let \( S \) be a non-empty set with least upper bound property and suppose that \( A \subset S \) is bounded below. Since \( A \) is bounded below, the set 
    \[ B= \{x \in S: x \le a \text{ for every } a \in A\} \]
    is well-defined and non-empty. By definition, every element of \( A \) is an upper bound for \( B \). So \( B \) is bounded above and since \( S \) has the least upper bound property, the supremum, call it \( \gamma \), of \( B \) exists. I claim that \( \gamma = \inf \left( A \right) \). We need to show that:
    \begin{enumerate}[label=\textbf{\roman*)}]
        \item \( \gamma \) is a lower bound of \( A \); 
        \item if \( \eta > \gamma \), then \( \eta \) is not a lower bound of \( A \).
    \end{enumerate}
    For the first part, suppose that \( \gamma \) is \textbf{not} a lower bound of \( A \). Then there is some \( a \in A \), such that \( a < \gamma \). But since every element of \( A \) is an upper bound of \( B \), we have found an upper bound of \( B \) less than \( \gamma \), which contradicts that \( \gamma \) is the \textbf{least} upper bound. So \( \gamma \) is a lower bound of \( A \).\\
    For the second part suppose that \( \eta > \gamma \) and that \( \eta \) is a lower bound of \( A \). It follows, by definition, that \( \eta \in B \). This contradicts that \( \gamma \) is an upper bound of \( B \), as we have just found an element of \( B \), namely \( \eta \), for which \( \eta > \gamma \). So \( \gamma \) is the greatest lower bound of \( A \).
\end{proof}


\chapter{Basic Topology}
\section{Metric Spaces}
\begin{dfn}\label{def: Metric Space}
    Let \( M \) be a set. A \vocab{metric} on \( M \) is a map \( \rho: M \times M \to [0, \infty) \) such that the following criteria are held:
    \begin{enumerate}[label=\textbf{\roman*)}]
        \item \( \rho \left( x,y \right) =0 \) if and only if \( x=y \). 
        \item \( \rho \left( x,y \right) = \rho \left( y,x \right)\) for every \( x, y \in M \). 
        \item \( \rho \left( x,z \right) \le \rho \left( x,y \right) + \rho \left( y,z \right)\), this is called the \vocab{triangle inequality}.
    \end{enumerate}
    We call the pair \( \left( M, \rho \right) \) a \vocab{metric space}. We will often omit mention of \( \rho \) if it is understood in context or mention of \( \rho \) is unnecessary.
\end{dfn}

\begin{lemma}
    A metric is non-negative. That is; 
    \[ \rho \left(  x,y  \right) \ge 0 \] for all \( x,y \in M \).
\end{lemma}
\begin{proof}
    \begin{align*}
        \rho \left( x,x \right) &\le \rho \left( x,y \right) + \rho \left( y,x \right) \tag{By the triangle ineqality}\\
        \rho \left( x,x \right) & \le 2 \rho \left( x,y \right) \\
        0 & \le 2 \rho \left( x,y \right) \\
        0 &\le \rho \left( x,y \right)
    \end{align*}
    
\end{proof}


\begin{dfn}
    Let \( M \) be a metric space and define the \vocab{\( \epsilon \)-ball centered at \( x \)} to be
    \[ B_{\rho} \left( x; \epsilon \right) = \left\{ y \in M: \rho \left( x,y \right)< \epsilon \right\}\]
The \vocab{topology induced by the metric} is a topological space on \( M \) with a basis consisting of all possible \( \epsilon \)-balls centered at every \( x \in M \).
\end{dfn}

\begin{lemma}
    The basis described above is indeed a basis.
\end{lemma}
\begin{proof}
    We will verify that the conditions of \Cref{def: basis for a topology} hold.
    Suppose that \( M \) is a metric space with basis elements. \( \left\{ B_{\rho} \left( x; \epsilon \right) : x \in M , \epsilon \in \mathbb{R} \right\} \). By definition, every element of \( M \) belongs to some basis element. So we just need to verify the intersection condition of the basis. Suppose that we fix \( x_{1}, x_{2} \in M \), \( \epsilon_{1}, \epsilon_{2} \in \mathbb{R} \) such that 
    \[ B_{\rho} \left( x_{1}; \epsilon_{1} \right) \cap B_{\rho} \left( x_{2}; \epsilon_{2} \right) \neq \varnothing. \]
    Pick any \( y \in  B_{\rho} \left( x_{1}; \epsilon_{1} \right) \cap B_{\rho} \left( x_{2}; \epsilon_{2} \right) \) and let 
    \[ \epsilon < \min \left\{ \epsilon_{1}- \rho \left( x_{1}, y \right), \epsilon_{2} - \rho \left( x_{2},y \right) \right\} .\]
    I claim that \( B_{\rho} \left( y; \epsilon \right) \subseteq B_{\rho} \left( x_{1}; \epsilon_{1} \right) \cap B_{\rho} \left( x_{2}; \epsilon_{2} \right) \). To show this pick any \( z \in B_{\rho} \left( y; \epsilon \right) \), then 
    \begin{align*}
        \rho \left( x_{1},z \right)  & \le \rho \left( x_{1},y \right) + \rho \left( y,z \right)\\
        &< \rho\left( x_{1},y \right) + \epsilon_{1} - \rho \left( x_{1},y \right) \tag{Since $z \in B_{\rho} \left( y, \epsilon \right)$}\\
       &< \epsilon_{1}
    \end{align*}
    so \( z \in B_{\rho} \left( x_{1}; \epsilon_{1} \right) \). Showing \( z \in B_{\rho} \left( x_{2};\epsilon_{2} \right) \) is similar. So \( z \in   B_{\rho} \left( x_{1}; \epsilon_{1} \right) \cap B_{\rho} \left( x_{2}; \epsilon_{2} \right)\), which shows that \( B_{\rho} \left( y; \epsilon \right) \subseteq   B_{\rho} \left( x_{1}; \epsilon_{1} \right) \cap B_{\rho} \left( x_{2}; \epsilon_{2} \right) \)
\end{proof}

\begin{figure}[H]
    \centering
    \includegraphics[width=0.8\textwidth]{figures/analysis/basis.png}
    \caption{Given \( y \in B_\rho(x_1; \varepsilon_1) \cap B_\rho(x_2; \varepsilon_2) \), we choose \( \varepsilon \) small enough so that \( B_\rho(y; \varepsilon) \subseteq B_\rho(x_1; \varepsilon_1) \cap B_\rho(x_2; \varepsilon_2) \), verifying the basis intersection property for the topology induced by a metric.}
    \label{fig:basis}
\end{figure}
\FloatBarrier



\begin{lemma}[Generalized Triangle Inequality]
    For any \( x_{1}, \dots, x_{n} \) in a metric space \( M \), we have
    \[
        \rho(x_{1}, x_{n}) \le \sum_{j=1}^{n-1} \rho(x_{j}, x_{j+1}).
    \]
\end{lemma}
\begin{proof}
    We prove this by induction on \( n \).\\
Although the base cases of \( n =2 \) or \( n=3 \) trivially hold, we will set the base case to be \( n =4 \) for illustrative purposes. 
    \textbf{Base case:} Let \( n = 4 \). Suppose \( x_1, x_2, x_3, x_4 \in M \). Then:
    \begin{align}
        \rho(x_1, x_4) 
            &\le \rho(x_1, x_3) + \rho(x_3, x_4) \tag*{[triangle inequality on \( x_1, x_3, x_4 \)]} \\
        &\le \rho(x_1, x_2) + \rho(x_2, x_3) + \rho(x_3, x_4) \tag*{[triangle inequality on \( x_1, x_2, x_3 \)]}
    \end{align}
    \textbf{Inductive step:} Assume the inequality holds for \( n - 1 \), i.e.,
    \[
        \rho(x_1, x_{n-1}) \le \sum_{j=1}^{n-2} \rho(x_j, x_{j+1}).
    \]
    Then:
    \begin{align}
        \rho(x_1, x_n) 
            &\le \rho(x_1, x_{n-1}) + \rho(x_{n-1}, x_n) \tag*{[triangle inequality]} \\
        &\le \sum_{j=1}^{n-2} \rho(x_j, x_{j+1}) + \rho(x_{n-1}, x_n) \tag*{[by induction hypothesis]} \\
        &= \sum_{j=1}^{n-1} \rho(x_j, x_{j+1}) \tag*{[combine and reindex]}
    \end{align}

    Therefore, the inequality holds for all \( n \ge 2 \).
\end{proof}

\begin{corollary}
    \[ \abs{\rho \left( w,x \right) - \rho \left( y,z \right)} \le \rho \left( x,y \right) + \rho \left( w,z \right) \]
\end{corollary}
\begin{proof}
    Applying the generalized triangle inequality, we have 
    \begin{align*}
        \rho \left( w,x \right) \le \rho \left( w,z \right) &+ \rho \left( z,y \right) + \rho \left( y,x \right)\\
        \rho \left( w,x \right)- \rho \left( y,z \right) &\le \rho \left( x,y \right)+ \rho \left( w,z \right)\\
        \abs{ \rho \left( w,x \right)- \rho \left( y,z \right) } &\le \abs{\rho \left( x,y \right)+ \rho \left( w,z \right)}\\
        \abs{ \rho \left( w,x \right)- \rho \left( y,z \right) } &\le \rho \left( x,y \right) + \rho \left( w,z \right)
    \end{align*}
    
\end{proof}


\begin{example}[Euclidean Metric on \( \mathbb{R}^{n} \)]
    Let \( \vb{x} = \left( x_{1}, \dots, x_{n} \right), \vb{y} = \left( y_{1}, \dots , y_{n} \right) \in \mathbb{R}^{n}\). We define the standard Euclidean metric on \( \mathbb{R}^{n} \) to be
    \[ \rho\left( \vb{x}, \vb{y} \right) : = \sqrt{ \sum_{j=1}^{n } \left( x_{j}-y_{j} \right)^{2}} \]
    We need only to verify \textbf{iii} of \Cref{def: Metric Space} as \textbf{i} and \textbf{ii} are fairly easy to see. Let \( \vb{x} = \left( x_{1},\dots , x_{n} \right), \vb{y} = \left( y_{1}, \dots, y_{n} \right), \vb{z}= \left( z_{1}, \dots, z_{n} \right) \in \mathbb{R}^{n} \)
    \begin{align*}
        \sum_{j=1}^{n} \left( x_{j}-z_{j} \right)^{2} &= \sum_{j=1}^{n} \left( x_{j}-y_{j}+y_{j}-z_{j} \right)^{2}\\
        &= \sum_{j=1}^{n} \left( x_{j}- y_{j} \right) ^{2} + \sum_{j=1}^{n } \left( y_{j}-z_{j} \right)^{2} + 2 \sum_{j=1}^{n } \left( x_{j}-y_{j} \right)\left( y_{j}-z_{j} \right)\\
        & \le \sum_{j=1}^{n} \left( x_{j}- y_{j} \right) ^{2} + \sum_{j=1}^{n } \left( y_{j}-z_{j} \right)^{2} + 2 \sqrt{\sum_{j=1}^{n} \left( x_{j}-y_{j} \right)^{2} }\sqrt{\sum_{j=1}^{n} \left( y_{j}-z_{j} \right)^{2} } \tag{Applying the \nameref{thrm: The Cauchy-Schwarz Inequality}}
    \end{align*}
   This gives 
   \[ \sum_{j=1}^{n} \left( x_{j}-z_{j} \right)^{2} \le  \sum_{j=1}^{n} \left( x_{j}- y_{j} \right) ^{2} + \sum_{j=1}^{n } \left( y_{j}-z_{j} \right)^{2} + 2 \sqrt{\sum_{j=1}^{n} \left( x_{j}-y_{j} \right)^{2} }\sqrt{\sum_{j=1}^{n} \left( y_{j}-z_{j} \right)^{2} } \]  or 
   \[ \sum_{j=1}^{n} \left( x_{j}-z_{j} \right)^{2} \le \left( \sqrt{\sum_{j=1}^{n} \left( x_{j}-y_{j} \right)^{2} }\ + \sqrt{\sum_{j=1}^{n} \left( y_{j}-z_{j} \right)^{2} } \right)^{2} \]
   Taking the square root of both sides verifies the triangle inequality 
   \[ \sqrt{\sum_{j=1}^{n} \left( x_{j}-z_{j} \right)^{2}} \le \sqrt{\sum_{j=1}^{n} \left( x_{j}-y_{j} \right)^{2} }\ + \sqrt{\sum_{j=1}^{n} \left( y_{j}-z_{j} \right)^{2} } \]
\end{example}

\begin{example}
  Let \( S \) be the set of all complex-valued sequences. Define a metric on \( S \) by
  \[
    \rho \left( \vb{x}, \vb{y} \right) = \sum_{j=1}^{\infty} \frac{1}{2^{j}} \cdot \frac{\abs{x_{j} - y_{j}}}{1 + \abs{x_{j} - y_{j}}}.
  \]
  Much like the previous example, we will verify the triangle inequality. Consider the function
  \[
    f(t) = \frac{t}{1 + t},
  \]
  which is increasing on \( [0, \infty) \) since
  \[
    f'(t) = \frac{1}{(1 + t)^2} > 0.
  \]
  Therefore, by the triangle inequality \( \abs{a + b} \le \abs{a} + \abs{b} \), we have
  \[
    f\left( \abs{a + b} \right) \le f\left(  \abs{a}+ \abs{b} \right) = f\left( \abs{a} \right) + f \left( \abs{b} \right).
  \]
  Setting \( a = z_j - y_j \) and \( b = y_j - x_j \), we get:
  \begin{align*}
    f\left( \abs{z_j - x_j} \right)
    &\le f\left( \abs{z_j - y_j} \right) + f\left( \abs{y_j - x_j} \right) \\
    \frac{1}{2^j} f\left( \abs{z_j - x_j} \right)
    &\le \frac{1}{2^j} f\left( \abs{z_j - y_j} \right) + \frac{1}{2^j} f\left( \abs{y_j - x_j} \right)
  \end{align*}
  Summing over all \( j \in \mathbb{N} \), we obtain:
  \begin{align*}
    \sum_{j=1}^\infty \frac{1}{2^j} f\left( \abs{z_j - x_j} \right)
    &\le \sum_{j=1}^\infty \frac{1}{2^j} f\left( \abs{z_j - y_j} \right)
    + \sum_{j=1}^\infty \frac{1}{2^j} f\left( \abs{y_j - x_j} \right) \\
    \sum_{j=1}^\infty \frac{1}{2^j} \cdot \frac{\abs{z_j - x_j}}{1 + \abs{z_j - x_j}}
    &\le \sum_{j=1}^\infty \frac{1}{2^j} \cdot \frac{\abs{z_j - y_j}}{1 + \abs{z_j - y_j}}
    + \sum_{j=1}^\infty \frac{1}{2^j} \cdot \frac{\abs{y_j - x_j}}{1 + \abs{y_j - x_j}} \\
    \rho(\vb{z}, \vb{x}) &\le \rho(\vb{z}, \vb{y}) + \rho(\vb{y}, \vb{x}).
  \end{align*}
  Hence, the triangle inequality holds.
\end{example}


\begin{lemma}
    Let \( \mathbf{V} \) be a  \hyperref[def:norm]{normed space}.  Then \( \mathbf{V} \) has a \vocab{metric induced by the norm} defined by 
    \[ \rho \left( \vb{x}, \vb{y} \right) : = \norm{ \vb{x} - \vb{y}}\quad \text{for all } \vb{x}, \vb{y} \in \mathbf{V} .\]
\end{lemma}
\begin{proof}
    We want to show that \( \rho \left( \vb{x}, \vb{y} \right) = \norm{ \vb{x} - \vb{y}} \) defines a metric.\\ 
    Suppose that \( \vb{x} = \vb{y} \). Then 
    \begin{align*}
        \rho \left( \vb{x} ,\vb{y} \right) &= \norm{\vb{x} - \vb{y}}\\
        &= \norm{\vb{0}} \\
        &=0
    \end{align*}
    So \( \vb{x} =\vb{y}  \Rightarrow \rho \left( \vb{x},\vb{y} \right) =0\). Conversely, if \( \rho \left( \vb{x}, \vb{y} \right) =0\) then \( \norm{\vb{x} - \vb{y}} =0 \) so \( \vb{x} - \vb{y} = \vb{0} \) or \( \vb{x} = \vb{y} \). \\ 
    For symmetry, 
    \begin{align*}
        \rho \left( \vb{y},\vb{x} \right) &= \norm{\vb{y} - \vb{x}}\\
        &= \norm{-1 \left( \vb{x} -\vb{y} \right)} \\
        &= \abs{-1} \norm{\vb{x} - \vb{y} }\\
        &= \norm{ \vb{x} - \vb{y}} \\
        &= \rho \left( \vb{x}, \vb{y} \right)
    \end{align*}
    Finally, for the triangle inequality, 
    \begin{align*}
        \rho \left( \vb{x}, \vb{z} \right) &= \norm{\vb{x} - \vb{z}}\\
        &= \norm{ \left( \vb{x} - \vb{y} \right) + \left( \vb{y} - \vb{z} \right)} \\
        &\le \norm{\vb{x} - \vb{y}} + \norm{ \vb{y} - \vb{z}} \\
        & = \rho \left( \vb{x}, \vb{y} \right) + \rho \left( \vb{y}, \vb{z} \right)
    \end{align*}
    So \( \rho \left( \vb{x}. \vb{z} \right) \le \rho \left( \vb{x}, \vb{y} \right) + \rho \left( \vb{y}, \vb{z} \right) \).\\
    So \( \rho \left( \vb{x}, \vb{y} \right) = \norm{ \vb{x} - \vb{y}} \) defines a metric.
\end{proof}



\begin{lemma}[The Reverse Triangle Inequality]\label{thm:Reverse-Triangle}
    In a normed space \( \abs{ \; \norm{\vb{x}} - \norm{\vb{y}}\; } \le \norm{\vb{x} - \vb{y}}. \)
\end{lemma}
\begin{proof}
       We can define the metric \( \rho \) induced by the norm as \( \rho \left( \vb{x}, \vb{y} \right) := \norm{ \vb{x} -\vb{y}}\). As such, we wish to show that \( \abs{ \rho \left( \vb{x}, \vb{0} \right) - \rho \left( \vb{y}, \vb{0} \right)} \le \rho \left( \vb{x}, \vb{y} \right) \). This is straightforward. 
    \begin{align*}
        &\rho \left( \vb{x}, \vb{0} \right)  \le \rho \left( \vb{x},\vb{y} \right) + \rho \left(  \vb{y}, \vb{0} \right) \tag*{By the triangle inequality.} \\
        &\rho \left( \vb{x} , \vb{0} \right) - \rho \left( \vb{y}, \vb{0} \right) \le \rho \left( \vb{x},\vb{y} \right)
    \end{align*}
    Applying the absolute value to both sides finishes this proof.
\end{proof}
\begin{altproof}
    We can set \( \vb{x} = \vb{x}- \vb{y} + \vb{y} \). Then 
    \begin{align*}
        \norm{\vb{x}} &= \norm{\left( \vb{x}- \vb{y} \right) + \vb{y}} \\
        &\le \norm{\vb{x}- \vb{y}} + \norm{\vb{y}}
    \end{align*}
    so 
    \[ \norm{\vb{x}} - \norm{\vb{y}} \le \norm{\vb{x}- \vb{y}}\]. Taking the absolute value of both sides finishes this proof.
\end{altproof}







\chapter{Sequences and Series}
\section{Cauchy Sequences}

\begin{dfn}
    A sequence \( \left\{ x_{n} \right\} \) in a metric space \( \left( M, \rho \right) \) is said to be a \vocab{Cauchy sequence} if for all \( \epsilon >0 \), there exists an \( N  \in \mathbb{N}\) such that if \( m,n > N \) then 
    \[ \rho \left( x_{m}, x_{n} \right) < \epsilon \]
\end{dfn}

\begin{lemma}
    If \( \left\{ x_{n} \right\} \) is a Cauchy sequence in a metric space \( \left( M, \rho \right) \), then for all \( y \in M \), there exists an \( \epsilon_{y} \in \mathbb{R} \) such that every element of the sequence is contained in the closed ball \( \overline{B_{\rho} \left( y ; \epsilon_{y} \right)} \).
\end{lemma}
\begin{proof}
    
\end{proof}



\begin{dfn}\label{def:complete-metric-space}
    A metric space \( M \) is said to be \vocab{complete} if every Cauchy sequence converges to a point in \( M \).
\end{dfn}

\chapter{Continuity}
\section{Continuity of Functions to \( \mathbb{R} \)}

\begin{lemma}\label{thm:reciprocal-is-continuous}
    Suppose \( f: I \to \mathbb{R} \) is continuous at \( x_{0} \in I \), \( f(x_{0}) =y \), and that for all \( x \in I \), \( f(x) \neq 0 \). Then \( \frac{1}{f(x)}: I \to \mathbb{R} \) is continuous at \( x_{0} \) and 
    \[ \lim_{x \to x_{0}} \frac{1}{f(x)} = \frac{1}{y} .\]
\end{lemma}
\begin{proof}
Pick \( \delta >0 \), such that 
\[ \abs{f(x) - y} < \min \left\{ \frac{\abs{y}}{2}, \frac{\abs{y}^{2} \epsilon}{2} \right\} .\] Notice
\begin{align*}
     \abs{y} - \abs{f(x)} &\le  \abs{\abs{y} - \abs{f(x)}} \\ 
     &\le \abs{f(x) - y} \tag{By the \hyperref[thm:Reverse-Triangle]{reverse triangle inequality} } \\ 
    &< \frac{\abs{y}}{2} \tag{By hypothesis}
\end{align*}
So \( \abs{y} - \abs{ f(x)} < \frac{\abs{y}}{2}\) This gives us \( \frac{\abs{y}}{2}<  \abs{ f(x)}\) or \( \boxed{\frac{1}{\abs{ f(x)}} < \frac{2}{\abs{y}}} \). Now 
    \begin{align*}
        \abs{\frac{1}{f(x)} - \frac{1}{y}} &= \abs{\frac{y - f(x)}{y f(x)}} \\
        &= \frac{1}{ \abs{y} \abs{f(x)}} \abs{y - f(x)} \\
        &< \frac{2}{\abs{y}^{2}} \abs{y-f(x)} \tag{By our earlier investigation}
        \\
        &< \frac{2}{\abs{y}^{2}} \frac{\abs{y}^{2} \epsilon}{2} \tag{By our choice of $\delta$} \\
        &= \epsilon
    \end{align*}
So we can bound \( \abs{\frac{1}{f(x)} - \frac{1}{y}} \) by arbitrarily small epsilon, which was required. 

\end{proof}


\begin{lemma}\label{thm:prod-of-continuous-functions-continuous}
    Suppose that \( f: I \to \mathbb{R} \) and \( g: I \to \mathbb{R} \) are continuous at \( x_{0} \in I \) and that \( f(x_{0}) = y_{f} \) and \( g(x_{0})=y_{g} \). Then, the product \( f \cdot g: I \to \mathbb{R} \) defined by \( \left( f \cdot g \right)(x) = f(x) \cdot g(x) \) for all \( x \in I \) is continuous at \( x_{0} \) and 
    \[ \lim_{x \to x_{0}} \left( f \cdot g\right)(x) = y_{f}  y_{g} \]
\end{lemma}
\begin{proof}
We want to show that for any \( \epsilon >0 \), there exists a \( \delta >0 \) such that whenever \( \abs{x_{0} -x} < \delta \), we have \( \abs{f(x_{0}) g(x_{0}) - f(x)  g(x)} =  \abs{f(x_{0})  g(x_{0}) - y_{f}  y_{g}} < \epsilon \). Before we choose our delta, let us do some manipulation: 
\begin{align*}
     \abs{f(x_{0}) g(x_{0}) - y_{f} y_{g}} &= \abs{ \left( f(x_{0}) - y_{f}  \right) \left( g (x_{0}) - y_{g} \right) + y_{g}  \left( f(x_{0}) -y_{f} \right) + y_{f} \left( g\left( x_{0} \right) -y_{g} \right)} \\
     & \le \abs{\left( f(x_{0}) - y_{f}  \right) \left( g (x_{0}) - y_{g} \right)} + \abs{ y_{g}  \left( f(x_{0}) -y_{f} \right)} + \abs{y_{f} \left( g\left( x_{0} \right) -y_{g} \right)} \\
     &= \abs{ f(x_{0}) - y_{f}} \abs{g (x_{0}) - y_{g}} + \abs{ y_{g} } \abs{f(x_{0}) -y_{f} } + \abs{y_{f} } \abs{g\left( x_{0} \right) -y_{g}}
\end{align*}
In that regard, we apply the continuity hypothesis and pick \( \delta >0 \) such that 
\begin{align*}
 \abs{f(x_{0}) -y_{f} } &<\min \left\{ \sqrt{\frac{\epsilon}{3}}, \frac{\epsilon}{3 \left( \abs{y_{g}}+1 \right)} \right\} \\
 \abs{g\left( x_{0} \right) -y_{g}} &< \min \left\{ \sqrt{\frac{\epsilon}{3}}, \frac{\epsilon}{3 \left( \abs{y_{f}}+1 \right)} \right\}
\end{align*}
So 
\begin{align*}
 \abs{f(x_{0}) g(x_{0}) - y_{f} y_{g}} &\le \abs{ f(x_{0}) - y_{f}} \abs{g (x_{0}) - y_{g}} + \abs{ y_{g} } \abs{f(x_{0}) -y_{f} } + \abs{y_{f} } \abs{g\left( x_{0} \right) -y_{g}} \\
 &<  \sqrt{\frac{\epsilon}{3}} \sqrt{\frac{\epsilon}{3}} + \frac{\epsilon \abs{y_{g}}}{3 \left( \abs{y_{g}}+1 \right)} + \frac{\epsilon \abs{y_{f}}}{3 \left( \abs{y_{f}}+1 \right)} \\
 &< \frac{\epsilon}{3} + \frac{\epsilon}{3} +\frac{\epsilon}{3} \\
 &= \epsilon
\end{align*}
So we can for any \( \epsilon >0\), we can find a \( \delta >0 \) such that \(  \abs{f(x_{0})  g(x_{0}) - y_{f}  y_{g}} < \epsilon  \) whenever \( \abs{x_{0} -x} < \delta \), which is what we wanted. 
\end{proof}


\begin{corollary}
     Suppose that \( f: I \to \mathbb{R} \) and \( g: I \to \mathbb{R} \) are continuous at \( x_{0} \in I \) and that \( f(x_{0}) = y_{f} \) and \( g(x_{0})=y_{g} \ne 0 \). Then, the quotient \( \frac{f}{g}: I \to \mathbb{R} \) defined by \( \left( \frac{f }{g} \right)(x)= \frac{f(x)}{g(x)} \) is continuous at \( x_{0} \) and 
    \[ \lim_{x \to x_{0}} \frac{f(x)}{g(x)} = \frac{y_{f}}{y_{g}} .\]
\end{corollary}
\begin{proof}
    Apply \cref{thm:reciprocal-is-continuous} to \( g \) and then apply \cref{thm:prod-of-continuous-functions-continuous} to \( f \) and the resultant \( \frac{1}{g} \).
\end{proof}

\section{Continuity of Functions Between Metric Spaces}

\begin{dfn}
    Let\( \left( X, \rho_{X} \right) \) and \( \left( Y, \rho_{Y} \right) \) be metric spaces. A function \( f: X \to Y \) is \vocab{continuous} at \( x \in X \) if for all \( \epsilon >0 \), there exists a \( \delta >0 \) such that 
    \[ \rho_{X} \left( t,x \right) < \delta \Rightarrow \rho_{Y} \left( f (t),f(x) \right) < \epsilon. \]
\end{dfn}

\chapter{Differentiation}
\section{The Derivative}
\begin{theorem}[The Chain Rule]\label{thm: chain rule}
    Suppose \( f: \left[ a,b \right] \to \left[ c,d \right] \) is continuous and that it is differentiable at \( x \). Also \( g:I \to \mathbb{R} \) (where \( \left[ c,d \right] \subseteq I \)) is continuous and that it is differentiable at \( f(x) \). Then the composition function \( g \circ f: \left[ a,b \right] \to \mathbb{R} \) is differentiable at \( x \) and 
    \[ \left( g \circ f \right)'(x) = g'\left( f(x) \right) \cdot f'(x). \]
\end{theorem}
\begin{proof}
    Our goal is to verify that the following limit 
    \[ \lim_{t \to x}  \frac{g\left( f (t) \right) - g \left( f (x) \right)}{t-x}\] exists and equals \( \left( g \circ f \right)'(x) = g'\left( f(x) \right) \cdot f'(x). \)\\
    Since \( g \) is differentiable at \( f(x) \), the limit 
    \[ \lim_{y \to f(x)} \frac{g(y) - g(f(x))}{y-f(x)} \]
    exists and equals \( g' \left( f(x) \right) \). As such, we can define a continuous auxiliary function 
    \[ \Psi \left( y \right) := \begin{cases}
        \frac{g(y) - g(f(x))}{y-f(x)} & y \neq f(x)\\
         g' \left( f(x) \right) & y = f(x)
    \end{cases} \]
    Note that $\Psi$ is continuous at $f(x)$ since 
    \[ \lim_{y \to f(x)} \Psi(y) = \lim_{y \to f(x)} \frac{g(y) - g(f(x))}{y-f(x)} = g'(f(x)) = \Psi(f(x)). \]
    For \( y \neq f(x) \), we have  
    \[ g(y) - g(f(x)) = \Psi(y) \left( y-f(x) \right)  .\]
    This equation also holds when $y = f(x)$ since both sides equal 0.
    
    Letting \( y= f(t) \), we have 
    \[  g\left( f(t) \right) - g \left( f(x) \right)= \Psi \left( f(t) \right) \left( f(t)- f(x) \right)  .\]
    Now we can divide both sides by \( t-x \) (for $t \neq x$) to get
    \[ \frac{ g\left( f(t) \right) - g \left( f(x) \right)}{t-x}= \Psi \left( f(t) \right) \frac{f(t)- f(x) }{t-x}  \]
    Finally we can take the limit as \( t \to x \) of both sides 
    \begin{align*}
        \lim_{t \to x}\frac{ g\left( f(t) \right) - g \left( f(x) \right)}{t-x} &= \lim_{t \to x}\left[\Psi \left( f(t) \right) \cdot \frac{f(t)- f(x) }{t-x}\right]\\
        &= \left(\lim_{t \to x}\Psi \left( f(t) \right)\right) \left(\lim_{t \to x} \frac{f(t)- f(x) }{t-x}\right) \tag{Since both limits exist}\\
        &= \Psi \left( \lim_{t \to x}f(t) \right) \left(\lim_{t \to x} \frac{f(t)- f(x) }{t-x}\right) \tag{By continuity of $\Psi$ and $f$}\\
        &= \Psi \left( f(x) \right) \left(\lim_{t \to x} \frac{f(t)- f(x) }{t-x}\right) \\
        &= g'(f(x)) \left(\lim_{t \to x} \frac{f(t)- f(x) }{t-x}\right) \\
        &=  g'(f(x)) \cdot f'(x) \tag{Since $f$ is differentiable at $x$}
    \end{align*}
    So we have shown that 
    \[  \lim_{t \to x}\frac{ g\left( f(t) \right) - g \left( f(x) \right)}{t-x}  \] exists and equals \(   g'(f(x)) \cdot f'(x) \), as desired.
\end{proof}


\section{The Mean Value Theorems}
\begin{dfn}
    Suppose that \( f \) is a continuous real-valued function. We say that \( f \) attains a \vocab{local maximum} at \( p \) if there exists a \( \delta >0 \) such that \( f(p) \ge f(q) \) whenever \( q \in B\left( p; \delta \right)\).\\

    Similarly, we say that \( f \) attains a \vocab{local minimum} at \( p \) if there exists a \( \delta >0 \) such that \( f(p) \le f(q) \) whenever \( q \in B\left( p; \delta \right)\).
\end{dfn}

\begin{theorem}
    Suppose that \( f: \left[ a,b \right]\to \mathbb{R}\) is differentiable on \( (a,b) \). If \( f \) attains a local maximum or minimum at \( x \), then \( f'(x) =0 \).
\end{theorem}
\begin{proof}
    Let the givens be as stated and suppose that \( f \) attains a local maximum at \( x \in \left( a,b \right) \). Then there is a \( \delta >0 \) such that whenever \( t \in B \left( x, \delta \right) \), we have \( f(x) \ge f(t) \). 
    
    First suppose that \( x- \delta < t < x \). Then \( f(x) - f(t) \ge 0 \) and \( x - t > 0 \), so
    \begin{equation}\label{eq:difference-quotient-positive}
        \frac{f(x) - f(t)}{x-t} \ge 0 
    \end{equation}
    as it is the quotient of two non-negative numbers where the denominator is positive.
    
    On the other hand, if \( x < t < x + \delta \), then \( f(x) - f(t) \ge 0 \) and \( x - t < 0 \), so
    \begin{equation}\label{eq:difference-quotient-negative}
        \frac{f(x) - f(t)}{x-t} \le 0 
    \end{equation}
    as it is the quotient of a non-negative and negative number.
    
    Since \( f \) is differentiable at \( x \), the limit 
    \[ f'(x)= \lim_{t \to x}\frac{f(x) - f(t)}{x-t} \] 
    exists. By \Cref{eq:difference-quotient-positive}, we have 
    \[ f'(x) = \lim_{t \to x^-}\frac{f(x) - f(t)}{x-t} \ge 0 \]
    and by \Cref{eq:difference-quotient-negative}, we have
    \[ f'(x) = \lim_{t \to x^+}\frac{f(x) - f(t)}{x-t} \le 0 \]
    Since both one-sided limits equal \( f'(x) \), we must have \( f'(x) = 0 \).
    
    The case where \( f \) attains a local minimum at \( x \) follows by similar reasoning, with the inequalities reversed.
\end{proof}


\chapter{Measure Theory}
\section{Outer Measure}
\begin{dfn}
    Let \( I \) be a basis element of the standard topology on \( \mathbb{R} \),
    where basis elements are intervals of the form
    \( (a,b) \), \( (-\infty,b) \), \( (a,\infty) \), or \( \mathbb{R} = (-\infty,\infty) \).
    We define the \vocab{length} of \( I \) to be
    \[
        \ell(I) = \begin{cases}
            b - a & \text{if } I = (a,b), \\
            0     & \text{if } I = \varnothing, \\
            \infty & \text{otherwise}.
        \end{cases}
    \]
\end{dfn}
\begin{dfn}
    Let \( A \subset \mathbb{R} \). The \vocab{outer measure} of \( A \), denoted 
    \( m^{*}(A) \), is defined by
    \[
        m^{*}(A) = \inf \left\{ \sum_{k=1}^{\infty} \ell(I_{k}) 
        \ \middle|\ 
        A \subseteq \bigcup_{k=1}^{\infty} I_{k},\ I_{k} \text{ are open} \right\}.
    \]
\end{dfn}






\part{Complex Analysis}
\label{Complex Analysis}
\parttoc
\chapter{Complex Numbers}

\begin{theorem}[Euler's Formula]
\[ e^{i\theta} = \cos(\theta) + i\sin(\theta) \]
\end{theorem}

\begin{proof}
Consider the power series representation of $e^x$:
\[ e^x = \sum_{j=0}^{\infty} \frac{x^j}{j!} \]

Substituting $x = i\theta$, we have
\begin{align*}
e^{i\theta} &= \sum_{j=0}^{\infty} \frac{(i\theta)^j}{j!} \\
&= \sum_{j=0}^{\infty} \frac{(i\theta)^{2j}}{(2j)!} + \sum_{j=0}^{\infty} \frac{(i\theta)^{2j+1}}{(2j+1)!} \tag{separating even and odd terms} \\
&= \sum_{j=0}^{\infty} \frac{i^{2j}\theta^{2j}}{(2j)!} + \sum_{j=0}^{\infty} \frac{i^{2j+1}\theta^{2j+1}}{(2j+1)!} \\
&= \sum_{j=0}^{\infty} \frac{(-1)^j\theta^{2j}}{(2j)!} + i\sum_{j=0}^{\infty} \frac{(-1)^j\theta^{2j+1}}{(2j+1)!} \tag{since $i^{2j} = (-1)^j$ and $i^{2j+1} = i(-1)^j$} \\
&= \cos(\theta) + i\sin(\theta)
\end{align*}

The last equality follows from the Taylor series expansions:
\[ \cos(\theta) = \sum_{j=0}^{\infty} \frac{(-1)^j\theta^{2j}}{(2j)!} \quad \text{and} \quad \sin(\theta) = \sum_{j=0}^{\infty} \frac{(-1)^j\theta^{2j+1}}{(2j+1)!} \]

Note that this manipulation is justified by the absolute convergence of the exponential series for all complex numbers.
\end{proof}










\chapter{Complex Differentiation}
\section{The Complex Derivative and the Cauchy-Riemann equations}
\begin{dfn}
    Suppose that \( U \) is an open subset of \( \mathbb{C} \) and \( f: U \to \mathbb{C} \). 
    We say that \( f \) is \vocab{differentiable} at \( z_{0} \in U \) if 
    \[
        \lim_{h \to 0} \frac{f(z_{0} + h) - f(z_{0})}{h}
    \]
    exists and is finite. We denote this limit by \( f'(z_{0}) \).
\end{dfn}

We can consider \( f(z) \) has having a real and imaginary part. That is, \( f(z) = u \left( x,y \right) + i v(x,y) \) where \( u, v: \mathbb{R}^{2} \to \mathbb{R} \) and \( z=x+iy \).

\begin{theorem}[The Cauchy-Riemann Equations]
    Let $f: U \to \mathbb{C}$ where $U$ is an open subset of $\mathbb{C}$, and write $f(z) = u(x,y)+i v(x,y)$ with $z=x+iy$. Suppose $u$ and $v$ have continuous partial derivatives in $U$. Then $f$ is complex differentiable at $z$ if and only if the \vocab{Cauchy-Riemann equations} hold at $z$:
    \[ \pdv{u}{x} = \pdv{v}{y}, \quad \pdv{u }{y} = - \pdv{v }{x} \]
\end{theorem}
\begin{proof}
    $(\Rightarrow)$ If $f$ is complex differentiable at $z$, then 
    \[ f'(z) = \lim_{\Delta z \to 0} \frac{f \left( z + \Delta z\right) - f(z)}{\Delta z} \] 
    exists and is finite. Letting $\Delta z = \Delta x + i \Delta y$, we have 
    \[ f'(z) = \lim_{ \left( \Delta x, \Delta y \right) \to  \left( 0,0 \right)} \frac{u \left( x + \Delta x , y + \Delta y \right) + i v \left( x + \Delta x, y + \Delta y \right) - u (x,y) -iv (x,y)}{\Delta x+i \Delta y} \]
    Since the limit is independent of the path chosen, we first choose the path with $\Delta y = 0$:
    \[ f'(z) = \lim_{\Delta x \to 0}  \frac{u \left( x + \Delta x , y \right) + i v \left( x + \Delta x,y\right) - u (x,y) -iv (x,y)}{ \Delta x}  \]
    Separating real and imaginary parts:
     \[ f'(z) = \lim_{\Delta x \to 0}  \frac{u \left( x + \Delta x , y \right) - u (x,y)}{ \Delta x} + i \lim_{\Delta x \to 0} \frac{v \left( x + \Delta x , y \right) - v (x,y)}{ \Delta x} \]
     Therefore:
     \[ \boxed{ f'(z) = \pdv{u }{x} + i \pdv{v }{x}} \]
     Now choosing the path with $\Delta x = 0$:
      \[ f'(z) = \lim_{\Delta y\to 0}  \frac{u \left( x, y + \Delta y  \right) + i v \left( x, y + \Delta y\right) - u (x,y) - iv (x,y)}{i \Delta y} \]
    Factoring out $\frac{1}{i} = -i$:
      \[ f'(z) = \lim_{\Delta y\to 0} \left( -i \right) \left[ \frac{u \left( x, y + \Delta y  \right) - u (x,y) }{ \Delta y}  + i \frac{ v \left( x, y + \Delta y\right)- v (x,y)}{\Delta y} \right] \] 
      Therefore:
      \[ \boxed{f'(z) = -i \pdv{u}{y} + \pdv{v }{y}} \]
      Since both expressions equal $f'(z)$, we have:
      \[ \pdv{u }{x} + i \pdv{v }{x} = \pdv{v }{y} - i \pdv{u}{y} \]
      Equating real and imaginary parts:
       \[ \boxed{\pdv{u}{x} = \pdv{v}{y}, \quad \pdv{v }{x} = - \pdv{u }{y}} \]
       
    $(\Leftarrow)$ Now suppose the Cauchy-Riemann equations hold and $u, v$ have continuous partial derivatives. We need to show that 
    \[ \lim_{\Delta z \to 0} \frac{f(z + \Delta z) - f(z)}{\Delta z} \]
    exists. Let $\Delta z = \Delta x + i\Delta y$. Since $u$ and $v$ are differentiable (having continuous partial derivatives), we can write:
    \begin{align*}
    u(x+\Delta x, y+\Delta y) - u(x,y) &= \pdv{u }{x} \Delta x + \pdv{u}{y} \Delta y + \epsilon_1(\Delta x, \Delta y)|\Delta z|\\
    v(x+\Delta x, y+\Delta y) - v(x,y) &= \pdv{v }{x} \Delta x + \pdv{v }{y}\Delta y + \epsilon_2(\Delta x, \Delta y)|\Delta z|
    \end{align*}
    where $\epsilon_1, \epsilon_2 \to 0$ as $(\Delta x, \Delta y) \to (0,0)$.
    
    Therefore:
    \begin{align*}
    \frac{f(z + \Delta z) - f(z)}{\Delta z} &= \frac{u(x+\Delta x, y+\Delta y) - u(x,y) + i \left(  v(x+\Delta x, y+\Delta y) - v(x,y) \right)}{ \Delta x + i \Delta y}\\ 
    &= \frac{\pdv{u }{x} \Delta x + \pdv{u}{y} \Delta y + \epsilon_1(\Delta x, \Delta y)|\Delta z| + i \left(  \pdv{v }{x} \Delta x + \pdv{v }{y}\Delta y + \epsilon_2(\Delta x, \Delta y)|\Delta z| \right)}{\Delta x + i \Delta y}\\
     &= \frac{\left( \pdv{u }{x} + i \pdv{v }{x} \right)\Delta x + \left( \pdv{u }{y} + i \pdv{v }{y} \right)\Delta y + \epsilon(\Delta z)|\Delta z|}{\Delta x + i\Delta y}
    \end{align*}
    where $\epsilon(\Delta z) = \epsilon_{1} (\Delta x, \Delta y) + i \epsilon_{2} \left( \Delta x, \Delta y \right) \to 0$ as $\Delta z =\Delta x + i \Delta y \to 0$.
    
    Using the Cauchy-Riemann equations $\pdv{u}{x} = \pdv{v}{y}$ and $\pdv{u}{y} = -\pdv{v}{x}$:
    \begin{align*}
    \frac{f(z + \Delta z) - f(z)}{\Delta z} &= \frac{\left( \pdv{u}{x} + i\pdv{v}{x} \right)\Delta x +\left(  -\pdv{v}{x} + i\pdv{u}{x} \right)\Delta y + \epsilon(\Delta z)|\Delta z|}{\Delta x + i\Delta y}\\
    &= \frac{\left( \pdv{u}{x} + i\pdv{v}{x} \right)(\Delta x + i\Delta y) + \epsilon(\Delta z)|\Delta z|}{\Delta x + i\Delta y}\\
    &= \frac{\left( \pdv{u}{x} + i\pdv{v}{x} \right)(\Delta x + i\Delta y) }{\Delta x + i\Delta y} + \frac{\epsilon(\Delta z)|\Delta z|}{\Delta z}\\
    &= \pdv{u}{x} + i\pdv{v}{x} + \epsilon(\Delta z)\frac{\abs{\Delta z}}{\Delta z}
    \end{align*}
    
    Since $ \abs{ \frac{\abs{\Delta z}}{\Delta z}}= 1$, we have:
    \[ \lim_{\Delta z \to 0}\left|\frac{f(z + \Delta z) - f(z)}{\Delta z} - \left(\pdv{u }{x} + i \pdv{v }{x}\right)\right| =  \lim_{\Delta z \to 0}|\epsilon(\Delta z)| = 0 \]
    
    Therefore $f'(z)  = \pdv{u}{x} + i\pdv{v}{x}$ exists.
    

\end{proof}
    \textbf{Geometric interpretation:} The Jacobian matrix of $f$ viewed as a real function $(x,y) \mapsto (u(x,y), v(x,y))$ is:
    \[ J_f = \begin{bmatrix}
        \pdv{u}{x} & \pdv{u}{y}\\ \\
        \pdv{v}{x} & \pdv{v}{y}
    \end{bmatrix} = \begin{bmatrix}
        \pdv{u}{x} &-\pdv{v}{x}\\ \\
        \pdv{v}{x} & \pdv{u}{x}
    \end{bmatrix} \]
    This matrix represents multiplication by the complex number $\pdv{u}{x} + i\pdv{v}{x} = f'(z)$, showing that complex differentiability corresponds to similarity transformations (combinations of scaling and rotation) in the plane.


\part{Group Theory}
\label{Group Theory}
\parttoc
\chapter{Groups: Definitions and Examples}

\section{Axioms and Basic Results}
\begin{dfn}
    Let \( G \) be a non-empty set. \( \str: G \times G \to G \) is a \vocab{binary operation} and we write \(a  \str b \) instead of \( \str \left( a,b \right) \). \( G \) is called a \vocab{group} if the following hold:
    \begin{enumerate}[label=\textbf{\roman*)}]
        \item \( G  \) is \textbf{associative}. For every \( a,b,c \in G \), we have \(  \left(  a \str b \right) \str c = a \str \left( b \str c \right)\). 
        \item There exists an element \( e \in G\) called the \vocab{identity element} with the property that every \( g \in G \), we have \( e \str g = g \str e = g. \)
        \item For every \( g \in G \), there is an associated element \( g^{-1} \) called the \vocab{inverse} with the property that \( g \str g^{-1}= g^{-1} \str g = e. \)
    \end{enumerate}
If for every \( g,h \in G \), we have \( g \str h = h \str g \), we say the group is \vocab{abelian} and the operations is commutative.
\end{dfn}

\begin{example}
    \( \mathbb{Z} ,\mathbb{Q}, \mathbb{R},\) and \( \mathbb{C} \) are all groups under addition with \( e =0 \) and \( x^{-1}=-x \).
\end{example}

\begin{lemma}
    The identity of a group \( G \) is unique.
\end{lemma}
\begin{proof}
    Suppose that \( G \) is a group and \( e \) as well as \( e' \) are both identity elements of \( G \). Then 
    \begin{align*}
        e &= e \str e' \tag{Since $e'$ is an identity element.}\\
        &= e' \tag{Since $e$ is an identity element.}
    \end{align*}
    
\end{proof}

\begin{lemma}
    For every \( g \in G \), \( g^{-1} \) is unique.
\end{lemma}
\begin{proof}
    Very similar to the previous proof, we will assume that \( g^{-1} \) and \( g'^{-1} \) are both inverse to \( g \). So we have 
\begin{align*}
    g^{-1} &= g^{-1} \str e \\
    &= g^{-1} \str \left( g \str g'^{-1} \right)\\
    &= \left( g^{-1} \str g\right) \str g'^{-1}\\
    &= g'^{-1}
\end{align*}

\end{proof}

\begin{corollary}
        \( \left( g^{-1} \right)^{-1} =g.\)
\end{corollary}

\begin{lemma}
    For every \( g,h \in G \), \( \left( g \str h \right) ^{-1} = h^{-1} \str g^{-1}.\)
\end{lemma}
\begin{proof}
    \begin{align*}
        \left( g \str h \right) \str \left( g \str h \right) ^{-1} & = e \tag{By definition}\\
        g^{-1} \str  \left[ \left( g \str h \right) \str \left( g \str h \right) ^{-1} \right] & = g^{-1} \str e  \tag{Applying $g^{-1}$ to both sides}\\
        \left( g^{-1} \str g \right) \str h  \str \left( g \str h \right) ^{-1} & = g^{-1} \tag{By associativity}\\
        h  \str \left( g \str h \right) ^{-1} & = g^{-1} \\
        h^{-1} \str \left[ h  \str \left( g \str h \right) ^{-1}  \right] & =h^{-1} \str g^{-1}\\
        \left( h^{-1} \str h \right) \str \left( g \str h \right) ^{-1}   & =h^{-1} \str g^{-1}\\
        \left( g \str h \right) ^{-1} &= h^{-1} \str g^{-1}.
    \end{align*}
\end{proof}

\begin{exercise}
    Suppose that \( G \) and \( H \) are groups. Then \( G \times H \) can be made into a group with 
    \[ \left( g_{1}, h_{1} \right) \str_{G \times  H} \left( g_{2}, h_{2} \right) :=  \left( g_{1} \str_{G}g_{2}, h_{1} \str_{H} h_{2}\right).\]
\end{exercise}
\begin{solution}
    We will verify each of the group axioms. 
    \begin{enumerate}[label=\textbf{\roman*)}]
        \item For associativity, we have 
        \begin{align*}
            \left[ \left( g_{1}, h_{1} \right) \str_{G \times H} \left( g_{2}, h_{2} \right)\right] \str_{G \times H} \left( g_{3}, h_{3} \right) &=  \left( g_{1} \str_{G}g_{2}, h_{1} \str_{H} h_{2}\right)\str_{G \times H} \left( g_{3}, h_{3} \right)\\
            &= \left( \left(  g_{1} \str_{G}g_{2} \right) \str_{G} g_{3},  \left( h_{1} \str_{H} h_{2} \right) \str_{H} h_{3}\right)\\
            &= \left( g_{1} \str_{G} \left( g_{2} \str_{G}g_{3} \right), h_{1} \str_{H} \left( h_{2}\str_{H}h_{3} \right)  \right)\\
            &= \left( g_{1},h_{1} \right) \str_{G \times H} \left( g_{2} \str_{G} g_{3}, h_{2} \str_{H} h_{3} \right)\\
            &= \left( g_{1}, h_{1} \right) \str_{G \times H} \left[ \left( g_{2}, h_{2} \right) \str_{G \times H} \left( g_{3}, h_{3} \right)\right]
        \end{align*}
        \item We choose the identity element to be \( \left( e_{G},e_{H} \right) \). Then 
        \[ \left( g,h \right) \str_{G \times H} \left( e_{G}, e_{H} \right) =\left( g \str_{G} e_{G}, h \str_{H} e_{H} \right) = \left( g,h \right)\]
        \item For the inverse element of \( \left( g,h \right) \), we choose \( \left( g^{-1},h^{-1} \right) \). 
        \[ \left( g,h \right) \str_{G \times H} \left( g^{-1}, h^{-1} \right) =\left( g \str_{G} g^{-1}, h \str_{H} h^{-1} \right) = \left( e_{G},e_{H} \right). \]
    \end{enumerate}
\end{solution}

\begin{lemma}
    If \( G \) is a group and \( g,h \in G \). The equations \( g \str x =h \) and \( y \str g =h \) have unique solutions in \( G \).
\end{lemma}
\begin{proof}
    We can very quickly see 
    \[ x= g^{-1} \str h \quad \text{ and } \quad y = h \str g^{-1} .\]
\end{proof}

\begin{dfn}
    For a group \( G \), we define the \vocab{order} of an element \( g \in G\) to be the smallest positive integer \( n \) for which \( g^{n} =e \). We will denote this by \( \abs{g}=n \). If no such integer exists, we say that \( g \) has infinite order. 
\end{dfn}

\begin{exercise}
    If \( x^{2} =e \) for every \( x \in G \), then \( G \) is abelian.
\end{exercise}
\begin{solution}
    If \( G \) has one or two elements, the result is trivial. So we will assume \( G \) has at least three elements. Pick any non-identity elements \( x,y \in G \). 
    \begin{align*}
        \left( x \str y \right) \str \left( x \str y \right) &=e \\
        \left( x \str y \right) \str \left( x \str y \right) \str y &= e \str y\\
        \left( x \str y \right) \str x \str \left( y \str y \right) &= y\\
        \left( x \str y \right) \str x \str \left( e \right) &= y\\
        \left( x \str y \right) \str x &= y \\
         \left[ \left( x \str y \right) \str x   \right]\str x &= y \str x \\
         \left( x \str y \right) \str \left( x \str \right) &= y \str x\\
         x \str y &= y \str x
    \end{align*}
The last equation is what was required to prove.
\end{solution}




\section{Symmetric Groups}
\begin{theorem}
    Let \( \Omega \) be any non-empty set. Then the set 
    \[ S_{\Omega}= \left\{f: \Omega \to \Omega: f \text{ is a bijection}. \right\} \]
is a group under the operation of function composition.
\end{theorem}
\begin{proof}
    The composition of bijective functions is a bijective function and composition is associative. We take the identity element of \( S_{\Omega} \) to be \( \mathrm{Id}_{\Omega} \). For the inverse of \( f \in S_{\Omega}, \) we take \( f^{-1} \).
\end{proof}

The above group is called the \vocab{symmetric group} on \( \Omega \). For the rest of this section, we will take the special case that \( \Omega = \left\{ 1,2, \dots n \right\} \). This symmetric group is called the \vocab{symmetric group of degree \( n \)} and is denoted by \( S_{n} \).

\begin{theorem}
    \( \abs{S_{n}} =n! \)
\end{theorem}
\begin{proof}
    Let \( \sigma \in S_{n} \). Then there are \( n \) choices for \( \sigma(1) \), \( n-1 \) choices for \( \sigma(2) \) and in general there are \( n+1-j \) choices for \( \sigma \left( j \right) \). Multiplying all these choices together, we get
    \[ \abs{S_{n}} = \prod_{j=1}^{n } \left( n+1-j \right) =n! \]
\end{proof}

\begin{example}
    We will use this example to motive the need for cycle notation as well as how to write a permutation in cycle notation.\\
    Let \( n =13 \) and \( \sigma \in S_{13} \) be given by 
    \begin{align*}
        &\sigma(1)= 12, &&\sigma(2)=13, &&&\sigma(3)=3, &&&&\sigma(4)=1, &&&&&\sigma(5)=11,\\
        &\sigma(6)=9, &&\sigma(7)=5, &&&\sigma(8) = 10, &&&&\sigma(9)=6, &&&&&\sigma(10)=4,\\
        &\sigma(11)=7, &&\sigma(12)=8, &&&\sigma(13)= 2
    \end{align*}
This is quite cumbersome to write and it is difficult to deduce information quickly from reading this. That is why we use cycle notation.\\
To convert a permutation into a cycle, we begin with the smallest element not yet in a cycle and then we follow where the permutation sends this element and repeat until we end up back to where we started. For this example, we have 
\[ 1 \to 12 \to 8 \to 10 \to 4 \to 1 .\]
So this cycle is written as 
\[ \left( 1 \quad  12\quad  8 \quad  10 \quad   4 \right) . \]
We repeat this process with the next smallest element that is not in a previous cycle until all elements are accounted for. Then we write all cycles next to each other. In this example, we have 
\[ \sigma = \left( 1 \quad  12\quad  8 \quad  10 \quad   4 \right) \left( 2 \quad 13 \right) \left( 3 \right) \left( 5 \quad 11 \quad 7 \right) \left( 6 \quad 9 \right).\]
As a final step, we will omit mention of fixed points since they can be understood by their absence. So finally, we have
\[ \sigma = \left( 1 \quad  12\quad  8 \quad  10 \quad   4 \right) \left( 2 \quad 13 \right)  \left( 5 \quad 11 \quad 7 \right) \left( 6 \quad 9 \right).\]
By construction, it is easy to find the inverse of permutation in cycle notation. We see that 
\[ \sigma^{-1}= \left( 4 \quad  10 \quad  8 \quad  12 \quad   1 \right) \left( 13 \quad 2 \right)  \left( 7 \quad 11 \quad 5 \right) \left( 9 \quad 6 \right).\]
\end{example}

\begin{example}
    Let \( \sigma, \tau \in S_{3} \) be given by 
    \[ \sigma = \left( 1 \ \  2 \right) \quad \quad \tau = \left( 1 \ \  3 \right) .\]
    We can compose by simply following the elements under consecutive permutations. For example, if we we wish to find \( \tau \circ \sigma \), we start with \( 1 \) and see that it gets sent to \( 2 \) by \( \sigma \). And we see the that \( \tau \) fixes \( 2 \). So \( \tau \circ \sigma \) sends \( 1 \) to \( 2 \). Now \( \sigma \) sends \( 2 \) to \( 1 \) and \( \tau \) sends \( 1 \) to \( 3 \). So \( \tau \circ \sigma \) sends \( 2 \) to \( 3 \). So we have 
    \[ \tau \circ \sigma = \left( 1 \ \ 2 \ \ 3 \right) .\]
We can do the same for \( \sigma \circ \tau \) and get 
\[ \sigma \circ \tau = \left( 1 \ \ 3 \ \ 2 \right). \]
In this example, we have shown that \( S_{3} \) is not abelian.
\end{example}

\begin{theorem}
    For all \( n \ge 3 \), \( S_{n} \) is non-abelian.
\end{theorem}
\begin{proof}
    Take the element \( \sigma \) that exchanges \( 1 \) and \( 2 \) and leaves everything else fixed and the element \( \tau \) that exchanges the elements \( 1 \) and \( 3 \) and leaves everything else fixed. By the above example, we know that 
    \[ \tau \circ \sigma \neq \sigma \circ \tau. \]
\end{proof}



\section{Dihedral Group}
\begin{example}
Consider an equilateral triangle with vertices colored {\color[HTML]{0000ff} \textbf{blue}}, {\color[HTML]{ff0000} \textbf{red}}, and {\color[HTML]{008000} \textbf{green}}, positioned at labels \( 1, 2 \), and \( 3 \), respectively.

\begin{center}
    \includegraphics[width=0.2\textwidth]{figures/algebra/group_theory/initialtriangle.png}
    \label{fig:initial-triangle}
\end{center}

Using the previously established cycle notation, we impose a group structure on the symmetries of the triangle. The cycle \( \left( a \ \ b \ \ c \right) \) means that the vertex at position \( a \) moves to position \( b \), the vertex at position \( b \) moves to position \( c \), and the vertex at position \( c \) moves to position \( a \).

For example, applying the cycle \( \left( 1 \ \ 2 \right) \) to the triangle:
\begin{itemize}
    \item the {\color[HTML]{0000ff} \textbf{blue vertex}} at position 1 moves to position 2,
    \item the {\color[HTML]{ff0000} \textbf{red vertex}} at position 2 moves to position 1,
    \item the {\color[HTML]{008000} \textbf{green vertex}} at position 3 remains fixed (as it is omitted from the cycle).
\end{itemize}

The resulting triangle appears as follows:
\begin{center}
    \includegraphics[width=0.2\textwidth]{figures/algebra/group_theory/triangle(12).png}
\end{center}

This visual framework aligns naturally with the group operation in the symmetric group. Suppose that after applying \( \left( 1 \ \ 2 \right) \), we then apply the cycle \( \left( 1 \ \ 3 \ \ 2 \right) \). Then:
\begin{itemize}
    \item the {\color[HTML]{ff0000} \textbf{red vertex}}, now at position 1, moves to position 3,
    \item the {\color[HTML]{008000} \textbf{green vertex}} at position 3 moves to position 2,
    \item the {\color[HTML]{0000ff} \textbf{blue vertex}} at position 2 moves back to position 1.
\end{itemize}

The final configuration is:
\begin{center}
    \includegraphics[width=0.2\textwidth]{figures/algebra/group_theory/triangle(23).png}
\end{center}

Note that this is the same result as applying the cycle \( \left( 2 \ \ 3 \right) \) to the \hyperref[fig:initial-triangle]{original triangle}. Indeed,
\[
\left( 1 \ \ 3 \ \ 2 \right)\left( 1 \ \ 2 \right) = \left( 2 \ \ 3 \right).
\]
We can continue this with every possible symmetry of the equilateral triangle to get the following multiplication table:
\begin{center}
    \includegraphics[width=\textwidth]{figures/algebra/group_theory/Dihedral3Table.png}
    \label{fig:Dihedral3Table}
\end{center}
\end{example}


Before we explore the dihedral group in general, let us work through another example: the symmetries of the square. 

\begin{example}
Consider a square with vertices colored 
{\color[HTML]{0b4dff} \textbf{blue}}, 
{\color[HTML]{ff0b33} \textbf{red}}, 
{\color[HTML]{34ff0b} \textbf{green}}, and 
{\color[HTML]{ff8b39} \textbf{orange}} 
and positioned at labels \( 1, 2, 3, 4 \) respectively.
\begin{center}
    \includegraphics[width=0.2\textwidth]{figures/algebra/group_theory/initialsquare.png}
\end{center}

As we will show in generality later, we can generate every element of the dihedral group using a rotation and a flip. It does not matter which flip we choose, so let us pick the \emph{vertical flip} (swapping the left and right vertices), represented by the permutation \( (1\ 4)(2\ 3) \). For our rotation, we will use the \(90^\circ\) clockwise rotation (sending each vertex to the next one in clockwise order), represented by \( (1\ 2\ 3\ 4) \).\\

Applying \( (1\ 2\ 3\ 4) \) to our square has the following effect:
\begin{itemize}
    \item the {\color[HTML]{0b4dff} \textbf{blue}} vertex at position 1 moves to position 2,
    \item the {\color[HTML]{ff0b33} \textbf{red}} vertex at position 2 moves to position 3,
    \item the {\color[HTML]{34ff0b} \textbf{green}} vertex at position 3 moves to position 4, and
    \item the {\color[HTML]{ff8b39} \textbf{orange}} vertex at position 4 moves to position 1.
\end{itemize}
\begin{center}
    \includegraphics[width=0.7\textwidth]{figures/algebra/group_theory/square1234.png}
\end{center}

Similarly, applying \( (1\ 4)(2\ 3) \) (the vertical flip) gives:
\begin{itemize}
    \item the {\color[HTML]{0b4dff} \textbf{blue}} vertex at position 1 moves to position 4,
    \item the {\color[HTML]{ff0b33} \textbf{red}} vertex at position 2 moves to position 3,
    \item the {\color[HTML]{34ff0b} \textbf{green}} vertex at position 3 moves to position 2, and
    \item the {\color[HTML]{ff8b39} \textbf{orange}} vertex at position 4 moves to position 1.
\end{itemize}
\begin{center}
    \includegraphics[width=0.7\textwidth]{figures/algebra/group_theory/square1423.png}
\end{center}

Much like the previous example, we can compose symmetries.  
If we apply a rotation and then a flip, we have  
\( (1\ 4)(2\ 3) \cdot (1\ 2\ 3\ 4) = (1\ 3) \),  
a reflection across the diagonal from vertex 1 to vertex 3.\\
If we first apply a flip followed by a rotation, we have  
\( (1\ 2\ 3\ 4) \cdot (1\ 4)(2\ 3) = (2\ 4) \),  
a reflection across the diagonal from vertex 2 to vertex 4.
\begin{center}
    \includegraphics[width=0.9\textwidth]{figures/algebra/group_theory/squarescompare.png}
\end{center}

By combining the rotation and the flip in different orders, we can generate all eight symmetries of the square. The complete multiplication table is shown on the next page:
\begin{itemize}
    \item The identity element has a white background. 
    \item The rotation elements have grey backgrounds with \( (1\ 2\ 3\ 4) \) having {\color[HTML]{71797e} \textbf{this color background}}, \( (1\ 3)(2\ 4) \) having {\color[HTML]{848884} \textbf{this color background}}, and \( (1\ 4\ 3\ 2) \) having {\color[HTML]{c0c0c1} \textbf{this color background}}.
    \item The reflection elements have brown backgrounds with \( (1\ 4)(2\ 3) \) having {\color[HTML]{d2b48c} \textbf{this color background}}, \( (1\ 2)(3\ 4) \) having {\color[HTML]{c2b280} \textbf{this color background}}, \( (2\ 4) \) having {\color[HTML]{988558} \textbf{this color background}}, and \( (1\ 3) \) having {\color[HTML]{6f4e37} \textbf{this color background}}.
\end{itemize}
\begin{center}
    \includegraphics[width=0.9\textwidth]{figures/algebra/group_theory/D8table.png}
\end{center}
\end{example}


\section{Matrix groups}
\input{content/algebra/group-theory/1-definition-and-examples/matrix-groups}





\chapter{Group Actions}
\epigraph{Groups, as men, will be known by their actions.}{\textit{Guillermo Moreno}}

\section{Definitions, Basic Results, and Examples}
\begin{dfn}
We say that a group \( G \) \vocab{acts} on a set \( A \) if there is a map
\[
    \cdot \colon G \times A \to A, \quad (g,a) \mapsto g \cdot a
\]
such that for all \(g_{1}, g_{2} \in G\) and \(a \in A\):
\begin{enumerate}[label=\textbf{\roman*)}]
    \item \( g_{1} \cdot \big( g_{2} \cdot a \big) = (g_{1} g_{2}) \cdot a \), where \(g_{1} g_{2}\) denotes the product in \(G\).
    \item \( e_{G} \cdot a = a \), where \(e_G\) is the identity of \(G\).
\end{enumerate}
We call this a \vocab{group action} of \( G \) on \( A \). We will denote this as \( G \acts A \).
\end{dfn}


Instead of viewing the group action as a map from \( G \times A \) to \( A \), we could actually adopt the view that a group action is a map from \( G \) to \( S_{A} \), where \( S_{A} \) is the collection of bijections on \( A \). Moreover, this map is a homomorphism! This requires proof. 

\begin{theorem}
    Let \(G\) be a group acting on a set \(A\) via a map
    \[
        \cdot \colon G \times A \to A.
    \]
    Then there is an associated map
    \[
        \varphi \colon G \to S_{A}, \quad g \mapsto \varphi \left( g \right)  \text{ such that }
        \left[ \varphi(g) \right] (a) = g \cdot a,
    \]
    where \(S_{A}\) is the symmetric group on \(A\) (the set of all bijections \(A \to A\)).
    Moreover, \(\varphi\) is a group homomorphism.

    Conversely, any group homomorphism \(\varphi \colon G \to S_{A}\) defines a group action of \(G\) on \(A\) via
    \[
        g \cdot a := \left[ \varphi(g) \right](a).
    \]
\end{theorem}
\begin{proof}
First, we will verify that for each \(g \in G\), the map \(\varphi(g)\colon A \to A\) is a bijection. To do this, it suffices to show that \(\varphi(g)\) has a two-sided inverse. The natural candidate is
\[
\left[ \varphi(g) \right]^{-1} := \varphi(g^{-1}).
\]

For any \(a \in A\), we have
\begin{align*}
\big(\varphi(g^{-1}) \circ \varphi(g)\big)(a) &= \varphi(g^{-1})\big(\varphi(g)(a)\big) \\
&= \varphi(g^{-1})(g \cdot a) && \text{(by definition of \(\varphi\))}\\
&= g^{-1} \cdot (g \cdot a) && \text{(by definition of the group action)}\\
&= (g^{-1} g) \cdot a \\
&= e \cdot a \\
&= a && \text{(identity property of the action).}
\end{align*}

Similarly, \(\varphi(g) \circ \varphi(g^{-1}) = \operatorname{id}_A\), so \(\varphi(g)\) is bijective.\\
Now to show that \( \varphi \) is a homomorphism, pick any \( g,h \in G \) and \( a \in A \), then 
\begin{align*}
   \left[  \varphi \left( gh \right) \right] \left( a \right) &= \left( gh \right) \cdot a\\
   &= g \cdot \left( h \cdot a \right)\\
   &= g \cdot \left( \varphi \left( h  \right) \left( a \right) \right)\\
   &= \varphi \left( g \right) \left( \varphi \left( h  \right)\left( a \right) \right) \\
   &= \left[ \varphi \left( g \right) \circ \varphi \left( h \right) \right] \left( a \right)
\end{align*}
which shows that \( \varphi \left( gh \right) = \varphi \left( g \right) \circ \varphi \left( h \right) \).\\
Finally, to show that any group homomorphism \( \varphi: G \to S_{A} \) defines a \( G \)-action on \( A \) by \( g \cdot a = \left[ \varphi \left( g \right) \right] \left( a \right) \), we just need to verify the axioms of a group action. For any \( g_{1}, g_{2} \in G \) and \( a \in A \), we have
\begin{align*}
    \left( g_{1} g_{2} \right) \cdot a &= \left[ \varphi \left( g_{1}g_{2} \right) \right] \left( a \right)\\
    &= \left[ \varphi \left( g_{1} \right) \circ \varphi \left( g_{2} \right) \right] \left( a \right) \tag{Since $\varphi$ is a homomorphism.}\\
    &= \varphi \left( g_{1} \right) \left( g_{2} \cdot a \right) \\
    &= g_{1} \cdot \left( g_{2} \cdot a \right)
\end{align*}
For the identity property,
\begin{align*}
    e \cdot a &= \left[ \varphi \left( e \right) \right] \left( a \right)\\
    &= \mathrm{Id}_{A} \left( a \right) \tag{Since $\varphi$ is a homomorphism.}\\
    &= a
\end{align*}
which completes the proof.
\end{proof}



The above result highlights why group theory is such a powerful tool. Any group action corresponds to a subgroup of the group of all bijections on \(A\). This means that groups provide a powerful angle of attack for tackling and understanding symmetries of any set. When we restrict our attention to bijections preserving additional structure, such as homeomorphisms in topology, biholomorphisms in complex analysis, or linear transformations in linear algebra (this is the focus of representation theory), the same framework applies, giving us a systematic way to understand and manipulate these transformations through their group properties.

\begin{dfn}
    If the homomorphism, \( \varphi: G \to S_{A} \) is injective, we say that the associated group action of \( G \) on \( A \) \vocab{acts faithfully}.
\end{dfn}

\begin{lemma}
    We define the \vocab{kernel} of a \( G \) action on \( A \) to be 
    \[ \mathrm{Ker}\left( G \acts A \right)=\left\{ g \in G \ \middle| \ g \cdot a =a \text{ for all } a \in A\right\} \]
    Then \(\mathrm{Ker}\left( G \acts A \right) \unlhd G \).
\end{lemma}
\begin{proof}
    This fact is readily apparent from the result that if \( \varphi: G \to S_{A} \) is a homomorphism, then \( \mathrm{Ker} \left( \varphi \right) \) is a normal subgroup of \( G \). However, we will prove this result using the group action perspective for practice. \\
    First we will show that \( \mathrm{Ker} \left( G \acts A \right) \) is a subgroup. Since \( G \acts A \), \( e \in \mathrm{Ker} \left( G \acts A \right) \). Pick any \( g,h \in \mathrm{Ker} \left( G \acts A \right) \). Then 
    \begin{align*}
        a &= g \cdot a \tag{Since $g\in \mathrm{Ker} \left( G \acts A \right)$}\\
        &= g \cdot \left( e \cdot a \right) \\
        &= g \cdot \left( \left( h^{-1} h \right) \cdot a \right) \\
        &= g \cdot \left( h^{-1} \cdot \left( h \cdot a \right)  \right) \\
        &= g \cdot \left( h^{-1} \cdot a \right)  \tag{Since $h\in \mathrm{Ker} \left( G \acts A \right)$} \\
        &= \left( gh^{-1} \right) \cdot a
    \end{align*}
   This shows that \( gh^{-1} \in \mathrm{Ker} \left( G \acts A \right) \). By the subgroup test, \( \mathrm{Ker} \left( G \acts A \right) \le G \). \\
   Now to show that \( \mathrm{Ker} \left( G \acts A \right) \unlhd G \), pick any \( g \in  \mathrm{Ker} \left( G \acts A \right) \), \( h \in G \), and \( a \in A \). Then 
   \begin{align*}
    \left( hgh^{-1} \right) \cdot a &= \left( hg \right) \cdot \left( h^{-1} \cdot a \right)\\
    &= h \cdot \left( g \cdot \left( h^{-1}  \cdot a\right) \right) \\
    &= h \cdot \left( h^{-1} \cdot a \right) \tag{Since $g\in \mathrm{Ker} \left( G \acts A \right)$} \\
    &= \left( h h^{-1} \right) \cdot a \\
    &= e \cdot a \\
    &= a
   \end{align*}
   This shows that \( \mathrm{Ker} \left( G \acts A \right) \) is a normal subgroup of \( G \).
\end{proof}

\begin{exercise}
    Show that the kernel of a \( G \) action on \( A \) contains only the identity if and only if \( G \) acts faithfully on \( A \).
\end{exercise}
\begin{solution}
    Let $\varphi: G \to S_A$ be the homomorphism associated to the action.

    $(\Rightarrow)$ Suppose $\ker \varphi \neq \{e\}$. Then there exists $g \in G$, $g \neq e$, such that $\varphi(g) = \mathrm{id}_A$. In particular, for all $a \in A$ we have $g \cdot a = a$, so $g$ and $e$ induce the same permutation of $A$. Hence $\varphi$ is not injective, and the action is not faithful.

    $(\Leftarrow)$ Conversely, if the action is not faithful, then $\varphi$ is not injective. Thus there exist distinct $g,h \in G$ such that $\varphi(g) = \varphi(h)$. Then
    \[
        \varphi(gh^{-1}) = \varphi(g)\varphi(h)^{-1} = \varphi(h)\varphi(h)^{-1} = \mathrm{id}_A,
    \]
    so $gh^{-1} \in \ker \varphi$ and $gh^{-1} \neq e$. Therefore the kernel is nontrivial.
\end{solution}




\chapter{Direct and Semidirect Products}
\section{Semidirect Products}
\begin{theorem}
    Let \( H \) and \( K \) be groups and \( \varphi: K \to \mathrm{Aut}(H) \) be a homomorphism. Let \( \cdot \) be the left \( K \)-action on \( H \) as determined by \( \varphi \). Let \( G = H \times K \) and define a multiplication on \( G \) by
    \[ \left( h_{1}, k_{1} \right) \str_{G} \left( h_{2},k_{2} \right) = \left( h_{1} \str_{H} \left( k_{1} \cdot h_{2} \right) , k_{1} \str_{K} k_{2} \right) \]
    Then 
    \begin{enumerate}[label=\textbf{\roman*)}]
        \item The is multiplication makes \( G \) into a group called the \vocab{semidirect product} of \( H \) and \( K \) with respect to \( \varphi \) and we write \( G = H \rtimes_{\varphi} K \) or \( H \rtimes K \) if \( \varphi \) is unambiguous. 
    \end{enumerate}
\end{theorem}
\begin{proof}
    
\end{proof}


\part{Ring Theory}
\label{Ring Theory}
\parttoc

\chapter{Introduction to Rings}
\begin{dfn}
A \vocab{ring} \(R\) is a non-empty set equipped with two binary operations \(+\) and \(\times\), called addition and multiplication, respectively, such that:
\begin{enumerate}[label=\textbf{\roman*)}]
    \item \(R\) is an abelian group under addition. 
    \item Multiplication \(\times\) is associative.
    \item The left and right distributive laws hold: for all \(r,s,t \in R\),
    \[
        r \times (s+t) = (r \times s) + (r \times t) \quad \text{and} \quad (r+s) \times t = (r \times t) + (s \times t).
    \]
\end{enumerate}
The ring \(R\) is \vocab{commutative} if multiplication is commutative. It is \vocab{unitary} if there exists an identity element \(1_R \in R\) such that \(1_R \times r = r \times 1_R = r\) for all \(r \in R\). Multiplication may be written as
\[
r \times s, \quad r \cdot s, \quad \text{or simply } rs,
\]
with juxtaposition \(rs\) usually being the default.
\end{dfn}

\begin{theorem}
    Every finite integral domain is a field. 
\end{theorem}
\begin{proof}
    We need only show that every non-zero element of \( R \) has a multiplicative inverse. \\
    Suppose that \( R \) is a finite integral domain. Pick \( a \in R \) non-zero and define 
    \[ \varphi_{a}(x) := ax .\]
    Suppose that \( \varphi_{a}(x) = \varphi_{a}(y) \). Then 
    \begin{align*}
        \varphi_{a}(x) &= \varphi_{a}(y)\\
        ax &=ay  \\
        ax- ay &= 0\\
        a \left( x-y \right) &=0
    \end{align*}
    Since \( R \) is an integral domain, either \( a \) or \( x-y \) must be \( 0 \). Since \( a \) was chosen to be non-zero, \( x-y \) must be \( 0 \) or \( x=y \). Since \(    \varphi_{a}(x) = \varphi_{a}(y) \Rightarrow x =y\), \( \varphi_{a} \) is injective. Since \( R \) is finite, \( \varphi_{a} \) is also surjective and hence has a unique two sided inverse, call it \( \varphi_{b} \). Therefore 
    \[ \varphi_{b} \left( \varphi_{a} (1) \right) = \varphi_{a} \left( \varphi_{b} (1) \right) = 1 \Rightarrow ba =ab =1 \]
    So \( a  \) has a multiplicative inverse and we are done. 
\end{proof}


\section{Ideals}
Much like in group theory, we want to develop some quotient structure on rings. Since a ring is a group under addition, that can be taken care of with existing tools. So let us contend with multiplication. \\ 

Given a subset \( I \) of a ring \( R \). What conditions do we need for the quotient \( \faktor{R}{I} \) to be a ring? Let \( r+I \) and \( s+I \) be left additive cosets. We want 
\[ \left( r+I \right) \left( s+I \right) = rs +I .\]
Let us "expand" the left side. 
\[ \left( r+I \right) \left( s+I \right) = rs + rI + sI + I \cdot I \]
If we impose the conditions, that 
\begin{enumerate}[label=\textbf{\roman*)}]
    \item \( I \) is a subring of \( R \), 
    \item For all \( r \in R \), \( rI \subseteq I \),
\end{enumerate}
we quickly see that sets \( \left( r+I \right)\left( s+I \right)  \) and \( rs + I \) are equal. This is the motivation for the definition of an \vocab{ideal}. 

\begin{dfn}
    Let \( I \subseteq R \) be a subset of a ring \( R \). We say \( I \) is a \vocab{left ideal} of \( R \) if \( I \) is an additive subgroup of \( R \) closed under multiplication, and for all \( r \in R \) and \( x \in I \), we have \( rx \in I \). \\ 
    Similarly, \( I \) is a \vocab{right ideal} of \( R \) if \( I \) is an additive subgroup of \( R \) closed under multiplication, and for all \( r \in R \) and \( x \in I \), we have \( xr \in I \). \\ 
    A \vocab{two-sided ideal} of \( R \) is a subset that is both a left ideal and a right ideal. \\ 
    When the sidedness is clear from context or irrelevant, we will simply use the term \vocab{ideal}.
\end{dfn}

\part{Point-Set Topology}
\label{part: Point-Set Topology}
\parttoc
\chapter*{Why Topology?}
It's 6th period geometry. Your teacher, yet un-assaulted by chalk dust (but give it time), is talking about when we know two triangles are congruent. SSS, SAS, ASA... wait what? Doesn't she mean equal? Why are we just making up new words for equality? Is Big Math™ just making stuff up for the sake of confusion? \\

You're picking up on a subtle point. Two triangles might \emph{seem} equal if they have the exact same side lengths and angles but they aren't for what seems to be a very nit-picky but important reason, they don't occupy the same points in space. However, we have an easy fix. We can find a map called an \vocab{isometry} that maps the set of points that make one triangle onto the set of points that make up the other. Under this rephrasing, the SSS, SAS, ASA conditions are sufficient conditions that allow us to declare the existence of an isometry that maps one triangle onto another without having to explicitly find one. \\

The idea of finding a "useful" class of maps and conditions that guarantee their existence cannot possibly be more prevalent than it is in topology. However, isometries, while useful, are too rigid, they preserve too much structure. That is why, in topology, we begin with the most general class of functions that preserve spatial (topological) information called \vocab{homeomorphisms}. We will discuss conditions that guarantee their existence and spatial information that they preserve. \\ 

While we will (usually) just stick to homeomorphisms in this part, it is important to mention that homeomorphisms are \emph{too} general. A complete classification of spaces up to homeomorphism will elude us as long as Sisyphus remains at his task. So, in practice, we impose additional structure on topological spaces and restrict to maps that preserve this structure, allowing us to detect much finer invariants.\\ 

While homeomorphisms are the maps, the primary actors for which homeomorphisms act on are \vocab{topological spaces}, which are supposed to capture the notion of closeness. It has taken mathematicians many years to pick out the right abstractions of closeness. A primitive approach might be "two points are close if the distance between them is small." But that just kicks the can down the road. "The distance between two points is small if they are close together." The stroke of genius here is that closeness is a contextual property. If I give you one point, you cannot tell me anything about closeness or farness. This is the first hint that sets are the natural tool to talk about closeness. But what \emph{types} of sets? If I just hand you the full collection of points, there just isn't enough information to construct a notion of closeness. We need to pick the \emph{right} sets. \\ 

Let's see how sets can encode closeness through a game. Suppose that we play a game of darts but the bar hasn't ordered the dartboard. Like the hooligans we are, we proceed anyways. We mark a bulls-eye on the blank wall and each throw our dart a single round before the bar owner gets mad at us. Before we are kicked out of the bar, we must determine a winner. The bar owner obliges us and hands us a bunch of empty cardboard in which we can cut out circles. We make a big circle with the first cardboard and place the center on our imagined bullseye. It covers the two holes in the wall we made so we conclude that our darts landed in the interior of our circle. We make a second circle but this time it is far too small. Both holes we made lie in the exterior of our circle. The third circle is our Goldilocks circle, we see that the hole you made lies in the interior and the hole I made lies in the exterior. You win again (how do you keep winning?). Notice that we arrived at a notion of closeness \emph{without measuring or even assigning a number to anything}. Each circle defines an \vocab{open set}: the collection of all points that would fall strictly inside it. By declaring which sets are "open," we encode the information about closeness we need.

\chapter{Topological Spaces and Continuous Functions}
\section{Definition, Open Sets}
\begin{dfn}
    Let \( X \) be any set. A \vocab{topology} on \( X \) is a collection \( \mathscr{T} \subseteq \mathcal{P}(X) \) such that the following criteria hold:
    \begin{enumerate}[label=\textbf{\roman*)}]
        \item \( \varnothing, X \in \mathscr{T} \).
        \item For any collection of sets \( \left\{ U_{\alpha} \right\}_{\alpha \in \mathcal{A}} \subseteq \mathscr{T} \), we have \[ \bigcup_{\alpha \in \mathcal{A}} U_{\alpha} \in \mathscr{T}. \]
        \item For any finite collection of sets \( \left\{ U_{j} \right\}_{j=1}^n \subseteq \mathscr{T}\), we have \[ \bigcap_{j=1}^{n} U_{j} \in \mathscr{T}. \]
    \end{enumerate}
If a set \( U \subseteq X \) belongs to \( \mathscr{T} \), we call \( U \) an \vocab{open} set. If \( x \in U \) and \( U \) is open, we sometimes refer to \( U \) as an \vocab{open neighborhood} of \( x \).
\end{dfn}

The topology on the empty set is not interesting, so from now on we will assume that \( X \) is non-empty.

\begin{dfn}
    Let \( X \) be a set and let \( \mathscr{T} \) and \( \mathscr{T}' \) be two topologies on \( X \).  If \( \mathscr{T} \subseteq \mathscr{T}' \), we say that \( \mathscr{T}' \) is \vocab{finer} than \( \mathscr{T} \), or equivalently that \( \mathscr{T} \) is \vocab{coarser} than \( \mathscr{T}' \). If \( \mathscr{T} \subset \mathscr{T}' \), we say that \( \mathscr{T}' \) is \vocab{strictly finer} than \( \mathscr{T} \), or that \( \mathscr{T} \) is \vocab{strictly coarser} than \( \mathscr{T}' \). We call the topologies \( \mathscr{T} \) and \( \mathscr{T}' \) \vocab{comparable} if one is finer than the other.
\end{dfn}

\begin{example}
    For any set \( X \), there are two obvious topologies. The \vocab{indiscrete} topology which is just 
    \[ \mathscr{T}_{\mathrm{indiscrete}} =\left\{ \varnothing, X \right\}\]
    and the \vocab{discrete} topology which is just 
    \[ \mathscr{T}_{\mathrm{discrete}} = \mathcal{P} \left( X \right).\]
\end{example}

\begin{exercise}
    Let \( X = \left\{ a,b,c \right\} \). What are all the possible topologies on \( X \)?
\end{exercise}
\begin{solution}
    We have the discrete and indiscrete topologies on \( X \). 
    \[ \mathscr{T}_{\mathrm{discrete}} = \left\{ \varnothing, \left\{ a \right\}, \left\{ b \right\}, \left\{ c \right\}, \left\{ a,b  \right\}, \left\{ a,c  \right\}, \left\{ b,c \right\}, \left\{ a,b,c \right\} \right\} \]
    \[ \mathscr{T}_{\mathrm{indiscrete}} = \left\{ \varnothing, \left\{ a,b,c \right\} \right\} \]
Then we have the topologies that augment the indiscrete topology with a singleton set 
\[ \mathscr{T}_{1} = \left\{ \varnothing, \left\{ a \right\}, \left\{ a ,b, c \right\} \right\},\  \mathscr{T}_{2} = \left\{ \varnothing, \left\{ b \right\}, \left\{ a ,b, c \right\} \right\}, \  \mathscr{T}_{3} = \left\{ \varnothing, \left\{ c \right\}, \left\{ a ,b, c \right\} \right\} \]
\[ \]
We can fill out the rest 
\[ \mathscr{T}_{4} = \left\{ \varnothing, \left\{ a \right\}, \left\{ a,b \right\}, \left\{ a ,b, c \right\} \right\}, \ \mathscr{T}_{5} = \left\{ \varnothing, \left\{ a \right\}, \left\{ a,c \right\},\left\{ a ,b, c \right\} \right\}, \  \mathscr{T}_{6} = \left\{ \varnothing, \left\{ a \right\}, \left\{ b,c \right\}, \left\{ a ,b, c \right\} \right\}\]
\[ \mathscr{T}_{7} = \left\{ \varnothing, \left\{ b \right\}, \left\{ a,b \right\}, \left\{ a ,b, c \right\} \right\}, \  \mathscr{T}_{8} = \left\{ \varnothing, \left\{ b \right\}, \left\{ a,c \right\},\left\{ a ,b, c \right\} \right\}, \ \mathscr{T}_{9} = \left\{ \varnothing, \left\{ b \right\}, \left\{ b,c \right\}, \left\{ a ,b, c \right\} \right\}  \]
\[ \mathscr{T}_{10} = \left\{ \varnothing, \left\{ c \right\}, \left\{ a,b \right\}, \left\{ a ,b, c \right\} \right\}, \ \mathscr{T}_{11} = \left\{ \varnothing, \left\{ c \right\}, \left\{ a,c \right\},\left\{ a ,b, c \right\} \right\}, \  \mathscr{T}_{12} = \left\{ \varnothing, \left\{ c \right\}, \left\{ b,c \right\}, \left\{ a ,b, c \right\} \right\} \]


\[ \mathscr{T}_{13} = \left\{ \varnothing, \left\{ a \right\}, \left\{ b \right\}, \left\{ a,b \right\}, \left\{ a ,b, c \right\} \right\}, \ \mathscr{T}_{14} = \left\{ \varnothing, \left\{ a \right\}, \left\{ c \right\}, \left\{ a,c \right\}, \left\{ a ,b, c \right\} \right\}, \  \mathscr{T}_{15} = \left\{ \varnothing, \left\{ b  \right\}, \left\{ c \right\}, \left\{ b,c \right\}, \left\{ a ,b, c \right\} \right\}\]
\[ \mathscr{T}_{16} = \left\{ \varnothing, \left\{ a \right\}, \left\{a, b \right\}, \left\{ a,c \right\}, \left\{ a ,b, c \right\} \right\}, \ \mathscr{T}_{17} = \left\{ \varnothing, \left\{ b \right\}, \left\{ a,b \right\}, \left\{ b,c \right\}, \left\{ a ,b, c \right\} \right\}, \  \mathscr{T}_{18} = \left\{ \varnothing, \left\{ c  \right\}, \left\{ a,c \right\}, \left\{ b,c \right\}, \left\{ a ,b, c \right\} \right\}\]

\[ \mathscr{T}_{19} = \left\{ \varnothing, \left\{ a \right\}, \left\{ b  \right\}, \left\{a, b \right\} ,\left\{ a,c \right\}, \left\{ a ,b, c \right\} \right\}, \ \mathscr{T}_{20} = \left\{ \varnothing, \left\{ a \right\}, \left\{ c  \right\}, \left\{a, b \right\} ,\left\{ a,c \right\}, \left\{ a ,b, c \right\} \right\}\]
\[ \mathscr{T}_{21} =  \left\{ \varnothing, \left\{ a \right\}, \left\{ b  \right\}, \left\{a, b \right\} ,\left\{ b,c \right\}, \left\{ a ,b, c \right\} \right\}, \ \mathscr{T}_{22} = \left\{ \varnothing, \left\{ a \right\}, \left\{ c  \right\}, \left\{a, c \right\}, \left\{ b,c \right\}, \left\{ a ,b, c \right\} \right\}\]
\[ \mathscr{T}_{23}= \left\{ \varnothing, \left\{ b \right\}, \left\{ c \right\}, \left\{ b,c \right\}, \left\{ a,b \right\}, \left\{ a,b,c \right\} \right\}, \mathscr{T}_{24}= \left\{ \varnothing, \left\{ b \right\}, \left\{ c \right\}, \left\{ b,c \right\}, \left\{ a,c \right\}, \left\{ a,b,c \right\} \right\} \]

\[ \mathscr{T}_{25} = \left\{ \varnothing, \left\{ a,b \right\}, \left\{ a,b,c \right\} \right\}, \  T_{26} = \left\{ \varnothing, \left\{ a,c \right\}, \left\{ a,b,c \right\} \right\}, \  \mathscr{T}_{27} = \left\{ \varnothing, \left\{ b,c \right\}, \left\{ a,b,c \right\} \right\}  \]


\end{solution}

\begin{exercise}
    Let \( X \) be a set and we define the co-finite topology \( \mathscr{T}_{\mathrm{cf}} \) as follows: \( U \) is open in \( \mathscr{T}_{\mathrm{cf}} \) if and only if \( X-U \) is finite or all of \( X \). Show that this is indeed a topology.
\end{exercise}
\begin{solution}
    Clearly \( \varnothing \) and \( X \) each belong to \( \mathscr{T}_{\mathrm{cf}} \), so we will jump right into verifying closure under arbitrary unions and finite intersections. \\
    Let \( U_{\alpha \in A} \in \mathscr{T}_{\mathrm{cf}} \). We want to show that 
    \[ \bigcup_{\alpha \in A} U_{\alpha} \in \mathscr{T}_{\mathrm{cf}} \]
    or that 
    \[ X - \left(  \bigcup_{\alpha \in A} U_{\alpha} \right) \]
    is finite. We can apply one of DeMorgan's laws to the above expression to get 
    \[ X - \left(  \bigcup_{\alpha \in A} U_{\alpha} \right) = \bigcap_{\alpha \in A } \left( X - U_{\alpha} \right)\]
Since each \( U_{\alpha} \) belongs to \( \mathscr{T}_{\mathrm{cf}} \), each \( X- U_{\alpha} \) is finite. Therefore \( \bigcap_{\alpha \in A } \left( X - U_{\alpha} \right) \) is certainly finite. This establishes that \( \bigcup_{\alpha \in A} U_{\alpha} \in \mathscr{T}_{\mathrm{cf}} \).\\
Now for finite intersections, suppose that \( \{U_{1}, \dots U_{n}\} \subseteq \mathscr{T}_{\mathrm{cf}} \). We want to show that 
\[ \bigcap_{j=1}^{n} U_{n} \in \mathscr{T}_{\mathrm{cf}}\] or 
\[ X - \left( \bigcap_{j=1}^{n} U_{j} \right) \]
is finite. Again, we apply one of DeMorgan's laws to get 
\[ X - \left( \bigcap_{j=1}^{n} U_{j} \right) = \bigcup_{j=1}^{n } \left( X- U_{j}\right).\]
Since each \( U_{j} \in \mathscr{T}_{\mathrm{cf}} \), each \( X-U_{j} \) is finite. This implies that \( \bigcup_{j=1}^{n }\left( X-U_{j} \right) \) is finite so \( \bigcap_{j=1}^{n }U_{j} \in \mathscr{T}_{\mathrm{cf}}\). This concludes the proof.
\end{solution}

\begin{dfn}
    Let \( X \) be a topological space. The \vocab{interior} of a set \( A \) of \( X \), denoted by \( \mathrm{Int} \left( A \right) \) is defined to be the set: 
    \[ \mathrm{Int} \left( A \right) = \bigcup \left\{ U \subseteq A \  \middle| \ U \text{ is an open set.} \right\} \]
    In other words, \( \mathrm{Int}(A) \) is the union of all open sets contained in \( A \). \\ 
    A point \( x \) of \( A \) is called an \vocab{interior point} of \( A \) it is a member of \( \mathrm{Int} \left( A \right) \).
\end{dfn}

\begin{dfn}
    Let \( X \) be a topological space. The \vocab{exterior} of a set \( A \), denoted by \( \mathrm{Ext}(A) \) is defined to be 
    \[ \mathrm{Ext} \left( A \right) = \mathrm{Int} \left( X-A \right) .\]
\end{dfn}

\begin{dfn}
    The \( X \) be a topological space. The \vocab{boundary} of a set \( A \), denoted as \( \partial \left( A \right) \), is the set 
    \[ \partial \left( A \right) = X - \left( \mathrm{Int} \left( A \right) \cup \mathrm{Ext} \left( A \right) \right).\]
\end{dfn}

\begin{lemma}\label{thm:open-set-equals-its-interior}
    A set \( U \) of a topological space \( X \) is open if and only if \( U = \mathrm{Int}(U) \).
\end{lemma}
\begin{proof}
    \( (\Rightarrow) \) Suppose that \( U \) is open. We want to show that \( U = \mathrm{Int}(U) \). \( \mathrm{Int}(U) \subseteq U \) is obvious by definition so we just need to show that \( U \subseteq \mathrm{Int}(U) \). Since \( \mathrm{Int}(U) \) is the union of all open subsets of \( U \) and \( U \) is open, it follows that \(U \subseteq  \mathrm{Int}(U) \). \\ 
    \( (\Leftarrow )\) Suppose \( \mathrm{Int}(U) = U \). Since \( \mathrm{Int}(U) \) is the union of all open subsets of \( U \), \( \mathrm{Int} \left( U \right) \) is open and hence, \( U \) is open.
\end{proof}

\begin{corollary}
    The exterior of a set \( A \) in a topological space is an open set. 
\end{corollary}












\section{Closed Sets}
\begin{dfn}
    Let \( X \) be a topological space. A set \( C \subseteq X \) is said to be \vocab{closed} in \( X \) if \( X - C \) is open in \( X \).
\end{dfn}

\begin{theorem}
    The following are equivalent: 
    \begin{enumerate}[label=\textbf{\Roman*)}]
        \item There is a set \( \mathscr{C} \subseteq \mathcal{P} \left( X \right) \) such that: 
        \begin{enumerate}[label=\textbf{\roman*)}]
            \item \( \varnothing, X \in \mathscr{C} \)
            \item For any arbitrary collection \( \left\{ C_{\alpha} \right\}_{\alpha \in A} \subseteq  \mathscr{C} \), \( \bigcap_{\alpha \in A } C_{\alpha} \in \mathscr{C} \)
            \item For any finite collection \( \left\{ C_{k} \right\}_{k=1}^{n} \), \( \bigcup_{k=1 }^{n } C_{k} \in \mathscr{C} \)
        \end{enumerate}
        \item There is a topology on \( X \) whose collection of closed sets is precisely \( \mathscr{C} \).
    \end{enumerate}
    This is to say that we could have defined a topology on \( X \) in terms of closed sets.
\end{theorem}
\begin{proof}
   \( \left(  \Rightarrow  \right) \) Assume the conditions of \textbf{(I)} and define 
   \[ \mathscr{T} = \left\{ U \in \mathcal{P} \left( X \right) \ \middle| \ X-U \in \mathscr{C} \right\} \]
   We want to show that \( \mathscr{T} \) is indeed a topology. \\ 
   \( \varnothing \in \mathscr{T}\) since \( X = X- \varnothing \in \mathscr{C} \) and similarly \( X \in \mathscr{T} \) since \( \varnothing = X-X \in \mathscr{C} \). \\ 
   Pick some arbitrary subcollection \( \left\{ U_{\alpha} \right\}_{\alpha \in A} \) of \( \mathscr{T} \). We want to show that \(  \bigcup_{\alpha \in A} U_{\alpha} \in \mathscr{T} \) or, equivalently, \( X - \left(  \bigcup_{\alpha \in A} U_{\alpha} \right) \in \mathscr{C} \)
   \begin{align*}
    X - \left(  \bigcup_{\alpha \in A} U_{\alpha} \right)  &= \bigcap_{\alpha \in A} \left( X-U_{\alpha} \right) \\
    & \in \mathscr{C}
   \end{align*}
   Similarly, if \( \left\{ U_{k} \right\}_{k=1}^{n} \) is a finite subcollection of \( \mathscr{T} \), we want to show that \( \bigcap_{k=1 }^{n} U_{k} \in \mathscr{T} \) or, equivalently, \( X - \left( \bigcap_{k=1}^{n} U_{k}\right) \in \mathscr{C} \)
   \begin{align*}
    X - \left( \bigcap_{k=1}^{n} U_{k}\right) &= \bigcup_{k=1}^{n} \left( X- U_{k} \right) \\
    & \in \mathscr{C}
   \end{align*}
   So \( \mathscr{T} \) is a topology defined on \( X \) whose collection of closed sets is, by construction, \( \mathscr{C} \). \\ 
 \(
\left( \Leftarrow \right)
\)
Assume \( X \) is equipped with a topology \( \mathscr{T} \), and let
\[
\mathscr{C} = \{\, X - U \mid U \in \mathscr{T} \,\}
\]
be the collection of closed sets. We check that \(\mathscr{C}\) satisfies \textbf{(i)}–\textbf{(iii)}.\\
\textbf{(i)} Since \( X \) and \( \varnothing \) are open, their complements
\[
X - X = \varnothing, \qquad X - \varnothing = X
\]
are closed, so \( \varnothing, X \in \mathscr{C} \).\\
\textbf{(ii)} Let \( \{ C_{\alpha} \}_{\alpha \in A} \subseteq \mathscr{C} \). For each \( \alpha \), choose an open set \( U_{\alpha} \) with
\( C_{\alpha} = X - U_{\alpha} \). Then
\[
\bigcap_{\alpha \in A} C_{\alpha}
= \bigcap_{\alpha \in A} (X - U_{\alpha})
= X - \bigcup_{\alpha \in A} U_{\alpha},
\]
and since an arbitrary union of open sets is open, the right-hand side is in \( \mathscr{C} \).\\
\textbf{(iii)} For a finite collection \( C_1, \dots, C_n \in \mathscr{C} \), write
\( C_k = X - U_k \) with each \( U_k \) open. Then
\[
\bigcup_{k=1}^n C_k
= \bigcup_{k=1}^n (X - U_k)
= X - \bigcap_{k=1}^n U_k,
\]
and because a finite intersection of open sets is open, this lies in \( \mathscr{C} \) as well.\\
Thus \(\mathscr{C}\) satisfies the three conditions, completing the proof.
\end{proof}


\begin{dfn}
    A point \( x \) of a topological space \( X \) is said to be a \vocab{limit point} of a set \( A \subseteq X \) if every deleted open neighborhood of \( x \) contains a point of \( A \). That is to say; for every open set \( U \) that contains \( x \): 
    \[ \left( U - \left\{ x \right\} \right) \cap A \neq \varnothing.\]
    The collection of limit points of \( A \) is denoted by \( A' \).
\end{dfn}

\begin{theorem}
    A subset \( C \) of a topological space \( X \) is closed if and only if it contains all its limit points. In other words,
    \[ C \text{ is closed} \iff C' \subseteq C .\]
\end{theorem}
\begin{proof}
    \( (\Rightarrow) \) Suppose that \( C \) is closed. Pick some \( x \notin C \). So \( x \in X - C \). Since \( C \) is closed, \( X - C \) is open. But that means we have found an open neighborhood, particularly \( X - C \), of \( x \) that is disjoint from \( C \). So \( x \) cannot be a limit point of \( C \). Since the assumption \( x \notin C \) leads to the conclusion that \( x \) is not a limit point of \( C \), it follows that \( C \) must contain all its limit points.\\
    \( (\Leftarrow) \) Suppose that \( C \) contains all its limit points. Consider the set \( X - C \). Since \( C \) contains all its limit points, for every element \( x \in X - C \), we know that \( x \) is not a limit point of \( C \). Therefore, there exists a neighborhood \( U \) of \( x \) for which \( (U - \{x\}) \cap C = \varnothing \). Since \( x \in X - C \) (so \( x \notin C \)), this implies that \( U \cap C = \varnothing \), and hence \( U \subseteq X - C \). This shows that every point of \( X - C \) is an interior point of \( X - C \), or equivalently \( X - C = \mathrm{Int}(X - C) \). By \cref{thm:open-set-equals-its-interior}, this implies \( X - C \) is open and hence \( C \) is closed.
\end{proof}

\begin{dfn}
    The \vocab{closure} of a set \( A \) in a topological space \( X \), denoted as \( \overline{A} \), is the set 
    \[ \overline{A} = \bigcap \; \left\{ A \subseteq C \; \middle| \; C \text{ is a closed set.} \right\} \]
\end{dfn}


\section{Basis for a Topology}
\begin{dfn}\label{def: basis for a topology}
    If \( X \) is a set, we define a \vocab{basis} \( \mathscr{B} \) for a topology to be a collection of subsets of \( X \) that satisfies the following criteria:
    \begin{enumerate}[label=\textbf{\roman*)}]
        \item For every \( x \in X \), there is some \( B \in \mathscr{B} \) such that \( x \in B \). 
        \item For every \( B_{1}, B_{2} \in \mathscr{B} \) and \( x \in B_{1} \cap B_{2} \) there is some \( B_{3} \in \mathscr{B} \) such that \( x \in B_{3} \) and \( B_{3} \subseteq B_{1} \cap B_{2} \). 
    \end{enumerate}
A subset \( U \) of \( X \) belongs to the topology  \( \mathscr{T} \) generated by \( \mathscr{B} \) if for each \( x \in U \), there is some \( B_{x} \in \mathscr{B} \) such that \( x \in B_{x} \) and \( B_{x} \subseteq U \).
\end{dfn}

\begin{theorem}
    The topology \( \mathscr{T} \) generated by \( \mathscr{B} \) is indeed a topology.
\end{theorem}
\begin{proof}
    \( \varnothing \in \mathscr{T} \) vacuously and \( X  \in \mathscr{T} \) by definition. Now let \( U_{\alpha \in A}  \) be an arbitrary collection of open sets. We wish to show that 
    \[ \bigcup_{\alpha \in A } U_{\alpha} \in \mathscr{T}. \]
Pick any \( x \in \bigcup_{\alpha \in A } U_{\alpha}  \). Then \( x \in U_{\beta} \) for at least one \( \beta \in A \). Since \( U_{\beta} \in \mathscr{T} \), there is some \( B \in \mathscr{B} \) for which \( x \in B \) and \( B \subseteq U_{\beta} \). So we can choose the same \( B \) to get 
\[ x \in B \subseteq  \bigcup_{\alpha \in A } U_{\alpha}.\]
Since this holds for all \( x \in \bigcup_{\alpha \in A } U_{\alpha}  \), this shows that \( \bigcup_{\alpha \in A } U_{\alpha} \in \mathscr{T} \).\\
Now let \( U_{1},\dots, U_{n} \) be a finite collection of open sets.  We wish to show that 
\[ \bigcap_{j=1}^{n} U_{j} \in \mathscr{T}. \]
We will proceed by induction. The case \( n=1 \) is trivial so our base case will be \( n=2 \). Suppose \( U_{1} \) and \( U_{2} \) are open sets. We wish to show that \( U_{1} \cap U_{2} \) is open. In other words, for any \( x \in U_{1} \cap U_{2} \), we wish to find a basis element that contains \( x \) and is contained in \( U_{1} \cap U_{2}. \) So pick any \( x \in U_{1} \cap U_{2} \). Then \( x \in U_{1} \) and \( x \in U_{2} \). Since \( U_{1} \) and \( U_{2} \) were already open, there exists \( B_{1}, B_{2} \in \mathscr{B} \) such that \( x \in B_{1} \subseteq U_{1} \) and \( x \in B_{2} \subseteq U_{2}\). Since \( \mathscr{B} \) is a basis, there is another basis element \( B_{3} \) containing \( x \) and contained in \( B_{1} \cap B_{2} \). It is clear that \( B_{3} \in U_{1} \cap U_{2} \), and hence\( U_{1} \cap U_{2} \) is open in \( \mathscr{T} \).\\
 Now for the inductive step, assume that we have shown that \( \bigcap_{j=1}^{n-1} U_{j} \) is open in \( \mathscr{T} \). We define \( V = \bigcap_{j=1}^{n-1} U_{j} \), which is open by the inductive hypothesis. Then \( V \cap U_{n} \) collapses to the base case The inductive step and the proof is completed.
\end{proof}

\begin{theorem}\label{thm:topology-equals-union-of-basis}
    Let \( \mathscr{B} \) be a basis for a topology \( \mathscr{T} \) on \( X \). Then \( \mathscr{T} \) equals the collection of all unions of elements of \( \mathscr{B} \).
\end{theorem}
\begin{proof}
    Suppose that \( \mathbf{B} \) is the collection of all unions of elements of \( \mathscr{B} \). We wish to show that \( \mathscr{T} = \mathbf{B} \).\\ \( \mathbf{B} \subseteq \mathscr{T} \) since each member of \( \mathscr{B} \) is open in \( \mathscr{T} \) and since \( \mathscr{T} \) is a topology, their unions are also members of \( \mathscr{T} \). \\ One the other hand, we pick any \( U \in \mathscr{T} \). Since \( \mathscr{B} \) generates the topology \( \mathscr{T} \), for every \( x \in U \), we may find \( B_{x} \in \mathscr{B} \) such that \( x \in B_{x} \subseteq U \). So we may write
    \[ U = \bigcup_{x \in U} B_{x}. \]
    This completes the proof.
\end{proof}

\begin{theorem}
    Let \( \left( X, \mathscr{T} \right) \) be a topological space. Suppose that \( \mathscr{C} \) is a collection of open sets of \( X \) such that for each open set \( U \) of  \( X \) and each element \( x \) of \( U \), there is a \( C \in \mathscr{C} \) such that \( x \in C \subseteq U \). This qualifies \( \mathscr{C} \) as a basis for the topology on \( X \).
\end{theorem}
\begin{proof}
    We need to first verify that the collection \( \mathscr{C} \) satisfies the conditions laid out in \cref{def: basis for a topology}. \\ 
    The first condition is easy, we simply take our open set to be \( X \) and the conditions of the statement of the theorem ensure that there is some element \( C \in \mathscr{C} \) for which \( x \in C \). \\ 
    For the second condition, pick \( C_{1}, C_{2} \in \mathscr{C} \) such that \( C_{1} \cap C_{2} \neq \varnothing \). Since \( C_{1} \) and \( C_{2} \) are open, it follows that \( C_{1} \cap C_{2} \) is also open. Hence for any \( x \in C_{1} \cap C_{3} \), there is some \( C_{3} \) for which \( x \in C_{3} \subseteq  C_{1} \cap C_{2} \). This shows that \( \mathscr{C} \) is a basis for a topology on \( X \). \\ \\ 
    All this shows is that \( \mathscr{C} \) generates \emph{some} topology \( \mathscr{T}' \). We want to show that \( \mathscr{T} = \mathscr{T}' \). If \( U \in \mathscr{T} \), then for each \( x \in U \), there is some \( C \in \mathscr{C} \) for which \( x \in C \subseteq U \). By the last sentence of \cref{def: basis for a topology}, \( U \) must belong to the topology generated \( \mathscr{C} \) and hence \( U \in \mathscr{T}' \). \\ 
    Conversely, if \( U' \in \mathscr{T}' \), by \cref{thm:topology-equals-union-of-basis}, 
    \[ U' = \bigcup_{x \in U'} C_{x} .\]
    Since each \( C_{x} \) is open in \( \mathscr{T} \), it follows that \( U' \) is also open in \( \mathscr{T} \) so \( U' \subseteq \mathscr{T} \). This completes the proof.
\end{proof}




\chapter{Connectedness and Compactness}


\section{Connectedness}
\begin{dfn}
    A topological space \( X \) is said to be \vocab{disconnected} if there exists non-empty open sets \( U \) and \( V \) of \( X \) such that 
    \[ U \cup V =X \]
    and 
    \[ U \cap V = \varnothing .\]
    A topological space is \vocab{connected} if it is not disconnected.
\end{dfn}

Here is a useful equivalent condition for connectedness. 

\begin{theorem}
    A topological space \( X \) is connected if and only if every continuous characteristic function \( \chi_{U}: X \to \left\{ 0,1 \right\} \) is constant.
\end{theorem}
\begin{proof}
    For a given subset \( U \subseteq X \), recall that the characteristic function \( \chi_{U} \) is defined by
    \[
        \chi_{U}(x) = \begin{cases}
            1 & \text{if } x \in U \\
            0 & \text{if } x \notin U
        \end{cases}.
    \]
    \( \Rightarrow \) Suppose there exists a continuous, non-constant characteristic function \( \chi_U \) for some \( U \subset X \). Then by definition,
    \[
        \chi_U^{-1}(1) = U \quad \text{and} \quad \chi_U^{-1}(0) = X - U.
    \]
    Since \( \chi_U \) is non-constant, both \( U \) and \( X - U \) are nonempty. And since \( \chi_U \) is continuous and \( \{0,1\} \) has the discrete topology, both \( U \) and \( X - U \) are open in \( X \).\\
    Thus, \( X \) is the union of two disjoint, nonempty open sets  so \( X \) is disconnected.\\
    \(\Leftarrow\) Now assume that \( X \) is disconnected. Then there exist disjoint, non-empty open sets \( U \) and \( V \) in \( X \) such that \( U \cup V = X \). \\
Consider the characteristic function \( \chi_U : X \to \{0,1\} \). Since \( U \) and \( V = X - U \) are both open, and \( \{0,1\} \) has the discrete topology, the preimages
\[
\chi_U^{-1}(1) = U \quad \text{and} \quad \chi_U^{-1}(0) = V
\]
are open. Hence, \( \chi_U \) is continuous. Moreover, it is non-constant, since both \( U \) and \( V \) are non-empty. This completes the proof.

\end{proof}

\begin{corollary}
    Suppose \( X \) is a connected topological space and \( Y \) is any topological space. If \( f: X \to Y \) is continuous, then the image \( f(X) \), equipped with the subspace topology from \( Y \), is connected.
\end{corollary}
\begin{proof}
   If \( f(X) \) contains exactly one point, the result holds trivially, since a one-point space is connected. So assume \( f(X) \) contains more than one point.\\
Suppose, for a contradiction, that \( f(X) \) is not connected. Then, by the previous theorem, there exists a non-constant continuous characteristic function \( \chi_U : f(X) \to \{0,1\} \) for some proper, nonempty clopen subset \( U \subset f(X) \).\\
Now consider the composition \( \chi_U \circ f : X \to \{0,1\} \). This map is continuous, since it is the composition of continuous functions. Moreover, since \( f \) is surjective onto \( f(X) \), the composition is also non-constant.\\
But this contradicts the connectedness of \( X \), since we've constructed a non-constant continuous characteristic function on \( X \). Hence, \( f(X) \) must be connected.
\end{proof}

\begin{exercise}
    Show that a finite set of points in a \( T_2 \) (Hausdorff) space is not connected.
\end{exercise}
\begin{solution}
    Let \( S = \{x_1, x_2, \dots, x_n\} \subseteq X \), where \( X \) is a \( T_2 \)  space and \( n \geq 2 \). We aim to show that \( S \), with the subspace topology inherited from \( X \), is not connected.\\
    Since \( X \) is Hausdorff, we can find pairwise disjoint open sets \( U_{x_1}, \dots, U_{x_n} \) in \( X \), each containing \( x_i \).\\
    Now consider the subspace topology on \( S \). Let
    \[
        U = U_{x_1} \cap S, \qquad V = \bigcup_{j = 2}^{n} \left( U_{x_j} \cap S \right).
    \]
    Then \( U \) and \( V \) are open in the subspace topology on \( S \), disjoint by construction, and cover \( S \), since each \( x_i \in S \) is contained in some \( U_{x_i} \).\\
    Thus, \( S = U \cup V \) is a separation of \( S \) into two nonempty, disjoint, open sets. Therefore, \( S \) is disconnected.
\end{solution}





\part{Functional Analysis}
\label{part: Functional Analysis}
\parttoc
\chapter{Examples of New Spaces}
\section{Sequence Spaces}
\begin{dfn}
    The space \( \ell_{0} \) (or \( c_{00} \left( \mathbb{K} \right) \)) consists of all \( \mathbb{K} \)-valued sequences \( (x_n)_{n=1}^{\infty} \) such that \( x_n = 0 \) for all but finitely many \( n \in \mathbb{N} \).
\end{dfn}

\begin{lemma}
    \( \ell_{0} \) is a \( \mathbb{K} \)-vector space where for all \( \vb{x}=  (x_n)_{n=1}^{\infty}, \vb{y} = (y_n)_{n=1}^{\infty} \in \ell_{0} \) and \( \lambda \in \mathbb{K} \), we have 
    \[ \vb{x} + \vb{y} := (x_n +y_{n})_{n=1}^{\infty}  \quad  \text{ and } \quad  \lambda \vb{x} : = ( \lambda x_n)_{n=1}^{\infty}.\]
\end{lemma}
\begin{proof}
    We need verify that the conditions laid out in \Cref{def:vector-space} hold.
    \begin{description}
        \item[\textbf{Closure under addition:}] Suppose that  \( \vb{x}=  (x_n)_{n=1}^{\infty}, \; \vb{y} = (y_n)_{n=1}^{\infty} \in \ell_{0} \). Then there exists \( k_{1}, k_{2} \in \mathbb{N} \) such that if \( n_{1} > k_{1} \), \( x_{n_{1}} =0 \) and if \( n_{2} > k_{2} \), \( y_{n_{2}} =0 \). So pick \( k = \max \left\{ k_{1}, k_{2} \right\} \). It is clear that if \( n >k \), we have \( x_{n}+y_{n}  =0\) so \( \vb{x}+\vb{y} \in \ell_{0} .\)
        \item[\textbf{Closure under scalar multiplication:}] Again, if \(  \vb{x}=  (x_n)_{n=1}^{\infty} \in \ell_{0} \), there exists a \( k \in \mathbb{N} \) such that if \( n >k \), \( x_{n} =0 \). So for any \( \lambda \in \mathbb{K} \), we can choose the same \( k \) so that if \( n >k \), \( \lambda x_{n} =0 \). So \( \lambda \vb{x} \in \ell_{0} \). 
        \item[\textbf{Commutativity:}] We have 
        \begin{align*}
            \vb{x} + \vb{y} &= (x_n +y_{n})_{n=1}^{\infty} \\
            &= (y_n +x_{n})_{n=1}^{\infty} \tag{Since $x_{n}$ and $y_{n}$ are elements of $\mathbb{K}.$} \\
            &= \vb{y} + \vb{x}
        \end{align*}
        \item[\textbf{Associativity:}] If \( \vb{x}=  (x_n)_{n=1}^{\infty}, \vb{y} = (y_n)_{n=1}^{\infty},  \vb{z}=  (z_n)_{n=1}^{\infty} \in \ell_{0} \), then 
        \begin{align*}
            \left( \vb{x} + \vb{y} \right) + \vb{z} &= (x_n +y_{n})_{n=1}^{\infty} + \vb{z} \\
            &= \left( (x_{n}+y_{n}) + z_{n} \right)_{n=1}^{\infty} \\
            &= \left( x_{n} + (y_{n} +z_{n}) \right)_{n=1}^{\infty} \tag{Inheritance of associativity from $\mathbb{K}$} \\
            &= \vb{x} + \left( y_{n} +z_{n} \right)_{n=1}^{\infty}\\
            &= \vb{x} + \left( \vb{y} + \vb{z} \right)
        \end{align*}
    Similarly if \( \alpha, \beta \in \mathbb{K} \), we have 
    \begin{align*}
        \left( \alpha \beta \right) \vb{x} &= \left( (\alpha \beta)x_{n} \right)_{n=1}^{\infty} \\
        &= \left( \alpha (\beta x_{n}) \right)_{n=1}^{\infty}\\
        &= \alpha \left( \beta x_{n} \right)_{n=1}^{\infty} \\
        &= \alpha \left( \beta \vb{x} \right)
    \end{align*}
    \item[\textbf{Distributive Properties:}] For \( \alpha \in \mathbb{K} \) and \( \vb{x}, \vb{y} \in \ell_{0} \), we have 
    \begin{align*}
        \alpha \left[ \vb{x} + \vb{y} \right] &= \alpha \left[ (x_n +y_{n})_{n=1}^{\infty} \right] \\
        &= \left( \alpha \left( x_{n} +y_{n} \right) \right)_{n=1}^{\infty} \\
        &= \left( \alpha x_{n} + \alpha y_{n}\right)_{n=1}^{\infty} \\
        &= \alpha \vb{x} + \alpha \vb{y}
    \end{align*}
    Similarly, if \( \beta \in \mathbb{K} \), we have 
    \begin{align*}
        \left( \alpha + \beta \right) \vb{x} &= \left( \alpha + \beta \right) \left( x_{n} \right)_{n=1}^{\infty} \\
        &= \left( (\alpha + \beta) x_{n} \right)_{n=1}^{\infty} \\
        &= \left( \alpha x_{n} + \beta x_{n} \right)_{n=1}^{\infty} \\
        &= \alpha \vb{x} + \beta \vb{x}
    \end{align*}
    
    \end{description}
\end{proof}

\begin{exercise}
    Find a basis for \( \ell^{0} \).
\end{exercise}

\begin{solution}
    For each \( k \in \mathbb{N} \), define the sequence 
    \[
        \vb{e}_{k} = (e_{n})_{n=1}^{\infty}, \quad 
        e_{n} = 
        \begin{cases}
            1, & n = k, \\[6pt]
            0, & n \neq k.
        \end{cases}
    \]
    Let 
    \[
        B = \{ \vb{e}_{k} \mid k \in \mathbb{N} \}.
    \]

    Each \(\vb{e}_{k}\) has exactly one nonzero coordinate, hence belongs to \(\ell_{0}\).

 
    Suppose
    \[
        \sum_{j=1}^{m} a_{j} \vb{e}_{k_{j}} = \vb{0}, 
        \qquad a_{j} \in \mathbb{K}, \ k_{j} \in \mathbb{N}.
    \]
    Looking at the \(k_{i}\)-th coordinate, we obtain \(a_{i} = 0\) for each \(i\).  
    Hence the family \(B\) is linearly independent.

    Let \(\vb{x} = (x_{n}) \in \ell^{0}\).  
    By definition, only finitely many \(x_{n}\) are nonzero. If \(x_{j} \neq 0\), then
    \[
        \vb{x} = \sum_{j : \, x_{j} \neq 0} x_{j} \vb{e}_{j},
    \]
    which is a finite linear combination of elements of \(B\). Thus, \(\vb{x}\) lies in the span of \(B\).

\end{solution}


\begin{dfn}
    A \( \mathbb{K} \) sequence \( \vb{x} = \left( x_{n} \right)_{n=1}^{\infty}\) is a member of \( \ell_{p} \) for \( 1 \le p < \infty \) if the sum 
    \[ \sum_{j=1}^{\infty} \abs{x_{j}}^{p} < \infty. \]
\end{dfn}

\begin{example}
    Let \( H \) denote the harmonic sequence \( \left( \frac{1}{n} \right)_{n=1}^\infty \). It is well-known that 
    \[ \sum_{n=1}^{\infty} \frac{1}{n^{2}} = \frac{\pi}{6}  \]
    so \( H \in \ell_{2} \). However, it is also know that 
    \[ \sum_{n =1}^{\infty} \frac{1}{n} \quad \text{ diverges} \]
    so \( H \not \in \ell_{1} \).
\end{example}

\begin{exercise}
    Let \( \mathbf{V} \) be any inner product space. If \( \norm{\vb{x}_{n}} \) converges to \( \norm{\vb{x}} \) and \( \left< \vb{x}_{n}, \vb{x} \right> \) converges to \( \left< \vb{x},\vb{x} \right> \), then \( \vb{x}_{n} \) converges to \( \vb{x} \).
\end{exercise}
\begin{solution}
    If \( \vb{x}_{n} \) is an eventually zero sequence and \( \vb{x} =0 \), the result is trivial so let us assume otherwise.\\
    Since \( \norm{\vb{x}_{n}} \) converges to \( \norm{\vb{x}} \) and and \( \left< \vb{x}_{n}, \vb{x} \right> \) converges to \( \left< \vb{x},\vb{x} \right> \), we can simultaneously chose a sufficiently large \( n \) such that 
    \[ \abs{ \; \norm{\vb{x}_{n}} - \norm{\vb{x}} \; } < \frac{\epsilon^{2}}{2 \left( \norm{\vb{x}_{n}} + \norm{\vb{x}} \right)}\quad \text{and} \quad \abs{\left< \vb{x},\vb{x} \right> -\left< \vb{x}_{n}, \vb{x} \right> } < \frac{\epsilon^{2}}{4}\]
    Now we apply the result from \Cref{exc:equality-for-norm-squared}.
\begin{align*}
    \norm{\vb{x}_{n} - \vb{x}}^{2} &= \norm{\vb{x}_{n}}^{2} + \norm{\vb{x}}^{2} - 2 \Re \left( \left< \vb{x}_n , \vb{x} \right> \right) \\
    &= \norm{\vb{x}_{n}}^{2} + \norm{\vb{x}}^{2} -2 \norm{\vb{x}}^{2}  + 2 \norm{\vb{x}}^{2}  - 2 \Re \left( \left< \vb{x}_n , \vb{x} \right> \right) \tag{add and subtract $2\norm{\vb{x}}^2$}\\
    &= \norm{\vb{x}_{n}}^{2} - \norm{\vb{x}}^{2} + 2 \norm{\vb{x}}^{2} - 2 \Re \left( \left< \vb{x}_n , \vb{x} \right> \right) \\
    & = \norm{\vb{x}_{n}}^{2} - \norm{\vb{x}}^{2} + 2 \left< \vb{x}, \vb{x} \right> - 2 \Re \left( \left< \vb{x}_n , \vb{x} \right> \right) \\
     & = \norm{\vb{x}_{n}}^{2} - \norm{\vb{x}}^{2} + 2 \Re \left( \left< \vb{x}, \vb{x} \right> \right)- 2 \Re \left( \left< \vb{x}_n , \vb{x} \right> \right) \tag{since $\left< \vb{x}, \vb{x} \right>$ is real}\\
     &= \norm{\vb{x}_{n}}^{2} - \norm{\vb{x}}^{2} + 2 \Re \left( \left< \vb{x}, \vb{x} \right> - \left< \vb{x}_n , \vb{x} \right>\right) \tag{linearity of $\Re$}\\
     &= \left( \norm{\vb{x}_{n}} + \norm{\vb{x}} \right) \left( \norm{\vb{x}_{n}} - \norm{\vb{x}} \right) + 2 \Re \left( \left< \vb{x}, \vb{x} \right> - \left< \vb{x}_n , \vb{x} \right>\right) \tag{difference of squares}\\
     & \le  \left( \norm{\vb{x}_{n}} + \norm{\vb{x}} \right)  \abs{ \; \norm{\vb{x}_{n}} - \norm{\vb{x}} \; }+ 2  \Re \left( \left< \vb{x}, \vb{x} \right> - \left< \vb{x}_n , \vb{x} \right>\right) \tag{since $a \le |a|$ for real $a$} \\
     & \le  \left( \norm{\vb{x}_{n}} + \norm{\vb{x}} \right)  \abs{ \; \norm{\vb{x}_{n}} - \norm{\vb{x}} \; } + 2\abs{\left< \vb{x},\vb{x} \right> -\left< \vb{x}_{n}, \vb{x} \right> } \tag{since $|\Re(z)| \le |z|$}\\
     & <  \left( \norm{\vb{x}_{n}} + \norm{\vb{x}} \right) \frac{\epsilon^{2}}{2 \left( \norm{\vb{x}_{n}} + \norm{\vb{x}} \right)} + 2 \frac{\epsilon^{2}}{4} \tag{by choice of $n$}\\
     &= \frac{\epsilon^{2}}{2} + \frac{\epsilon^{2}}{2} \\
    &= \epsilon^{2}
\end{align*}
    So \(  \norm{\vb{x}_{n} - \vb{x}}^{2} < \epsilon^{2} \) for sufficiently large \( n \) or \(  \norm{\vb{x}_{n} - \vb{x}} < \epsilon \), which is what we wanted.
\end{solution}


\section{Function Spaces}
\begin{exercise}
    Suppose \( f \in \mathscr{C} \left[ 0,1 \right] \) has sup norm \( \epsilon \). What is the largest possible value for the norm of \( f \) induced by the inner product 
    \[ \left< f,g  \right> = \int_{0}^{1} f(x) g(x) \dd{x} .\]
\end{exercise}
\begin{solution}
    Since \( f \) has sup norm \( \epsilon \) it follows that \( \abs{f(x)} \le \epsilon\) for all \( x \in \left[ 0,1 \right] \). So let us choose \( f(x) = \epsilon\) for all \( x \in \left[ 0,1 \right] \) Then 
    \begin{align*}
        \norm{f} &= \sqrt{ \left< f,f \right>} \\
        &= \sqrt{ \int_{0}^{1 } f(x)^{2} \dd{x}} \\
        &= \sqrt{ \epsilon^{2}}
    \end{align*}
    So \( \norm{f} = \epsilon \).
\end{solution}



\chapter{Hilbert Spaces}
\section{Hilbert Spaces}


Hilbert spaces provide the natural setting for extending geometric and analytical results from finite-dimensional inner product spaces to infinite dimensions. While finite-dimensional inner product spaces are automatically complete, this fails in infinite dimensions. We need an additional completeness assumption to ensure Cauchy sequences converge. This completeness allows us to perform limiting operations and develop a robust theory of approximation, projection, and convergence.


\subsection{Continuity of the Inner Product and Norm}

Before defining Hilbert spaces, we establish some fundamental continuity properties. Although the metric on an inner product space is induced by the inner product (via the norm), the continuity of the inner product and norm with respect to this metric requires proof. These results will be used in the chapter on Banach spaces as well.

\begin{theorem}\label{thm:inner-product-continuity}
   Let \( \mathbf{V} \) be an inner product space. If the sequences \( \left\{ \vb{x}_{n} \right\} \) converges to \( \vb{x} \) and \( \left\{ \vb{y}_{n} \right\} \) converges to \( \vb{y} \), then \( \left\{ \left< \vb{x}_{n}, \vb{y}_{n} \right> \right\} \) converges to \( \left< \vb{x}, \vb{y} \right> \).
\end{theorem}
\begin{proof}
    We have
    \begin{align*}
        \abs{\left< \vb{x}_{n}, \vb{y}_{n} \right> - \left< \vb{x}, \vb{y} \right>} &= \abs{ \left< \vb{x}_{n} - \vb{x}, \vb{y}_{n} - \vb{y} \right> + \left< \vb{x}, \vb{y}_{n} -\vb{y} \right> + \left< \vb{x}_{n}- \vb{x}, \vb{y} \right>} \tag{\Cref{exc:expand-ip-diff}}\\
        & \le \abs{\left< \vb{x}_{n} - \vb{x}, \vb{y}_{n} - \vb{y} \right>} + \abs{\left< \vb{x}, \vb{y}_{n} -\vb{y} \right>} + \abs{\left< \vb{x}_{n}- \vb{x}, \vb{y} \right>} \tag{Triangle inequality} \\
        & \le \norm{\vb{x}_{n} - \vb{x}} \norm{\vb{y}_{n} - \vb{y}} + \norm {\vb{x}} \norm{\vb{y}_{n} - \vb{y}} + \norm{\vb{x}_{n} - \vb{x}} \norm{\vb{y}}. \tag{Cauchy-Schwarz inequality}
    \end{align*}
    Given \( \epsilon >0 \), choose \( N \) sufficiently large so that for all \( n \ge N \),
    \[ \norm{\vb{x}_{n} - \vb{x}} < \min \left\{ \sqrt{ \frac{\epsilon}{3}}, \frac{\epsilon}{ 3 \left( \norm{\vb{y}} +1 \right)} \right\} \]
    and 
     \[ \norm{\vb{y}_{n} - \vb{y}} < \min \left\{ \sqrt{ \frac{\epsilon}{3}}, \frac{\epsilon}{ 3 \left( \norm{\vb{x}} +1 \right)} \right\}. \]
     Then for \( n \ge N \),
     \begin{align*}
        \abs{\left< \vb{x}_{n}, \vb{y}_{n} \right> - \left< \vb{x}, \vb{y} \right>} &\le \norm{\vb{x}_{n} - \vb{x}} \norm{\vb{y}_{n} - \vb{y}} + \norm {\vb{x}} \norm{\vb{y}_{n} - \vb{y}} + \norm{\vb{x}_{n} - \vb{x}} \norm{\vb{y}}  \\
        & <  \sqrt{\frac{\epsilon}{3}} \sqrt{\frac{\epsilon}{3}} + \frac{\norm{\vb{x}}}{\norm{\vb{x}} +1} \cdot \frac{\epsilon}{3} + \frac{\norm{\vb{y}}}{\norm{\vb{y}} +1} \cdot \frac{\epsilon}{3} \\
        & < \frac{\epsilon}{3} +  \frac{\epsilon}{3} +  \frac{\epsilon}{3} = \epsilon.
     \end{align*}
     Therefore \( \left< \vb{x}_{n}, \vb{y}_{n} \right> \to \left< \vb{x}, \vb{y} \right> \).
\end{proof}

\begin{theorem}\label{thm:continuity-of-norm}
    If \( \left\{ \vb{x}_{n} \right\} \) converges to \( \vb{x} \) in a normed vector space \( \mathbf{V} \), then \( \norm{\vb{x}_{n}} \) converges to \( \norm{\vb{x}} \).
\end{theorem}
\begin{proof}
    If \( \mathbf{V} \) is an inner product space and the norm is induced by the inner product, then the result is simply a corollary of \Cref{thm:inner-product-continuity}. \\ 
    Otherwise, we proceed directly. For any \( \epsilon > 0 \), pick \( N \) sufficiently large so that if \( n \ge N \), then \( \norm{\vb{x}_{n} - \vb{x}} < \epsilon \). Then for \( n \ge N \), 
  \begin{align*}
       \abs{\norm{\vb{x}_{n}} - \norm{\vb{x}}} &\le \norm{\vb{x}_{n} -\vb{x}} \tag{reverse triangle inequality}\\
       &< \epsilon
   \end{align*}
    Hence, \( \norm{\vb{x}_{n}} \to \norm{\vb{x}} \).
\end{proof}



\begin{theorem}
    If \( \left\{ \vb{x}_{n} \right\} \) and \( \left\{ \vb{y}_{n} \right\} \) is a Cauchy sequence of vectors in an inner product space \( \mathbf{V} \), then \( \left< \vb{x}_{n}, \vb{y}_{n} \right> \) is a Cauchy sequence of scalars.
\end{theorem}
\begin{proof}
    The proof is similar to the proof of \Cref{thm:inner-product-continuity} and, as such, can be skipped. \\ 
    For any \( m,n \in \mathbb{N} \), we have 
    \begin{align*}
        \abs{ \left< \vb{x}_{m}, \vb{y}_{m} \right> - \left< \vb{x}_{n}, \vb{y}_{n} \right>} &= \abs{ \left< \vb{x}_{m}- \vb{x}_{n}, \vb{y}_{m} - \vb{y}_{n} \right> + \left< \vb{x}_{n}, \vb{y}_{m} - \vb{y}_{n}  \right> + \left< \vb{x}_{m}- \vb{x}_{n}, \vb{y}_{n} \right>} 
    \end{align*}
    
\end{proof}



\subsection{Definition and Basic Examples}
\begin{dfn}
    A \hyperref[def:complete-metric-space]{complete} \hyperref[def:inner-product]{\textcolor[RGB]{180, 109, 214}{inner-product space}} is called a \vocab{Hilbert space}.
\end{dfn}


\chapter{Linear Maps}
\section{Linear Maps between Hilbert Spaces}
\begin{theorem}
   Let \( T \in \mathcal{L} \left( \mathbf{H}, \mathbf{K} \right) \). The following are equivalent:
    \begin{enumerate}[label=\textbf{\roman*)}]
        \item \( T \) is continuous 
        \item \( T \) is continuous at some \( \vb{x}_{0} \)
        \item \( T \) is continuous at \( \vb{0} \)
        \item \( \left\{ \norm{T \left( \vb{x} \right)}_{\mathbf{K}} \ \middle| \; \norm{\vb{x}}_{\mathbf{H}} = 1 \right\} \) is bounded 
        \item There exists an \( M \ge 0 \) such that \( \norm{T \left( \vb{x} \right)}_{\mathbf{K}} < M \norm{\vb{x}}_{\mathbf{H}} \) for all \( \vb{x} \in \mathbf{H} \)
    \end{enumerate}
\end{theorem}
\begin{proof}
    
\end{proof}


\part{Fourier Analysis}
\label{part:Fourier-Analysis}
\parttoc 
\chapter{Basic Properties of the Fourier Series}
\section{Definition and Examples}
\begin{dfn}
    Suppose that \( f: [a,b] \to \mathbb{R} \) is Riemann integrable on \( [a,b] \). Set \( L = b-a \). Given \( n \in \mathbb{Z} \), the \vocab{nth Fourier coefficient} is defined to be: 
    \[ \hat{f}(n):= \frac{1}{L} \int_{a}^{b} f(x) \exp \left( - \frac{2 n \pi i x}{L} \right) \dd{x}.\]
    The \vocab{Fourier series} of \( f(x) \) is given by 
    \[ f(x) \sim \sum_{n = -\infty}^{\infty} \hat{f}(n) \exp \left( \frac{2 n \pi i x}{L} \right).\]
\end{dfn}

\begin{exercise}
    Show that for each integer \( n \), 
    \[ \exp \left( n \pi i \right)  = \exp \left( - n \pi i \right)  \]
\end{exercise}
\begin{solution}
    Using Euler's formula, we have
    \[ \exp(n\pi i) = \cos(n\pi) + i\sin(n\pi). \]
    For any integer \( n \), we know that \( \cos(n\pi) = (-1)^n \) and \( \sin(n\pi) = 0 \).
    Therefore,
    \[ \exp(n\pi i) = (-1)^n. \]
    
    Similarly,
    \[ \exp(-n\pi i) = \cos(-n\pi) + i\sin(-n\pi). \]
    Since cosine is an even function and sine is an odd function, we have
    \[ \cos(-n\pi) = \cos(n\pi) = (-1)^n \quad \text{and} \quad \sin(-n\pi) = -\sin(n\pi) = 0. \]
    Thus,
    \[ \exp(-n\pi i) = (-1)^n. \]
    
    Therefore, \( \exp(n\pi i) = \exp(-n\pi i) = (-1)^n \) for all integers \( n \).
\end{solution}


\begin{example}
    Suppose we want to find the Fourier series of \( f(x) =x \) over the interval \( \left[ -a,a \right] \). Then 
    \[ \hat{f}(n) = \frac{1}{2a} \int_{-a }^{a} x  \exp \left( - \frac{2 n \pi i x}{2a} \right) \dd{x} \]
    This is routine integration by parts, so we get 
    \begin{align*}
        \hat{f}(n) &= \frac{1}{2a} \left( \frac{1 }{- \frac{2 n \pi i }{2a}} x  \exp \left( - \frac{2 n \pi i x}{2a} \right)  - \frac{1}{ \left( - \frac{2 n \pi i }{2a} \right)^{2}} \exp \left( - \frac{2 n \pi i x}{2a} \right) \eval_{-a}^{a}\right) \\
        &= \frac{1}{2a} \left( - \frac{a}{n \pi i} x  \exp \left( - \frac{2 n \pi i x}{2a} \right)  + \frac{a^{2}}{ \left( n \pi  \right)^{2}}  \exp \left( - \frac{2 n \pi i x}{2a} \right) \eval_{-a}^{a} \right) \\
        &= \left( \frac{a }{2 \left( n \pi  \right)^{2}}  \exp \left( - \frac{ n \pi i x}{a} \right) - \frac{1}{2 n \pi i} x  \exp \left( - \frac{ n \pi i x}{a} \right)  \right)\eval_{-a}^{a} \\
        &= \frac{a }{2 \left( n \pi  \right)^{2}}  \exp \left( -n \pi i \right)  - \frac{1}{2 n \pi i} a \exp \left( -n \pi i \right) - \frac{a }{2 \left( n \pi  \right)^{2}}  \exp \left( n \pi i \right)  -   \frac{1}{2 n \pi i} a \exp \left( n \pi i \right) \\
        &= \frac{a}{2 \left( n \pi  \right)^{2}} \left( \exp \left( - n \pi i \right) - \exp \left( n \pi i \right) \right) - \frac{a}{2 n \pi i } \left( \exp \left( - n \pi i \right) +\exp \left( n \pi i \right) \right) \\ 
        &= - \frac{a }{n \pi i} \left( \exp \left( n \pi i \right)  \right) \\
        &= \left( -1 \right)^{n+1} \frac{a}{n \pi i} \quad n \in \mathbb{Z}
    \end{align*}
    So 
  \begin{align*}
        x &\sim \sum_{n \in \mathbb{Z} - \{ 0 \}} \hat{f}(n) \exp \left( \frac{n \pi i x }{a}\right) \\ 
        &= \sum_{n \in \mathbb{Z} - \{ 0 \}}  \left( -1 \right)^{n+1} \frac{a}{n \pi i} \exp \left( \frac{n \pi i x }{a}\right) \\
        &= \sum_{n \in \mathbb{Z} - \{ 0 \}}  \left( -1 \right)^{n+1} \frac{a}{n \pi i} \left( \cos{ \left( \frac{n \pi  x }{a} \right) } + i \sin{ \left( \frac{n \pi  x }{a} \right) }\right) \\
       &= \sum_{n \in \mathbb{Z} - \{ 0 \}}  \left( -1 \right)^{n+1} \frac{a}{n \pi i} \cos{ \left( \frac{n \pi  x }{a} \right) } +  \left( -1 \right)^{n+1} \frac{a}{n \pi i} i \sin{ \left( \frac{n \pi  x }{a} \right) } \\
       &= \sum_{n \in \mathbb{Z} - \{ 0 \}} 0 +  \left( -1 \right)^{n+1} \frac{a}{n \pi i} i \sin{ \left( \frac{n \pi  x }{a} \right) } \tag{Since $\cos{ \left( \frac{n \pi  x }{a} \right) }$ is even in $n$ but $  \frac{a}{n \pi i}$ is odd.} \\
       &= 2 \sum_{n =1}^{\infty} \left( -1 \right)^{n+1} \frac{a}{n \pi i} i \sin{ \left( \frac{n \pi  x }{a} \right) } \\
       &= 2\sum_{n =1}^{\infty} \left( -1 \right)^{n+1} \frac{a}{n \pi } \sin{ \left( \frac{n \pi  x }{a} \right) }
\end{align*}
    Here is a \href{https://www.desmos.com/calculator/f8ey4wrfu4}{nice Desmos visualization}.
\end{example}


\part{Partial Differential Equations}
\label{part:Partial-Differential-Equations}
\parttoc
\input{content/analysis/partial-differential-equation/partial-differential-equations-master}

\part{ Algebraic Topology}
\vspace*{2em} The primary resource for this part is \cite{ref:hatcher_topology}.  \vspace*{1em}
\label{Algebraic Topology}
\parttoc 
\chapter{The Fundamental Group}
\section{Homotopy and Paths}
\begin{dfn}
    Let \( f, g : X \to Y \) be continuous maps. We say that \( f \) and \( g \) are \vocab{homotopic} if there exists a \vocab{homotopy} between them; that is, there exists a continuous function 
    \[
        H : [0,1] \times X \to Y
    \]
    such that
    \[
        H(0, x) = f(x) \quad \text{and} \quad H(1, x) = g(x) \quad \text{for all } x \in X.
    \]
    Moreover, we say that two spaces \( X \) and \( Y \) are \vocab{homotopy equivalent} if there exist continuous maps \( f: X \to Y \) and \( g: Y \to X \) such that \( g \circ f \) is homotopic to \( \mathrm{Id}_{X} \) and \( f \circ g \) is homotopic to \( \mathrm{Id}_{Y} \).
\end{dfn}

\begin{lemma}
    Homotopy defines an equivalence relation on the set of continuous maps \( X \to Y \).
\end{lemma}

\begin{proof}
    We verify the three properties of an equivalence relation.

    \textbf{Reflexivity:}  
    For any \( f: X \to Y \), we have \( f \sim f \) via the constant homotopy
    \[
        H(t, x) = f(x) \quad \text{for all } t \in [0,1], \, x \in X.
    \]
    As a bonus, this shows that every continuous function lies in some equivalence class.

    \textbf{Symmetry:}  
    Suppose \( f \sim g \) via a homotopy \( H \). Define
    \[
        H'(t, x) := H(1 - t, x).
    \]
    Then \( H' \) is continuous, \( H'(0, x) = g(x) \), and \( H'(1, x) = f(x) \), so \( g \sim f \).

    \textbf{Transitivity:}  
    Suppose \( f_{1} \sim f_{2} \) via \( H_{1} \) and \( f_{2} \sim f_{3} \) via \( H_{2} \). Define
    \[
        H(t, x) =
        \begin{cases}
            H_{1}(2t, x), & 0 \le t \le \frac{1}{2}, \\
            H_{2}(2t - 1, x), & \frac{1}{2} < t \le 1.
        \end{cases}
    \]
    The function \( H \) is continuous since \( H_{1} \) and \( H_{2} \) are continuous and match at \( t = \frac{1}{2} \), where \( H_{1}(1, x) = H_{2}(0, x) = f_{2}(x) \). Moreover, \( H(0, x) = f_{1}(x) \) and \( H(1, x) = f_{3}(x) \), so \( f_{1} \sim f_{3} \).
\end{proof}

For now, we will limit our attention to homotopies of loops; that is, we will consider maps
\[
    \gamma: [0,1] \to X
\]
such that \( \gamma \) is continuous and \( \gamma(0) = \gamma(1) \).


\begin{dfn}
Let $\Gamma_x$ denote the set of loops $\gamma: [0,1] \to X$ with $\gamma(0) = \gamma(1) = x$. We define an equivalence relation $\sim$ on $\Gamma_x$ by setting $\gamma_1 \sim \gamma_2$ if there exists a homotopy between them, leaving the endpoints fixed. A \textbf{homotopy class} is an equivalence class of loops under this relation, denoted $[\gamma]$.
\end{dfn}

\begin{lemma}
Concatenation of loops induces a well-defined group operation on the homotopy classes $\Gamma_x/\sim$.
\end{lemma}
\begin{proof}
We must verify that concatenation gives $\Gamma_x/\sim$ the structure of a group. This requires showing well-definedness and the three group axioms.\\

 Suppose $\alpha_1 \sim \alpha_2$ and $\beta_1 \sim \beta_2$. We need to show that $\alpha_1 \cdot \beta_1 \sim \alpha_2 \cdot \beta_2$.

First, suppose $\alpha_1 \sim \alpha_2$. We show that $\alpha_1 \cdot \beta \sim \alpha_2 \cdot \beta$ for any loop $\beta$. By definition:
\[ (\alpha_i \cdot \beta)(t) = \begin{cases}
    \alpha_i(2t) & 0 \leq t \leq \frac{1}{2}\\
    \beta(2t-1) & \frac{1}{2} \leq t \leq 1
\end{cases} \]

Since $\alpha_1 \sim \alpha_2$, there exists a homotopy $H(s,t)$ with $H(0,t) = \alpha_1(t)$ and $H(1,t) = \alpha_2(t)$. Define
\[ H'(s,t) = \begin{cases}
    H(s,2t) & 0 \leq t \leq \frac{1}{2} \\
    \beta(2t-1) & \frac{1}{2} \leq t \leq 1
\end{cases} \]

Then $H'(0,t) = (\alpha_1 \cdot \beta)(t)$ and $H'(1,t) = (\alpha_2 \cdot \beta)(t)$, so $\alpha_1 \cdot \beta \sim \alpha_2 \cdot \beta$.

Similarly, if $\beta_1 \sim \beta_2$, then $\alpha \cdot \beta_1 \sim \alpha \cdot \beta_2$ for any $\alpha$. Combining these results shows that the operation $[\alpha] \cdot [\beta] = [\alpha \cdot \beta]$ is well-defined on homotopy classes.\\

 Let $\mathbf{1}$ denote the constant loop $\mathbf{1}(t) = x$ for all $t \in [0,1]$. For any loop $\gamma$:
\[ (\mathbf{1} \cdot \gamma)(t) = \begin{cases}
    \mathbf{1}(2t) = x & 0 \leq t \leq \frac{1}{2}\\
    \gamma(2t-1) & \frac{1}{2} \leq t \leq 1
\end{cases} \]

Define a homotopy $F(s,t)$ by:
\[ F(s,t) = \begin{cases}
    x & 0 \leq t \leq \frac{1-s}{2}\\
    \gamma\left(\frac{2t-(1-s)}{1+s}\right) & \frac{1-s}{2} \leq t \leq 1
\end{cases} \]

Then $F(0,t) = (\mathbf{1} \cdot \gamma)(t)$ and $F(1,t) = \gamma(t)$, so $\mathbf{1} \cdot \gamma \sim \gamma$. Similarly, $\gamma \cdot \mathbf{1} \sim \gamma$.

 For any loop $\gamma$, define its inverse $\overline{\gamma}$ by $\overline{\gamma}(t) = \gamma(1-t)$. We show that $\gamma \cdot \overline{\gamma} \sim \mathbf{1}$.

The concatenation $\gamma \cdot \overline{\gamma}$ is given by:
\[ (\gamma \cdot \overline{\gamma})(t) = \begin{cases}
    \gamma(2t) & 0 \leq t \leq \frac{1}{2}\\
    \gamma(2(1-t)) & \frac{1}{2} \leq t \leq 1
\end{cases} \]

Define a homotopy $G(s,t)$ by:
\[ G(s,t) = \begin{cases}
    \gamma(2t(1-s)) & 0 \leq t \leq \frac{1}{2}\\
    \gamma(2(1-t)(1-s)) & \frac{1}{2} \leq t \leq 1
\end{cases} \]

Then $G(0,t) = (\gamma \cdot \overline{\gamma})(t)$ and $G(1,t) = \gamma(0) = x$ for all $t$, so $\gamma \cdot \overline{\gamma} \sim \mathbf{1}$. Similarly, $\overline{\gamma} \cdot \gamma \sim \mathbf{1}$.\\

 For loops $\alpha, \beta, \gamma$, we need $(\alpha \cdot \beta) \cdot \gamma \sim \alpha \cdot (\beta \cdot \gamma)$.

The loop $(\alpha \cdot \beta) \cdot \gamma$ is defined by:
\[ ((\alpha \cdot \beta) \cdot \gamma)(t) = \begin{cases}
    \alpha(4t) & 0 \leq t \leq \frac{1}{4}\\\\
    \beta(4t-1) & \frac{1}{4} \leq t \leq \frac{1}{2}\\\\
    \gamma(2t-1) & \frac{1}{2} \leq t \leq 1
\end{cases} \]

The loop $\alpha \cdot (\beta \cdot \gamma)$ is defined by:
\[ (\alpha \cdot (\beta \cdot \gamma))(t) = \begin{cases}
    \alpha(2t) & 0 \leq t \leq \frac{1}{2}\\\\
    \beta(4t-2) & \frac{1}{2} \leq t \leq \frac{3}{4}\\\\
    \gamma(4t-3) & \frac{3}{4} \leq t \leq 1
\end{cases} \]

These differ only in the timing of transitions between the three constituent loops. A linear reparametrization homotopy can interpolate between these two parameterizations, establishing the required homotopy equivalence.

Therefore, $\Gamma_x/\sim$ forms a group under the concatenation operation.
\end{proof}

\begin{dfn}
    Let \( X \) be a topological space and \( x \in X \).
    We will call \( \Gamma_{x}/ \sim \), the \vocab{fundamental group with basepoint at \( x \)} and denote it by \( \pi_{1}\left( X, x \right) \).
\end{dfn}






\chapter{Homology}
\input{content/geometry-and-topology/algebraic-topology/homology/homology}

\part{Manifolds}
\vspace*{2em} The primary resources for this part are \cite{ref:lee_smooth} and \cite{ref:lee_topological}. \cite{ref:lee_riemannian}, \cite{ref:lee_complex},\cite{ref:janich_vector}, \cite{ref:hubbard_vector}, \cite{ref:jost_riemannian}, \cite{ref:baez_knots}, \cite{ref:warner_manifolds} have been used as well. \vspace*{-1em}
\label{Manifolds}
\parttoc
\chapter{Definitions and Examples}


\begin{dfn}
    A \vocab{topological manifold} of dimension $n$ is a second-countable Hausdorff topological space \( M \) that is locally homeomorphic to \( \mathbb{R}^n \), together with a collection of open sets \( U_{\alpha} \) and corresponding homeomorphisms \( \varphi_{\alpha}: U_{\alpha} \to \mathbb{R}^{n} \) where \( \widetilde{U}_{\alpha} = \varphi_{\alpha}(U_{\alpha}) \) denotes the image of \( U_{\alpha} \) under \( \varphi_{\alpha} \) such that 
    \begin{enumerate}[label=\textbf{\roman*)}]
        \item The collection \( \left\{ U_{\alpha} \right\} \) covers \( M \). 
        \item For any \( \alpha, \beta \) such that \( U_{\alpha} \cap U_{\beta} \neq \emptyset \), the \vocab{transition maps} \( \varphi_{\alpha} \circ \varphi_{\beta}^{-1}: \varphi_{\beta}(U_{\alpha} \cap U_{\beta}) \to \varphi_{\alpha}(U_{\alpha} \cap U_{\beta})\) are continuous.
    \end{enumerate}
    The pairs \( (U_{\alpha}, \varphi_{\alpha}) \) are called \vocab{charts}, and the collection \( \{(U_{\alpha}, \varphi_{\alpha})\} \) forms an \vocab{atlas} for \( M \).
    
    Applying additional conditions to the second criterion will give us different types of manifolds. 
    \begin{itemize}
        \item If the transition maps are differentiable, we say that the manifold is differentiable. 
        \item[!] If the transition maps are smooth, then the manifold is smooth. 
        \item If the transition maps are analytic, then the manifold is analytic. 
        \item If the transition maps are holomorphic, then the manifold is complex.
    \end{itemize}
    When we say manifold without any other adjective, we will typically mean that it is a smooth manifold.
\end{dfn}

\begin{exercise}
    Show that the following three definitions of topological manifold are equivalent:
    \begin{enumerate}[label=\textbf{\roman*)}]
        \item The definition given above (where $\widetilde{U}_{\alpha} = \varphi_{\alpha}(U_{\alpha})$ can be any open subset of $\mathbb{R}^{n}$).
        \item The same definition, but requiring each $\varphi_{\alpha}: U_{\alpha} \to \mathbb{R}^{n}$ to be a homeomorphism onto all of $\mathbb{R}^{n}$.
        \item The same definition, but requiring each $\widetilde{U}_{\alpha} = \varphi_{\alpha}(U_{\alpha})$ to be an open ball in $\mathbb{R}^{n}$.
    \end{enumerate}
\end{exercise}
\begin{solution}
     It is sufficient to just show \textbf{(i)} \( \iff \) \textbf{(iii)} and \textbf{(ii)} \( \iff \) \textbf{(iii)}. \\
    \textbf{(iii)} \( \Rightarrow \)  \textbf{(i)} Trivial: open balls are themselves open subsets of \( \mathbb{R}^{n} \).\\
    \textbf{(i)} \( \Rightarrow \) \textbf{(iii)} Suppose \( p \in M \) and pick \( \varphi_{p} \) and \( U_{p} \) containing \( p \) compliant with the conditions of \textbf{(i)}. Given 
    \[ \varphi_{p}^{-1} : \widetilde{U_{p}} \to U_{p}  \]
    we may restrict \( \varphi_{p}^{-1}\) to an an open ball \( B_{\varphi_{p}(p)} \) containing \( \varphi_{p}(p) \) since \( \widetilde{U_{p}} \) is open in \( \mathbb{R}^{n} \). The restriction map 
    \[ \varphi_{p}^{-1} \eval_{B_{\varphi_{p}(p)}} \hspace{-2.5em}: B_{\varphi_{p} \left( p \right)} \to  \varphi_{p}^{-1} \left( B_{\varphi_{p}(p)} \right) \]
    is a local homeomorphism that satisfies the goal conditions. \\ 
    \textbf{(iii)} \( \Rightarrow \) \textbf{(ii)} Suppose that \( \varphi_{\alpha} \) maps the open set \( U_{\alpha} \) to \( B\left( \vb{x} ; \epsilon \right) \) with \( \varphi_{\alpha} (p) = \vb{x} \). We then further define a function \( f: B \left( \vb{x}; \epsilon \right) \to \mathbb{R}^{n} \) by 
    \[ f \left( \vb{y} \right) = \begin{cases}
        \tan{ \left( \frac{\pi}{2 \epsilon} \cdot \norm{\vb{y} -\vb{x}} \right) } \cdot \frac{\vb{y} -\vb{x}}{\norm{\vb{y} - \vb{x}}} \quad  & \text{if } \vb{y} \neq \vb{x} \\
        \vb{0}\quad & \text{if } \vb{y} = \vb{x}
    \end{cases} \]
Although the definition of \( f \) looks a bit unwieldy, all it is really doing is shifting the epsilon ball so it is centered at \( \vb{0}\), then stretching it outward along each ray with a tangent-based scaling that sends the boundary to infinity. The center is set to map to \( \vb{0} \) by hand.  We leave it to the reader to verify that \( f \) is doing what we claim and that it is a homeomorphism between \( B \left( \vb{x} ; \epsilon \right) \) and \( \mathbb{R}^{n} \). The composition \( f \circ \varphi_{\alpha} \) is the desired homeomorphism between \( U_{\alpha} \) and \( \mathbb{R}^{n} \).\\
\textbf{(ii)} \( \Rightarrow \) \textbf{(iii)} Assume the conditions of \textbf{(ii)}. Composing \( \varphi_\alpha \) with \( f^{-1} \) yields a chart whose image is an open ball.
\end{solution}


\chapter{Maps}
\section{Smooth maps to \( \mathbb{R}^{n} \)}

\begin{dfn}
    Let \( M  \) be a manifold. A function \( f: M \to \mathbb{R} \) is smooth at \( p \in M \) if there exists a chart, \( \left( U_{\alpha}, \varphi_{\alpha} \right) \) the map 
    \[ f \circ \varphi_{\alpha}^{-1} : \widetilde{U_{\alpha}} \to \mathbb{R} \] is smooth at \( \varphi_{\alpha}(p) .\) A map is smooth if it smooth everywhere.
\end{dfn}

\begin{exercise}
    If \( f:M \to \mathbb{R} \) is smooth at \( p  \) for a particular chart, show it is smooth for all charts containing \( p \).
\end{exercise}
\begin{solution}
    Suppose that \( f \) is smooth at \( p \) in the chart \( \left( U_{\alpha}, \varphi_{\alpha} \right) \). Pick chart \( \left( U_{\beta}, \varphi_{\beta} \right) \) that contains \( p \). Then 
    \[ f \circ \varphi_{\beta}^{-1} = \left( f \circ \varphi^{-1}_\alpha \right) \circ \left( \varphi_{\alpha} \circ \varphi^{-1}_\beta \right): \varphi_{\beta} \left( U_{\alpha} \cap U_{\beta} \right) \to \mathbb{R} \] is smooth at \( \varphi_{\beta} \left( p \right) \) as it is the composition of smooth maps.
\end{solution}

\begin{exercise}
    Suppose that \( f: M \to \mathbb{R} \) and \( g : \mathbb{R} \to \mathbb{R} \) are both smooth. Show that \( g \circ f: M \to \mathbb{R} \) is smooth.
\end{exercise}
\begin{solution}
    
\end{solution}


\begin{theorem}
    Let \( \mathcal{C}^{\infty} \left( M \right) \) be the set of all smooth maps \( M \to \mathbb{R} \). Then \( \mathcal{C}^{\infty} \left( M \right) \) is a commutative ring as well as a commutative and associative \( \mathbb{R} \)-algebra where 
    \begin{align*}
        \left( f +g \right)(p) &:= f(p) +g(p) \\
        \left( fg \right)(p) &:= f(p)g(p) \\
        \left( \lambda f \right) &:= \lambda f(p)
    \end{align*}
    For every \( f,g \in \mathcal{C}^{\infty}(M) \), \( p \in M \) and \( \lambda \in \mathbb{R} \).
\end{theorem}
\begin{proof}
  
\end{proof}


\chapter{Tangent Vectors}


The goal of this part is to develop and expand the mechanisms of calculus to smooth manifolds. This chapter aims to introduce one of the main actors: the tangent vector. At first, it might be unclear how one defines a vector on a manifold. We will present several ways to do so, as well as convince you of their equivalence and show that they really do behave like vectors in "flat space."

\section{Different Ways of Defining Tangent Vectors}
\subsection{Algebraically}


If you go back to the multivariable calculus part, you will see a rather inconspicuous operation called the "directional derivative." Given a function \( f: \mathbb{R}^{n} \to \mathbb{R} \), a point \( a \in \mathbb{R}^{n} \), and a direction \( \vb{v} \in \mathbb{R}^{n} \), the directional derivative tells you the rate of change of \( f \) at \( a \) in the direction of \( \vb{v} \). It has the benefit of being easy to compute, namely \( \left( D_{\vb{v}}(f) \right)(a)= \left( \largetriangledown f \right)(a) \cdot \vb{v} \), and it satisfies established derivative rules, namely linearity and the \textbf{Leibniz rule.} \\
To define a vector on a manifold, we will leverage a surprising fact: if we have a linear map \( \vb{v}: \mathcal{C}^{\infty} \left( M \right) \to \mathbb{R} \) that satisfies the Leibniz rule, there is sufficient information to \emph{define} a vector from this.

\begin{dfn}
    Let \( M \) be a manifold and let \( p \in M \).  
    A \vocab{derivation at \( p \)} is a linear map 
    \[
        \vb{v} : C^{\infty}_{p}(M) \to \mathbb{R},
    \]
    where \( C^{\infty}_{p}(M) \) denotes the algebra of germs of smooth functions at \( p \), such that the Leibniz rule is satisfied: for all \( f,g \in C^{\infty}_{p}(M) \),
    \[
        \vb{v}(fg) = f(p)\,\vb{v}(g) + g(p)\,\vb{v}(f).
    \]
\end{dfn}



\subsection{As an Equivalence Class of Tangents to Paths}


For this method, we will leverage one of the gifts that manifolds give us: comparison to \( \mathbb{R}^{n} \). If we are given a smooth path \( \gamma: \left( - \epsilon, \epsilon \right) \to M \), where \( M \) is an \( n \)-manifold, then we may look at \( \gamma(0) \in U \subseteq M \) and consider \( \left(  \varphi \circ \gamma  \right)(0) \in \mathbb{R}^{n} \). Since \( \varphi \circ \gamma: (-\epsilon, \epsilon) \to \mathbb{R}^{n} \), the concept of a derivative exists and we can define \( \left( \varphi \circ \gamma \right)'(0) \), which is the usual tangent vector to a path in \( \mathbb{R}^{n} \). We then regard this derivative as the definition of a tangent vector to \( M \) at \( p = \gamma(0) \), and we can break for lunch, right? \\

Unfortunately, our gazpacho must wait. We have some details to iron out. Suppose that we have two curves \( \gamma_{1} \) and \( \gamma_{2} \) that both pass through \( p \) and have the same slope at \( 0 \). We want these two paths to represent the same tangent vector. This can be remedied by attaching an equivalence relation
\[
    \gamma_{1} \sim \gamma_{2} \quad \text{if there exists a chart } \varphi \text{ such that } \left( \varphi \circ \gamma_{1} \right)'(0) = \left( \varphi \circ \gamma_{2} \right)'(0).
\]
It might seem problematic that our definition relies on a particular choice of charts. We will show that as soon as we have a chart containing \( p \) that satisfies our definition, then all charts containing \( p \) will satisfy it as well.\\

Finally, we will need to give a vector space structure to our construction. This is done by declaring that addition and scalar multiplication of tangent vectors are defined in local coordinates.
\begin{lemma}
    Let \( M \) be a manifold and \( p \in M \). Let 
    \[ V = \left\{ \gamma: \left( - \epsilon, \epsilon \right) \to M \ \middle| \ \gamma(0) = p, \ \gamma \text{ is smooth.} \right\}. \]
    Define a relation \( \sim \) on \( V \), \( \gamma_{1} \sim \gamma_{2} \) if there is a chart \( \varphi \) such that \(  \left( \varphi \circ \gamma_{1} \right)'(0) = \left( \varphi \circ \gamma_{2} \right)'(0) \). Then \( \sim \) is an equivalence relation on \( V \).
\end{lemma}
\begin{proof}
    Reflexivity and symmetry are easy to check.\\
    Transitivity requires a bit more work. Suppose that \( \gamma_{1} \sim \gamma_{2} \) and \( \gamma_{2} \sim \gamma_{3} \). Then, by definition, there are coordinate charts \( \varphi \) and \( \psi \) such that 
    \[ \left( \varphi \circ \gamma_{1} \right)'(0)= \left( \varphi \circ \gamma_{2} \right)'(0) \quad \text{and} \quad \left( \psi \circ \gamma_{2} \right)'(0) = \left( \psi \circ \gamma_{3} \right)'(0).  \]
    
    Since \( \varphi \) and \( \psi \) are charts containing \( \gamma_{i}(0) = p \), we have that \( \psi \circ \varphi^{-1} \) is a smooth map from \( \mathbb{R}^{n} \) to \( \mathbb{R}^{n} \). By the chain rule,
    \begin{align*}
         \left( \psi \circ \gamma_{1} \right)'(0) &= \left( \psi \circ \varphi^{-1} \circ \varphi \circ \gamma_{1} \right)'(0)\\
         &= \left( \left[ \psi \circ \varphi^{-1} \right] \left( \varphi \left( \gamma_{1}(0) \right) \right) \right)'\\
         &=\left( \left[ \psi \circ \varphi^{-1} \right] \left( \varphi \left( p\right) \right) \right)'\\
         &= D(\psi \circ \varphi^{-1})|_{\varphi(p)} \cdot \left( \varphi \circ \gamma_{1} \right)'(0).
    \end{align*}
    
    
    Since \( \gamma_1(0) = \gamma_2(0) = p \), we have \( \varphi(\gamma_1(0)) = \varphi(\gamma_2(0)) = \varphi(p) \). Therefore, \( D(\psi \circ \varphi^{-1})|_{\varphi(\gamma_1(0))} = D(\psi \circ \varphi^{-1})|_{\varphi(\gamma_2(0))} = D(\psi \circ \varphi^{-1})|_{\varphi(p)} \).
    
    Using this and the fact that \( \left( \varphi \circ \gamma_{1} \right)'(0)= \left( \varphi \circ \gamma_{2} \right)'(0) \), we get:
    \begin{align*}
        \left( \psi \circ \gamma_{1} \right)'(0) &= D(\psi \circ \varphi^{-1})|_{\varphi(p)} \cdot \left( \varphi \circ \gamma_{1} \right)'(0) \\
        &= D(\psi \circ \varphi^{-1})|_{\varphi(p)} \cdot \left( \varphi \circ \gamma_{2} \right)'(0) \tag{Since $\left( \varphi \circ \gamma_{1} \right)'(0)= \left( \varphi \circ \gamma_{2} \right)'(0)$}\\
        &=  \left( \psi \circ \varphi^{-1} \circ \varphi \circ \gamma_{2} \right)'(0) \tag{By the chain rule}\\
        &=  \left( \psi \circ \gamma_{2} \right)'(0) \\
        &=  \left( \psi \circ \gamma_{3} \right)'(0) \tag{Since $\left( \psi \circ \gamma_{2} \right)'(0) = \left( \psi \circ \gamma_{3} \right)'(0)$.}
    \end{align*}
    
    Therefore, \( \gamma_{1} \sim \gamma_{3} \), establishing transitivity.
\end{proof}






\part{ Lie Groups, Lie Algebras, and their Representations}
\label{part: Lie Groups, Lie Algebras, and their Representations}
\parttoc
\chapter{Definition of a Lie Group and Basic Examples}


This chapter is intended to be read in parallel with the chapter on Lie algebras.

\section{Definitions}
\begin{dfn}
We say that a group \( G \) \vocab{acts} on a set \( A \) if there is a map
\[
    \cdot \colon G \times A \to A, \quad (g,a) \mapsto g \cdot a
\]
such that for all \(g_{1}, g_{2} \in G\) and \(a \in A\):
\begin{enumerate}[label=\textbf{\roman*)}]
    \item \( g_{1} \cdot \big( g_{2} \cdot a \big) = (g_{1} g_{2}) \cdot a \), where \(g_{1} g_{2}\) denotes the product in \(G\).
    \item \( e_{G} \cdot a = a \), where \(e_G\) is the identity of \(G\).
\end{enumerate}
We call this a \vocab{group action} of \( G \) on \( A \). We will denote this as \( G \acts A \).
\end{dfn}


Instead of viewing the group action as a map from \( G \times A \) to \( A \), we could actually adopt the view that a group action is a map from \( G \) to \( S_{A} \), where \( S_{A} \) is the collection of bijections on \( A \). Moreover, this map is a homomorphism! This requires proof. 

\begin{theorem}
    Let \(G\) be a group acting on a set \(A\) via a map
    \[
        \cdot \colon G \times A \to A.
    \]
    Then there is an associated map
    \[
        \varphi \colon G \to S_{A}, \quad g \mapsto \varphi \left( g \right)  \text{ such that }
        \left[ \varphi(g) \right] (a) = g \cdot a,
    \]
    where \(S_{A}\) is the symmetric group on \(A\) (the set of all bijections \(A \to A\)).
    Moreover, \(\varphi\) is a group homomorphism.

    Conversely, any group homomorphism \(\varphi \colon G \to S_{A}\) defines a group action of \(G\) on \(A\) via
    \[
        g \cdot a := \left[ \varphi(g) \right](a).
    \]
\end{theorem}
\begin{proof}
First, we will verify that for each \(g \in G\), the map \(\varphi(g)\colon A \to A\) is a bijection. To do this, it suffices to show that \(\varphi(g)\) has a two-sided inverse. The natural candidate is
\[
\left[ \varphi(g) \right]^{-1} := \varphi(g^{-1}).
\]

For any \(a \in A\), we have
\begin{align*}
\big(\varphi(g^{-1}) \circ \varphi(g)\big)(a) &= \varphi(g^{-1})\big(\varphi(g)(a)\big) \\
&= \varphi(g^{-1})(g \cdot a) && \text{(by definition of \(\varphi\))}\\
&= g^{-1} \cdot (g \cdot a) && \text{(by definition of the group action)}\\
&= (g^{-1} g) \cdot a \\
&= e \cdot a \\
&= a && \text{(identity property of the action).}
\end{align*}

Similarly, \(\varphi(g) \circ \varphi(g^{-1}) = \operatorname{id}_A\), so \(\varphi(g)\) is bijective.\\
Now to show that \( \varphi \) is a homomorphism, pick any \( g,h \in G \) and \( a \in A \), then 
\begin{align*}
   \left[  \varphi \left( gh \right) \right] \left( a \right) &= \left( gh \right) \cdot a\\
   &= g \cdot \left( h \cdot a \right)\\
   &= g \cdot \left( \varphi \left( h  \right) \left( a \right) \right)\\
   &= \varphi \left( g \right) \left( \varphi \left( h  \right)\left( a \right) \right) \\
   &= \left[ \varphi \left( g \right) \circ \varphi \left( h \right) \right] \left( a \right)
\end{align*}
which shows that \( \varphi \left( gh \right) = \varphi \left( g \right) \circ \varphi \left( h \right) \).\\
Finally, to show that any group homomorphism \( \varphi: G \to S_{A} \) defines a \( G \)-action on \( A \) by \( g \cdot a = \left[ \varphi \left( g \right) \right] \left( a \right) \), we just need to verify the axioms of a group action. For any \( g_{1}, g_{2} \in G \) and \( a \in A \), we have
\begin{align*}
    \left( g_{1} g_{2} \right) \cdot a &= \left[ \varphi \left( g_{1}g_{2} \right) \right] \left( a \right)\\
    &= \left[ \varphi \left( g_{1} \right) \circ \varphi \left( g_{2} \right) \right] \left( a \right) \tag{Since $\varphi$ is a homomorphism.}\\
    &= \varphi \left( g_{1} \right) \left( g_{2} \cdot a \right) \\
    &= g_{1} \cdot \left( g_{2} \cdot a \right)
\end{align*}
For the identity property,
\begin{align*}
    e \cdot a &= \left[ \varphi \left( e \right) \right] \left( a \right)\\
    &= \mathrm{Id}_{A} \left( a \right) \tag{Since $\varphi$ is a homomorphism.}\\
    &= a
\end{align*}
which completes the proof.
\end{proof}



The above result highlights why group theory is such a powerful tool. Any group action corresponds to a subgroup of the group of all bijections on \(A\). This means that groups provide a powerful angle of attack for tackling and understanding symmetries of any set. When we restrict our attention to bijections preserving additional structure, such as homeomorphisms in topology, biholomorphisms in complex analysis, or linear transformations in linear algebra (this is the focus of representation theory), the same framework applies, giving us a systematic way to understand and manipulate these transformations through their group properties.

\begin{dfn}
    If the homomorphism, \( \varphi: G \to S_{A} \) is injective, we say that the associated group action of \( G \) on \( A \) \vocab{acts faithfully}.
\end{dfn}

\begin{lemma}
    We define the \vocab{kernel} of a \( G \) action on \( A \) to be 
    \[ \mathrm{Ker}\left( G \acts A \right)=\left\{ g \in G \ \middle| \ g \cdot a =a \text{ for all } a \in A\right\} \]
    Then \(\mathrm{Ker}\left( G \acts A \right) \unlhd G \).
\end{lemma}
\begin{proof}
    This fact is readily apparent from the result that if \( \varphi: G \to S_{A} \) is a homomorphism, then \( \mathrm{Ker} \left( \varphi \right) \) is a normal subgroup of \( G \). However, we will prove this result using the group action perspective for practice. \\
    First we will show that \( \mathrm{Ker} \left( G \acts A \right) \) is a subgroup. Since \( G \acts A \), \( e \in \mathrm{Ker} \left( G \acts A \right) \). Pick any \( g,h \in \mathrm{Ker} \left( G \acts A \right) \). Then 
    \begin{align*}
        a &= g \cdot a \tag{Since $g\in \mathrm{Ker} \left( G \acts A \right)$}\\
        &= g \cdot \left( e \cdot a \right) \\
        &= g \cdot \left( \left( h^{-1} h \right) \cdot a \right) \\
        &= g \cdot \left( h^{-1} \cdot \left( h \cdot a \right)  \right) \\
        &= g \cdot \left( h^{-1} \cdot a \right)  \tag{Since $h\in \mathrm{Ker} \left( G \acts A \right)$} \\
        &= \left( gh^{-1} \right) \cdot a
    \end{align*}
   This shows that \( gh^{-1} \in \mathrm{Ker} \left( G \acts A \right) \). By the subgroup test, \( \mathrm{Ker} \left( G \acts A \right) \le G \). \\
   Now to show that \( \mathrm{Ker} \left( G \acts A \right) \unlhd G \), pick any \( g \in  \mathrm{Ker} \left( G \acts A \right) \), \( h \in G \), and \( a \in A \). Then 
   \begin{align*}
    \left( hgh^{-1} \right) \cdot a &= \left( hg \right) \cdot \left( h^{-1} \cdot a \right)\\
    &= h \cdot \left( g \cdot \left( h^{-1}  \cdot a\right) \right) \\
    &= h \cdot \left( h^{-1} \cdot a \right) \tag{Since $g\in \mathrm{Ker} \left( G \acts A \right)$} \\
    &= \left( h h^{-1} \right) \cdot a \\
    &= e \cdot a \\
    &= a
   \end{align*}
   This shows that \( \mathrm{Ker} \left( G \acts A \right) \) is a normal subgroup of \( G \).
\end{proof}

\begin{exercise}
    Show that the kernel of a \( G \) action on \( A \) contains only the identity if and only if \( G \) acts faithfully on \( A \).
\end{exercise}
\begin{solution}
    Let $\varphi: G \to S_A$ be the homomorphism associated to the action.

    $(\Rightarrow)$ Suppose $\ker \varphi \neq \{e\}$. Then there exists $g \in G$, $g \neq e$, such that $\varphi(g) = \mathrm{id}_A$. In particular, for all $a \in A$ we have $g \cdot a = a$, so $g$ and $e$ induce the same permutation of $A$. Hence $\varphi$ is not injective, and the action is not faithful.

    $(\Leftarrow)$ Conversely, if the action is not faithful, then $\varphi$ is not injective. Thus there exist distinct $g,h \in G$ such that $\varphi(g) = \varphi(h)$. Then
    \[
        \varphi(gh^{-1}) = \varphi(g)\varphi(h)^{-1} = \varphi(h)\varphi(h)^{-1} = \mathrm{id}_A,
    \]
    so $gh^{-1} \in \ker \varphi$ and $gh^{-1} \neq e$. Therefore the kernel is nontrivial.
\end{solution}




\section{Quaternions}
A useful source of examples of Lie groups involves the quaternions. We will dedicate this section to defining them and exploring some of their properties.

\subsection{Basic Definition}
\begin{dfn}
    The \vocab{quaternions}, denoted \( \mathbb{H} \), are defined to be the associative real algebra
    \[
        \mathbb{H} = \left\{ a + b\vb{i} + c\vb{j} + d\vb{k} \ \middle\vert\  a, b, c, d \in \mathbb{R} \right\},
    \]
    where the fundamental quaternion units satisfy
    \[
        \vb{i}^{2} = \vb{j}^{2} = \vb{k}^{2} = \vb{i} \vb{j} \vb{k} = -1.
    \]
\end{dfn}

\begin{lemma}
    The quaternion units obey the following multiplication rules:
    \begin{align*}
        \vb{i}\vb{j} &= \vb{k}, & \vb{j}\vb{i} &= -\vb{k}, \\
        \vb{j}\vb{k} &= \vb{i}, & \vb{k}\vb{j} &= -\vb{i}, \\
        \vb{k}\vb{i} &= \vb{j}, & \vb{i}\vb{k} &= -\vb{j}.
    \end{align*}
\end{lemma}

\begin{proof}
    We will just show one of these equalities. The rest are very similar.  \\
    We have 
    \[ \vb{i}\vb{j}\vb{k}=-1 \]
    Multiplying on the right by \( \vb{k} \) gives us 
    \[ \vb{i}\vb{j}\vb{k}\vb{k}=-\vb{k} \]
    \[ -1\vb{i}\vb{j}=-\vb{k} \text{ since } \vb{k}^{2}=-1 \]
    \[ \vb{i}\vb{j}=\vb{k} \]
\end{proof}
The products of \( \vb{i}, \vb{j} \) and \( \vb{k} \) can be memorized with the following diagram:
\[\begin{tikzcd}
	& i \\
	\\
	k && j
	\arrow[curve={height=-6pt}, from=1-2, to=3-3]
	\arrow[curve={height=-6pt}, from=3-1, to=1-2]
	\arrow[curve={height=-6pt}, from=3-3, to=3-1]
\end{tikzcd}\]

\begin{dfn}
    The \vocab{real quaternions} is the set 
    \[ \Re (\mathbb{H}) = \left\{  a =a1 \in \mathbb{H} \middle\vert a \in \mathbb{R}\right\}. \]
    The \vocab{imaginary quaternions}  are similarly defined 
    \[ \Im \left( \mathbb{H} \right) = \left\{ b \vb{i} + c \vb{j} + d \vb{k} \middle\vert b,c,d \in \mathbb{R} \right\} \]
\end{dfn}

\begin{dfn}
    The \vocab{conjugate} of a quaternion \( w = a + b \vb{i} + c \vb{j} + d \vb{k} \) is 
    \[ \overline{w}= a - b \vb{i} - c \vb{j} - d\vb{k} \]
    and the \vocab{norm squared} is 
    \[ \norm{w}^{2} =a^{2}+b^{2}+c^{2}+d^{2} \]
\end{dfn}

\begin{lemma}
    For a quaternion \( w = a + b\vb{i} + c\vb{j} + d\vb{k} \), we have:
    \[
    \norm{w}^2 = w \overline{w} = \overline{w} w.
    \]
\end{lemma}

\begin{proof}
    Let \( w = a + b\vb{i} + c\vb{j} + d\vb{k} \). Then the conjugate of \( w \) is:
    \[
    \overline{w} = a - b\vb{i} - c\vb{j} - d\vb{k}.
    \]
    Now compute:
    \begin{align*}
        w \overline{w} &= (a + b\vb{i} + c\vb{j} + d\vb{k})(a - b\vb{i} - c\vb{j} - d\vb{k}) \\
        &= a^2 - ab\vb{i} - ac\vb{j} - ad\vb{k} 
        + ab\vb{i} - b^2\vb{i}^2 - bc\vb{i}\vb{j} - bd\vb{i}\vb{k} \\
        &\quad + ac\vb{j} - bc\vb{j}\vb{i} - c^2\vb{j}^2 - cd\vb{j}\vb{k} \\
        &\quad + ad\vb{k} - bd\vb{k}\vb{i} - cd\vb{k}\vb{j} - d^2\vb{k}^2
    \end{align*}

    Substituting in:
    \begin{align*}
        w \overline{w} &= a^2 + b^2 + c^2 + d^2 \\
        &\quad - bc\vb{k} - bd(-\vb{j}) - bc(-\vb{k}) - cd\vb{i} \\
        &\quad + bd\vb{j} + cd(-\vb{i}) \\
        &= a^2 + b^2 + c^2 + d^2 \quad \text{(all imaginary terms cancel out)}.
    \end{align*}

    Therefore:
    \[
    w \overline{w} = \norm{w}^2 = a^2 + b^2 + c^2 + d^2.
    \]
    Similarly, \( \overline{w}w = \norm{w}^2 \), since quaternion norm is preserved under conjugation order.

    \qedhere
\end{proof}


\subsection{Quaternionic Matrices}

Since any quaternion \( q \in \mathbb{H} \) can be written uniquely as \( q=q_{1}+\vb{j}q_{2} \) for \( q_{1}, q_{2} \in \mathbb{C} \), any \( n \times n \) quaternionic matrix \( A \in \mathrm{Mat} \left( \mathbb{H}, n \times n \right) \) can be written uniquely as \( A = A_{1}+\vb{j}A_{2} \) for \( A_{1}, A_{2} \in \mathrm{Mat} \left( \mathbb{C}, n \times n \right) \). This allows us to define the following. 

\begin{dfn}
    Let \( A \in \mathrm{Mat} \left( \mathbb{H}, n \times n \right) \) and \( A_{1}, A_{2} \in \mathrm{Mat} \left( \mathbb{C}, n \times n \right) \) such that \( A = A_{1}+\vb{j}A_{2} \). The \vocab{adjoint} of \( A \) is the matrix \( \chi_{A} \in \mathrm{Mat} \left( \mathbb{C} , 2n \times 2n\right) \) defined to be 
    \[ \chi_{A} = \begin{pmatrix}
    A_{1}  & - \overline{A_{2}}\\
    A_{2} & \overline{A_{1}}
    \end{pmatrix} \]
\end{dfn}




\chapter{Definition of Lie algebras}


\section{Definitions and Examples}
\begin{dfn}
    A \vocab{Lie algebra} is a vector space \( \mathfrak{g} \) equipped with a bilinear map \( [\cdot, \cdot]: \mathfrak{g} \times \mathfrak{g} \to \mathfrak{g} \) such that the following properties hold:
    \begin{description}[style=nextline, leftmargin=3em]
        \item[Skew-symmetry] For all \( x, y \in \mathfrak{g} \), we have
        \[
        [x, y] = -[y, x].
        \]
        
        \item[Jacobi identity] For all \( x, y, z \in \mathfrak{g} \), we have
        \[
        [x, [y, z]] + [y, [z, x]] + [z, [x, y]] = 0.
        \]
    \end{description}
\end{dfn}

\begin{example}
    Suppose that \( \mathbf{V} \) is a vector space. The set of linear maps \( \mathcal{L} \left( \mathbf{V} \right) \) can be made into a Lie algebra where 
    \[  \left[ \varphi, \psi \right]:= \varphi \circ \psi - \psi \circ \varphi \] for all \( \varphi, \psi \in \mathcal{L} \left(  \mathbf{V} \right) \).
\end{example}


\begin{example}
    Suppose that \( \mathbf{V} \) is vector space with some multiplication equipped. Recall that a derivation on \( V \) is any linear map \( \varphi: \mathbf{V} \to \mathbf{V} \) such that 
    \[ \varphi \left( \vb{x} \cdot \vb{y} \right) = \varphi \left( \vb{x} \right) \cdot \vb{y} + \vb{x} \cdot \varphi \left( \vb{y} \right) .\]
    The collection of all such linear maps is denoted by \( \mathrm{Der} \left(  \mathbf{V} \right) .\) Then \( \mathrm{Der} \left( \mathbf{V} \right) \) is a \vocab{Lie subalgebra} of \( \mathcal{L} \left( \mathbf{V} \right) \). To show this, pick \( \varphi, \psi \in \mathrm{Der} \left( \mathbf{V} \right) \). Then,
    \begin{align*}
        \left[ \varphi, \psi \right] \left( \vb{x} \cdot \vb{y} \right) &= \left( \varphi \circ \psi - \psi \circ \varphi \right) \left( \vb{x} \cdot \vb{y} \right) \\\\
        &= \left( \varphi \circ \psi \right) \left( \vb{x} \cdot \vb{y} \right) - \left( \psi \circ \varphi \right) \left( \vb{x} \cdot \vb{y} \right) \\\\
        &= \varphi \left( \psi \left( \vb{x} \cdot \vb{y} \right) \right) - \psi \left( \varphi \left( \vb{x} \cdot \vb{y} \right) \right) \\
        &= \varphi\left( \psi (\vb{x}) \cdot \vb{y} + \vb{x} \cdot \psi (\vb{y}) \right) - \psi \left( \varphi(\vb{x}) \cdot \vb{y} + \vb{x} \cdot \varphi (\vb{y}) \right)\\\\
        &=\varphi \left( \psi (\vb{x}) \cdot \vb{y} \right) + \varphi \left(  \vb{x} \cdot \psi ( \vb{y})\right) - \psi \left( \varphi (\vb{x}) \cdot \vb{y} \right) - \psi\left( \vb{x} \cdot \varphi(\vb{y}) \right)\\\\
        &= \varphi \left( \psi (\vb{x}) \right) \cdot \vb{y} + \psi (\vb{x}) \cdot \varphi (\vb{y}) +\varphi (\vb{x}) \cdot \psi (\vb{y}) + \vb{x} \cdot \varphi \left( \psi (\vb{y}) \right)\\
        &-\psi \left( \varphi(\vb{x}) \right) \cdot \vb{y} - \varphi (\vb{x}) \cdot \psi (\vb{y}) - \psi\left( \vb{x} \right) \cdot \varphi \left( \vb{y} \right) - \vb{x} \cdot \psi \left( \varphi (\vb{y}) \right)\\\\
        &= \varphi \left( \psi (\vb{x}) \right) \cdot \vb{y} + \Ccancel[red]{\psi (\vb{x}) \cdot \varphi (\vb{y})} + \Ccancel[orange]{\varphi (\vb{x}) \cdot \psi (\vb{y})} + \vb{x} \cdot \varphi \left( \psi (\vb{y}) \right)\\
        &-\psi \left( \varphi(\vb{x}) \right) \cdot \vb{y} - \Ccancel[orange]{\varphi (\vb{x}) \cdot \psi (\vb{y})} -\Ccancel[red]{\psi (\vb{x}) \cdot \varphi (\vb{y})} - \vb{x} \cdot \psi \left( \varphi (\vb{y}) \right) \\\\
        &= \left(  \varphi \circ \psi \right) (\vb{x}) \cdot \vb{y} - \left( \psi \circ \varphi \right)(\vb{x}) \cdot \vb{y} + \vb{x} \cdot\left(  \varphi \circ \psi \right) (\vb{y})  -  \vb{x} \cdot \left( \psi \circ \varphi \right)(\vb{y}) \\\\
        &= \left[ \varphi, \psi \right] \left(  \vb{x} \right) \cdot \vb{y} + \vb{x} \cdot \left[ \varphi, \psi \right] \left( \vb{y} \right)
    \end{align*}
Hence $[\varphi,\psi] \in \mathrm{Der}(\mathbf{V})$.
\end{example}


\part{Gauge Theory}
\vspace*{2em} The primary resource for this part is \cite{ref:hamilton_gauge}.  The author's personal understanding has greatly benefited from use of \cite{ref:baez_knots} and \cite{ref:garrity_em}.\vspace*{-1em}
\label{part:Gauge-Theory}
\parttoc
\chapter{Fiber Bundles}

\section{General Fiber Bundles}
\begin{dfn}
    Let \( \rho: E \to M \) be surjective smooth map between smooth manifolds. We call \( M \) the \vocab{base space}, \( E \) the \vocab{total space}, and \( \rho \) the \vocab{projection}.\\

    We call the set 
    \[ \rho^{-1} \left( x \right) = \left\{ e \in E \  \middle| \ \rho(e)=x \right\} \]
    the \vocab{fiber of \( \rho \) over \( x \)}. \\
    Similarly if \( U \subset M \) we call the set 
    \[ \rho^{-1} \left( U \right) = \left\{ e \in E \  \middle| \ \rho(e) \in U \right\} \]
     the \vocab{fiber of \( \rho \) over \( U \)}. 

    If \( s: M \to E \) is a differentiable map such that 
    \[ \rho \circ s = \mathrm{Id}_{M} \]
    we call \( s \) a \vocab{global section} of \( \rho \).\\
    If \( U \subset M \) is open and \( s: U \to E \) is a differentiable map such that 
    \[ \rho \circ s = \mathrm{Id}_{U} \]
    we call \( s \) a \vocab{local section} of \( \rho \).\\
    
    The quartet \( \left( E, \rho, M;F \right) \) is called a \vocab{fiber bundle} if for every \( x \in M \), there is an open set \( U \subseteq M \) and diffeomorphism \[ \varphi_{U}: \rho^{-1} \left( U \right) \to U \times F \] such that \[ \mathrm{pr}_{1} \circ \varphi_{U} = \rho \] Where \( \mathrm{pr}_{1}: U \times F \to U \) is given by \( \mathrm{pr}_{1} \left( u,f \right) = u\) for all \( u \in U \) and \( f \in F \). In other words, the following diagram commutes
    \[\begin{tikzcd}
	{\rho^{-1}(U)} && {U \times F} \\
	\\
	&& U
	\arrow["{\varphi_U}", from=1-1, to=1-3]
	\arrow["\rho"', from=1-1, to=3-3]
	\arrow["{\mathrm{pr}_1}", from=1-3, to=3-3]
\end{tikzcd}\]
\end{dfn} 






\part{ A First Course in Probability and Statistics}
\label{part:First-Course-Probability}
\parttoc
\chapter{Probability}
\section{Definition and Properties}
\begin{dfn}
    The set \( \Omega \) of all possible outcomes of an experiment is called the \vocab{sample space}.  
    A subset \( A \subseteq \Omega \) is called an \vocab{event}.
\end{dfn}

\begin{example}
    Suppose that we conduct an experiment of rolling a six-sided die. Then 
    \[ \Omega = \{\text{roll a }1,\text{ roll a }2,\dots,\text{ roll a }6\},\]
    An example event is "roll a \( 1 \) or a \( 4 \).
\end{example}


\begin{dfn}
    A \vocab{probability space} is a triplet \( (\Omega, \mathscr{F}, \mathbb{P}) \) where
    \begin{enumerate}[label=\textbf{\roman*)}]
        \item \( \Omega \) is the sample space,
        \item \( \mathscr{F} \) is a \hyperref[def:sigma-algebra]{\( \sigma \)-algebra on \( \Omega \)} whose elements are events,
        \item \( \mathbb{P} : \mathscr{F} \to [0,1] \) is a \vocab{probability measure} or just \vocab{probability}. 
    \end{enumerate}
    The function \( \mathbb{P} \) satisfies:
    \begin{enumerate}[label=\textbf{\roman*)}]
        \setcounter{enumi}{3}
        \item \( \mathbb{P}(\Omega) = 1 \),
        \item If \( A_1, A_2, \dots \) are pairwise disjoint events, then
        \[
        \mathbb{P}\!\left( \bigcup_{k=1}^{\infty} A_k \right)
        = \sum_{k=1}^{\infty} \mathbb{P}(A_k).
        \]
    \end{enumerate}
    We read \( \mathbb{P}(A) \) as “the probability that \( A \) occurs.”
\end{dfn}



If you are not yet familiar with the formal definition of a \( \sigma \)-algebra, do not worry.  
For our purposes, it is enough to know that it is a collection of events with the following properties:

\begin{enumerate}[label=\textbf{\roman*)}]
    \item The impossible event \( \varnothing \) is included.
    \item The certain event \( \Omega \) is included.
    \item If \( A \) is an event, then its complement \( A^c \) (the event that \( A \) does not occur) is also an event.
    \item If \( A_1, A_2, \dots \) are events, then their union \( \bigcup_{k=1}^{\infty} A_k \) is also an event.
\end{enumerate}

These properties ensure that we can consistently assign probabilities to events and combinations of events.

\begin{lemma}
    \( \mathbb{P} \left( \varnothing \right)=0 \).
\end{lemma}
\begin{proof}
    Since \( \Omega \cap \varnothing = \varnothing \), they are pairwise disjoint. Then
    \begin{align*}
        1 &= \mathbb{P} \left( \Omega \right)\\ 
        &= \mathbb{P} \left( \Omega \cup \varnothing \right) \\
        &= \mathbb{P} \left( \Omega \right) + \mathbb{P} \left( \varnothing \right) \\
       1 &= 1 + \mathbb{P} \left( \varnothing \right)
    \end{align*}
    So  \( \mathbb{P} \left( \varnothing \right)=0 \).
\end{proof}

\begin{lemma}
    If \( A^{c} \) denotes \( \Omega - A \) where \( A \in \mathscr{F} \). Then 
    \[ \mathbb{P} \left( A^{c} \right) = 1- \mathbb{P} \left( A \right) .\]
\end{lemma}
\begin{proof}
    Very similar to above (In fact, the previous result is just a specific case of this result.) \( A^{c} \cap A = \varnothing \) and \( A^{c} \cup A = \Omega \) 
    \begin{align*}
        1 &= \mathbb{P} \left( \Omega \right)\\ 
        &= \mathbb{P} \left( A \cup A^{c} \right) \\
        &= \mathbb{P} \left( A \right) + \mathbb{P} \left( A^{c} \right)
    \end{align*}
    So \( \mathbb{P} \left( A^{c} \right) = 1- \mathbb{P} \left( A \right) \).
\end{proof}



\part{A First Course in Number Theory}
\label{part:Basic-Number-Theory}
\parttoc
\chapter{Introduction}
\section{Well-Ordered Sets and Induction}

\begin{dfn}
     An \hyperref[def:order-on-a-set]{ordered}  set \( X \) is said to be \vocab{well-ordered} if every non-empty subset of \( X \) contains a minimal element.
\end{dfn}

\begin{dfn}
    An ordered set \( X \) is said to satisfy the \vocab{principle of (strong) induction} if for every statement \( P(x) \) (that is, a map \( P: X \to \{\text{True}, \text{False}\} \)) and every element \( x_0 \in X \) such that: for all \( x \geq x_{0} \), if \( P(y) \) is true for all \( y \) with \( x_{0} \leq y < x \), then \( P(x) \) is true, it follows that \( P(x) \) is true for all \( x \geq x_{0} \).
\end{dfn}

\textbf{Note:} In dealing whether to include \( 0 \) as a natural number, we adopt \vocab{the axiom of convinience}. For any statement involving \( \mathbb{N} \) as the universal set, we leave it to the reader to deduce from context if \( 0 \in \mathbb{N} \) or \( 0 \not \in \mathbb{N} \).

\section{The Division Algorithm}

\begin{theorem}[The Division Algorithm]
    For any \( a,b \in \mathbb{N} \), with \( b >0 \), there exists unique \( q \in \mathbb{N} \) called the \vocab{quotient} and unique \( r \in \mathbb{N} \), with \( 0 \le r < b \) called the \vocab{remainder} for which 
    \[ a = bq +r. \]
\end{theorem}
\begin{proof}
Let
\[
S=\{\,n\in\mathbb{N}\mid n=a-bq\text{ for some }q\in\mathbb{N}\,\}.
\]
Since \(q=0\) gives \(a-b\cdot 0=a\), the set \(S\) is nonempty, so it has a least element; call that element \(r\).

We claim \(0\le r<b\). Certainly \(r\ge0\) because \(r\in\mathbb{N}\). If \(r\ge b\), write \(r=a-bq_1\) for some \(q_1\in\mathbb{N}\). Then
\[
r-b=a-b(q_1+1).
\]
Since \(r-b\ge0\), the element \(r-b\) lies in \(S\) and satisfies \(r-b<r\), contradicting the minimality of \(r\). Hence \(r<b\).

Now suppose there are two such remainders \(r_1\) and \(r_2\) with \(0\le r_1<r_2<b\). Write
\[
r_1=a-bq_1,\qquad r_2=a-bq_2.
\]
Subtracting gives
\[
r_2-r_1=b(q_1-q_2).
\]
Because \(r_2-r_1>0\), we have \(q_1-q_2>0\), so \(q_1-q_2\in\mathbb{N}\). But then \(b(q_1-q_2)\ge b\), contradicting \(r_2-r_1<b\). Thus the remainder \(r\) is unique.\\ 
Finally, once \(r\) is fixed, the quotient \(q\) is determined by \(q=\frac{a-r}{b}\), so \(q\) is also unique.
\end{proof}

\begin{dfn}
    Suppose that \( a, b \in \mathbb{N} \) with \( b>0 \). We say that \( b \) \vocab{divides} \( a \), denoted as \( b  \mid   a \) if the remainder given by the division algorithm of \( a \) by \( b \) is \( 0 \). If \( b \) does not divide \( a \), we write \( b   \nmid  a \).
\end{dfn}

\begin{example}
    For any natural number \( n \), \( 1  \mid  n \) and \( n   \mid  n \).
\end{example}

\begin{exercise}
    Suppose that \( n >2 \). Show that \( n-1 \nmid n \).
\end{exercise}
\begin{solution}
    Since \(n>2\), we have \(n-1\ge 2\). By the division algorithm,
    \[
        n = 1\cdot(n-1) + 1.
    \]
    The remainder is \(1\), which satisfies \(0<1<n-1\). Because the remainder is not \(0\), we conclude that \(n-1\nmid n\).
\end{solution}



\begin{dfn}
    Suppose \( n \in \mathbb{N} \) with \( n>1 \). \( n \) is \vocab{prime} if the only natural numbers that divide \( a \) are \( 1 \) and \( n\). \( n \) is \vocab{composite} if it is not prime. \\ 
    \textbf{Note:} \( 1 \) is neither composite nor prime. It is \vocab{unit}. 
\end{dfn}





\appendix
\part{ Appendix} 
\label{part: Appendix}

\parttoc
\chapter{Basic Logic}
\label{ch:Logic}
\input{content/appendix/logic/logic-master}
\chapter{Naive Approach to Set Theory}
\label{ch:Naive Sets}

\section{Sets and Their Operations}
\subsection{Sets}
For much of modern math, the "bottom turtle" for the objects we study are sets.
\begin{dfn}
    A \vocab{set}, roughly speaking, is a collection of elements. We typically denote a set with a capital letter \( X \) and their \vocab{elements} are often denoted with a lowercase letter. \( x \). \\
    If some element \( x  \) belongs to \( X  \), we will write this as 
    \[ x \in X. \] 
Otherwise, if the element \( x  \) is \textbf{not} a member of the set \( X  \), we will denote this by 
\[ x \not \in X. \] 
\end{dfn}
There are multiple ways to describe a set. 
\\For example, we may explicitly list out the elements. 
\begin{example}
    Let 
    \[ X = \{a,b,c \} .\] 
This is precisely the set \( X \) which contains only the elements \( a,b,c \).
\end{example}
We might begin with an existing set and then apply a filter to generate a new set. In other words, if we are given a set \( Y \), we may construct a new set \( X \) of the form 
\[ X = \{x \in Y \mid P(x)\} \] 
where \( P(x) \) is some statement involving the element \( x \) and whenever \( P(x) \) is true, we include the element \( x \) in the set \( X  \).
\begin{example}
Let \( \mathbb{R} \) denote the collection of real numbers. Then we may construct \( \mathbb{R}^{+} \) to be 
\[ \mathbb{R}^{+} = \{x \in \mathbb{R} \mid x > 0\} \] 
which is the set of all positive real numbers.
\end{example}
\begin{dfn}
Let \( X  \) and \( Y \) each be sets. If \( X  \) and \( Y \) each contain the exact same elements, we say that they are the same set and write \( X =Y \).
\end{dfn}
\begin{example}
    If \( X = \left\{ 1,2,3 \right\} \) and \( Y = \{1,2,3\} \), then \( X  \) and \( Y  \) are the same set and we write \( X=Y. \) \\
    If \( Z = \{1,2\} \), it is clear that \( X \neq Z \) and \( Y \neq Z \) since \( 3 \in X \) and \( 3 \in Y \) but \( 3 \not \in Z. \) 
\end{example}

\begin{dfn}
Let \( X  \) and \( Y \) each be sets. We say that \( X \)  is a \vocab{subset} of \( Y \) if \( x \in Y \) whenever \( x \in X \). We write this as 
\[ X \subseteq Y \]. If \( Y \) contains at least one element not contained in \( X \), we say that \( X \) is a \vocab{subset}  subset of \( Y \) and denote it by \( X \subset Y \).
\end{dfn}
\begin{example}
    Using our previous example of \( X,Y \) and \( Z \), it is clear that 
    \[ Z \subseteq X \quad \text{and} \quad  Z \subseteq Y \] 
\end{example}

 \begin{theorem}
For any two sets \( X  \) and \( Y \),
\[ X=Y \iff X \subseteq Y \text{ and } Y \subseteq X. \] 
 \end{theorem}
 \begin{proof}
     \( \Rightarrow \) Suppose that \( X =Y \). Then, by definition, \( X  \) and \( Y \) contain exactly the same elements. Therefore, each element of \( X \) must also be in \(  Y  \) and vice-versa. So it must be the case that \( X \subseteq Y \) and \( Y \subseteq X \). \\
     \( \Leftarrow \) Now suppose that \( X \subseteq Y \) and \( Y \subseteq X \). If \( x \in X \), then, by the fact that \( X \subseteq Y \), we have that \( x \in Y \). In other words, \( X \) contains nno elements that are not also in \( Y \). We can just as easily conclude that \( Y \) contains no elements that cannot be found in \( X \). Since \( X \) and \( Y \) contain exactly the same elements, it follows that \( X=Y \).
 \end{proof}
\begin{dfn}
    The \vocab{empty set}, denoted as \( \varnothing \), is the set with no elements. 
\end{dfn}
\begin{lemma}
    The empty set is unique and it is a subset of every set.
\end{lemma}
\begin{proof}
    It is not hard to show that any other set that contains no elements must be identical to the empty set. To show for any set \( X \), that \( \varnothing \subseteq X \), we will write out explicitly what it means for \( \varnothing \subseteq X \). We have 
    \[ x \in \varnothing \Rightarrow x \in X. \] 
However, the statement \( x \in \varnothing \) is false ,by definition. So the entire implication becomes true. 
\end{proof}
\subsection{Unions, Intersections, and Compliments}
\begin{dfn}
    Let \( X  \) and \( Y  \) be any sets. Then we define the \vocab{union} of the sets \( X \) and \( Y \), denoted by \( X \cup Y \) is the new set defined by 
    \[ X \cup Y = \{z \mid z \in X \text{ or } z \in Y\} \] 
\end{dfn}

\begin{example}
    Let \( X = \left\{ 1,2,3 \right\} \) and \( Y = \left\{ 3,4,5 \right\} \). Then,
    \[ X \cup Y = \left\{ 1,2,3,4,5 \right\} \] 
\end{example}

\begin{dfn}
    Let \( X \) and \( Y \) be sets. Then the \vocab{intersection} of \( X \) and \( Y \), denoted by \( X \cap Y \) is the set 
    \[ X \cap Y = \{z \mid z \in X \text{ and } z \in Y\} \] 
\end{dfn}

\begin{example}
    Again letting \( X = \left\{ 1,2,3 \right\} \) and \( Y = \left\{ 3,4,5 \right\} \). Then,
    \[ X \cap Y = \left\{ 3 \right\} \] 
\end{example}

\begin{theorem}
    For any sets \( X \), \( Y \), and \( Z \), we have 
    \[ X \cap \left( Y \cup Z \right) = \left( X \cap Y \right) \cup \left( X \cap Z \right).\]
\end{theorem}
\begin{proof}
Suppose \( x \in X \cap \left( Y \cup Z \right)\) . Then \(  x \in X \) and \( x \in Y \cup Z \). Since \(  x \in Y \cup Z \), \( x \in Y \) or \( x \in Z \). Without loss of generality, we can assume \( x \in Y \). Since \( x \in X \) and \( x \in Y \), we have \( x \in X \cap Y \). So \( x \in  \left( X \cap Y \right) \cup \left( X \cap Z \right) \). This demonstrates that 
\[ X \cap \left( Y \cup Z \right) \subseteq  \left( X \cap Y \right) \cup \left( X \cap Z \right)\]
For the reverse direction, assume that \( x \in  \left( X \cap Y \right) \cup \left( X \cap Z \right) \). The \( x \in \left( X \cap Y \right) \) or \( x \in \left( X \cap Z \right) \). Again, without loss of generality, assume that \( x \in X \cap Z \). Then \( x \in X \) and \( x \in Z \). Therefore, \( x \in Y \cup Z \). So we can conclude that \(  x \in X \cap \left( Y \cup Z \right) \). Together with the previous conclusion, we have 
\[ X \cap \left( Y \cup Z \right) = \left( X \cap Y \right) \cup \left( X \cap Z \right),\]
as desired.
\end{proof}

\begin{theorem}
    For any sets \( X \), \( Y \), \(  Z \), we have 
    \[ X \cup \left( Y \cap Z \right) = \left( X \cup Y \right) \cap \left( X \cup Z \right)\]
\end{theorem}
\begin{proof}
    Suppose \( x \in  X \cup \left( Y \cap Z \right).\) Then \( x \in X \) or \( x \in Y \cap Z \). Take the case that \( x \in X \). Then clearly \( x \in X \cup Y \) and \( x \in X \cup Z \). So \( x \in \left(  X \cup Y \right) \cap \left( X \cup Z \right) \). If we take the case that \( x \in Y \cap Z \). Then \( x \in Y \) and \( x \in Z \). So clearly \( x \in X \cup Y \) and \( x \in X \cup Z \). Therefore, \( x \in \left(  X \cup Y \right) \cap \left( X \cup Z \right) \). We have shown that 
\[ X \cup \left( Y \cap Z \right) \subseteq \left( X \cup Y \right) \cap \left( X \cup Z \right)\]
Now suppose that \( x \in \left( X \cup Y \right) \cap \left( X \cup Z \right).\) Then \( x \in \left( X \cup Y \right) \) and \( x \in \left( X \cup Z \right) \). Now if \( x \in X \), we trivially have that \(  x \in X \in X \cup \left( Y \cap Z \right) \). So we take the case that \( x \not \in X \). So \( x \) must be contained in \( Y \) and in \( Z \). So \(  x \in Y \cap Z \). So \( x \in X \cup \left(  Y \cap Z \right) \). With the previous conclusion, we have 
 \[ X \cup \left( Y \cap Z \right) = \left( X \cup Y \right) \cap \left( X \cup Z \right),\]
 as desired.
\end{proof}

\begin{dfn}
    Suppose that \( X \) and \( Y \) are sets such that \( Y \subseteq X \). Then the \vocab{set compliment} of \( Y \) relative to \( X \) is the set \( X \minus Y \) (or \( Y^{C} \) if the set \( X \) is understood in context), defined by 
    \[ X-Y = Y^{C }= \{x \in X \mid x \not \in Y\} \] 
\end{dfn}

\begin{example}
    Let \( X = \left\{ 1,2,3,4 \right\} \) and \( Y = \{1.3\} \). Then 
    \[ X \minus Y = \{2,4\} \] 
\end{example}

\begin{exercise}
    For any set \( X \), we have
    \[ X \cap \varnothing = \varnothing \text{ and } X \cup \varnothing = X\] 
\end{exercise}
\begin{solution}
Trivial.
\end{solution}

\begin{theorem}[DeMorgan's Laws]
    Suppose \( X,Y \), and \( Z \) are sets with \( X \subseteq Z \) and \( Y \subseteq Z \). Then 
    \[ \left( X \cup Y \right)^{C}= X^{C} \cap Y^{C} \] and 
    \[ \left( X \cap Y \right)^{C}= X^{C} \cup Y^{C} \] 
\end{theorem}
\begin{proof}
If either \( X \) or \( Y \) are the empty set or the universal set \( Z \), then the conclusion is trivial so we will assume otherwise.\\
We will first show 
\[ \left( X \cup Y \right)^{C}= X^{C} \cap Y^{C} \]
Suppose \( x \in \left( X \cup Y \right)^{C} \). Then, by definition, \( x \in Z \) but \( x \not \in X \cup Y \). Therefore, \( x \not \in X \) and \( x \not \in Y \). This gives us \( x \in X^{C } \) and \( x \in Y^{C} \). So \( x \in X^{C } \cap Y^{C}. \) So 
 \[ \left( X \cup Y \right)^{C}  \subseteq X^{C} \cap Y^{C} \] 
Now assume \( x \in X^{C} \cap Y^{C} \). By definition, \( x \in X^{C} \) and \( x \in Y^{C} \). In other words, \( x \in Z \) but \( x \not \in X \) and \( x \not \in Y \). So \( x \not \in X \cup Y \) or \( x \in \left( X \cup Y \right)^{C} \). Taken with the earlier conclusion, we have 
\[ \left( X \cup Y \right)^{C}= X^{C} \cap Y^{C}. \]
Now we will show 
\[ \left( X \cap Y \right)^{C}= X^{C} \cup Y^{C} \]
Assume that \( x \in \left( X \cap Y \right)^{C} \). Then \( x \not \in X \cap Y\). So it must be the case that \( x \not \in X \) or \( x \not \in Y \) or both. Without loss of generality, we will assume that \( x \not  \in X  \).  Then \( x \in X^{C} \). So then, \( x \in X^{C} \cup Y^{C} \). This gives us 
\[ \left( X \cap Y \right)^{C} \subseteq  X^{C} \cup Y^{C}. \] 
Now assume \( x \in X^{C} \cup Y^{C} \). Without loss of generality, we may assume that \( x \in Y^{C} \). This gives us that \( x \not \in Y \) so \( x \not \in X \cap Y \). So then \( x \in \left( X \cap Y \right)^{C} \). Together with the previous conclusion, we have
\[ \left( X \cap Y \right)^{C}= X^{C} \cup Y^{C} \] 
which concludes this proof.
\end{proof}


\section{Cartesian Products, Relations, and Functions}
\subsection{Relations}

\begin{dfn}
    Let \( X \) and \( Y \) be non-empty sets. Then the \vocab{Cartesian product} of \( X  \) and \( Y \) is the following set: 
    \[ X \times Y = \{(x,y)  \mid x \in X, \ y \in Y\}. \] 
\end{dfn}

\begin{example}
    Suppose \( X = \{1,2,3\} \) and \( Y = \{a,b\} \). Then 
    \[ X \times  Y = \{(1,a), (1,b), (2,a), (2,b), (3,a), (3,b)\} \] 
\end{example}

\begin{dfn}
    A \vocab{relation} on a non-empty set \( X \) is any subset \( R \) of the Cartesian product \( X \times X \). If the ordered pair \( (x,y) \in R \), we will often express this as \( xRy \).
\end{dfn}

\begin{dfn}
    Let \( S \) be a set. An \vocab{order} on \( S \) is a relation \( < \) with following two properties.
    \begin{enumerate}[label=\textbf{\roman*)}]
        \item For every \( x,y \in S \) exactly one of the following hold:
        \[ x <y \quad y <x \quad x=y \]
        \item If \( x <y \) and \( y <z \), then \( x <z \).
    \end{enumerate}
\end{dfn}

\begin{dfn}
    An \vocab{equivalence relation} is a relation \( \sim \) on a non-empty set \( X \) such that the following criteria hold:
    \begin{enumerate}[label=\textbf{\roman*)}]
        \item For every \( x \in X \), \( x \sim x \). This is called the \vocab{reflexive} property.
        \item For every \( x,y \in X \), \( x \sim y \) implies \( y \sim x \). This is called the \vocab{symmetric} property.
        \item For every \( x,y,z \in X \), \( x \sim y \) and \( y \sim z \) implies \( x \sim z \). This is called the \vocab{transitive} property.
    \end{enumerate}
The set 
\[ A_{x} = \{ y \in X \mid x \sim y\} \] 
is called the \vocab{equivalence class} of \( x \).
\end{dfn}

\begin{example}
Let \( X = \{a,b,c\} \). The reader can verify that the set 
\[ \sim = \left\{ (a,a), (a,b), (b,a), (b,b), (c,c) \right\} \] is an equivalence relation.
\end{example}

\begin{example}[Construction of the rational numbers]
For a non-trivial example, suppose \( \mathbb{Z} \) is the set of integers. We define an equivalence relation on \( \mathbb{Z} \times  \mathbb{Z} - \{ (x,0) \mid x \in \mathbb{Z} \} \) by 
\[ \left( x_{1},y_{1} \right) \sim \left( x_{2}, y_{2} \right) \iff x_{1} \cdot y_{2} = x_{2} \cdot y_{1 }. \] 
The reflexive and symmetric properties are trivial. So we will just demonstrate transitivity.\\
Suppose that 
\[ \left( x_{1},y_{1} \right) \sim \left( x_{2}, y_{2} \right) \text{ and } \left( x_{2}, y_{2} \right) \sim \left( x_{3}, y_{3} \right)\] 
Then, 
\begin{align*}
    x_{1} \cdot y_{2} &= x_{2} \cdot y_{1} \comment{Since $\left( x_{1},y_{1} \right) \sim \left( x_{2}, y_{2} \right)$}\\
    y_{3} \cdot\left( x_{1} \cdot y_{2} \right) &= y_{3} \cdot \left( x_{2} \cdot y_{1} \right) \comment{Since $y_{3} \neq 0$}\\
    \left( x_{1} \cdot y_{3} \right) \cdot y_{2} &= \left( x_{2} \cdot y_{3} \right) \cdot y_{1} \comment{Rearranging}\\
    \left( x_{1} \cdot y_{3} \right) \cdot y_{2} &= \left( x_{3} \cdot y_{2} \right) \cdot y_{1} \comment{Since $\left( x_{2},y_{2} \right) \sim \left( x_{3}, y_{3} \right)$}\\
    \left( x_{1} \cdot y_{3} \right) \cdot y_{2} &= \left( x_{3} \cdot y_{1} \right) \cdot y_{2} \comment{Rearranging}\\
    \left( x_{1} \cdot y_{3} \right) \cdot y_{2} &- \left( x_{3} \cdot y_{1} \right) \cdot y_{2}=0 \\
    \left( x_{1} \cdot y_{3} \right . &- \left . x_{3} \cdot y_{1} \right) \cdot y_{2} =0 \comment{By the disributive law of the integers}\\
    x_{1} \cdot y_{3}  &-  x_{3} \cdot y_{1} =0 \comment{Since the $\mathbb{Z}$ is an integral domain and $y_{2} \neq 0$}\\
    x_{1} \cdot y_{3} &= x_{3} \cdot y_{1}\\
    \left( x_{1},y_{1} \right) & \sim \left( x_{3}, y_{3} \right) 
\end{align*}
We will finish this demonstration by writing \( \frac{x_{1}}{y_{1}} \) instead of \( \left( x_{1}, y_{1} \right) \).\\
We call this set \(  \mathbb{Z} \times  \mathbb{Z} - \{ (x,0)  \} \) together with this identification, the \vocab{rational numbers} and denote it by \( \mathbb{Q} \) .
\end{example}

Building on the previous example, we notice that even if \( \left( 1,2 \right)  \) and \( \left( 3,6 \right) \) are distinct elements of \( \mathbb{Z} \times \mathbb{Z} - \{(x,0)\} \), they are the same as elements of \( \mathbb{Q} \). This motivates the following definition and theorem.

\begin{dfn}
    A \vocab{partition} \( P \)  on a set \( X \) is a collection of subsets \( X_{\alpha} \) of \( X \) such that the following criteria hold:
    \begin{enumerate}[label=\textbf{\roman*)}]
        \item \( \bigcup_{\alpha}^{} X_{\alpha} = X \)
        \item For any distinct sets \( X_{\alpha}, X_{\beta} \) in the partition \( P \), we have \( X_{\alpha} \cap X_{\beta} = \varnothing \) 
    \end{enumerate}
\end{dfn}

\begin{theorem}
    Any equivalence class \( \sim \) on a non-empty set \( X \) forms a partition. And similarly, for every partition \( P \)  on a non-empty set \( X \), there exists an equivalence relation \( \sim \) whose equivalence classes are precisely the subsets described by the partition.
\end{theorem}
\begin{proof}
    Let \( X_{\alpha}, \alpha \in A \) denote the collection of equivalence classes as defined by the equivalence relation \( \sim \) on \( X \). The reflexive condition guarantees that every element of \( X \) is in some equivalence class so 
    \[ \bigcup_{\alpha \in A}^{} X_{\alpha} =X \] 
    is automatically satisfied.
Now suppose
\[ X_{a} = \{x \in X \mid a \sim x\} \] 
\[ X_{b} = \{x \in X \mid b \sim x\} \] 
are the equivalence classes of \( a \) and \( b \), each members of \( X \). We want to show that they are either disjoint or identical.\\
So suppose that \( X_{a} \cap X_{b} \) contained at least one element call it \( c \). Now pick any element \( x_{a} \in X_{a} \). Then,
\begin{align*}
     a & \sim c \quad a \sim x_{a} \comment{Since $c,x_{a} \in X_{a}$}\\
    c & \sim a \quad a \sim x_{a} \comment{Applying symmetry}\\
    c & \sim x_{a} \comment{Applying transitivity}\\
    b & \sim c \quad c  \sim x_{a} \comment{Since $c \in X_{b}$}\\
    b & \sim x_{a} \comment{Applying transitivity}
\end{align*}
This shows that \( x_{a} \in X_{b} \) so \( X_{a} \subseteq X_{b} \). It is similar to show that \( X_{b} \subseteq  X_{a} \). In other words, we have shown that \( X_{a} \) and \( X_{b} \) contain at least one element in common, then \( X_{a}= X_{b} \). \\
The other part of this theorem is trivial. If \( P = \{X_{\alpha}\}\) is a partition, then simply say \( x \sim y \) if they belong to same subset defined by the partition. Verifying that this defines an equivalence relation is an exercise.
\end{proof}

\subsection{Functions}

We arrive at one of the most crucial objects of study in mathematics. 
\begin{dfn}
    A \vocab{function} between two non-empty sets \( X \) and \( Y \) is a relation \( f \) that satisfies the following conditions:
    \begin{enumerate}[label=\textbf{\roman*)}]
        \item For every \( x \in X \), there exists some \( y \in Y \) such that \( \left( x,y \right) \in f\). 
        \item \( \left( x,y \right) \) and \( \left( x,y' \right) \) are members of \( f \) if and only if \( y = y' \).
    \end{enumerate}
We will write \( f: X \to Y \) if \( f \) is function from \( X \) to \( Y \). We will also write \( f: x \mapsto y \) (read: \( f  \) maps \( x \) to \( y \)) or \( y= f \left( x \right) \)(read: \( y = \) \( f \) of \( x \)) if \( \left( x,y \right) \in f\).\\
Moreover, we call \( X \) the \vocab{domain} of \( f \) and \( Y \) the \vocab{codomain}.
\end{dfn}

\begin{dfn}
    Suppose \( X \), \( Y \), and \( Z \) are sets and \( f: X \to Y \), \( g: Y \to Z \). We define the \vocab{composite} function \( g \circ f: X \to Z \) as follows: 
    \[ \left( x,z \right) \in g \circ f \text{ if } \left( x,y \right) \in f \text{ and } \left( y,z \right) \in g .\]
In other words, if \( y =f(x) \) and \( z= g(y) \), we define \( z= \left( g \circ f \right) \left( x \right) \).
\end{dfn}

\begin{dfn}
    Suppose \( X \) and \( Y \) are sets, \( f: X \to Y \) and \( A \subset X \). We define the \vocab{restriction} of \( f \) to \( A \), denoted by \( f \vert_{A}: A \to Y \), as a new function
    \[ f \vert_{A}(a) := f(a) \quad \text{for every } a \in A.\]
In other words, 
\[ f \vert_{A} = f \cap \left( A \times Y \right) .\]
\end{dfn}

\begin{example}
    Let \( X \) be any set. We have a natural function called the \vocab{identity} function on \( X \), \( \mathrm{Id}_{X}:X \to X \), where 
    \[ \mathrm{Id}_{X} \left( x \right):=x \quad  \text{   for every } x \in X.\]
\end{example}

\begin{example}
    Let \( X \) be any non-empty set and \( A \subseteq X \). We have the \vocab{characteristic function} of \( A \), \( \chi_{A}: X \to \{0,1\} \), where 
    \[ \chi_{A} \left( x  \right): = 
    \begin{cases}
        1 & x \in A\\
        0 & x \not \in A.
    \end{cases}
    \]
\end{example}

\begin{dfn}
    Let \( X \) and \( Y \) be sets and \( f: X \to Y \) be a function. We say that \( f \) is \vocab{injective}, is \vocab{one-to-one}, or is an \vocab{injection} if for every \( x_{1}, x_{2} \in X \), we have 
    \[ x_{1} \neq x_{2} \Rightarrow f \left( x_{1} \right) \neq f \left( x_{2} \right). \]
\end{dfn}

\begin{dfn}
    Let \( X \) and \( Y \) be sets and \( f: X \to Y \) be a function. We say that \( f \) is \vocab{surjective}, is \vocab{onto}, or is a \vocab{surjection} if for every \( y \in Y \), there exists an \( x \in X \) such that \( y = f(x) \).
\end{dfn}

\begin{dfn}
    Let \( X \) and \( Y \) be sets and \( f: X \to Y \) be a function. We say that \( f \) is \vocab{bijective} or is a \vocab{bijection} if it is both injective and surjective.
\end{dfn}

\begin{dfn}
Suppose \( X \) and \( Y \) are sets and \( f: X \to Y \), we say:
\begin{enumerate}[label=\textbf{\roman*)}]
    \item \( f \) has a \vocab{left-inverse} or is \vocab{left-invertable}  if there exists a function \( g: Y \to X \) such that \( g \circ f = \mathrm{Id}_{X} \). 
    \item \( f \) has a \vocab{right-inverse} or is \vocab{right-invertable} if there exists a function \( g: Y \to X \) such that \( f \circ g = \mathrm{Id}_{Y} \).
    \item \( f \) has a \vocab{two-sided-inverse} or is \vocab{invertable} if there exists a function \( g: Y \to X \) such that the above two criteria hold.
\end{enumerate}
\end{dfn}

\begin{theorem}
    Suppose \( X \) and \( Y \) are sets and \( f:X \to Y \) is a function. \( f \) has a left-inverse if and only \( f \) is an injection.
\end{theorem}
\begin{proof} 
    \( \Rightarrow \) For the forward implication, suppose that \( f \) is \textbf{not} injective. Then there are \( x_{1}, x_{2} \in X \) such that \( f \left( x_{1} \right) = f \left(  x_{2} \right)\) but \( x_{1} \neq x_{2} \). But then no possible function \( g: Y \to X \) could be a left inverse since \( \left( g \circ f \right)(x_{1})= x_{1} \) and \( \left( g \circ f \right)(x_{2})= x_{2}  \). cannot be simultaneously true. \\
    \( \Leftarrow \) Now suppose that \( f \) is injective. Then for any \( y \in \mathrm{Img}\left( f \right) \), we define \( g(y)=x \) whenever \( y= f(x) \). If \( y \not \in \mathrm{Img}(f) \), then we fix an element \( a \in X \) simply define \( g(y)=a \). It is clear from construction that \( g \circ f = \mathrm{Id}_{X} \) and the proof is complete.
\end{proof}

\begin{theorem}
    Suppose \( X \) and \( Y \) are sets and \( f: X \to Y \) is a function. \( f \) has a right-inverse if and only if \( f \) is a surjection.
\end{theorem}
\begin{proof}
    \( \Rightarrow \) Suppose that \( f \) is \textbf{not} surjective. Then, by definition, there exists at least one \( y \in Y \) such that \( y \not \in \mathrm{Img} \left( f \right) \). Then for no function \(  g : Y \to X \) can we have \( \left( f \circ g \right) \left(  y \right)= y\). So \( f \) cannot have a right inverse.\\
    \( \Leftarrow \) Now suppose that \( f \) is surjective. Then for every \( y \in Y \), there is some \( x \in X \) such that \( f(x)=y \). We then define \( g(y)=x \) and \( f \) is not injective, we simply choose a particular \( x \). By construction, it is clear that \( \left( f \circ g \right) = \mathrm{Id}_{Y} \) and the proof is complete.
\end{proof}

\begin{theorem}
    Suppose \( X \) and \( Y \) are finite sets with the same number of elements. Then the following are equivalent for \( f: X \to Y \):
    \begin{enumerate}[label=\textbf{\roman*)}]
        \item \( f \) is bijective. 
        \item \( f \) is injective. 
        \item \( f \) is surjective.
    \end{enumerate}
\end{theorem}

\begin{proof}
From the statement of the theorem, it suffices to show \( \textbf{ii} \iff \textbf{iii}\). We will proceed by induction on the cardinality of both sets.

The base case \( |X| = |Y| = 1 \) is trivial.

Suppose that the result holds up to \( n-1 \).

\( \Rightarrow \) Now suppose \( |X| = |Y| = n \) and \( f:X \to Y \) is injective. We fix some \( a \in X \) and hence, some \( f(a) \in Y \) and consider the restriction map \( f \vert_{X \setminus \{a\}}: X \setminus \{a\} \to Y \setminus \{f(a)\} \). Since \( f \) is injective, the restriction is also injective. By the inductive hypothesis, \( f \vert_{X \setminus \{a\}} \) is surjective onto \( Y \setminus \{f(a)\} \). Therefore, \( f \) is surjective onto \( Y \).

\( \Leftarrow \) Now suppose \( |X| = |Y| = n \) and \( f:X \to Y \) is surjective. We fix some \( a \in X \) and consider \( f(a) \in Y \). Let's define the restriction \( f \vert_{X \setminus \{a\}}: X \setminus \{a\} \to Y \setminus \{f(a)\} \). To ensure this restriction is well-defined, we need to verify that for all \( b \in X \setminus \{a\} \), we have \( f(b) \neq f(a) \).

Suppose for contradiction that there exists some \( b \in X \setminus \{a\} \) such that \( f(b) = f(a) \). Then \( f \) maps at least two elements \( a \) and \( b \) to the same value, meaning \( f \) maps at most \( n-1 \) elements of \( X \) to distinct elements of \( Y \). Since \( |Y| = n \), there must be at least one element of \( Y \) that is not in the image of \( f \), contradicting the assumption that \( f \) is surjective.

Therefore, the restriction \( f \vert_{X \setminus \{a\}} \) is well-defined. Since \( |X \setminus \{a\}| = |Y \setminus \{f(a)\}| = n-1 \), and the restriction is surjective by construction, we can apply the inductive hypothesis to conclude that \( f \vert_{X \setminus \{a\}} \) is also injective. This implies that \( f \) is injective on all of \( X \).
\end{proof}


\chapter{Inequalities}
\label{ch:Inequalities}

\section{Some Basic Algebraic Inequalities}
\begin{exercise}\label{ex:ez-first-ineq}
    Show that \( a^{2} +b ^{2} \geq 2ab \) for all \( a,b \in \mathbb{R} \).
\end{exercise}
\begin{solution}
    Since \( \left( a-b \right)^{2} \geq 0 \), we have 
    \[ a^{2}-2ab+b^{2} \geq 0 \] or 
    \[  a^{2} +b ^{2} \geq 2ab .\]
\end{solution}

\begin{exercise}
    Show that 
    \[ \frac{a+b}{2} \ge \sqrt{ab} \] for \( a,b \ge 0 \). 
\end{exercise}
\begin{solution}
    Apply \cref{ex:ez-first-ineq} to \( \sqrt{a} \) and \( \sqrt{b} \). \\ 
    \begin{align*}
        \left( \sqrt{a} \right)^{2} + \left( \sqrt{b} \right)^{2} &\ge 2 \sqrt{a} \sqrt{b} \\
        a +b &\ge 2 \sqrt{a} \sqrt{b} \tag*{Since $ a,b \ge 0$.}\\
        \frac{a+b}{2} &\ge \sqrt{ab}
    \end{align*}
    
\end{solution}



\section{Introduction to the Cauchy-Schwarz Inequality}
\begin{theorem}[Cauchy--Schwarz Inequality]\label{thrm: The Cauchy-Schwarz Inequality}
  Suppose \( \mathbf{V} \) is a real inner product space. For all \( \vb{v}, \vb{w} \in \mathbf{V} \), we have:
  \[
    \abs{\left< \vb{v}, \vb{w} \right>} \le \left< \vb{v}, \vb{v} \right>^{1/2} \left< \vb{w}, \vb{w} \right>^{1/2}.
  \]
\end{theorem}
\begin{proof}
  Consider the vector
  \[
    \vb{u} = \frac{\vb{v}}{\left< \vb{v}, \vb{v} \right>^{1/2}} - \frac{\vb{w}}{\left< \vb{w}, \vb{w} \right>^{1/2}}.
  \]
  Since inner products are nonnegative on real inner product spaces,
  \[
    \left< \vb{u}, \vb{u} \right> \ge 0.
  \]
  Expanding \( \left< \vb{u}, \vb{u} \right> \) gives:
  \begin{align*}
    \left< \vb{u}, \vb{u} \right>
    &= \left< \frac{\vb{v}}{\left< \vb{v}, \vb{v} \right>^{1/2}}, \frac{\vb{v}}{\left< \vb{v}, \vb{v} \right>^{1/2}} \right>
     - 2 \left< \frac{\vb{v}}{\left< \vb{v}, \vb{v} \right>^{1/2}}, \frac{\vb{w}}{\left< \vb{w}, \vb{w} \right>^{1/2}} \right>
     + \left< \frac{\vb{w}}{\left< \vb{w}, \vb{w} \right>^{1/2}}, \frac{\vb{w}}{\left< \vb{w}, \vb{w} \right>^{1/2}} \right> \\
    &= \frac{\left< \vb{v}, \vb{v} \right>}{\left< \vb{v}, \vb{v} \right>} 
     - 2 \cdot \frac{\left< \vb{v}, \vb{w} \right>}{\left< \vb{v}, \vb{v} \right>^{1/2} \left< \vb{w}, \vb{w} \right>^{1/2}} 
     + \frac{\left< \vb{w}, \vb{w} \right>}{\left< \vb{w}, \vb{w} \right>} \\
    &= 1 - 2 \cdot \frac{\left< \vb{v}, \vb{w} \right>}{\left< \vb{v}, \vb{v} \right>^{1/2} \left< \vb{w}, \vb{w} \right>^{1/2}} + 1 \\
    &= 2 - 2 \cdot \frac{\left< \vb{v}, \vb{w} \right>}{\left< \vb{v}, \vb{v} \right>^{1/2} \left< \vb{w}, \vb{w} \right>^{1/2}}.
  \end{align*}
  Since this is \( \ge 0 \), we conclude:
  \[
    2 - 2 \cdot \frac{\left< \vb{v}, \vb{w} \right>}{\left< \vb{v}, \vb{v} \right>^{1/2} \left< \vb{w}, \vb{w} \right>^{1/2}} \ge 0,
  \]
  which implies:
  \[
    \frac{\left< \vb{v}, \vb{w} \right>}{\left< \vb{v}, \vb{v} \right>^{1/2} \left< \vb{w}, \vb{w} \right>^{1/2}} \le 1.
  \]
  If \( \left< \vb{v}, \vb{w} \right> < 0 \), the same argument applied to \( -\vb{w} \) yields the inequality
  \[
    \abs{\left< \vb{v}, \vb{w} \right>} \le \left< \vb{v}, \vb{v} \right>^{1/2} \left< \vb{w}, \vb{w} \right>^{1/2}.
  \]
\end{proof}

\begin{example}
  In \( \mathbb{R}^n \) with the standard dot product
  \[
    \left< \vb{v}, \vb{w} \right> := \sum_{j=1}^n v_j w_j,
  \]
  the Cauchy--Schwarz inequality becomes
  \[
    \left| \sum_{j=1}^n v_j w_j \right| \le 
    \left( \sum_{j=1}^n v_j^2 \right)^{1/2} 
    \left( \sum_{j=1}^n w_j^2 \right)^{1/2}.
  \]
\end{example}

\begin{theorem}[Cauchy--Schwarz Inequality (Complex Case)]\label{thrm:Complex Cauchy-Schwarz}
  Let \( \mathbf{V} \) be a complex inner product space. For all \( \vb{v}, \vb{w} \in \mathbf{V} \), we have:
  \[
    \abs{\left< \vb{v}, \vb{w} \right>} \le \left< \vb{v}, \vb{v} \right>^{1/2} \left< \vb{w}, \vb{w} \right>^{1/2}.
  \]
\end{theorem}
\begin{proof}
  Let \( \vb{v}, \vb{w} \in \mathbf{V} \) be non-zero vectors and define for \( \lambda \in \mathbb{C} \) the function:
  \[
    f(\lambda) = \left< \vb{v} - \lambda \vb{w}, \vb{v} - \lambda \vb{w} \right>.
  \]
Note that \( f \left(  \lambda \right) \ge 0 \) by positive semi-definiteness. We can expand \( f \left( \lambda\right)\) as 
\begin{align*}
  f \left(  \lambda \right) &= \left< \vb{v} - \lambda \vb{w}, \vb{v}- \lambda \vb{w} \right> \\
&=  \left< \vb{v}, \vb{v}- \lambda \vb{w} \right> + \left< -\lambda \vb{w}, \vb{v}- \lambda \vb{w} \right>\\
&=  \left< \vb{v}, \vb{v}- \lambda \vb{w} \right> - \lambda \left< \vb{w}, \vb{v}- \lambda \vb{w} \right>\\
&= \left< \vb{v}, \vb{v} \right> + \left< \vb{v}, - \lambda \vb{w} \right> - \lambda \left< \vb{w}, \vb{v} \right> - \lambda \left< \vb{w}, - \lambda \vb{w} \right> \\
&= \left< \vb{v}, \vb{v} \right> - \overline{\lambda} \left< \vb{v}, \vb{w} \right> - \lambda \left< \vb{w}, \vb{v} \right> + \abs{\lambda}^{2}  \left< \vb{w}, \vb{w} \right>
\end{align*}

Now choose
  \[
    \lambda = \frac{\left< \vb{v}, \vb{w} \right>}{\left< \vb{w}, \vb{w} \right>}.
  \]
  Substituting into the expanded \( f(\lambda) \), we have
  \begin{align*}
    f \left( \frac{\left< \vb{v}, \vb{w} \right>}{\left< \vb{w}, \vb{w} \right>}\right) &=  \left< \vb{v}, \vb{v} \right> - \overline{ \frac{\left< \vb{v}, \vb{w} \right>}{\left< \vb{w}, \vb{w} \right>}} \left< \vb{v}, \vb{w} \right> -  \frac{\left< \vb{v}, \vb{w} \right>}{\left< \vb{w}, \vb{w} \right>} \left< \vb{w}, \vb{v} \right> + \abs{ \frac{\left< \vb{v}, \vb{w} \right>}{\left< \vb{w}, \vb{w} \right>}}^{2}  \left< \vb{w}, \vb{w} \right> \\
    &= \left< \vb{v}, \vb{v} \right> - \overline{ \frac{\left< \vb{v}, \vb{w} \right>}{\left< \vb{w}, \vb{w} \right>}} \left< \vb{v}, \vb{w} \right> -  \frac{\left< \vb{v}, \vb{w} \right>}{\left< \vb{w}, \vb{w} \right>} \overline{ \left< \vb{v}, \vb{w} \right>} + \abs{ \frac{\left< \vb{v}, \vb{w} \right>}{\left< \vb{w}, \vb{w} \right>}}^{2}  \left< \vb{w}, \vb{w} \right> \\
    &=\left< \vb{v}, \vb{v} \right> -  2 \frac{\overline{\left< \vb{v}, \vb{w} \right> } \left< \vb{v}, \vb{w} \right>  }{\left< \vb{w}, \vb{w} \right>} + \abs{ \frac{\left< \vb{v}, \vb{w} \right>}{\left< \vb{w}, \vb{w} \right>}}^{2}  \left< \vb{w}, \vb{w} \right> \\
    &=\left< \vb{v}, \vb{v} \right> -  2 \frac{\overline{\left< \vb{v}, \vb{w} \right> } \left< \vb{v}, \vb{w} \right>  }{\left< \vb{w}, \vb{w} \right>} + \frac{\overline{\left< \vb{v}, \vb{w} \right> } \left< \vb{v}, \vb{w} \right>  }{\left< \vb{w}, \vb{w}\right> \left< \vb{w}, \vb{w} \right>} \left< \vb{w}, \vb{w} \right>  \\
    &=\left< \vb{v}, \vb{v} \right> -  2 \frac{\overline{\left< \vb{v}, \vb{w} \right> } \left< \vb{v}, \vb{w} \right>  }{\left< \vb{w}, \vb{w} \right>} + \frac{\overline{\left< \vb{v}, \vb{w} \right> } \left< \vb{v}, \vb{w} \right>  }{\left< \vb{w}, \vb{w} \right>} \\
    & =\left< \vb{v}, \vb{v} \right> -  \frac{\overline{\left< \vb{v}, \vb{w} \right> } \left< \vb{v}, \vb{w} \right>  }{\left< \vb{w}, \vb{w} \right>}\\
    &=\left< \vb{v}, \vb{v} \right> -  \frac{  \abs{\left< \vb{v}, \vb{w} \right>}^2  }{\left< \vb{w}, \vb{w} \right>}
  \end{align*}
  Finally we can apply the non-negativity of \( f \),
  \begin{align*}
   0 &\le \left< \vb{v}, \vb{v} \right> -  \frac{  \abs{\left< \vb{v}, \vb{w} \right>}^2  }{\left< \vb{w}, \vb{w} \right>} \\
    \frac{  \abs{\left< \vb{v}, \vb{w} \right>}^2  }{\left< \vb{w}, \vb{w} \right>} & \le  \left< \vb{v}, \vb{v} \right> \\
    \abs{\left< \vb{v}, \vb{w} \right>}^2 & \le \left< \vb{v}, \vb{v} \right> \left< \vb{w}, \vb{w} \right>
  \end{align*}
  
  and taking square roots
  \[
  \boxed{  \abs{\left< \vb{v}, \vb{w} \right>} \le \left< \vb{v}, \vb{v} \right>^{1/2} \left< \vb{w}, \vb{w} \right>^{1/2}}.
  \]
\end{proof}

\begin{example}
  Consider the complex inner product on \( \mathbb{C}^n \) defined by:
  \[
    \left< \vb{v}, \vb{w} \right> := \sum_{j=1}^n v_j \overline{w_j}.
  \]
  Then the Cauchy--Schwarz inequality becomes:
  \[
    \left| \sum_{j=1}^n v_j \overline{w_j} \right| 
    \le \left( \sum_{j=1}^n \abs{v_j}^2 \right)^{1/2}
         \left( \sum_{j=1}^n \abs{w_j}^2 \right)^{1/2}.
  \]
\end{example}

\begin{example}
  Let \( f, g \in L^2(\gamma) \), where \( \gamma \subset \mathbb{C} \) is a piecewise smooth contour. Define the inner product:
  \[
    \left< f, g \right> := \int_{\gamma} f(z) \, \overline{g(z)} \, \abs{\dd{z}}.
  \]
  Then the Cauchy--Schwarz inequality gives:
  \[
    \left| \int_{\gamma} f(z) \, \overline{g(z)} \, \abs{\dd{z}} \right|
    \le \left( \int_{\gamma} \abs{f(z)}^2 \, \abs{\dd{z}} \right)^{1/2}
         \left( \int_{\gamma} \abs{g(z)}^2 \, \abs{\dd{z}} \right)^{1/2}.
  \]
\end{example}

\begin{exercise}
    Show that 
    \[
    \sum_{j=1}^{n } a_{j} \le \sqrt{n \sum_{j=1}^{n} \left( a_{j } \right)^{2} }.
    \]
\end{exercise}
\begin{solution}
    Apply the Cauchy–Schwarz inequality to the vectors \( \vec{a} = (a_1, a_2, \dots, a_n) \) and \( \vec{b} = (1, 1, \dots, 1) \in \mathbb{R}^n \). Then:
    \[
    \left( \sum_{j=1}^n a_j \cdot 1 \right)^2 
    \le \left( \sum_{j=1}^n a_j^2 \right) \left( \sum_{j=1}^n 1^2 \right)
    = \left( \sum_{j=1}^n a_j^2 \right) \cdot n.
    \]
    Taking square roots of both sides gives:
    \[
    \sum_{j=1}^n a_j 
    \le \sqrt{n \sum_{j=1}^n a_j^2}.
    \]
    This proves the inequality.
\end{solution}






\printbibliography[
heading=bibintoc,
title={Bibliography}
] 


\end{document}