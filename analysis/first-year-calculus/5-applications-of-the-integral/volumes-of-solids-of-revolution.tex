Suppose we wish to calculate the volume of a solid obtained by rotating the graph of a function \( f(x) \geq 0 \) about the \( x \)-axis for \( x \) between \( a \) and \( b \). We think of slicing the solid perpendicular to the \( x \)-axis. At position \( x \), the cross-section is a circle of radius \( f(x) \), so its area is 
\[
A(x) = \pi \big(f(x)\big)^{2}.
\]
The volume is then obtained by integrating these cross-sectional areas:
\[
\boxed{V = \int_a^b A(x)\,\dd{x} 
= \pi \int_a^b \big(f(x)\big)^{2} \,\dd{x}.}
\]

Now suppose we want to find the volume of the solid formed by rotating the region bounded by two functions $f(x)$ and $g(x)$ about the $x$-axis. Suppose that $f(x)$ and $g(x)$ intersect at $x=a$ and $x=b$, and that $f(x) \geq g(x)$ on the interval $[a,b]$.

When we take a vertical cross-section at any point $x$ in $[a,b]$ and rotate it about the $x$-axis, we obtain a \textbf{washer} (a disk with a hole in the middle). The outer radius of this washer is $f(x)$, and the inner radius is $g(x)$.

The area of a washer is the area of the outer circle minus the area of the inner circle:
\[\text{Area of washer} = \pi R^2 - \pi r^2 = \pi(R^2 - r^2)\]

In our case, this gives us:
\[\text{Area of cross-section} = \pi[f(x)]^2 - \pi[g(x)]^2 = \pi\left([f(x)]^2 - [g(x)]^2\right)\]

To find the total volume, we integrate these cross-sectional areas from $x = a$ to $x = b$:
\[ \boxed{V = \int_a^b \pi\left([f(x)]^2 - [g(x)]^2\right) \, \dd{x} = \pi \int_a^b \left([f(x)]^2 - [g(x)]^2\right) \, \dd{x}}\]

\begin{exercise}
    Use the washer method to find the volume of the region above the curve \( y=x^{3} \) and below \( y=1 \) between \( x=0 \) and \( x=1 \) about the \( x \)-axis.
\end{exercise}
\begin{solution}
    We have 
    \begin{align*}
        V &= \pi \int_{0}^{1} 1^{2}- \left( x^{3 } \right)^{2 } \dd{x}\\
        &= \pi \int_{0}^{1} 1 - x^{6} \dd{x}\\
        &= \pi \left( x - \frac{x^{7}}{7} \right)\eval_{0}^{1}
    \end{align*}
    \[ \boxed{V = \frac{6 \pi}{7}} \]
\end{solution}

\begin{exercise}
    Calculate the same volume as above using cylindrical shells.
\end{exercise}
\begin{solution}
    We calculate the inverse function of \( y=x^{3} \) to get \( y= x^{\frac{1}{3}}. \) Then using the cylindrical shells formula, we get 
    \begin{align*}
        V &= 2\pi \int_{0}^{1} x \cdot x^{\frac{1}{3}} \dd{x}\\
        &= 2 \pi \int_{0}^{1} x^{\frac{4}{3}} \dd{x}\\
        &= 2 \pi \cdot \frac{3}{7} \cdot x^{\frac{7}{3}} \eval_{0}^{1}
    \end{align*}
    \[ \boxed{V = \frac{6 \pi}{7}} \]
\end{solution}

\begin{exercise}
    Now rotate the same region but about the \( y \)-axis using cylindrical shells.
\end{exercise}
\begin{solution}
   Our "height" is \( 1-x^{3} \). So we have 
   \begin{align*}
    V &= 2 \pi \int_{0}^{1} x \left( 1-x^{3} \right) \dd{x} \\
    &= 2 \pi \int_{0}^{1} x -x^{4} \dd{x} \\
    &= 2 \pi \left( \frac{x^{2}}{2}- \frac{x^{5}}{5} \eval_{0}^{1} \right)
   \end{align*}
    \[ \boxed{V = \frac{3 \pi}{5}} \]
\end{solution}

\begin{exercise}
    Calculate the same volume as above but using the washer method.
\end{exercise}
\begin{solution}
    \begin{align*}
        V &= \pi \int_{0}^{1} \left( x^{\frac{1}{3}} \right)^{2} \dd{x} \\
        &= \pi \int_{0}^{1} x^{\frac{2}{3}} \dd{x} \\
        &= \pi \cdot \frac{3}{5} \cdot x ^{\frac{5}{3}} \eval_{0}^{1}
    \end{align*}
    \[ \boxed{V = \frac{3 \pi}{5}} \]
\end{solution}

\begin{exercise}
    Let \( R \) be the region above the \( x \)-axis, below the graph of \( y= 1- \frac{1}{x} \) and to the right of \( x=3 \). Find the volume of \( R \) rotated about the \( x \)-axis by the disk method.
\end{exercise}
\begin{solution}
    We have 
    \begin{align*}
        V &= \pi \int_{1}^{3} \left( 1 - \frac{1}{x} \right)^{2} \dd{x}\\
        &= \pi \int_{1}^{3} 1- \frac{2}{x}+ \frac{1}{x^{2}} \dd{x} \\
        &= \pi \left( x - 2 \ln(x) - \frac{1}{x} \right) \eval_{1}^{3}
    \end{align*}
    \[ \boxed{V = \left( \frac{8}{3} - \ln(9) \right) \pi} \]
\end{solution}

\begin{exercise}
    Find the same volume as above but with cylindrical shells. 
\end{exercise}
\begin{solution}
    With shells, our height is \( 3 - \frac{1}{1-y} \) so our integral is 
    \begin{align*}
        V &= 2 \pi \int_{0}^{\frac{2}{3}} y \left( 3- \frac{1}{1-y} \right) \dd{y} \\
        &= 2 \pi \int_{0}^{\frac{2}{3}} 3y - \frac{y}{1-y} \dd{y} \\
        &= 2 \pi \left( \frac{3}{2}y^{2} + \ln(1-y) +y \right)\eval_{0}^{\frac{2}{3}} \\
        &= 2 \pi \left( \frac{2}{3} + \ln \left( \frac{1}{3} \right) + \frac{2}{3} \right)
    \end{align*}
    It can be easily verified that 
    \[ \boxed{V = \left( \frac{8}{3} - \ln(9) \right) \pi} \]
\end{solution}

