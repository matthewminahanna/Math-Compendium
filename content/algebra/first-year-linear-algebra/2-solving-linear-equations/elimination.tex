We want a systematic way of solving systems of linear equations. Gaussian elimination provides this. 
We first note that the system
\begin{align*}
  3x+y-z &= 2 \\
  4x+2y +3z &=23 \\
  x-3y+2z &=19 
\end{align*}
can be written more compactly as an \vocab{augmented matrix}.
\[ \begin{bmatrix}[*2cr@{\quad}|@{\quad}>{\color{red}}r]
  3 & 1 & -1  &  2 \\
  4 & 2 & 3 & 23\\
  1 & -3 & 2 & 19
\end{bmatrix} \] 
Here each equation gets assigned a row. We will dedicate this section to motivating the row operations needed for Gaussian elimination. \\
Let us take, for example the first two equations
\begin{align*}
  3x+y-z &= 2 \\
  4x+2y +3z &=23 \\
\end{align*}
We can take arbitrary linear combinations of these two equations and \emph{replace one equation}. For example, we can take three copies of the second equation and subtract from it four copies of the first. This gives us 
\[ 3 \left( 4x+2y +3z =23 \right) - 4 \left(  3x+y-z = 2 \right) \Rightarrow \left( 12x + 6y + 9z =69 \right) - \left( 12x +4y -4z =8 \right) \Rightarrow \boxed{0x +2y +13z = 61} \]
Notice that because we are taking linear combinations of these two equations, the solution we found earlier of \( \left( x,y,z \right) = \left( 3,-2,5 \right) \) is a solution to \[ 0x +2y +13z = 61. \] 
We can verify this: \( 2(-2) + 13(5) = -4 + 65 = 61 \). As such the system 
\begin{align*}
  3x+y-z &= 2 \\
  0x +2y +13z &= 61 \\
  x-3y+2z &=19 
\end{align*}
still has the same solution as above. So we can write our augmented system as
\[ \begin{bmatrix}[*2cr@{\quad}|@{\quad}>{\color{red}}r]
  3 & 1 & -1  &  2 \\
  0 & 2 & 13 & 61\\
  1 & -3 & 2 & 19
\end{bmatrix} \]
Finally note that row exchange is trivially acceptable since our solution shouldn't depend on the order in which we write out equations.\\
Notice that if we have an augmented matrix of the form
\[ \begin{bmatrix}[*2cr@{\quad}|@{\quad}>{\color{red}}r]
  a & b & c  &  s  \\
  0 & d &e  & t \\
  0 & 0 & f & u
\end{bmatrix} \]
This corresponds to the system of equations
\begin{align*}
  ax+by+ cz &= s \\
  dy +ez &=t  \\
  fz &=u
\end{align*} 
So then we can simply solve for \( z \) in the third equation, then back substitute our solution for \( z \) into the second equation to solve for \( y \), and finally solve for \( x \) by substituting the solutions for \( y \) and \( z \). \emph{This is the goal for Gaussian elimination}. We wish to convert an arbitrary linear system into an \vocab{upper triangular matrix.}

\begin{example}
  Let us try Gaussian elimination with a system we are very familiar with 
  \begin{align*}
  3x+y-z &= 2 \\
  4x+2y +3z &=23 \\
  x-3y+2z &=19 
\end{align*}
Converting to an augmented matrix, we get
\[ \begin{bmatrix}[*2cr@{\quad}|@{\quad}>{\color{red}}r]
  3 & 1 & -1  &  2 \\
  4 & 2 & 3 & 23\\
  1 & -3 & 2 & 19
\end{bmatrix} \] 

First, we eliminate the $x$-term in the second row by replacing $R_2$ with $(3R_2 - 4R_1)$:
\[\begin{bmatrix}[*2cr@{\quad}|@{\quad}>{\color{red}}r]
  3 & 1 & -1  &  2 \\
  4 & 2 & 3 & 23\\
  1 & -3 & 2 & 19
\end{bmatrix}\quad  \underrightarrow{(3R_{2} - 4R_{1}) \to R_{2}} \quad \begin{bmatrix}[*2cr@{\quad}|@{\quad}>{\color{red}}r]
  3 & 1 & -1  &  2 \\
  0 & 2 & 13 & 61\\
  1 & -3 & 2 & 19
\end{bmatrix} \]

Next, we eliminate the $x$-term in the third row by replacing $R_3$ with $(3R_3 - R_1)$:
\[  \begin{bmatrix}[*2cr@{\quad}|@{\quad}>{\color{red}}r]
  3 & 1 & -1  &  2 \\
  0 & 2 & 13 & 61\\
  1 & -3 & 2 & 19
\end{bmatrix} \quad  \underrightarrow{(3R_{3} - R_{1}) \to R_{3}} \quad  \begin{bmatrix}[*2cr@{\quad}|@{\quad}>{\color{red}}r]
  3 & 1 & -1  &  2 \\
  0 & 2 & 13 & 61\\
  0 & -10 &7 & 55
\end{bmatrix} \]

Finally, we eliminate the $y$-term in the third row by replacing $R_3$ with $(R_3 + 5R_2)$:
\[ \begin{bmatrix}[*2cr@{\quad}|@{\quad}>{\color{red}}r]
  3 & 1 & -1  &  2 \\
  0 & 2 & 13 & 61\\
  0 & -10 &7 & 55
\end{bmatrix}  \quad  \underrightarrow{(R_{3} + 5R_{2}) \to R_{3}} \quad \begin{bmatrix}[*2cr@{\quad}|@{\quad}>{\color{red}}r]
  3 & 1 & -1  &  2 \\
  0 & 2 & 13 & 61\\
  0 & 0 & 72 & 360
\end{bmatrix}\]

We now have the upper triangular system 
\begin{align*}
  3x +y -z &= 2\\
  2y + 13z &= 61\\
  72z &= 360
\end{align*}

Now we use back substitution. From the third equation, we see that $z = \frac{360}{72} = 5$, so $\boxed{z=5}$. 

From the second equation:
\begin{align*}
  2y + 13(5) &= 61\\
  2y + 65 &= 61\\
  2y &= -4\\
  y &= -2
\end{align*}
so $\boxed{y=-2}$. 

Finally, from the first equation:
\begin{align*}
  3x + (-2) - 5 &= 2\\
  3x - 7 &= 2\\
  3x &= 9\\
  x &= 3
\end{align*}
so $\boxed{x=3}$.

Therefore, our solution is $(x,y,z) = (3,-2,5)$.
\end{example}

\begin{example}
  Use Gaussian elimination to solve the system 
  \begin{align*}
    3w +2x +11y + 5z &=25 \\
    -2 w +7x -8y +z &=13\\
    12w-3x+9y + 4z &=-21 \\
    -6w +x -4y -3z &= 15
  \end{align*}
  We have the following augmented matrix 
  \[ \begin{bmatrix}[*3rr@{\quad}|@{\quad}>{\color{black}}r]
  3 & 2& 11  &  5 & 25\\
  -2 & 7 & -8 & 1 & 13\\
  12 & -3 & 9 & 4 & -21 \\
  -6 & 1 & -4 & -3 & 15
\end{bmatrix} \]
For the first column, we have 
  \[ \begin{bmatrix}[*3rr@{\quad}|@{\quad}>{\color{black}}r]
  3 & 2& 11  &  5 & 25\\
  -2 & 7 & -8 & 1 & 13\\
  12 & -3 & 9 & 4 & -21 \\
  -6 & 1 & -4 & -3 & 15
\end{bmatrix} \quad  \underrightarrow{(3R_{2} + 2R_{1}) \to R_{2}} \quad \begin{bmatrix}[*3rr@{\quad}|@{\quad}>{\color{black}}r]
  3 & 2& 11  &  5 & 25\\
  0 & 25 & -2 & 13 & 89\\
  12 & -3 & 9 & 4 & -21 \\
  -6 & 1 & -4 & -3 & 15
\end{bmatrix}\]
\[ \begin{bmatrix}[*3rr@{\quad}|@{\quad}>{\color{black}}r]
  3 & 2& 11  &  5 & 25\\
  0 & 25 & -2 & 13 & 89\\
  12 & -3 & 9 & 4 & -21 \\
  -6 & 1 & -4 & -3 & 15
\end{bmatrix}  \quad  \underrightarrow{(R_{3} -4 R_{1}) \to R_{3}} \quad \begin{bmatrix}[*3rr@{\quad}|@{\quad}>{\color{black}}r]
  3 & 2& 11  &  5 & 25\\
  0 & 25 & -2 & 13 & 89\\
  0 & -11 & -35 & -16 & -121 \\
  -6 & 1 & -4 & -3 & 15
\end{bmatrix} \]
\[ \begin{bmatrix}[*3rr@{\quad}|@{\quad}>{\color{black}}r]
  3 & 2& 11  &  5 & 25\\
  0 & 25 & -2 & 13 & 89\\
  0 & -11 & -35 & -16 & -121 \\
  -6 & 1 & -4 & -3 & 15
\end{bmatrix}\quad  \underrightarrow{(R_{4} +2 R_{1}) \to R_{3}} \quad  \begin{bmatrix}[*3rr@{\quad}|@{\quad}>{\color{black}}r]
  3 & 2& 11  &  5 & 25\\
  0 & 25 & -2 & 13 & 89\\
  0 & -11 & -35 & -16 & -121 \\
  0 & 5 & 18 & 7 & 65
\end{bmatrix} \]
For the second column, we have 
\[  \begin{bmatrix}[*3rr@{\quad}|@{\quad}>{\color{black}}r]
  3 & 2& 11  &  5 & 25\\
  0 & 25 & -2 & 13 & 89\\
  0 & -11 & -35 & -16 & -121 \\
  0 & 5 & 18 & 7 & 65
\end{bmatrix} \quad  \underrightarrow{(25R_{3} +11 R_{2}) \to R_{3}} \quad  \begin{bmatrix}[*3rr@{\quad}|@{\quad}>{\color{black}}r]
  3 & 2& 11  &  5 & 25\\
  0 & 25 & -2 & 13 & 89\\
  0 & 0 & -897 & -257 & -2046 \\
  0 & 5 & 18 & 7 & 65
\end{bmatrix} \]
\[ \begin{bmatrix}[*3rr@{\quad}|@{\quad}>{\color{black}}r]
  3 & 2& 11  &  5 & 25\\
  0 & 25 & -2 & 13 & 89\\
  0 & 0 & -897 & -257 & -2046 \\
  0 & 5 & 18 & 7 & 65
\end{bmatrix}  \quad  \underrightarrow{(5R_{4} - R_{2}) \to R_{4}} \quad \begin{bmatrix}[*3rr@{\quad}|@{\quad}>{\color{black}}r]
  3 & 2& 11  &  5 & 25\\
  0 & 25 & -2 & 13 & 89\\
  0 & 0 & -897 & -257 & -2046 \\
  0 & 0 & 92 & 22 & 236
\end{bmatrix} \]
\[ \begin{bmatrix}[*3rr@{\quad}|@{\quad}>{\color{black}}r]
  3 & 2& 11  &  5 & 25\\
  0 & 25 & -2 & 13 & 89\\
  0 & 0 & -897 & -257 & -2046 \\
  0 & 0 & 92 & 22 & 236
\end{bmatrix} \quad  \underrightarrow{ \left( \frac{1}{2}R_{4} \right) \to R_{4}} \quad \begin{bmatrix}[*3rr@{\quad}|@{\quad}>{\color{black}}r]
  3 & 2& 11  &  5 & 25\\
  0 & 25 & -2 & 13 & 89\\
  0 & 0 & -897 & -257 & -2046 \\
  0 & 0 & 46 & 11 & 118
\end{bmatrix}\]
Finally for the third column, we have 
\[  \begin{bmatrix}[*3rr@{\quad}|@{\quad}>{\color{black}}r]
  3 & 2& 11  &  5 & 25\\
  0 & 25 & -2 & 13 & 89\\
  0 & 0 & -897 & -257 & -2046 \\
  0 & 0 & 46 & 11 & 118
\end{bmatrix}\quad  \underrightarrow{ \left( 897R_{4} + 46 R_{3} \right) \to R_{4}} \quad \begin{bmatrix}[*3rr@{\quad}|@{\quad}>{\color{black}}r]
  3 & 2& 11  &  5 & 25\\
  0 & 25 & -2 & 13 & 89\\
  0 & 0 & -897 & -257 & -2046 \\
  0 & 0 & 0 & -1955 & 11730
\end{bmatrix} \]
We now have the upper triangular system 
\begin{align*}
  3w + 2x + 11y + 5z &= 25\\
  25x - 2y + 13z &= 89\\
  -897y - 257z &= -2046\\
  -1955z &= 11730
\end{align*}

Now we use back substitution. From the fourth equation, we see that $z = \frac{11730}{-1955} = -6$, so $\boxed{z=-6}$. 

From the third equation:
\begin{align*}
  -897y - 257(-6) &= -2046\\
  -897y + 1542 &= -2046\\
  -897y &= -3588\\
  y &= 4
\end{align*}
so $\boxed{y=4}$. 

From the second equation:
\begin{align*}
  25x - 2(4) + 13(-6) &= 89\\
  25x - 8 - 78 &= 89\\
  25x - 86 &= 89\\
  25x &= 175\\
  x &= 7
\end{align*}
so $\boxed{x=7}$. 

Finally, from the first equation:
\begin{align*}
  3w + 2(7) + 11(4) + 5(-6) &= 25\\
  3w + 14 + 44 - 30 &= 25\\
  3w + 28 &= 25\\
  3w &= -3\\
  w &= -1
\end{align*}
so $\boxed{w=-1}$.

Therefore, our solution is $(w,x,y,z) = (-1,7,4,-6)$.
\end{example}

\subsection{Elimination Matrices}

Much like how in \( \mathbb{R}^{3} \) every vector can be written as a linear combination of the standard basis vectors
\( \vb{e}_{1}, \vb{e}_{2}, \vb{e}_{3} \),
we can describe the steps of Gaussian elimination using linear combinations of elimination matrices.

\begin{dfn}
By an \vocab{elimination matrix}, denoted \( \vb{E}_{ji} \) with \( j > i \), we mean a matrix that is equal to the identity except for a scalar \( m \) in the \( (j,i) \) entry.
\end{dfn}
