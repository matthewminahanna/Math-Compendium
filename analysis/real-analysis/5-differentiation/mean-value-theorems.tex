\begin{dfn}
    Suppose that \( f \) is a continuous real-valued function. We say that \( f \) attains a \vocab{local maximum} at \( p \) if there exists a \( \delta >0 \) such that \( f(p) \ge f(q) \) whenever \( q \in B\left( p; \delta \right)\).\\

    Similarly, we say that \( f \) attains a \vocab{local minimum} at \( p \) if there exists a \( \delta >0 \) such that \( f(p) \le f(q) \) whenever \( q \in B\left( p; \delta \right)\).
\end{dfn}

\begin{theorem}
    Suppose that \( f: \left[ a,b \right]\to \mathbb{R}\) is differentiable on \( (a,b) \). If \( f \) attains a local maximum or minimum at \( x \), then \( f'(x) =0 \).
\end{theorem}
\begin{proof}
    Let the givens be as stated and suppose that \( f \) attains a local maximum at \( x \in \left( a,b \right) \). Then there is a \( \delta >0 \) such that whenever \( t \in B \left( x, \delta \right) \), we have \( f(x) \ge f(t) \). 
    
    First suppose that \( x- \delta < t < x \). Then \( f(x) - f(t) \ge 0 \) and \( x - t > 0 \), so
    \begin{equation}\label{eq:difference-quotient-positive}
        \frac{f(x) - f(t)}{x-t} \ge 0 
    \end{equation}
    as it is the quotient of two non-negative numbers where the denominator is positive.
    
    On the other hand, if \( x < t < x + \delta \), then \( f(x) - f(t) \ge 0 \) and \( x - t < 0 \), so
    \begin{equation}\label{eq:difference-quotient-negative}
        \frac{f(x) - f(t)}{x-t} \le 0 
    \end{equation}
    as it is the quotient of a non-negative and negative number.
    
    Since \( f \) is differentiable at \( x \), the limit 
    \[ f'(x)= \lim_{t \to x}\frac{f(x) - f(t)}{x-t} \] 
    exists. By \Cref{eq:difference-quotient-positive}, we have 
    \[ f'(x) = \lim_{t \to x^-}\frac{f(x) - f(t)}{x-t} \ge 0 \]
    and by \Cref{eq:difference-quotient-negative}, we have
    \[ f'(x) = \lim_{t \to x^+}\frac{f(x) - f(t)}{x-t} \le 0 \]
    Since both one-sided limits equal \( f'(x) \), we must have \( f'(x) = 0 \).
    
    The case where \( f \) attains a local minimum at \( x \) follows by similar reasoning, with the inequalities reversed.
\end{proof}
