Although this substitution technique is not typically introduced in standard calculus courses, it is remarkably elegant and powerful for integrating rational functions of trigonometric expressions.

The main idea is to re-parameterize the unit circle using the half-angle tangent. We begin with the substitution 
\begin{equation}
    u = \tan\left( \frac{x}{2} \right). \label{eqn:6/21/25/2}
\end{equation}

From this substitution, we can find the differential:
\begin{align*}
    \dd{u} &= \dd{\left( \tan\left( \frac{x}{2} \right) \right)} \\
    &= \frac{1}{2} \sec^2\left( \frac{x}{2} \right) \dd{x}\\
    &= \frac{1}{2} \left( 1+ \tan^2\left( \frac{x}{2} \right) \right) \dd{x}\\
    &= \frac{1}{2} \left( 1+u^2 \right) \dd{x}.
\end{align*}

Solving for $\dd{x}$:
\begin{equation}
    \boxed{ \dd{x} = \frac{2}{1 +u^2} \dd{u}}. \label{eqn:6/21/25/3}
\end{equation}

Now we derive the substitution formulas for the main trigonometric functions.


Using the double angle formula for tangent:
\begin{align*}
    \tan(x) &= \tan\left( 2 \cdot \frac{x}{2} \right)\\
    &= \frac{2\tan\left( \frac{x}{2} \right)}{1 - \tan^2\left( \frac{x}{2} \right)}\\
    &= \frac{2u}{1-u^2}.
\end{align*}

Therefore:
\begin{equation}
    \boxed{ \tan(x) = \frac{2u}{1-u^2} }. \label{eqn:6/21/25/4}
\end{equation}


To derive the substitutions for sine and cosine, we interpret $u = \tan\left( \frac{x}{2} \right)$ geometrically. Consider a right triangle with angle $\frac{x}{2}$, where the opposite side has length $u$ and the adjacent side has length $1$. By the Pythagorean theorem, the hypotenuse has length $\sqrt{1+u^2}$.

This gives us:
\[ \sin\left( \frac{x}{2} \right) = \frac{u}{\sqrt{1+u^2}}, \quad \cos\left( \frac{x}{2} \right) = \frac{1}{\sqrt{1+u^2}}. \]

Using the double angle formula for sine:
\begin{align*}
    \sin(x) &= \sin\left( 2 \cdot \frac{x}{2} \right)\\
     &= 2 \sin\left( \frac{x}{2} \right) \cos\left( \frac{x}{2} \right)\\
     &= 2 \cdot \frac{u}{\sqrt{1+u^2}} \cdot \frac{1}{\sqrt{1+u^2}}\\
     &= \frac{2u}{1+u^2}.
\end{align*}

Therefore:
\begin{equation}
    \boxed{ \sin(x) = \frac{2u}{1+u^2} }. \label{eqn:6/21/25/5}
\end{equation}

Using the double angle formula for cosine:
\begin{align*}
    \cos(x) &= \cos\left( 2 \cdot \frac{x}{2} \right)\\
    &= \cos^2\left( \frac{x}{2} \right) - \sin^2\left( \frac{x}{2} \right)\\
    &= \frac{1}{1+u^2} - \frac{u^2}{1+u^2}\\
    &= \frac{1-u^2}{1+u^2}.
\end{align*}

Therefore:
\begin{equation}
    \boxed{ \cos(x) = \frac{1-u^2}{1+u^2} }. \label{eqn:6/21/25/6}
\end{equation}


The Weierstrass substitution $u = \tan\left( \frac{x}{2} \right)$ transforms any trigonometric rational function into an algebraic rational function, which can then be integrated using partial fractions or other algebraic techniques.

\begin{example}
    Find \( \int \sec(x) \, dx \).
    
    We have 
    \[ \int \frac{1}{\cos(x)} \, dx. \]
    
    Applying \Cref{eqn:6/21/25/3,eqn:6/21/25/6}, we have  
    \begin{align*}
        \int \sec(x) \, dx &= \int \frac{1+u^{2}}{1-u^{2}} \cdot \frac{2}{1 +u^2} \, du \\
        &= \int \frac{2}{1-u^{2}} \, du\\
        &= \int \left(\frac{1}{1+u} + \frac{1}{1-u}\right) \, du\\
        &= \ln|1+u| - \ln|1-u| + C \\
        &= \ln\left|\frac{1+u}{1-u}\right| + C
    \end{align*}
    
    It is tempting to substitute \( u = \tan\left(\frac{x}{2}\right) \) and it would still be correct, but we can arrive at the standard answer with some patience: 
    \begin{align*}
         \ln\left|\frac{1+u}{1-u}\right| &= \ln\left|\frac{1+u}{1-u} \cdot \frac{1+u}{1+u}\right|\\
         &= \ln\left|\frac{(1 +u)^{2}}{1-u^{2}}\right|\\
         &= \ln\left|\frac{1+2u+u^{2}}{1-u^{2}}\right|\\
         &= \ln\left|\frac{1+u^{2}}{1-u^{2}} + \frac{2u}{1-u^{2}}\right|\\
         &= \ln \abs{\sec(x) + \tan(x)}
    \end{align*}
So 
\[ \boxed{ \int \sec \left( x  \right) \dd{x} = \ln \abs{\sec \left( x  \right) + \tan{ \left( x  \right) }} +C.} \]
\end{example}
