Now that we have compiled a list of important functions, we will use them as building blocks to produce all the functions typically considered in a first year calculus course. 

\subsection{Transformations of Functions}

\begin{dfn}
    Let \( y=f(x) \) be any function and \( c >0 \), then we can apply the following transformations to \( f(x) \).
\begin{description}[style=nextline]
 \item[Upward shift:] 
  \( y = f(x) + c \) shifts the graph \textbf{up} by \( c \) units.  
  If \( (x_0, y_0) \) is on the graph of \( y = f(x) \), then \( (x_0, y_0 + c) \) is on the graph of \( y = f(x) + c \).

  \item[Downward shift:] 
  \( y = f(x) - c \) shifts the graph \textbf{down} by \( c \) units.  
  If \( (x_0, y_0) \) is on the graph of \( y = f(x) \), then \( (x_0, y_0 - c) \) is on the graph of \( y = f(x) - c \).

  \item[Left shift:] 
  \( y = f(x + c) \) shifts the graph \textbf{left} by \( c \) units.  
  If \( (x_0, y_0) \) is on the graph of \( y = f(x) \), then \( (x_0 - c, y_0) \) is on the graph of \( y = f(x + c) \).

  \item[Right shift:] 
  \( y = f(x - c) \) shifts the graph \textbf{right} by \( c \) units.  
  If \( (x_0, y_0) \) is on the graph of \( y = f(x) \), then \( (x_0 + c, y_0) \) is on the graph of \( y = f(x - c) \).

    \item[Vertical stretch:] 
  \( y = a f(x) \) \textbf{stretches} the graph vertically by a factor of \( |a| \) if \( |a| > 1 \).  
  If \( (x_0, y_0) \) lies on the graph of \( y = f(x) \), then \( (x_0, a y_0) \) lies on the graph of \( y = a f(x) \).

  \item[Vertical compression:] 
  \( y = a f(x) \) \textbf{compresses} the graph vertically if \( 0 < |a| < 1 \).  
  If \( (x_0, y_0) \) lies on the graph of \( y = f(x) \), then \( (x_0, a y_0) \) lies on the graph of \( y = a f(x) \).

  \item[Horizontal compression:] 
  \( y = f(ax) \) \textbf{compresses} the graph horizontally by a factor of \( \frac{1}{|a|} \) when \( |a| > 1 \).  
  If \( (x_0, y_0) \) lies on the graph of \( y = f(x) \), then \( \left( \frac{x_0}{a}, y_0 \right) \) lies on the graph of \( y = f(ax) \).

  \item[Horizontal stretch:] 
  \( y = f(ax) \) \textbf{stretches} the graph horizontally by a factor of \( \frac{1}{|a|} \) when \( 0 < |a| < 1 \).  
  If \( (x_0, y_0) \) lies on the graph of \( y = f(x) \), then \( \left( \frac{x_0}{a}, y_0 \right) \) lies on the graph of \( y = f(ax) \).

  \item[Reflection across \( x \)-axis:] 
  \( y = -f(x) \) reflects the graph across the \( x \)-axis.  
  If \( (x_0, y_0) \) lies on the graph of \( y = f(x) \), then \( (x_0, -y_0) \) lies on the graph of \( y = -f(x) \).

  \item[Reflection across \( y \)-axis:] 
  \( y = f(-x) \) reflects the graph across the \( y \)-axis.  
  If \( (x_0, y_0) \) lies on the graph of \( y = f(x) \), then \( (-x_0, y_0) \) lies on the graph of \( y = f(-x) \).


\end{description}
\end{dfn}

\begin{dfn}
Because functions output real numbers, we can combine them using familiar algebraic operations. Given two functions \( f(x) \) and \( g(x) \), we define:

\begin{description}[style=nextline, leftmargin=3.2cm]

  \item[Addition:] 
  \( (f + g)(x) := f(x) + g(x) \)

  \item[Subtraction:] 
  \( (f - g)(x) := f(x) - g(x) \)

  \item[Multiplication:] 
  \( (f \cdot g)(x) := f(x) \cdot g(x) \)

  \item[Division:] 
  \( \left( \frac{f}{g} \right)(x) := \frac{f(x)}{g(x)} \quad \) (as long as \( g(x) \ne 0 \))

\end{description}
\end{dfn}

\vspace{1em}

Functions also support a new operation that is unique and fundamental to calculus: \vocab{function composition}.

\begin{dfn}[Function Composition]
Given two functions \( f(x) \) and \( g(x) \), the \textbf{composition} of \( f \) with \( g \), written as \( (f \circ g)(x) \), means:
\[
(f \circ g)(x) := f(g(x))
\]
In words, we first evaluate \( g(x) \), and then plug that result into \( f \).

\vspace{0.5em}
\textbf{Important:} Composition is generally \emph{not commutative}, meaning \( f(g(x)) \ne g(f(x)) \) in general.
\end{dfn}


\begin{exercise}
  Suppose that \( g(x) =x-1 \) and that \( f(x) = g(x^{2}-1) \). Find all \( x \) such that 
  \[ \left( f \circ g \right)(x) = \left( g \circ f  \right)(x) \]
\end{exercise}
\begin{solution}
    We can explicitly find \( f(x) \) 
    \begin{align*}
          f(x) &= g(x^{2}-1) \\
          &= \left[ x^{2}-1 \right]-1 \\
          f(x) &= x^{2} -2
    \end{align*}
    Now to find \( \left( f \circ g \right)(x) \)
    \begin{align*}
      \left( f \circ g \right)(x)  &= \left( x-1 \right)^{2} -2\\
      &= x^{2}-2x+1 -2 \\
      \left( f \circ g \right)(x) &= x^{2}-2x-1
    \end{align*}
    For \( \left( g \circ f \right)(x) \) 
    \begin{align*}
       \left( g \circ f \right)(x)  &= \left( x^{2}-2 \right)-1 
        \left( g \circ f \right)(x) &= x^{2}-3
    \end{align*}
    Finally, 
    \begin{align*}
       \left( g \circ f \right)(x)  &=  \left( f \circ g \right)(x) \\
       x^{2}-3 &= x^{2}-2x-1 \\
       -3 &=-2x -1 \\
       -2 &= -2x \\
       1 &= x
    \end{align*}
    
\end{solution}

\begin{example}
  Let \( f(x)= \dfrac{b}{x-a}+a \). Find \( (f \circ f)(x) \), that is, compute \( f(f(x)) \).
  
  \vspace{1em}
  First, recall that composition means we evaluate:
  \[
  (f \circ f)(x) = f(f(x))
  \]

  Step 1: Start with the inner function:
  \[
  f(x) = \frac{b}{x - a} + a
  \]

  Step 2: Plug this into the outer \( f \). That is, replace every \( x \) in \( f(x) \) with \( f(x) \):
  \[
  f(f(x)) = \frac{b}{\left( \frac{b}{x - a} + a \right) - a} + a
  \]

  \vspace{0.5em}
  Step 3: Simplify the expression inside the denominator:
  \[
  f(f(x)) = \frac{b}{\frac{b}{x - a}} + a
  \]

  Step 4: Invert the inner fraction:
  \[
  \frac{b}{\frac{b}{x - a}} = x - a
  \]

  So:
  \[
  f(f(x)) = (x - a) + a = x
  \]

  \vspace{0.5em}
 \( (f \circ f)(x) = \boxed{x} \)
\end{example}

In the above example, we can say \( f(x) \) is its own \vocab{inverse}. 
\begin{dfn}
Given a function \( f(x) \), another function \( g(y) \) is called the \vocab{inverse function} of \( f(x) \) if each undoes the effect of the other:
\[
g(f(x)) = x \quad \text{for all } x \text{ in the domain of } f, 
\quad \text{and} \quad
f(g(y)) = y \quad \text{for all } y \text{ in the domain of } g.
\]
We often write \( g = f^{-1} \).
\end{dfn}

A function is invertible if and only if it passes the \textbf{Horizontal Line Test}, or equivalently, if it is \textbf{strictly monotonic} (either strictly increasing or strictly decreasing) on its domain.

\begin{example}
  Let us test (without trying to find an explicit inverse) if the function 
  \[ f(x) = x^{3}+ x- 4 \]
  has an inverse. \\ 
  Let \( h \neq 0 \), if \( f(x) \) is invertible then 
  \( f(x+ h) - f(x) \) cannot be zero. 
  \begin{align*}
    f(x + h) - f(x) &= \left[ \left( x+ h \right)^{3} + \left( x +h \right) -4\right] - \left[ x^{3} +x -4 \right] \\
    &= \left[ \left( x+h  \right)^{3} - x^{3} \right] + \left[ \left( x+h \right) -x \right] + \left[ -4 - \left( -4 \right) \right] \\
    &= \left( x+h -x \right)\left( \left( x+h \right)^{2}  +x \left( x+h \right) +x^{2 }\right) +h \tag{Applying difference of cubes.} \\
    &= h \left( 3x^{2} + 2hx + h^{2} \right) +h
  \end{align*}
  It can be easily verified that when \( h \neq 0 \), the quadratic \( 3x^{2} +2hx + h^{2} \ge 0 \) for every \( x \in \mathbb{R} \) (Setting \(a = 3, b =2h, c =h^{2}\) gives you a negative discriminant.) Since \( f(x+h) - f(x) \) is the sum of positive terms (if \( h \) is positive) or the sum of negatives terms (if \( h  \) is negative), \( f(x+h)- f(x) \neq 0 \) every \( x \in \mathbb{R} \) and \( h \neq 0 \). As such, \( f^{-1} (x) \) exists. \\ 
  Now suppose we wanted to find \( f^{-1} (2) \). Again, we don't need to calculate an explicit inverse. If \( f^{-1}(2)=a \) then, by definition, \( f(a) =-2 \) So 
  \begin{align*}
    f(a) &= -2 \\
    a^{3} + a -4 &= -2 \\
    a^{3} + a =2 
  \end{align*}
We can see that \( f(1) =-2 \) so \(  f^{-1} (-2) = 1 \).
\end{example}

\begin{exercise}
  Suppose that \( f: \left( -1,1  \right) \to \mathbb{R} \) be defined by 
  \[ f(x) = \frac{x}{1-x^{2}} .\] 
  Find \( f^{-1} \).
\end{exercise}
\begin{solution}
    We apply the algorithm of exchanging \( y \) and \( x \)
    \[ x = \frac{y}{1-y^{2}} .\]
    
    

    
\end{solution}
