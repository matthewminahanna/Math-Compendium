\begin{theorem}[Cauchy--Schwarz Inequality]\label{thrm: The Cauchy-Schwarz Inequality}
  Suppose \( \mathbf{V} \) is a real inner product space. For all \( \vb{v}, \vb{w} \in \mathbf{V} \), we have:
  \[
    \abs{\left< \vb{v}, \vb{w} \right>} \le \left< \vb{v}, \vb{v} \right>^{1/2} \left< \vb{w}, \vb{w} \right>^{1/2}.
  \]
\end{theorem}
\begin{proof}
  Consider the vector
  \[
    \vb{u} = \frac{\vb{v}}{\left< \vb{v}, \vb{v} \right>^{1/2}} - \frac{\vb{w}}{\left< \vb{w}, \vb{w} \right>^{1/2}}.
  \]
  Since inner products are nonnegative on real inner product spaces,
  \[
    \left< \vb{u}, \vb{u} \right> \ge 0.
  \]
  Expanding \( \left< \vb{u}, \vb{u} \right> \) gives:
  \begin{align*}
    \left< \vb{u}, \vb{u} \right>
    &= \left< \frac{\vb{v}}{\left< \vb{v}, \vb{v} \right>^{1/2}}, \frac{\vb{v}}{\left< \vb{v}, \vb{v} \right>^{1/2}} \right>
     - 2 \left< \frac{\vb{v}}{\left< \vb{v}, \vb{v} \right>^{1/2}}, \frac{\vb{w}}{\left< \vb{w}, \vb{w} \right>^{1/2}} \right>
     + \left< \frac{\vb{w}}{\left< \vb{w}, \vb{w} \right>^{1/2}}, \frac{\vb{w}}{\left< \vb{w}, \vb{w} \right>^{1/2}} \right> \\
    &= \frac{\left< \vb{v}, \vb{v} \right>}{\left< \vb{v}, \vb{v} \right>} 
     - 2 \cdot \frac{\left< \vb{v}, \vb{w} \right>}{\left< \vb{v}, \vb{v} \right>^{1/2} \left< \vb{w}, \vb{w} \right>^{1/2}} 
     + \frac{\left< \vb{w}, \vb{w} \right>}{\left< \vb{w}, \vb{w} \right>} \\
    &= 1 - 2 \cdot \frac{\left< \vb{v}, \vb{w} \right>}{\left< \vb{v}, \vb{v} \right>^{1/2} \left< \vb{w}, \vb{w} \right>^{1/2}} + 1 \\
    &= 2 - 2 \cdot \frac{\left< \vb{v}, \vb{w} \right>}{\left< \vb{v}, \vb{v} \right>^{1/2} \left< \vb{w}, \vb{w} \right>^{1/2}}.
  \end{align*}
  Since this is \( \ge 0 \), we conclude:
  \[
    2 - 2 \cdot \frac{\left< \vb{v}, \vb{w} \right>}{\left< \vb{v}, \vb{v} \right>^{1/2} \left< \vb{w}, \vb{w} \right>^{1/2}} \ge 0,
  \]
  which implies:
  \[
    \frac{\left< \vb{v}, \vb{w} \right>}{\left< \vb{v}, \vb{v} \right>^{1/2} \left< \vb{w}, \vb{w} \right>^{1/2}} \le 1.
  \]
  If \( \left< \vb{v}, \vb{w} \right> < 0 \), the same argument applied to \( -\vb{w} \) yields the inequality
  \[
    \abs{\left< \vb{v}, \vb{w} \right>} \le \left< \vb{v}, \vb{v} \right>^{1/2} \left< \vb{w}, \vb{w} \right>^{1/2}.
  \]
\end{proof}

\begin{example}
  In \( \mathbb{R}^n \) with the standard dot product
  \[
    \left< \vb{v}, \vb{w} \right> := \sum_{j=1}^n v_j w_j,
  \]
  the Cauchy--Schwarz inequality becomes
  \[
    \left| \sum_{j=1}^n v_j w_j \right| \le 
    \left( \sum_{j=1}^n v_j^2 \right)^{1/2} 
    \left( \sum_{j=1}^n w_j^2 \right)^{1/2}.
  \]
\end{example}

\begin{theorem}[Cauchy--Schwarz Inequality (Complex Case)]\label{thrm:Complex Cauchy-Schwarz}
  Let \( \mathbf{V} \) be a complex inner product space. For all \( \vb{v}, \vb{w} \in \mathbf{V} \), we have:
  \[
    \abs{\left< \vb{v}, \vb{w} \right>} \le \left< \vb{v}, \vb{v} \right>^{1/2} \left< \vb{w}, \vb{w} \right>^{1/2}.
  \]
\end{theorem}
\begin{proof}
  Let \( \vb{v}, \vb{w} \in \mathbf{V} \) be non-zero vectors and define for \( \lambda \in \mathbb{C} \) the function:
  \[
    f(\lambda) = \left< \vb{v} - \lambda \vb{w}, \vb{v} - \lambda \vb{w} \right>.
  \]
Note that \( f \left(  \lambda \right) \ge 0 \) by positive semi-definiteness. We can expand \( f \left( \lambda\right)\) as 
\begin{align*}
  f \left(  \lambda \right) &= \left< \vb{v} - \lambda \vb{w}, \vb{v}- \lambda \vb{w} \right> \\
&=  \left< \vb{v}, \vb{v}- \lambda \vb{w} \right> + \left< -\lambda \vb{w}, \vb{v}- \lambda \vb{w} \right>\\
&=  \left< \vb{v}, \vb{v}- \lambda \vb{w} \right> - \lambda \left< \vb{w}, \vb{v}- \lambda \vb{w} \right>\\
&= \left< \vb{v}, \vb{v} \right> + \left< \vb{v}, - \lambda \vb{w} \right> - \lambda \left< \vb{w}, \vb{v} \right> - \lambda \left< \vb{w}, - \lambda \vb{w} \right> \\
&= \left< \vb{v}, \vb{v} \right> - \overline{\lambda} \left< \vb{v}, \vb{w} \right> - \lambda \left< \vb{w}, \vb{v} \right> + \abs{\lambda}^{2}  \left< \vb{w}, \vb{w} \right>
\end{align*}

Now choose
  \[
    \lambda = \frac{\left< \vb{v}, \vb{w} \right>}{\left< \vb{w}, \vb{w} \right>}.
  \]
  Substituting into the expanded \( f(\lambda) \), we have
  \begin{align*}
    f \left( \frac{\left< \vb{v}, \vb{w} \right>}{\left< \vb{w}, \vb{w} \right>}\right) &=  \left< \vb{v}, \vb{v} \right> - \overline{ \frac{\left< \vb{v}, \vb{w} \right>}{\left< \vb{w}, \vb{w} \right>}} \left< \vb{v}, \vb{w} \right> -  \frac{\left< \vb{v}, \vb{w} \right>}{\left< \vb{w}, \vb{w} \right>} \left< \vb{w}, \vb{v} \right> + \abs{ \frac{\left< \vb{v}, \vb{w} \right>}{\left< \vb{w}, \vb{w} \right>}}^{2}  \left< \vb{w}, \vb{w} \right> \\
    &= \left< \vb{v}, \vb{v} \right> - \overline{ \frac{\left< \vb{v}, \vb{w} \right>}{\left< \vb{w}, \vb{w} \right>}} \left< \vb{v}, \vb{w} \right> -  \frac{\left< \vb{v}, \vb{w} \right>}{\left< \vb{w}, \vb{w} \right>} \overline{ \left< \vb{v}, \vb{w} \right>} + \abs{ \frac{\left< \vb{v}, \vb{w} \right>}{\left< \vb{w}, \vb{w} \right>}}^{2}  \left< \vb{w}, \vb{w} \right> \\
    &=\left< \vb{v}, \vb{v} \right> -  2 \frac{\overline{\left< \vb{v}, \vb{w} \right> } \left< \vb{v}, \vb{w} \right>  }{\left< \vb{w}, \vb{w} \right>} + \abs{ \frac{\left< \vb{v}, \vb{w} \right>}{\left< \vb{w}, \vb{w} \right>}}^{2}  \left< \vb{w}, \vb{w} \right> \\
    &=\left< \vb{v}, \vb{v} \right> -  2 \frac{\overline{\left< \vb{v}, \vb{w} \right> } \left< \vb{v}, \vb{w} \right>  }{\left< \vb{w}, \vb{w} \right>} + \frac{\overline{\left< \vb{v}, \vb{w} \right> } \left< \vb{v}, \vb{w} \right>  }{\left< \vb{w}, \vb{w}\right> \left< \vb{w}, \vb{w} \right>} \left< \vb{w}, \vb{w} \right>  \\
    &=\left< \vb{v}, \vb{v} \right> -  2 \frac{\overline{\left< \vb{v}, \vb{w} \right> } \left< \vb{v}, \vb{w} \right>  }{\left< \vb{w}, \vb{w} \right>} + \frac{\overline{\left< \vb{v}, \vb{w} \right> } \left< \vb{v}, \vb{w} \right>  }{\left< \vb{w}, \vb{w} \right>} \\
    & =\left< \vb{v}, \vb{v} \right> -  \frac{\overline{\left< \vb{v}, \vb{w} \right> } \left< \vb{v}, \vb{w} \right>  }{\left< \vb{w}, \vb{w} \right>}\\
    &=\left< \vb{v}, \vb{v} \right> -  \frac{  \abs{\left< \vb{v}, \vb{w} \right>}^2  }{\left< \vb{w}, \vb{w} \right>}
  \end{align*}
  Finally we can apply the non-negativity of \( f \),
  \begin{align*}
   0 &\le \left< \vb{v}, \vb{v} \right> -  \frac{  \abs{\left< \vb{v}, \vb{w} \right>}^2  }{\left< \vb{w}, \vb{w} \right>} \\
    \frac{  \abs{\left< \vb{v}, \vb{w} \right>}^2  }{\left< \vb{w}, \vb{w} \right>} & \le  \left< \vb{v}, \vb{v} \right> \\
    \abs{\left< \vb{v}, \vb{w} \right>}^2 & \le \left< \vb{v}, \vb{v} \right> \left< \vb{w}, \vb{w} \right>
  \end{align*}
  
  and taking square roots
  \[
  \boxed{  \abs{\left< \vb{v}, \vb{w} \right>} \le \left< \vb{v}, \vb{v} \right>^{1/2} \left< \vb{w}, \vb{w} \right>^{1/2}}.
  \]
\end{proof}

\begin{example}
  Consider the complex inner product on \( \mathbb{C}^n \) defined by:
  \[
    \left< \vb{v}, \vb{w} \right> := \sum_{j=1}^n v_j \overline{w_j}.
  \]
  Then the Cauchy--Schwarz inequality becomes:
  \[
    \left| \sum_{j=1}^n v_j \overline{w_j} \right| 
    \le \left( \sum_{j=1}^n \abs{v_j}^2 \right)^{1/2}
         \left( \sum_{j=1}^n \abs{w_j}^2 \right)^{1/2}.
  \]
\end{example}

\begin{example}
  Let \( f, g \in L^2(\gamma) \), where \( \gamma \subset \mathbb{C} \) is a piecewise smooth contour. Define the inner product:
  \[
    \left< f, g \right> := \int_{\gamma} f(z) \, \overline{g(z)} \, \abs{\dd{z}}.
  \]
  Then the Cauchy--Schwarz inequality gives:
  \[
    \left| \int_{\gamma} f(z) \, \overline{g(z)} \, \abs{\dd{z}} \right|
    \le \left( \int_{\gamma} \abs{f(z)}^2 \, \abs{\dd{z}} \right)^{1/2}
         \left( \int_{\gamma} \abs{g(z)}^2 \, \abs{\dd{z}} \right)^{1/2}.
  \]
\end{example}

\begin{exercise}
    Show that 
    \[
    \sum_{j=1}^{n } a_{j} \le \sqrt{n \sum_{j=1}^{n} \left( a_{j } \right)^{2} }.
    \]
\end{exercise}
\begin{solution}
    Apply the Cauchy–Schwarz inequality to the vectors \( \vec{a} = (a_1, a_2, \dots, a_n) \) and \( \vec{b} = (1, 1, \dots, 1) \in \mathbb{R}^n \). Then:
    \[
    \left( \sum_{j=1}^n a_j \cdot 1 \right)^2 
    \le \left( \sum_{j=1}^n a_j^2 \right) \left( \sum_{j=1}^n 1^2 \right)
    = \left( \sum_{j=1}^n a_j^2 \right) \cdot n.
    \]
    Taking square roots of both sides gives:
    \[
    \sum_{j=1}^n a_j 
    \le \sqrt{n \sum_{j=1}^n a_j^2}.
    \]
    This proves the inequality.
\end{solution}



