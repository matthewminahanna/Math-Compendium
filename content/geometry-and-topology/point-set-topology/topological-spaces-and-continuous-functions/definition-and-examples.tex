\begin{dfn}
    Let \( X \) be any set. A \vocab{topology} on \( X \) is a collection \( \mathscr{T} \subseteq \mathcal{P}(X) \) such that the following criteria hold:
    \begin{enumerate}[label=\textbf{\roman*)}]
        \item \( \varnothing, X \in \mathscr{T} \).
        \item For any collection of sets \( \left\{ U_{\alpha} \right\}_{\alpha \in \mathcal{A}} \subseteq \mathscr{T} \), we have \[ \bigcup_{\alpha \in \mathcal{A}} U_{\alpha} \in \mathscr{T}. \]
        \item For any finite collection of sets \( \left\{ U_{j} \right\}_{j=1}^n \subseteq \mathscr{T}\), we have \[ \bigcap_{j=1}^{n} U_{j} \in \mathscr{T}. \]
    \end{enumerate}
If a set \( U \subseteq X \) belongs to \( \mathscr{T} \), we call \( U \) an \vocab{open} set. If \( x \in U \) and \( U \) is open, we sometimes refer to \( U \) as an \vocab{open neighborhood} of \( x \).
\end{dfn}

The topology on the empty set is not interesting, so from now on we will assume that \( X \) is non-empty.

\begin{dfn}
    Let \( X \) be a set and let \( \mathscr{T} \) and \( \mathscr{T}' \) be two topologies on \( X \).  If \( \mathscr{T} \subseteq \mathscr{T}' \), we say that \( \mathscr{T}' \) is \vocab{finer} than \( \mathscr{T} \), or equivalently that \( \mathscr{T} \) is \vocab{coarser} than \( \mathscr{T}' \). If \( \mathscr{T} \subset \mathscr{T}' \), we say that \( \mathscr{T}' \) is \vocab{strictly finer} than \( \mathscr{T} \), or that \( \mathscr{T} \) is \vocab{strictly coarser} than \( \mathscr{T}' \). We call the topologies \( \mathscr{T} \) and \( \mathscr{T}' \) \vocab{comparable} if one is finer than the other.
\end{dfn}

\begin{example}
    For any set \( X \), there are two obvious topologies. The \vocab{indiscrete} topology which is just 
    \[ \mathscr{T}_{\mathrm{indiscrete}} =\left\{ \varnothing, X \right\}\]
    and the \vocab{discrete} topology which is just 
    \[ \mathscr{T}_{\mathrm{discrete}} = \mathcal{P} \left( X \right).\]
\end{example}

\begin{exercise}
    Let \( X = \left\{ a,b,c \right\} \). What are all the possible topologies on \( X \)?
\end{exercise}
\begin{solution}
    We have the discrete and indiscrete topologies on \( X \). 
    \[ \mathscr{T}_{\mathrm{discrete}} = \left\{ \varnothing, \left\{ a \right\}, \left\{ b \right\}, \left\{ c \right\}, \left\{ a,b  \right\}, \left\{ a,c  \right\}, \left\{ b,c \right\}, \left\{ a,b,c \right\} \right\} \]
    \[ \mathscr{T}_{\mathrm{indiscrete}} = \left\{ \varnothing, \left\{ a,b,c \right\} \right\} \]
Then we have the topologies that augment the indiscrete topology with a singleton set 
\[ \mathscr{T}_{1} = \left\{ \varnothing, \left\{ a \right\}, \left\{ a ,b, c \right\} \right\},\  \mathscr{T}_{2} = \left\{ \varnothing, \left\{ b \right\}, \left\{ a ,b, c \right\} \right\}, \  \mathscr{T}_{3} = \left\{ \varnothing, \left\{ c \right\}, \left\{ a ,b, c \right\} \right\} \]
\[ \]
We can fill out the rest 
\[ \mathscr{T}_{4} = \left\{ \varnothing, \left\{ a \right\}, \left\{ a,b \right\}, \left\{ a ,b, c \right\} \right\}, \ \mathscr{T}_{5} = \left\{ \varnothing, \left\{ a \right\}, \left\{ a,c \right\},\left\{ a ,b, c \right\} \right\}, \  \mathscr{T}_{6} = \left\{ \varnothing, \left\{ a \right\}, \left\{ b,c \right\}, \left\{ a ,b, c \right\} \right\}\]
\[ \mathscr{T}_{7} = \left\{ \varnothing, \left\{ b \right\}, \left\{ a,b \right\}, \left\{ a ,b, c \right\} \right\}, \  \mathscr{T}_{8} = \left\{ \varnothing, \left\{ b \right\}, \left\{ a,c \right\},\left\{ a ,b, c \right\} \right\}, \ \mathscr{T}_{9} = \left\{ \varnothing, \left\{ b \right\}, \left\{ b,c \right\}, \left\{ a ,b, c \right\} \right\}  \]
\[ \mathscr{T}_{10} = \left\{ \varnothing, \left\{ c \right\}, \left\{ a,b \right\}, \left\{ a ,b, c \right\} \right\}, \ \mathscr{T}_{11} = \left\{ \varnothing, \left\{ c \right\}, \left\{ a,c \right\},\left\{ a ,b, c \right\} \right\}, \  \mathscr{T}_{12} = \left\{ \varnothing, \left\{ c \right\}, \left\{ b,c \right\}, \left\{ a ,b, c \right\} \right\} \]


\[ \mathscr{T}_{13} = \left\{ \varnothing, \left\{ a \right\}, \left\{ b \right\}, \left\{ a,b \right\}, \left\{ a ,b, c \right\} \right\}, \ \mathscr{T}_{14} = \left\{ \varnothing, \left\{ a \right\}, \left\{ c \right\}, \left\{ a,c \right\}, \left\{ a ,b, c \right\} \right\}, \  \mathscr{T}_{15} = \left\{ \varnothing, \left\{ b  \right\}, \left\{ c \right\}, \left\{ b,c \right\}, \left\{ a ,b, c \right\} \right\}\]
\[ \mathscr{T}_{16} = \left\{ \varnothing, \left\{ a \right\}, \left\{a, b \right\}, \left\{ a,c \right\}, \left\{ a ,b, c \right\} \right\}, \ \mathscr{T}_{17} = \left\{ \varnothing, \left\{ b \right\}, \left\{ a,b \right\}, \left\{ b,c \right\}, \left\{ a ,b, c \right\} \right\}, \  \mathscr{T}_{18} = \left\{ \varnothing, \left\{ c  \right\}, \left\{ a,c \right\}, \left\{ b,c \right\}, \left\{ a ,b, c \right\} \right\}\]

\[ \mathscr{T}_{19} = \left\{ \varnothing, \left\{ a \right\}, \left\{ b  \right\}, \left\{a, b \right\} ,\left\{ a,c \right\}, \left\{ a ,b, c \right\} \right\}, \ \mathscr{T}_{20} = \left\{ \varnothing, \left\{ a \right\}, \left\{ c  \right\}, \left\{a, b \right\} ,\left\{ a,c \right\}, \left\{ a ,b, c \right\} \right\}\]
\[ \mathscr{T}_{21} =  \left\{ \varnothing, \left\{ a \right\}, \left\{ b  \right\}, \left\{a, b \right\} ,\left\{ b,c \right\}, \left\{ a ,b, c \right\} \right\}, \ \mathscr{T}_{22} = \left\{ \varnothing, \left\{ a \right\}, \left\{ c  \right\}, \left\{a, c \right\}, \left\{ b,c \right\}, \left\{ a ,b, c \right\} \right\}\]
\[ \mathscr{T}_{23}= \left\{ \varnothing, \left\{ b \right\}, \left\{ c \right\}, \left\{ b,c \right\}, \left\{ a,b \right\}, \left\{ a,b,c \right\} \right\}, \mathscr{T}_{24}= \left\{ \varnothing, \left\{ b \right\}, \left\{ c \right\}, \left\{ b,c \right\}, \left\{ a,c \right\}, \left\{ a,b,c \right\} \right\} \]

\[ \mathscr{T}_{25} = \left\{ \varnothing, \left\{ a,b \right\}, \left\{ a,b,c \right\} \right\}, \  T_{26} = \left\{ \varnothing, \left\{ a,c \right\}, \left\{ a,b,c \right\} \right\}, \  \mathscr{T}_{27} = \left\{ \varnothing, \left\{ b,c \right\}, \left\{ a,b,c \right\} \right\}  \]


\end{solution}

\begin{exercise}
    Let \( X \) be a set and we define the co-finite topology \( \mathscr{T}_{\mathrm{cf}} \) as follows: \( U \) is open in \( \mathscr{T}_{\mathrm{cf}} \) if and only if \( X-U \) is finite or all of \( X \). Show that this is indeed a topology.
\end{exercise}
\begin{solution}
    Clearly \( \varnothing \) and \( X \) each belong to \( \mathscr{T}_{\mathrm{cf}} \), so we will jump right into verifying closure under arbitrary unions and finite intersections. \\
    Let \( U_{\alpha \in A} \in \mathscr{T}_{\mathrm{cf}} \). We want to show that 
    \[ \bigcup_{\alpha \in A} U_{\alpha} \in \mathscr{T}_{\mathrm{cf}} \]
    or that 
    \[ X - \left(  \bigcup_{\alpha \in A} U_{\alpha} \right) \]
    is finite. We can apply one of DeMorgan's laws to the above expression to get 
    \[ X - \left(  \bigcup_{\alpha \in A} U_{\alpha} \right) = \bigcap_{\alpha \in A } \left( X - U_{\alpha} \right)\]
Since each \( U_{\alpha} \) belongs to \( \mathscr{T}_{\mathrm{cf}} \), each \( X- U_{\alpha} \) is finite. Therefore \( \bigcap_{\alpha \in A } \left( X - U_{\alpha} \right) \) is certainly finite. This establishes that \( \bigcup_{\alpha \in A} U_{\alpha} \in \mathscr{T}_{\mathrm{cf}} \).\\
Now for finite intersections, suppose that \( \{U_{1}, \dots U_{n}\} \subseteq \mathscr{T}_{\mathrm{cf}} \). We want to show that 
\[ \bigcap_{j=1}^{n} U_{n} \in \mathscr{T}_{\mathrm{cf}}\] or 
\[ X - \left( \bigcap_{j=1}^{n} U_{j} \right) \]
is finite. Again, we apply one of DeMorgan's laws to get 
\[ X - \left( \bigcap_{j=1}^{n} U_{j} \right) = \bigcup_{j=1}^{n } \left( X- U_{j}\right).\]
Since each \( U_{j} \in \mathscr{T}_{\mathrm{cf}} \), each \( X-U_{j} \) is finite. This implies that \( \bigcup_{j=1}^{n }\left( X-U_{j} \right) \) is finite so \( \bigcap_{j=1}^{n }U_{j} \in \mathscr{T}_{\mathrm{cf}}\). This concludes the proof.
\end{solution}

\begin{dfn}
    Let \( X \) be a topological space. The \vocab{interior} of a set \( A \) of \( X \), denoted by \( \mathrm{Int} \left( A \right) \) is defined to be the set: 
    \[ \mathrm{Int} \left( A \right) = \bigcup \left\{ U \subseteq A \  \middle| \ U \text{ is an open set.} \right\} \]
    In other words, \( \mathrm{Int}(A) \) is the union of all open sets contained in \( A \). \\ 
    A point \( x \) of \( A \) is called an \vocab{interior point} of \( A \) it is a member of \( \mathrm{Int} \left( A \right) \).
\end{dfn}

\begin{dfn}
    Let \( X \) be a topological space. The \vocab{exterior} of a set \( A \), denoted by \( \mathrm{Ext}(A) \) is defined to be 
    \[ \mathrm{Ext} \left( A \right) = \mathrm{Int} \left( X-A \right) .\]
\end{dfn}

\begin{dfn}
    The \( X \) be a topological space. The \vocab{boundary} of a set \( A \), denoted as \( \partial \left( A \right) \), is the set 
    \[ \partial \left( A \right) = X - \left( \mathrm{Int} \left( A \right) \cup \mathrm{Ext} \left( A \right) \right).\]
\end{dfn}

\begin{lemma}\label{thm:open-set-equals-its-interior}
    A set \( U \) of a topological space \( X \) is open if and only if \( U = \mathrm{Int}(U) \).
\end{lemma}
\begin{proof}
    \( (\Rightarrow) \) Suppose that \( U \) is open. We want to show that \( U = \mathrm{Int}(U) \). \( \mathrm{Int}(U) \subseteq U \) is obvious by definition so we just need to show that \( U \subseteq \mathrm{Int}(U) \). Since \( \mathrm{Int}(U) \) is the union of all open subsets of \( U \) and \( U \) is open, it follows that \(U \subseteq  \mathrm{Int}(U) \). \\ 
    \( (\Leftarrow )\) Suppose \( \mathrm{Int}(U) = U \). Since \( \mathrm{Int}(U) \) is the union of all open subsets of \( U \), \( \mathrm{Int} \left( U \right) \) is open and hence, \( U \) is open.
\end{proof}

\begin{corollary}
    The exterior of a set \( A \) in a topological space is an open set. 
\end{corollary}










