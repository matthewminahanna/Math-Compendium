\subsection{Non-negative Integer Powers}

We will derive the derivative rules for functions of the form \( x^n \), where \( n = 0, 1, 2, \dots \).

The simplest case is \( y = 1 \), an example of a constant function:
\begin{align*}
f'(a) &= \lim_{h \to 0} \frac{f(a + h) - f(a)}{h} \\
&= \lim_{h \to 0} \frac{1 - 1}{h} \tag*{Since \( f(a) = f(a+h) = 1 \)} \\
&= \lim_{h \to 0} 0 \\
&= 0
\end{align*}

So,
\begin{equation}\label{eqn: Derivative of 1}
\boxed{\dv{x}\left[ 1 \right] = 0}
\end{equation}

\begin{theorem}[The Power Rule for Derivatives]
   For any \( n \in \mathbb{N} \), we have 
   \[ \dv{x}\left[ x^{n} \right] = n \cdot x^{n-1} \]
\end{theorem}
\begin{proof}
    We have taken care of the case \( n=0 \) earlier. 
    Now, if \( n \neq 0 \), consider \( f(x) = x^n \). Then:
\begin{align*}
f'(a) &= \lim_{h \to 0} \frac{f(a + h) - f(a)}{h} \\
&= \lim_{h \to 0} \frac{(a + h)^n - a^n}{h} \tag*{Substitute \( f(x) = x^n \)} \\
&= \lim_{h \to 0} \frac{\displaystyle \sum_{j = 0}^n {n \choose j} a^{n - j} h^j - a^n}{h} \tag*{Apply the binomial expansion to \( (a + h)^n \)} \\
&= \lim_{h \to 0} \frac{\displaystyle \sum_{j = 1}^n {n \choose j} a^{n - j} h^j}{h} \tag*{Cancel the \( a^n \) terms} \\
&= \lim_{h \to 0} \sum_{j = 1}^n {n \choose j} a^{n - j} h^{j - 1} \tag*{Factor out \( \frac{1}{h} \)} \\
&= \lim_{h \to 0} \left[ {n \choose 1} a^{n - 1} + \sum_{j = 2}^n {n \choose j} a^{n - j} h^{j - 1} \right] \tag*{Separate the \( j = 1 \) term} \\
&= {n \choose 1} a^{n - 1} + 0 \tag*{Higher-order terms vanish as \( h \to 0 \)} \\
&= n \cdot a^{n - 1} \tag*{Since \( {n \choose 1} = n \)}
\end{align*}

Therefore,
\begin{equation}\label{eqn: Derivative of power function}
\boxed{\dv{x}\left[ x^n \right] = n \cdot x^{n - 1}}
\end{equation}
\end{proof}




\subsection{Linearity Properties of the Derivative}

The derivative satisfies two important linearity properties:
\[
\dv{x} \left[ c \cdot f(x) \right] = c \cdot \dv{x} \left[ f(x) \right] \quad \text{and} \quad
\dv{x} \left[ f(x) + g(x) \right] = \dv{x} \left[ f(x) \right] + \dv{x} \left[ g(x) \right].
\]
For every \( c \in \mathbb{R} \) and differentiable functions \( f(x) \) and \( g(x) \).
As always in mathematics, we require proof.

\medskip



\begin{align*}
\dv{x} \left[ c \cdot f(x) \right] 
&= \lim_{h \to 0} \frac{c \cdot f(x + h) - c \cdot f(x)}{h} \tag*{Substitute into the definition of the derivative} \\
&= \lim_{h \to 0} c \cdot \frac{f(x + h) - f(x)}{h} \tag*{Factor out the constant \( c \)} \\
&= c \cdot \lim_{h \to 0} \frac{f(x + h) - f(x)}{h} \tag*{Limit of a constant times a function is the constant times the limit} \\
&= c \cdot \dv{x} \left[ f(x) \right].
\end{align*}

\medskip



\begin{align*}
\dv{x} \left[ f(x) + g(x) \right] 
&= \lim_{h \to 0} \frac{\left[ f(x + h) + g(x + h) \right] - \left[ f(x) + g(x) \right]}{h} \tag*{Apply the definition of the derivative to the sum} \\
&= \lim_{h \to 0} \frac{f(x + h) - f(x) + g(x + h) - g(x)}{h} \tag*{Group terms by function} \\
&= \lim_{h \to 0} \left( \frac{f(x + h) - f(x)}{h} + \frac{g(x + h) - g(x)}{h} \right) \tag*{Split the single fraction into two terms} \\
&= \lim_{h \to 0} \frac{f(x + h) - f(x)}{h} + \lim_{h \to 0} \frac{g(x + h) - g(x)}{h} \tag*{Limit of a sum is the sum of the limits (when both exist)} \\
&= \dv{x} \left[ f(x) \right] + \dv{x} \left[ g(x) \right].
\end{align*}

With this new rule in our tool belt, we are now capable of taking derivatives of any polynomial function. 

\subsection{Derivatives of Products and Quotients of Functions}

It is \textbf{not} true that
\[
\dv{x} \left[ f(x) \cdot g(x) \right] = \dv{x} \left[ f(x) \right] \cdot \dv{x} \left[ g(x) \right]
\quad \text{or} \quad
\dv{x} \left[ \frac{f(x)}{g(x)} \right] = \frac{\dv{x} \left[ f(x) \right]}{\dv{x} \left[ g(x) \right]}.
\]

In fact, such rules would contradict the linearity of the derivative. For example, take any constant \( c \neq 0 \). Then,
\begin{align*}
\dv{x} \left[ c \cdot f(x) \right] &= \dv{x} \left[ c \right] \cdot \dv{x} \left[ f(x) \right] \\
&= 0 \cdot \dv{x} \left[ f(x) \right] = 0,
\end{align*}
which contradicts the established rule:
\[
\dv{x} \left[ c \cdot f(x) \right] = c \cdot \dv{x} \left[ f(x) \right].
\]

Instead, the correct rules are derived as follows.

\begin{theorem}[The Product Rule for Derivatives]\label{thrm: Product Rule}
 Let \( f \) and \( g \) be differentiable functions. Then   \[ \dv{x} \left[ f(x)g(x) \right] = f'(x)g(x) + f(x)g'(x) \]
\end{theorem}
\begin{proof}
    \begin{align*}
\dv{x} \left[ f(x) \cdot g(x) \right] 
&= \lim_{h \to 0} \frac{f(x+h)g(x+h) - f(x)g(x)}{h} \tag*{Apply the definition of the derivative} \\
&= \lim_{h \to 0} \frac{f(x+h)g(x+h) - f(x)g(x+h) + f(x)g(x+h) - f(x)g(x)}{h} \tag*{Add and subtract \( f(x)g(x+h) \)} \\
&= \lim_{h \to 0} \left[
\frac{f(x+h) - f(x)}{h} \cdot g(x+h)
+ f(x) \cdot \frac{g(x+h) - g(x)}{h}
\right] \tag*{Group and factor each difference} \\
&= \dv{x} \left[ f(x) \right] \cdot g(x) + f(x) \cdot \dv{x} \left[ g(x) \right]. \tag*{Take the limit of each term separately}
\end{align*}
\end{proof}

\begin{exercise}
Prove from the product rule that 
\[ \dv{x} \left[ f(x)^n \right] = n \cdot f(x)^{n-1} \cdot f'(x). \]
\end{exercise}
\begin{solution}
We will prove this by induction. The case $n=1$ is trivial so, for practice, let us set the base case equal to $n=2$. So 
\begin{align*}
\dv{x} \left[ f(x)^2 \right] &= \dv{x} \left[ f(x) \cdot f(x) \right]\\
&= f'(x) \cdot f(x) + f(x) \cdot f'(x) \tag{By \nameref{thrm: Product Rule}}\\
&= 2 \cdot f'(x) \cdot f(x)
\end{align*}

Now for the inductive step, suppose that we have shown the result up to $n=k$. Then for the case $n=k+1$, we have 
\begin{align*}
\dv{x} \left[ f(x)^{k+1} \right] &= \dv{x} \left[ f(x) \cdot f(x)^k \right]\\
&= f'(x) \cdot f(x)^k + f(x) \cdot \left( f(x)^k \right)' \tag{By \nameref{thrm: Product Rule}}\\
&= f'(x) \cdot f(x)^k + f(x) \cdot \left( k \cdot f(x)^{k-1} \cdot f'(x) \right) \tag{Applying the inductive hypothesis}\\
&= f'(x) \cdot f(x)^k + k \cdot f(x)^k \cdot f'(x)\\
&= f'(x) \cdot f(x)^k \cdot (1 + k)\\
&= (k+1) \cdot f(x)^k \cdot f'(x)
\end{align*}

This completes the induction.
\end{solution}

\begin{exercise}
   Use the preceding exercise to extend the power rule to positive rational exponents.
\end{exercise}
\begin{solution}
    Suppose that we have \( x^{\frac{p}{q}} \), where \( p,q \in \mathbb{Z}_{>0} \). Then, by definition, 
    \[ \left(  x^{\frac{p }{q}} \right)^{q} = x^{p} .\] Now we differentiate both sides to get 
    \begin{align*}
      \dv{x} \left[ \left(  x^{\frac{p }{q}} \right)^{q} \right] &= \dv{x} \left[ x^{p} \right]\\
      q \cdot  \left( x^{\frac{p }{q}} \right)^{q-1} \cdot \dv{x} \left[   x^{\frac{p }{q}}\right] &= \dv{x} \left[ x^{p} \right] \tag{Applying the result of the previous exercise.} \\
       q \cdot  \left( x^{\frac{p }{q}} \right)^{q-1} \cdot \dv{x} \left[   x^{\frac{p }{q}}\right] &= p x^{p-1} \tag{By the Power Rule}\\
        q \cdot x^{ \frac{p \cdot \left( q-1 \right)}{q}}\cdot \dv{x} \left[   x^{\frac{p }{q}}\right] &= p x^{p-1} 
    \end{align*}
    Now we can solve for \(  \dv{x} \left[   x^{\frac{p }{q}}\right] \).
    \begin{align*}
       \dv{x} \left[   x^{\frac{p }{q}}\right] & = \frac{p }{q} \cdot x^{p-1} \cdot x^{ -\frac{p \cdot \left( q-1 \right)}{q}} \\
        \dv{x} \left[   x^{\frac{p }{q}}\right] & = \frac{p }{q} \cdot x^{ \frac{p}{q}-1} \tag{Verify this.}
    \end{align*}
    This completes the proof.
\end{solution}


\begin{exercise}
   Use the product rule to show that the power rule extends to negative rational exponents.
\end{exercise}
\begin{solution}
    Let \( r \) be a positive rational number. Then
    \[
        1 = x^{-r} \cdot x^r.
    \]
    Taking the derivative of both sides with respect to \( x \), we have
    \begin{align*}
        \dv{x}[1] &= \dv{x} \left[ x^{-r} \cdot x^r \right] \\
        0 &= \dv{x} \left[ x^{-r} \cdot x^r \right] \tag*{Derivative of a constant is zero} \\
        &= \dv{x} \left[ x^{-r} \right] \cdot x^r + x^{-r} \cdot \dv{x} \left[ x^r \right] \tag*{By the Product Rule} \\
        &= \dv{x} \left[ x^{-r} \right] \cdot x^r + r x^{-r} \cdot x^{r-1} \\
        &= \dv{x} \left[ x^{-r} \right] \cdot x^r + r x^{-1}.
    \end{align*}
    Solving for \( \dv{x} \left[ x^{-r} \right] \), we find:
    \begin{align*}
        \dv{x} \left[ x^{-r} \right] \cdot x^r &= -r x^{-1} \\
        \dv{x} \left[ x^{-r} \right] &= -r x^{-r-1}.
    \end{align*}
    Therefore the power rule
    \[
        \dv{x} \left[ x^s \right] = s x^{s - 1}
    \]
    holds for all negative rational numbers \( s = -r \).
\end{solution}




\begin{theorem}[The Quotient Rule for Derivatives]\label{them: Quotient Rule}
   Let \( f \) and \( g \) be differentiable functions, and suppose \( g(x) \neq 0 \). Then \[ \dv{x} \left[ \frac{f(x)}{g(x)} \right] = \frac{f'(x)g(x) - f(x)g'(x)}{g(x)^2} \]
\end{theorem}
\begin{proof}
   \begin{align*}
\dv{x} \left[ \frac{f(x)}{g(x)} \right] 
&= \lim_{h \to 0} \frac{\dfrac{f(x+h)}{g(x+h)} - \dfrac{f(x)}{g(x)}}{h} \\
&= \lim_{h \to 0} \frac{\displaystyle\frac{f(x+h)g(x) - f(x)g(x+h) }{g(x+h)g(x)}}{h  }  \\\\
&= \lim_{h \to 0} \frac{\displaystyle\frac{f(x+h)g(x) -f(x)g(x) - f(x)g(x+h) + f(x)g(x)}{g(x+h)g(x)}}{h  }  \tag*{Add and subtract \( f(x)g(x) \) in the numerator} \\\\
&=\lim_{h \to 0} \frac{\displaystyle\frac{f(x+h)g(x) -f(x)g(x) - f(x)g(x+h) + f(x)g(x)}{h}}{g(x+h)g(x) } \\\\
&=\lim_{h \to 0} \frac{\displaystyle\frac{f(x+h)g(x) -f(x)g(x) }{h} - \frac{f(x)g(x+h) -f(x)g(x)}{h}}{g(x+h)g(x) } \tag*{Seperating the fractions} \\\\
&= \frac{ \left[ \displaystyle\lim_{h \to 0}\frac{f(x+h)g(x) -f(x)g(x) }{h}  \right]-  \left[ \displaystyle\lim_{h \to 0} \frac{f(x)g(x+h) -f(x)g(x)}{h} \right]}{ \displaystyle\lim_{h \to 0}g(x+h)g(x) } \tag*{Moving the limit inside.} \\\\
&= \frac{ \left[ \displaystyle\lim_{h \to 0}\frac{f(x+h) -f(x) }{h}  \right] g(x)-  f(x)\left[\displaystyle \lim_{h \to 0} \frac{g(x+h) -g(x)}{h} \right]}{ \displaystyle\lim_{h \to 0}g(x+h)g(x) }  \\\\
&= \frac{\displaystyle\dv{x} \left[ f(x) \right]g(x) - f(x)  \dv{x} \left[ g(x) \right]}{g(x)^2}. \tag*{Take the limit and simplify the denominator}
\end{align*} 
\end{proof}

\begin{exercise}
    Provide an alternate proof of the quotient rule using the product rule. \\
    \textbf{Hint:} If \( h(x) = \frac{f(x )}{g(x)} \), then we may write \( h(x) g(x)=f(x) \).
\end{exercise}
\begin{solution}
    Since \( h(x) g(x) = f(x) \), we have 
    \[ \dv{x} \left[ h(x) g(x) \right] = \dv{x} \left[ f(x) \right] .\]
    Applying the product rule, we have 
    \[ h'(x)g(x) + h(x)g'(x) = f'(x).\]
    Multiply both sides by \( g(x) \) to get 
    \[ h'(x) \left( g(x)  \right)^{2} + \left[ h(x)g(x) \right] g'(x) = f'(x) g(x).\]
    So 
     \[ h'(x) \left( g(x)  \right)^{2} + \left[ f(x)\right] g'(x) = f'(x)g(x) .\]
     Now solving for \( h'(x) \), we have 
     \[ \boxed{h'(x) = \frac{f'(x) g(x) - f(x)g'(x)}{\left( g(x) \right)^{2}} } \]
\end{solution}


\subsection{Chain Rule}

We will state the chain rule here without proof, as it is fairly more involved than the proofs of the other rules. If you are curious about the proof, check out \Cref{thm: chain rule}.\\
\textbf{The Chain Rule} If \( f \) and \( g \) are functions with \( f'(x) \) and \( g'(f(x)) \) both existing, then 
\[ \boxed{\left( g \circ f\right)'(x) = g'\left( f(x) \right) \cdot f'(x)}. \]

\begin{example}
    Verify the chain rule for the function \( h(x) = \left( x^{2}+4 \right)^{3} .\)\\
    We can take the derivative of this function first without the chain rule. However, we have to expand it using the binomial theorem.
    \begin{align*}
        \left( x^{2}+4 \right)^{3} &= \sum_{j=0}^{3} {3 \choose j} \left( x^{2} \right)^{3-j} \cdot 4^{j}\\
        &= {3 \choose 0}(x^2)^3 \cdot 4^0 + {3 \choose 1}(x^2)^2 \cdot 4^1 + {3 \choose 2}(x^2)^1 \cdot 4^2 + {3 \choose 3}(x^2)^0 \cdot 4^3\\
        &= x^{6} +12 x^{4} + 48 x^{2} +64
    \end{align*}
    Taking the derivative we get 
    \[ h'(x) = 6x^{5} + 48 x^{3} + 96x . \]
    
    If we instead consider \( h(x) = g\left( f(x) \right) \), where \( g(x) = x^{3} \) and \( f(x)= x^{2}+4 \), then the chain rule tells us 
    \begin{align*}
        h'(x) &= g'(f(x)) \cdot f'(x)\\
        &= 3 \left( x^{2}+4 \right)^{2} \cdot 2x\\
        &= 6x\left( x^{2}+4 \right)^{2}
    \end{align*}
    Notice how much simpler the chain rule approach is, we didn't need to expand a cubic!
    
    Let's verify that both expressions are equivalent by expanding the chain rule result:
    \begin{align*}
        6x\left( x^{2}+4 \right)^{2} &= 6x\left( x^{4} + 8x^{2} + 16 \right)\\
        &= 6x^{5} + 48x^{3} + 96x
    \end{align*}
    Indeed, both methods give the same answer.
\end{example}


\subsection{Derivatives of Trigonometric Functions}

\begin{theorem}
    \[
    \dv{x} \left[ \sin(x) \right] = \cos(x).
    \]
\end{theorem}
\begin{proof}
    \begin{align*}
        \dv{x} \left[ \sin(x) \right]
        &= \lim_{h \to 0} \frac{\sin(x + h) - \sin(x)}{h}  \\
        &= \lim_{h \to 0} \frac{\sin(x)\cos(h) + \cos(x)\sin(h) - \sin(x)}{h} \\
        &= \lim_{h \to 0} \left[ \cos(x) \cdot \frac{\sin(h)}{h} + \sin(x) \cdot \frac{\cos(h) - 1}{h} \right] 
    \end{align*}

    Now we apply known trigonometric limits:
    \begin{align*}
        \dv{x} \left[ \sin(x) \right]
        &= \cos(x) \cdot \lim_{h \to 0} \frac{\sin(h)}{h}
           + \sin(x) \cdot \lim_{h \to 0} \frac{\cos(h) - 1}{h} \\
        &= \cos(x) \cdot 1 + \sin(x) \cdot 0 \tag*{\( \boxed{\lim_{h \to 0} \frac{\sin(h)}{h} = 1 , \lim_{h \to 0} \frac{\cos(h) - 1}{h} = 0} \)} \\
        &= \cos(x)
    \end{align*}

    Therefore,
    \[
    \dv{x} \left[ \sin(x) \right] = \cos(x).
    \]
\end{proof}

\begin{theorem}
    \[
    \dv{x} \left[ \cos{ \left( x \right) } \right] = - \sin{ \left( x \right) }.
    \]
\end{theorem}
\begin{proof}
    \begin{align*}
        \dv{x} \left[ \cos{ \left( x \right) } \right] &= \lim_{h \to 0} \frac{ \cos{ \left( x +h  \right) } - \cos{ \left( x \right) }}{h} \\
        &=  \lim_{h \to 0}\frac{ \cos{ \left( x  \right) } \cos{ \left( h  \right) } - \sin{ \left( x  \right) } \sin{ \left( h  \right) } - \cos{ \left( x  \right) }}{h} \\
        &=\lim_{h \to 0} \frac{ \cos{ \left( x  \right) } \cos{ \left( h  \right) } - \cos{ \left( x \right) }}{h} - \lim_{h \to 0} \frac{ \sin{ \left( x  \right) } \sin{ \left( h  \right) }}{h} \\
        &= \cos{ \left( x  \right) } \lim_{h \to 0} \frac{ \cos{ \left( h  \right) } -1}{h} - \sin{ \left( x \right) } \lim_{ h \to 0} \frac{ \sin{ \left( h \right) }}{h}\\
        &= \cos{ \left( x \right) } \cdot 0 - \sin{ \left( x \right) } \cdot 1
    \end{align*}
    so 
    \[ \boxed{ \dv{x} \left[ \cos{ \left( x \right) } \right] = - \sin{ \left( x \right) }.} \]
\end{proof}

\begin{theorem}
    \[ \dv{x} \left[ \tan{ \left( x \right) } \right] = \sec^{2} \left( x \right). \]
\end{theorem}
\begin{proof}
    \begin{align*}
        \dv{x} \left[ \tan{ \left( x \right) } \right] &= \lim_{h \to 0} \frac{ \tan{ \left( x +h  \right) } - \tan{ \left( x \right) }}{h}\\
        &= \lim_{h \to 0} \frac{ \displaystyle \frac{ \tan{ \left( x \right) } + \tan{ \left( h \right) }}{1 - \tan{ \left( x  \right) } \tan{ \left( h \right) }} - \tan{ \left( x \right) }}{h} \\
        &= \lim_{h \to 0} \frac{ \displaystyle \frac{ \tan{ \left( x \right) } + \tan{ \left( h \right) }}{1 - \tan{ \left( x  \right) } \tan{ \left( h \right) }} - \frac{ \tan{ \left( x  \right) } \left( 1 - \tan{ \left( x \right) } \tan{ \left( h \right) } \right)}{1- \tan{ \left( x \right) } \tan{ \left( h \right) }}}{h}\\
        &= \lim_{h \to 0} \frac{ \tan{ \left( x  \right) } + \tan{ \left( h  \right) } - \tan{ \left( x \right) } + \tan^{2}{ \left( x \right) } \tan{ \left( h \right) }}{h \left( 1 - \tan{ \left( x \right) } \tan{ \left( h \right) } \right)} \\
        &= \lim_{h \to 0} \frac{ \tan{ \left( h \right) } + \tan^{2}{ \left( x \right) } \tan{ \left( h \right) }}{h \left( 1 - \tan{ \left( x \right) } \tan{ \left( h \right) } \right)} \\
        &= \lim_{h \to 0} \frac{ \tan{ \left( h  \right) } \left( 1 + \tan^{2}{ \left( x  \right) } \right)}{h \left( 1 - \tan{ \left( x \right) } \tan{ \left( h \right) } \right)} \\
        &= \left( 1 + \tan^{2}{ \left( x \right) } \right) \cdot \left( \lim_{h \to 0} \frac{ \tan{ \left( h \right) }}{h} \right) \cdot \left( \lim_{h \to 0} \frac{1}{1- \tan{ \left( x \right) } \tan{ \left(  h \right) }} \right) \\
        &= \left( 1 + \tan^{2}{ \left( x \right) } \right) \cdot 1 \cdot 1 \\
        &= \sec^{2} \left( x \right)
    \end{align*}
   so 
   \[ \boxed{\dv{x} \left[ \tan{ \left( x \right) } \right] = \sec^{2} \left( x \right).} \] 
\end{proof}

\begin{exercise}
    Use the quotient rule to find an easier way to take the derivative of \( \tan{ \left( x \right) } \).
\end{exercise}
\begin{solution}
    We write \( \tan{ \left( x  \right) } = \frac{ \sin{ \left( x \right) }}{ \cos{ \left( x \right) }} \). Then, 
    \begin{align*}
        \dv{x} \left[ \tan{ \left( x \right) } \right] &= \dv{x } \left[ \frac{ \sin{ \left( x  \right) }}{ \cos{ \left( x \right) }} \right] \\
        &= \frac{\displaystyle \dv{x} \left[ \sin{ \left( x \right) } \right] \cos{ \left( x \right) } -  \dv{x } \left[ \cos{ \left( x  \right) } \right] \sin{ \left( x \right) }}{ \cos^{2}{ \left( x \right) }} \\
        &= \frac{ \cos^{2}{ \left( x \right) } + \sin^{2}{ \left( x \right) }}{ \cos^{2}{ \left( x \right) }}
    \end{align*}
    So 
      \[ \boxed{\dv{x} \left[ \tan{ \left( x \right) } \right] = \sec^{2} \left( x \right).} \] 
\end{solution}

\begin{exercise}
    Use the product rule and the fact that 
    \[ \tan{ \left( x \right) } \cos{ \left( x \right) } = \sin{ \left( x \right) } \] to show that \( \dv{x} \left[ \tan{ \left( x \right) } \right] = \sec^{2} \left( x \right). \)
\end{exercise}
\begin{solution}
    We have 
    \begin{align*}
        \dv{x} \left[ \tan{ \left( x  \right) } \cos{ \left( x  \right) } \right] &= \dv{x} \left[ \sin{ \left( x \right) } \right] \\
         \dv{x } \left[ \tan{ \left( x  \right) } \right] \cos{ \left( x  \right) } + \tan{ \left( x  \right) } \dv{x } \left[ \cos{ \left( x  \right) } \right]&= \dv{x } \left[ \sin{ \left( x \right) } \right] \\
         \dv{x } \left[ \tan{ \left( x \right) } \right] \cos{ \left( x \right) } - \tan{ \left( x \right) } \sin{ \left( x \right) } &= \cos{ \left( x \right) } \\
         \dv{x } \left[ \tan{ \left( x \right) } \right] - \tan^{2}{ \left( x \right) } &=1 \\
         \dv{x } \left[ \tan{ \left( x \right) } \right] &= 1 + \tan^{2}{ \left( x \right) }
    \end{align*}
  \[ \boxed{\dv{x} \left[ \tan{ \left( x \right) } \right] = \sec^{2} \left( x \right).} \]    
\end{solution}


\begin{theorem}
    \[ \dv{x} \left[ \sec(x) \right] = \sec(x)\tan(x) .\]
\end{theorem}
\begin{proof}
    \begin{align*}
        \dv{x} \left[ \sec(x) \right] &= \lim_{h \to 0} \frac{ \sec{ \left( x +h \right) } - \sec{ \left( x \right) }}{h} \\
        &= \lim_{h \to 0} \frac{\displaystyle \frac{1}{ \cos{ \left( x+h  \right) }} -  \frac{1}{ \cos{ \left( x \right) }}}{h} \\
        &= \lim_{h \to 0} \frac{1}{h} \cdot\frac{ \cos{ \left( x  \right) } - \cos{ \left( x+h \right) }}{ \cos{ \left( x  \right) } \cos{ \left( x+h \right) }} \\
        &= \lim_{h \to 0} \frac{ \cos{ \left( x  \right) } - \cos{ \left( x+h  \right) }}{h} \cdot \lim_{ h \to 0} \frac{1}{ \cos{ \left( x \right) } \cos{ \left( x+h \right) }} \\
        &= \left( - \dv{x} \left[ \cos{ \left( x \right) } \right] \right) \cdot \left( \lim_{h \to 0} \frac{1}{ \cos{ \left( x \right) } \cos{ \left( x+h \right) }} \right) \\
        &= \sin{ \left( x  \right) } \sec^{2}{ \left( x \right) }
    \end{align*}
    or 
    \[ \boxed{\dv{x} \left[ \sec(x) \right] = \sec(x)\tan(x)} \]
\end{proof}

\begin{exercise}
    Prove that \( \dv{x } \left[ \sec{ \left( x \right) } \right] = \sec{ \left( x \right) } \tan{ \left( x \right) } \) \emph{the long way}. (Don't use prior knowledge of the derivative of cosine.)
\end{exercise}
\begin{solution}
    The first part of this proof starts off in the same way as above.
     \begin{align*}
        \dv{x} \left[ \sec(x) \right] &= \lim_{h \to 0} \frac{ \sec{ \left( x +h \right) } - \sec{ \left( x \right) }}{h} \\
        &= \lim_{h \to 0} \frac{\displaystyle \frac{1}{ \cos{ \left( x+h  \right) }} -  \frac{1}{ \cos{ \left( x \right) }}}{h} \\
        &= \lim_{h \to 0} \frac{1}{h} \cdot\frac{ \cos{ \left( x  \right) } - \cos{ \left( x+h \right) }}{ \cos{ \left( x  \right) } \cos{ \left( x+h \right) }} 
    \end{align*}
Here is where the steps diverge. 
\begin{align*}
    \dv{x} \left[ \sec(x) \right] &= \lim_{h \to 0} \frac{1}{h} \cdot \frac{ \cos{ \left( x \right) } - \cos{ \left( x  \right) } \cos{ \left( h  \right) } + \sin{ \left( x \right) } \sin{ \left( h \right) }}{\cos{ \left( x  \right) } \cos{ \left( x+h \right) }} \\
    &=\left[  \lim_{h \to 0} \left( \frac{1 - \cos{ \left( h \right) }}{h}  \right) \left( \frac{ \cos{ \left( x \right) }}{ \cos{ \left( x  \right) } \cos{ \left( x+h \right) }} \right) \right] + \left[ \lim_{h \to 0} \left( \frac{ \sin{ \left( h \right) }}{h} \right) \left( \frac{ \sin{ \left( x \right) }}{ \cos{ \left( x  \right) } \cos{ \left( x+h \right) }} \right)\right] \\
    &= \left(  \lim_{h \to 0}\frac{1 - \cos{ \left( h \right) }}{h}  \right)  \left( \lim_{h \to 0} \frac{ \cos{ \left( x \right) }}{ \cos{ \left( x  \right) } \cos{ \left( x+h \right) }} \right) + \left( \lim_{h \to 0} \frac{ \sin{ \left( h \right) }}{h} \right)  \left(  \lim_{h \to 0}\frac{ \sin{ \left( x \right) }}{ \cos{ \left( x  \right) } \cos{ \left( x+h \right) }} \right)\\
    &= 0 \cdot \sec{ \left( x \right) } + 1 \cdot \sec{ \left( x \right) } \tan{ \left( x \right) }
\end{align*}
So 
 \[ \boxed{\dv{x} \left[ \sec(x) \right] = \sec(x)\tan(x)} \]
\end{solution}

\subsection{Derivatives of Logarithmic and Exponential Functions}

\begin{theorem}
    \[
    \dv{x} \left[ \log_{a}(x) \right] = \frac{1}{\ln(a)x}
    \]
\end{theorem}
\begin{proof}
    \begin{align*}
        \dv{x} \left[ \log_{a}(x) \right]
        &= \lim_{h \to 0} \frac{\log_{a}(x + h) - \log_{a}(x)}{h}  \\
        &= \lim_{h \to 0} \frac{1}{h} \log_{a} \left( \frac{x + h}{x} \right)  \\
        &= \lim_{h \to 0} \log_{a} \left( \left( 1 + \frac{h}{x} \right)^{\frac{1}{h}} \right) \\
    \end{align*}

    We now handle the right-hand limit. Let \( t = \frac{1}{h} \), so that \( h \to 0^{+} \) corresponds to \( t \to \infty \). Then:
    \begin{align*}
        \lim_{h \to 0^{+}} \log_{a} \left( \left( 1 + \frac{h}{x} \right)^{\frac{1}{h}} \right)
        &= \lim_{t \to \infty} \log_{a} \left( \left( 1 + \frac{1}{xt} \right)^{t} \right) \\
        &= \log_{a} \left( \lim_{t \to \infty} \left( 1 + \frac{\frac{1}{x}}{t} \right)^{t} \right) \\
        &= \log_{a} \left( e^{\frac{1}{x}} \right) \tag*{Definition of \( e \): \( \lim_{n \to \infty} \left(1 + \frac{a}{n} \right)^n = e^{a} \)} \\
        &= \frac{1}{x} \log_{a}(e)  \\
        &= \frac{1}{x} \cdot \frac{\ln(e)}{\ln(a)}  \\
        &= \frac{1}{\ln(a)x}
    \end{align*}

    The left-hand limit as \( h \to 0^{-} \) proceeds similarly and yields the same result. (Hint: Negatives conspire to cancel.)

    Hence, the derivative exists and is given by:
    \[
    \dv{x} \left[ \log_{a}(x) \right] = \frac{1}{\ln(a)x}.
    \]
\end{proof}

\begin{theorem}
    \[ \dv{x } \left[ a^{x} \right] =  \ln{ \left( a  \right) } \cdot a^{x}. \]
\end{theorem}
\begin{proof}
    We have 
    \begin{align*}
        \dv{x } \left[ a^{x} \right] &= \lim_{h \to 0} \frac{a^{x+h} - a^{x}}{h} \\
        &= a^{x} \lim_{h \to 0} \frac{a^{h} -1}{h} \\
        &=a^{x} \lim_{h \to 0} \frac{ e^{ \displaystyle h \ln{ \left( a \right) }} -1}{h} \\
        &=a^{x} \lim_{k \to 0} \frac{ e^{ \displaystyle k} -1}{k} \cdot \ln{ \left( a \right) }\tag*{Setting $k = h \ln{ \left( a \right) }$.} \\
        &= \ln{ \left( a  \right) } \cdot a^{x} \lim_{k \to 0} \frac{ e^{ \displaystyle k} -1}{k} \\
        &= \ln{ \left( a  \right) } \cdot a^{x } \cdot 1
    \end{align*}
    So 
    \[ \boxed{\dv{x } \left[ a^{x} \right] =  \ln{ \left( a  \right) } \cdot a^{x}.} \]
\end{proof}
