\begin{dfn}\label{def:vector-space}
    A \vocab{vector space} over a field \( \mathbb{K} \) is a set \( \mathbf{V} \) with elements called \vocab{vectors} such that all of the following criteria are held:
\begin{description}
  \item[\textbf{Closure under addition:}] 
  There is a binary operation called \vocab{vector addition} where 
  \begin{align*}
    + : \; &\mathbf{V} \times \mathbf{V} \to \mathbf{V} \\
    &(\vb{u}, \vb{v}) \mapsto \vb{u} + \vb{v},
  \end{align*}
    \item[\textbf{Closure under scalar multiplication:}] There is an operation called \vocab{scalar multiplication} where
    \begin{align*}
     \cdot : \; & \mathbb{K} \times \mathbf{V} \to \mathbf{V} \\
    &(\lambda, \vb{v}) \mapsto \lambda \vb{v},
  \end{align*}
  \item[\textbf{Commutativity:}] \( \vb{u} + \vb{v} = \vb{v} + \vb{u} \) for all \( \vb{u}, \vb{v} \in \mathbf{V} \). 
  \item[\textbf{Associativity:}] \( \left( \vb{u} + \vb{v} \right) + \vb{w} = \vb{u} + \left( \vb{v} + \vb{w} \right) \) and \( \alpha \left( \beta \vb{v}   \right)  = \left( \alpha \beta \right) \vb{v} \) for all \( \vb{u}, \vb{v},\vb{w} \in \mathbf{V} \) and \( \alpha, \beta \in \mathbb{K} \).
  \item[\textbf{Distributive Properties:}] \( \alpha \left( \vb{u} + \vb{v} \right) = \alpha \vb{u} + \alpha \vb{v} \) and \( \left( \alpha + \beta \right)\vb{v} = \alpha \vb{v} + \beta \vb{v} \) for all \( \vb{u}, \vb{v} \in \mathbf{V} \) and \( \alpha, \beta \in \mathbb{K} \).
  \item[\textbf{Existence of an Additive Identity:}] There exists an element  \( \vb{0} \in \mathbf{V} \) called the \vocab{additive identity} such that for all \( \vb{v} \in \mathbf{V} \), we have \( \vb{v} + \vb{0} = \vb{v} \).
  \item[\textbf{Existence of an Additive Inverse:}] For every \( \vb{v} \in \mathbf{V} \), there exists a \( \vb{w} \in \mathbf{V} \) called the \vocab{additive inverse} such that \( \vb{v} + \vb{w} = \vb{0} \).
  \item[\textbf{Multiplicative Identity:}] For \( 1 \in \mathbb{K} \), we have that \( 1 \vb{v} = \vb{v} \) for all \( \vb{v} \in \mathbf{V} \).
\end{description}
We will usually pick \( \mathbb{K} \) to be \( \mathbb{R} \) or \( \mathbb{C} \). If \( \mathbb{K} = \mathbb{R} \), we will sometimes call \( \mathbf{V} \) a \vocab{real vector space}. If \( \mathbb{K} = \mathbb{C} \), we will sometimes call \( \mathbf{V} \) a \vocab{comples vecotr space}. 

\end{dfn}



\begin{example}
  Let \( \mathbb{K} \) be a field and let \( \mathbb{K}^{n} \) be the set of all \( n \)-tuples whose entries come from \( \mathbb{K} \). We define 
  \[ \left( a_{1}, a_{2}, \dots, a_{n} \right) + \left( b_{1}, b_{2}, \dots , b_{n} \right) = \left( a_{1} +b_{1}, a_{2} +b_{2}, \dots, a_{n} +b_{n} \right) \] and 
  \[ \lambda \left( a_{1}, a_{2}, \dots, a_{n} \right) = \left( \lambda a_{1}, \lambda a_{2}, \dots , \lambda a_{n} \right) \] for all \( \lambda , a_{1}, a_{2}, \dots, a_{n},b_{1}, b_{2}, \dots , b_{n} \in \mathbb{K} \). \\ 
  Closure under addition and scalar multiplication is immediate from inheritance from the field \( \mathbb{K} \). \\ 
  For \textbf{commutativity} , we have if \( \vb{a} =  \left( a_{1}, a_{2}, \dots, a_{n} \right) \) and \( \vb{b} =  \left( b_{1}, b_{2}, \dots , b_{n} \right) \) then 
  \begin{align*}
    \vb{a} + \vb{b} &= \left( a_{1}, a_{2}, \dots, a_{n} \right) + \left( b_{1}, b_{2}, \dots , b_{n} \right) \\
    &= \left( a_{1} +b_{1}, a_{2} +b_{2}, \dots, a_{n} +b_{n} \right) \\
    &=  \left( b_{1} +a_{1}, b_{2} +a_{2}, \dots, b_{n} +a_{n} \right) \tag{Since $\mathbb{K}$ is a field.}\\
    &= \left( b_{1}, b_{2}, \dots , b_{n} \right) + \left( a_{1}, a_{2}, \dots, a_{n} \right) \\
    &= \vb{b} + \vb{a}
  \end{align*}
  For \textbf{associativity}, let \( \vb{c} = \left( c_{1}, c_{2}. \dots, c_{n} \right) \) so 
  \begin{align*}
    \left[ \vb{a} + \vb{b} \right] + \vb{c} &= \left[ \left( a_{1}, a_{2}, \dots, a_{n} \right) + \left( b_{1}, b_{2}, \dots , b_{n} \right) \right] + \left( c_{1}, c_{2}. \dots, c_{n} \right) \\
    &= \left( a_{1} +b_{1}, a_{2} +b_{2}, \dots, a_{n} +b_{n} \right) + \left( c_{1}, c_{2}. \dots, c_{n} \right) \\
    &= \left( \left[ a_{1} +b_{1} \right]  + c_{1},  \left[ a_{2} +b_{2} \right]  + c_{2}, \dots, \left[ a_{n} +b_{n} \right]  + c_{n} \right) \\
    &= \left( a_{1} + \left[ b_{1} + c_{1} \right] , a_{2} + \left[ b_{2} + c_{2} \right], \dots, a_{n} + \left[ b_{n} + c_{n} \right]  \right) \tag{Since $\mathbb{K}$ is a field.}  \\
    &= \left( a_{1}, a_{2}, \dots, a_{n} \right) +\left( b_{1} + c_{1}, b_{2} + c_{2}, \dots, b_{n} + c_{n} \right) \\
    &=  \left( a_{1}, a_{2}, \dots, a_{n} \right) + \left[  \left( b_{1}, b_{2}, \dots , b_{n} \right) +  \left( c_{1}, c_{2}. \dots, c_{n} \right)\right] \\
    &=\vb{a} + \left[ \vb{b} + \vb{x} \right]
  \end{align*}
  For the second part of associativity, pick \( \lambda, \mu \in \mathbb{K} \). Then 
  \begin{align*}
    \lambda \left[ \mu \vb{a} \right] &= \lambda \left( \mu a _{1}, \mu a_{2}, \dots, \mu a_{n} \right) \\
    &= \left( \lambda \left[ \mu a_{1} \right], \lambda \left[ \mu a_{2} \right], \dots, \lambda \left[ \mu a_{n} \right] \right) \\
    &= \left( \left[ \lambda \mu \right] a_{1},  \left[ \lambda \mu \right] a_{2}, \dots , \left[ \lambda \mu \right] a_{n} \right) \tag{Since $\mathbb{K}$ is a field.} \\
    &= \left[ \lambda \mu \right] \left( a_{1}, a_{2}, \dots, a_{n} \right) \\ 
    &= \left[ \lambda \mu \right] \vb{a}
  \end{align*}
  For the \textbf{distributive properties}, we have 
  \begin{align*}
    \lambda \left[ \vb{a} + \vb{b} \right] &= \lambda \left( a_{1} +b_{1}, a_{2} +b_{2}, \dots, a_{n} +b_{n} \right) \\
    &= \left( \lambda \left[ a_{1} + b_{1} \right],  \lambda \left[ a_{2} + b_{2} \right], \dots  \lambda \left[ a_{n} + b_{n} \right] \right) \\
    &= \left( \lambda a_{1} + \lambda b_{1}, \lambda a_{2} + \lambda b_{2} , \dots , \lambda a_{n} + \lambda b_{n}  \right) \tag{Since $\mathbb{K}$ is a field.} \\
    &= \left( \lambda a_{1}, \lambda a_{2}, \dots, \lambda a_{n} \right) +\left( \lambda b_{1}, \lambda b_{2}, \dots, \lambda b_{n} \right) \\
    &= \lambda \left( a_{1}, a_{2}, \dots, a_{n} \right) + \lambda\left( b_{1}, b_{2}, \dots , b_{n} \right) \\
    &= \lambda \vb{a} + \lambda \vb{b}
  \end{align*}
For the second distributive property, we have 
\begin{align*}
  \left[ \lambda + \mu \right] \vb{a} &= \left[ \lambda + \mu \right] \left( a_{1}, a_{2}, \dots, a_{n} \right) \\
  &= \left(\left[ \lambda + \mu \right] a_{1},\left[ \lambda + \mu \right]a_{2}, \dots, \left[ \lambda + \mu \right]a_{n} \right) \\
  &= \left( \lambda a_{1} + \mu a_{1}, \lambda a_{2} + \mu a_{2}, \dots, \lambda a_{n} + \mu a_{n} \right) \tag{Since $\mathbb{K}$ is a field.} \\ 
  &= \left( \lambda a_{1}, \lambda a_{2}, \dots, \lambda a_{n} \right) + \left( \mu a_{1}, \mu a_{2}, \dots, \mu a_{n} \right) \\
  &= \lambda \left( a_{1}, a_{2}, \dots, a_{n} \right) + \mu \left( a_{1}, a_{2}, \dots, a_{n} \right) \\
  &= \lambda \vb{a} + \mu \vb{a}
\end{align*}
For the \textbf{additive identity}, let \( \vb{0} = \left( 0, 0, \dots, 0 \right) \) so for any \( \vb{a} \), we have 
\begin{align*}
  \vb{a} + \vb{0} &= \left( a_{1}, a_{2}, \dots, a_{n} \right)  +  \vb{0} = \left( 0, 0, \dots, 0 \right) \\
  &=  \left( a_{1} + 0, a_{2} + 0, \dots, a_{n} +0 \right) \\
  &=  \left( a_{1}, a_{2}, \dots, a_{n} \right) \tag{Since $\mathbb{K}$ is a field.} \\ 
  &= \vb{a}
\end{align*}
For the \textbf{additive inverse}, pick for any \( \vb{a} = \left( a_{1}, a_{2}, \dots, a_{n} \right) \) the vector \( - \vb{a} = \left(- a_{1}, -a_{2}, \dots, -a_{n} \right) \). So 
\begin{align*}
  \vb{a} + \left[ - \vb{a} \right] &= \left( a_{1}, a_{2}, \dots, a_{n} \right) + \left(- a_{1}, -a_{2}, \dots, -a_{n} \right) \\
  &= \left( a_{1} - a_{1},  a_{2} - a_{2}, \dots,  a_{n} - a_{n} \right) \\
  &= \left( 0, 0, \dots, 0 \right) \tag{Since $\mathbb{K}$ is a field.} \\
  &= \vb{0}
\end{align*}
For the \textbf{multiplicative identity} , we have 
\begin{align*}
  1 \vb{a} &= 1 \left( a_{1}, a_{2}, \dots, a_{n} \right)  \\
  &= \left(1 a_{1}, 1a_{2}, \dots, 1a_{n} \right) \\
  &= \left( a_{1}, a_{2}, \dots, a_{n} \right) \tag{Since $\mathbb{K}$ is a field.}  \\
  &= \vb{a}
\end{align*}

\end{example}


\begin{lemma}
  In a vector space \( \mathbf{V} \), the additive identity is unique.
\end{lemma}
\begin{proof}
    Suppose that \( \vb{0} \) and \( \vb{0}' \) are each additive identities for a vector space \( \mathbf{V} \). Then 
    \begin{align*}
      \vb{0} &= \vb{0} + \vb{0}' \tag{Since $\vb{0}'$ is an additive identity. }\\
      &= \vb{0}' \tag{Since $\vb{0}$ is an additive identity. }
    \end{align*}
    
\end{proof}

\begin{lemma}
For each \( \vb{v} \in \mathbf{V} \), its inverse is unique.
\end{lemma}
\begin{proof}
    Suppose that \( \vb{w} \) and \( \vb{w}' \) are each inverses to \( \vb{v} \). Then 
    \begin{align*}
      \vb{w} &= \vb{w} + \vb{0}\\
      &= \vb{w} + \left( \vb{v} + \vb{w}' \right) \\
      &= \left( \vb{w} + \vb{v} \right) + \vb{w}'\\
      &= \vb{0} + \vb{w}' \\
      &= \vb{w}'
    \end{align*}
    
\end{proof}


\begin{exercise}
  Suppose that \( \mathbf{V} \) is a real vector space. We define the \vocab{complexification} of \( \mathbf{V} \), denoted as \( \mathbf{V}_{\mathbb{C}} \), as \( \mathbf{V} \times  \mathbf{V} \) where we write \( \left( \vb{v}, \vb{w} \right) \) as \( \vb{v} + i \vb{w} \). We define addition on \( \mathbf{V}_{\mathbb{C}} \) as 
  \[ \left( \vb{v}_{1} + i \vb{v}_{2} \right) + \left( \vb{w}_{1} + i \vb{w}_{2} \right) = \left( \vb{v}_{1} + \vb{w}_{1} \right) + i \left( \vb{v}_{2} + \vb{w}_{2} \right) \]
  and scalar multiplication by 
  \[ \left( \alpha + i \beta \right) \left( \vb{v} + i \vb{w} \right) = \left( \alpha \vb{v} - \beta \vb{w} \right) + i \left( \beta \vb{v} + \alpha \vb{w} \right) \]
  for \( \alpha, \beta \in \mathbb{R} \). \\
  Show that \( \mathbf{V}_{\mathbb{C}}\) is a complex space. 
\end{exercise}
\begin{solution}
    
\end{solution}

