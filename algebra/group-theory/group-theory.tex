\chapter{Groups: Definitions and Examples}

\section{Axioms and Basic Results}
\begin{dfn}
    Let \( G \) be a non-empty set. \( \star: G \times G \to G \) is a \vocab{binary operation} and we write \(a  \star b \) instead of \( \star \left( a,b \right) \). \( G \) is called a \vocab{group} if the following hold:
    \begin{enumerate}[label=\textbf{\roman*)}]
        \item \( G  \) is \textbf{associative}. For every \( a,b,c \in G \), we have \(  \left(  a \star b \right) \star c = a \star \left( b \star c \right)\). 
        \item There exists an element \( e \in G\) called the \vocab{identity element} with the property that every \( g \in G \), we have \( e \star g = g \star e = g. \)
        \item For every \( g \in G \), there is an associated element \( g^{-1} \) called the \vocab{inverse} with the property that \( g \star g^{-1}= g^{-1} \star g = e. \)
    \end{enumerate}
If for every \( g,h \in G \), we have \( g \star h = h \star g \), we say the group is \vocab{abelian} and the operations is commutative.
\end{dfn}

\begin{example}
    \( \mathbb{Z} ,\mathbb{Q}, \mathbb{R},\) and \( \mathbb{C} \) are all groups under addition with \( e =0 \) and \( x^{-1}=-x \).
\end{example}

\begin{lemma}
    The identity of a group \( G \) is unique.
\end{lemma}
\begin{proof}
    Suppose that \( G \) is a group and \( e \) as well as \( e' \) are both identity elements of \( G \). Then 
    \begin{align*}
        e &= e \star e' \comment{Since $e'$ is an identity element.}\\
        &= e' \comment{Since $e$ is an identity element.}
    \end{align*}
    
\end{proof}

\begin{lemma}
    For every \( g \in G \), \( g^{-1} \) is unique.
\end{lemma}
\begin{proof}
    Very similar to the previous proof, we will assume that \( g^{-1} \) and \( g'^{-1} \) are both inverse to \( g \). So we have 
\begin{align*}
    g^{-1} &= g^{-1} \star e \\
    &= g^{-1} \star \left( g \star g'^{-1} \right)\\
    &= \left( g^{-1} \star g\right) \star g'^{-1}\\
    &= g'^{-1}
\end{align*}

\end{proof}

\begin{corollary}
        \( \left( g^{-1} \right)^{-1} =g.\)
\end{corollary}

\begin{lemma}
    For every \( g,h \in G \), \( \left( g \star h \right) ^{-1} = h^{-1} \star g^{-1}.\)
\end{lemma}
\begin{proof}
    \begin{align*}
        \left( g \star h \right) \star \left( g \star h \right) ^{-1} & = e \comment{By definition}\\
        g^{-1} \star  \left[ \left( g \star h \right) \star \left( g \star h \right) ^{-1} \right] & = g^{-1} \star e  \comment{Applying $g^{-1}$ to both sides}\\
        \left( g^{-1} \star g \right) \star h  \star \left( g \star h \right) ^{-1} & = g^{-1} \comment{By associativity}\\
        h  \star \left( g \star h \right) ^{-1} & = g^{-1} \\
        h^{-1} \star \left[ h  \star \left( g \star h \right) ^{-1}  \right] & =h^{-1} \star g^{-1}\\
        \left( h^{-1} \star h \right) \star \left( g \star h \right) ^{-1}   & =h^{-1} \star g^{-1}\\
        \left( g \star h \right) ^{-1} &= h^{-1} \star g^{-1}.
    \end{align*}
\end{proof}

\begin{exercise}
    Suppose that \( G \) and \( H \) are groups. Then \( G \times H \) can be made into a group with 
    \[ \left( g_{1}, h_{1} \right) \str_{G \times  H} \left( g_{2}, h_{2} \right) :=  \left( g_{1} \str_{G}g_{2}, h_{1} \str_{H} h_{2}\right).\]
\end{exercise}
\begin{solution}
    We will verify each of the group axioms. 
    \begin{enumerate}[label=\textbf{\roman*)}]
        \item For associativity, we have 
        \begin{align*}
            \left[ \left( g_{1}, h_{1} \right) \str_{G \times H} \left( g_{2}, h_{2} \right)\right] \str_{G \times H} \left( g_{3}, h_{3} \right) &=  \left( g_{1} \str_{G}g_{2}, h_{1} \str_{H} h_{2}\right)\str_{G \times H} \left( g_{3}, h_{3} \right)\\
            &= \left( \left(  g_{1} \str_{G}g_{2} \right) \str_{G} g_{3},  \left( h_{1} \str_{H} h_{2} \right) \str_{H} h_{3}\right)\\
            &= \left( g_{1} \str_{G} \left( g_{2} \str_{G}g_{3} \right), h_{1} \str_{H} \left( h_{2}\str_{H}h_{3} \right)  \right)\\
            &= \left( g_{1},h_{1} \right) \str_{G \times H} \left( g_{2} \str_{G} g_{3}, h_{2} \str_{H} h_{3} \right)\\
            &= \left( g_{1}, h_{1} \right) \str_{G \times H} \left[ \left( g_{2}, h_{2} \right) \str_{G \times H} \left( g_{3}, h_{3} \right)\right]
        \end{align*}
        \item We choose the identity element to be \( \left( e_{G},e_{H} \right) \). Then 
        \[ \left( g,h \right) \str_{G \times H} \left( e_{G}, e_{H} \right) =\left( g \str_{G} e_{G}, h \str_{H} e_{H} \right) = \left( g,h \right)\]
        \item For the inverse element of \( \left( g,h \right) \), we choose \( \left( g^{-1},h^{-1} \right) \). 
        \[ \left( g,h \right) \str_{G \times H} \left( g^{-1}, h^{-1} \right) =\left( g \str_{G} g^{-1}, h \str_{H} h^{-1} \right) = \left( e_{G},e_{H} \right). \]
    \end{enumerate}
\end{solution}

\begin{lemma}
    If \( G \) is a group and \( g,h \in G \). The equations \( g \str x =h \) and \( y \str g =h \) have unique solutions in \( G \).
\end{lemma}
\begin{proof}
    We can very quickly see 
    \[ x= g^{-1} \str h \quad \text{ and } \quad y = h \str g^{-1} .\]
\end{proof}

\begin{dfn}
    For a group \( G \), we define the \vocab{order} of an element \( g \in G\) to be the smallest positive integer \( n \) for which \( g^{n} =e \). We will denote this by \( \abs{g}=n \). If no such integer exists, we say that \( g \) has infinite order. 
\end{dfn}

\begin{exercise}
    If \( x^{2} =e \) for every \( x \in G \), then \( G \) is abelian.
\end{exercise}
\begin{solution}
    If \( G \) has one or two elements, the result is trivial. So we will assume \( G \) has at least three elements. Pick any non-identity elements \( x,y \in G \). 
    \begin{align*}
        \left( x \str y \right) \str \left( x \str y \right) &=e \\
        \left( x \str y \right) \str \left( x \str y \right) \str y &= e \str y\\
        \left( x \str y \right) \str x \str \left( y \str y \right) &= y\\
        \left( x \str y \right) \str x \str \left( e \right) &= y\\
        \left( x \str y \right) \str x &= y \\
         \left[ \left( x \str y \right) \str x   \right]\str x &= y \str x \\
         \left( x \str y \right) \str \left( x \str \right) &= y \str x\\
         x \str y &= y \str x
    \end{align*}
The last equation is what was required to prove.
\end{solution}

\section{Examples of Groups}

\subsection{Symmetric Groups}

\begin{theorem}
    Let \( \Omega \) be any non-empty set. Then the set 
    \[ S_{\Omega}= \left\{f: \Omega \to \Omega: f \text{ is a bijection}. \right\} \]
is a group under the operation of function composition.
\end{theorem}
\begin{proof}
    The composition of bijective functions is a bijective function and composition is associative. We take the identity element of \( S_{\Omega} \) to be \( \mathrm{Id}_{\Omega} \). For the inverse of \( f \in S_{\Omega}, \) we take \( f^{-1} \).
\end{proof}

The above group is called the \vocab{symmetric group} on \( \Omega \). For the rest of this section, we will take the special case that \( \Omega = \left\{ 1,2, \dots n \right\} \). This symmetric group is called the \vocab{symmetric group of degree \( n \)} and is denoted by \( S_{n} \).

\begin{theorem}
    \( \abs{S_{n}} =n! \)
\end{theorem}
\begin{proof}
    Let \( \sigma \in S_{n} \). Then there are \( n \) choices for \( \sigma(1) \), \( n-1 \) choices for \( \sigma(2) \) and in general there are \( n+1-j \) choices for \( \sigma \left( j \right) \). Multiplying all these choices together, we get
    \[ \abs{S_{n}} = \prod_{j=1}^{n } \left( n+1-j \right) =n! \]
\end{proof}

\begin{example}
    We will use this example to motive the need for cycle notation as well as how to write a permutation in cycle notation.\\
    Let \( n =13 \) and \( \sigma \in S_{13} \) be given by 
    \begin{align*}
        &\sigma(1)= 12, &&\sigma(2)=13, &&&\sigma(3)=3, &&&&\sigma(4)=1, &&&&&\sigma(5)=11,\\
        &\sigma(6)=9, &&\sigma(7)=5, &&&\sigma(8) = 10, &&&&\sigma(9)=6, &&&&&\sigma(10)=4,\\
        &\sigma(11)=7, &&\sigma(12)=8, &&&\sigma(13)= 2
    \end{align*}
This is quite cumbersome to write and it is difficult to deduce information quickly from reading this. That is why we use cycle notation.\\
To convert a permutation into a cycle, we begin with the smallest element not yet in a cycle and then we follow where the permutation sends this element and repeat until we end up back to where we started. For this example, we have 
\[ 1 \to 12 \to 8 \to 10 \to 4 \to 1 .\]
So this cycle is written as 
\[ \left( 1 \quad  12\quad  8 \quad  10 \quad   4 \right) . \]
We repeat this process with the next smallest element that is not in a previous cycle until all elements are accounted for. Then we write all cycles next to each other. In this example, we have 
\[ \sigma = \left( 1 \quad  12\quad  8 \quad  10 \quad   4 \right) \left( 2 \quad 13 \right) \left( 3 \right) \left( 5 \quad 11 \quad 7 \right) \left( 6 \quad 9 \right).\]
As a final step, we will omit mention of fixed points since they can be understood by their absence. So finally, we have
\[ \sigma = \left( 1 \quad  12\quad  8 \quad  10 \quad   4 \right) \left( 2 \quad 13 \right)  \left( 5 \quad 11 \quad 7 \right) \left( 6 \quad 9 \right).\]
By construction, it is easy to find the inverse of permutation in cycle notation. We see that 
\[ \sigma^{-1}= \left( 4 \quad  10 \quad  8 \quad  12 \quad   1 \right) \left( 13 \quad 2 \right)  \left( 7 \quad 11 \quad 5 \right) \left( 9 \quad 6 \right).\]
\end{example}

\begin{example}
    Let \( \sigma, \tau \in S_{3} \) be given by 
    \[ \sigma = \left( 1 \ \  2 \right) \quad \quad \tau = \left( 1 \ \  3 \right) .\]
    We can compose by simply following the elements under consecutive permutations. For example, if we we wish to find \( \tau \circ \sigma \), we start with \( 1 \) and see that it gets sent to \( 2 \) by \( \sigma \). And we see the that \( \tau \) fixes \( 2 \). So \( \tau \circ \sigma \) sends \( 1 \) to \( 2 \). Now \( \sigma \) sends \( 2 \) to \( 1 \) and \( \tau \) sends \( 1 \) to \( 3 \). So \( \tau \circ \sigma \) sends \( 2 \) to \( 3 \). So we have 
    \[ \tau \circ \sigma = \left( 1 \ \ 2 \ \ 3 \right) .\]
We can do the same for \( \sigma \circ \tau \) and get 
\[ \sigma \circ \tau = \left( 1 \ \ 3 \ \ 2 \right). \]
In this example, we have shown that \( S_{3} \) is not abelian.
\end{example}

\begin{theorem}
    For all \( n \ge 3 \), \( S_{n} \) is non-abelian.
\end{theorem}
\begin{proof}
    Take the element \( \sigma \) that exchanges \( 1 \) and \( 2 \) and leaves everything else fixed and the element \( \tau \) that exchanges the elements \( 1 \) and \( 3 \) and leaves everything else fixed. By the above example, we know that 
    \[ \tau \circ \sigma \neq \sigma \circ \tau. \]
\end{proof}

\subsection{Dihedral Group}


\begin{dfn}
    Consider a regular \( n \)-gon. The \vocab{dihedral group \( D_{2n} \)} is the group of all symmetries of the \( n \)-gon—that is, the set of all isometries of the plane that map the polygon onto itself.
\end{dfn}

If we label the vertices of the polygon as \( 1, 2, \dots, n \), then we can describe \( D_{2n} \) as a subgroup of the symmetric group \( S_n \). This group consists of:
\begin{itemize}
    \item \( n \) \textbf{rotations}, generated by the cycle \( \rho = (1 \ 2 \ \dots \ n) \),
    \item \( n \) \textbf{reflections}, one of which can be written as the product \( \tau = (1 \ n)(2 \ n{-}1)(3 \ n{-}2) \dots \)
\end{itemize}

These two elements, \( \rho \) and \( \tau \), generate the entire group. The group \( D_{2n} \) has order \( 2n \).

\begin{lemma}
    The dihedral group as presented above is indeed the dihedral group.
\end{lemma}
\begin{proof}
    In most other textbooks, the dihedral group is presented as being uniquely determined by two key properties:
    \begin{enumerate}[label=\textbf{\roman*)}]
        \item \( \abs{D_{2n}}=2n \)
        \item For any rotation (we will verify that all rotations are of the form \( \rho^{k}, k=0,1,\dots n-1 \)) and reflection \( \tau \), we have \( \rho^{k} \str \tau = \tau \str \rho^{-k} \)
    \end{enumerate}
For \textbf{i}, we will show that \( \abs{\rho} =n \) and that if \( i \neq j \mod n \), then \( \tau \str \rho^{i} \neq \tau \str \rho^{j} \). \\
We observe that \( \rho^{i} \left( k \right) = k+ i \mod n\). Thus it is readily apparent that \( \abs{\rho} = n \). To show the second part, note that \( \tau \left( k \right)=n+1-k \) so we have 
\[  \left( \tau \str \rho^{i} \right) (k) = n+1 - \left( k+i \mod{n} \right) \quad  \left( \tau \str \rho^{j} \right) (k) = n+1 - \left( k+j \mod n \right)\]
which are not equal if \( i \neq j \). To complete the proof that \( \abs{D_{2n}}=2n \), we need to show that all elements of \( D_{2n} \) are of the form \( \rho^{j} \) or \( \tau \str \rho^{j} \). This will require a proof of \textbf{ii}, so we will begin with that.\\
Slightly re-writing our goal, we will show that \( \tau \str \rho^{j} \str \tau = \rho^{-j} \). 
\begin{align*}
    \left( \tau \str \rho^{j} \str \tau \right) \left( k \right) &= \left( \tau \str \rho^{j} \right) \left( n+1-k \right)\\
    &= \tau \left( n+1 -k +j \right)\\
    &= \tau \left( 1-k+j \right) \comment{Since everything here is modulo n}\\
    &= \left( n+1 \right)- \left( 1-k+j \right)\\
    &=n+k-j\\
    &=k-j\\
    &= \rho^{-j}\left( k \right)
\end{align*}
With the last equation, the proof is completed.
\end{proof}

\begin{lemma}
    The subgroup of \( S_n \) generated by the cycle \[  \rho = (1\ 2\ \dots\ n)  \] and the reflection\[ \tau = (1\ n)(2\ n{-}1)(3\ n{-}2)\dots  \] is isomorphic to the classical dihedral group \( D_{2n} \).
\end{lemma}

\begin{proof}
    The classical presentation of the dihedral group \( D_{2n} \) involves two generators \( r \) and \( s \), subject to the relations:
    \[
    r^n = s^2 = e,\quad srs = r^{-1}
    \]
    We will verify that our elements \( \rho \) and \( \tau \) satisfy these relations and that the group generated by them has order \( 2n \).

    First, observe that \( \rho \) is an \( n \)-cycle. Then:
    \[
    \rho^k(i) = i + k \mod n \quad \Rightarrow \quad \rho^n = e, \text{ and } \rho^k \ne e \text{ for } 1 \le k < n
    \]
    Thus, \( \rho \) has order \( n \).

    Now, define \( \tau = (1\ n)(2\ n{-}1)(3\ n{-}2)\dots \). Since \( \tau \) is its own inverse, we have \( \tau^2 = e \).

    Next, we verify that the reflections \( \tau \rho^j \) are all distinct for \( 0 \le j < n \). Compute:
    \[
    (\tau \circ \rho^j)(k) = \tau(k + j \mod n) = n + 1 - (k + j \mod n)
    \]
    If \( i \ne j \), then the resulting maps differ, so these \( n \) reflections are distinct.

    Finally, we check the key identity:
    \[
    \tau \rho^j \tau = \rho^{-j}
    \]
    Indeed:
    \begin{align*}
        (\tau \circ \rho^j \circ \tau)(k) &= \tau \left( \rho^j(n + 1 - k) \right)\\
        &= \tau(n + 1 - k + j \mod n)\\
        &= n + 1 - (n + 1 - k + j \mod n) = k - j \mod n\\
        &= \rho^{-j}(k)
    \end{align*}
    So the conjugation identity is satisfied.

    Therefore, the group generated by \( \rho \) and \( \tau \) contains \( n \) distinct rotations and \( n \) distinct reflections, totaling \( 2n \) elements. Since they satisfy the same relations as \( D_{2n} \), we conclude that this group is isomorphic to the dihedral group.
\end{proof}





\section{Subgroups}

\subsection{Definition and Examples}

\begin{dfn}
    A subset \( H \) of a group \( G \) is called a \vocab{subgroup} if 
    \begin{enumerate}[label=\textbf{\roman*)}]
        \item \( e \in H \). 
        \item If \( g \in H \), then \( g^{-1} \in H \). 
        \item If \( g,h \in H \), then \( g \str h \in H \).
    \end{enumerate}
If \( H \) is a subgroup of \( G \), we write \( H \le G \).
\end{dfn}

\section{Homomorphisms and Isomorphisms}

\begin{dfn}
    Let \( G \) and \( H \) be groups. A \vocab{homomorphism} from \( G \) to \( H \) is a map \( \varphi: G \to H \) such that for all \( g_{1}, g_{2} \in G \), we have 
    \[ \varphi \left( g_{1} \str_{G} g_{2} \right) = \varphi \left( g_{1} \right)\str_{H} \varphi \left( g_{2} \right).\]
If \( \varphi \) is a bijection, we call \( \varphi \) an \vocab{isomorphism}.
\end{dfn}

\chapter{Group Actions}

\section{Group Actions}

\begin{dfn}
    A \vocab{group action} of a group \( G \) on a set \( A \) is a function \( \cdot \ :G \times A \to A \) with \( \left( g,a \right) \mapsto g \cdot a \) that satisfies the following criteria: 
\begin{enumerate}[label=\textbf{\roman*)}]
    \item \( g_{1} \cdot \left( g_{2} \cdot a \right) = \left( g_{1} \str g_{2} \right) \cdot a \) for all \( g_{1}, g_{2} \in G \) and \( a \in A \).
    \item \( e \cdot a= a\) for all \( a \in A. \)
\end{enumerate}
\end{dfn}

The key observation about group actions is that we can change our point of view from view a group action as a map from \( G \times A \) to \( A \) to a map from \( G  \) to \( \mathrm{Sym}_{A} \).

\begin{theorem}
    We define a map \( \psi: G \to A^{A} \) by \( \psi: g \mapsto \sigma_{g} \), where \( \sigma_{g} \left( a \right) =g \cdot a \), a \( G \) action on \( A \). We will prove the following properties of \( \psi \).
\begin{enumerate}[label=\textbf{\roman*)}]
    \item \( \mathrm{Img}_{\psi} \left( G \right) \subseteq \mathrm{Sym}_{A} \).
    \item \( \psi \) is a homomorphism. 
\end{enumerate}
\end{theorem}
\begin{proof} $ $
 \begin{enumerate}[label=\textbf{\roman*)}]
    \item For the first part, we will find a two-sided inverse for \( \sigma_{g} \). The candidate is \( \sigma_{g^{-1}} \). We have 
    \begin{align*}
        \left( \sigma_{g^{-1}} \circ \sigma_{g} \right)(a) &= \sigma_{g^{-1}} \left( \sigma_{g} \left( a \right) \right)\\
        &= g^{-1} \cdot \left( g \cdot a \right)\\
        &= \left( g^{-1} \str g \right) \cdot a \comment{Since $G$ acts on $A$.}\\
        &= e \cdot a\\
        &= a \\
        &= \mathrm{Id}_{A} \left( a \right)
    \end{align*}
So \( \sigma_{g^{-1}} \) is a left inverse to \( \sigma_{g} \), we can similarly show that it also a right inverse. In particular, this shows that \( \sigma_{g} \) is a bijection and hence it is an element of \( \mathrm{Sym}_{A} \)
\item To show it is a group homomorphism, take \( g_{1}, g_{2} \in G \) and \( a \in A \) so we have 
    \begin{align*}
        \psi \left( g_{1} \str g_{2} \right) (a) &= \sigma_{g_{1} \str g_{2}} \left( a \right)\\
         &= \left( g_{1} \str g_{2}\right) \cdot \left( a \right)\\
         &= g_{1} \cdot \left( g_{2} \cdot \left( a \right) \right)\\
         &= \sigma_{g_{1}} \left( \sigma_{g_{2}} \left( a \right) \right)\\
         &= \left( \sigma_{g_{1}} \circ \sigma_{g_{2}} \right) \left( a \right)\\
         &= \left( \psi \left( g_{1} \right) \circ \psi \left( g_{2} \right) \right) \left( a \right)
    \end{align*}
    
 \end{enumerate}
\end{proof}

\begin{exercise}
    Show that \( \mathbb{Z} \) acts on its by \( z \cdot a = z+a \) for every \( z,a \in \mathbb{Z} \).
\end{exercise}
\begin{solution}
    The identity element \( 0 \) acts trivially on \( \mathbb{Z} \) since 
    \[ 0 \cdot a = 0+a =a \]
    for every \( a \in \mathbb{Z} \). Now consider \( z_{1}, z_{2} \in \mathbb{Z} \). Then 
    \begin{align*}
        z_{1} \cdot \left( z_{2} \cdot a \right) &= z_{1} \cdot \left( z_{2} +a \right)\\
        &= z_{1} + \left( z_{2} + a \right)\\
        &= \left( z_{1} +z_{2} \right) +a\\
        &= \left( z_{1}+z_{2} \right) \cdot a
    \end{align*}
    
\end{solution}

\begin{lemma}
    Every group \( G \) acts on itself by 
    \[ g \cdot h = g \str h .\]
\end{lemma}
\begin{proof}
    This result is a generalization of the above exercise and the proof is very similar to the solution.
\end{proof}

\begin{lemma}
    For a \( G \) action on a set \( A \), we define the \vocab{kernel} of the action to be 
    \[ K = \left\{ g \in G : g \cdot a =a \text{ for all } a \in A \right\} .\]
    We have \( N \trianglelefteq G \).
\end{lemma}
\begin{proof}
    We first show that \( K \) is a subgroup of \( G \). Clearly, \( e \in G \). Now if \( g \in K \) and \( a \in A \), we have
    \begin{align*}
        a&= e \cdot a\\
        &= \left( g^{-1} \str g \right) \cdot a\\
        &= g^{1} \cdot \left( g \cdot a \right)\\
        &= g^{-1} \cdot a
    \end{align*}
So \( g^{-1} \in K \). Now if \( h \) also belongs in \( N \), we have 
\begin{align*}
    \left( g \str h \right) \cdot a &= g \cdot \left( h \cdot a \right)\\
    &= g \cdot a \\
    &= a
\end{align*}
So \( g \str h \in K \). This shows that \( K \) is a subgroup. \\
Now to show that \( K \) is normal, pick any \( k \in K \), \( g \in G \), and \( a \in A \). Then 
    \begin{align*}
        \left( g^{-1} \str k \str g \right) \cdot a &= \left( g^{-1} \str k \right) \cdot \left( g \cdot a \right)\\
        &=g^{1} \cdot \left( k \cdot \left( g \cdot a \right) \right)\\
        &= g^{-1} \cdot \left( g \cdot a \right)\\
        &= \left( g^{-1} \str g \right) \cdot a\\
        &= e \cdot a\\
        &= a
    \end{align*}
  So \( g^{-1} \str k \str g \in K \) which shows that \( K \) is normal. 
\end{proof}

\begin{lemma}
    Let \( G \) be a group and \( A \) a set.  We fix some \( a \in A \) and define the \vocab{stabilizier of \( a \) in \( G \)} to be 
    \[ S_{a} = \left\{ g \in G : g \cdot a =a \right\} .\]
    The stabilizier is a subgroup of \( G \).
\end{lemma}
\begin{proof}
    Clearly \( e \in S_{a} \). For any \( s ,t \in S_{a} \), we have 
    \begin{align*}
        a &= t \cdot a \\
        &= \left( t \str e \right) \cdot a \\
        &= \left( t \str s^{-1} \str s \right) \cdot a \\
        &= \left( t \str s^{-1} \right) \cdot \left( s \cdot a \right)\\
        &= \left( t \str s^{-1} \right) \cdot a
    \end{align*}
which shows that \( t \str s^{-1} \) stabilizes \( a \). So \( S_{a} \) is a subgroup of \( G \).
\end{proof}

