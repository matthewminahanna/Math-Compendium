Recall that an inverse function for \( f(x) \) is a function \( f^{-1}(x) \) with the property that 
\[ f(f^{-1}(x)) = x \quad \text{and} \quad f^{-1}(f(x)) = x \]
for all \( x \) in the appropriate domains. \\ 
This lets us derive information about an inverse function's derivative given information about the derivative of the original function. 

\begin{theorem}
    Suppose that \( f \) is an invertible function with inverse \( g \), that \( f \) is differentiable at \( g(a) \), and that \( f'(g(a)) \neq 0\). Then \( g \) is differentiable at \( a \) and
    \[ g'(a) = \frac{1}{f'(g(a))} .\]
\end{theorem}
\begin{proof} 
    We apply the chain rule to the identity \( f(g(x)) = x \).
    \begin{align*}
        \dv{x} \left[ f(g(x)) \right] &= \dv{x} \left[ x \right] \\
        f'(g(x)) \cdot g'(x) &= 1 \\
        g'(x) &= \frac{1}{f'(g(x))}
    \end{align*}
    The result is achieved by setting \( x = a \). 
\end{proof}