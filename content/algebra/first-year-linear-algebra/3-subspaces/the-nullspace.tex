\begin{dfn}
    Let \( A \) be an \( m \times n \) matrix. The \vocab{nullspace} of \( A \), denoted by \( \mathrm{null}\left( A \right) \) is the set of vectors in \(\vb{v} \in \mathbb{R}^{n} \) such that \( A\vb{v} =\vb{0} \in \mathbb{R}^{m} .\)
\end{dfn}

\begin{example}
    What is the nullspace of the matrix 
    \[ A = \begin{bmatrix*}[r]
        3  & 1 & -7\\
        1 & 0 & -2\\
        2 & 1 & -5\\
    \end{bmatrix*} .\]
    Gaussian elimination comes to the rescue!
    \begin{align*}
         \begin{bmatrix}
        3  & 1 & -7\\
        1 & 0 & -2\\
        2 & 1 & -5\\
    \end{bmatrix} \quad  &\underrightarrow{(3R_{2} - R_{1}) \to R_{2}} \quad  \begin{bmatrix*}[r]
        3  & 1 & -7\\
        0 & -1 & 1\\
        2 & 1 & -5\\
    \end{bmatrix*} \\
    \begin{bmatrix*}[r]
        3  & 1 & -7\\
        0 & -1 & 1\\
        2 & 1 & -5\\
    \end{bmatrix*}  \quad  &\underrightarrow{(3R_{3} - 2R_{1}) \to R_{3}} \quad \begin{bmatrix*}[r]
        3  & 1 & -7\\
        0 & -1 & 1\\
        0 & 1 & -1\\
    \end{bmatrix*}\\
    \begin{bmatrix*}[r]
        3  & 1 & -7\\ 
        0 & -1 & 1\\
        0 & 1 & -1\\
    \end{bmatrix*} \quad  &\underrightarrow{(R_{3} + R_{2}) \to R_{3}}  \quad  \begin{bmatrix*}[r]
        3  & 1 & -7\\
        0 & -1 & 1\\
        0 & 0 & 0\\
    \end{bmatrix*}
    \end{align*}
    Reading off the equations from this reduced matrix, we have the system 
    \begin{align*}
        3x_{1} + x_{2} - 7x_{3} &=0\\
        -x_{2} + x_{3} &= 0
    \end{align*}
    From the second equation, \( x_{2} = x_{3} \). Since \( x_{3} \) is a free variable, we can set it to any value. Let's choose \( x_{3} = 1 \) for convenience, so \( x_{2} = 1 \) as well. Then 
    \begin{align*}
        3x_{1} +1 -7 &=0 \\
        3x_{1} &= 6\\
        x_{1} &= 2
    \end{align*}
    So 
    \[ \boxed{\mathrm{null} \left( A \right) = \mathrm{span} \left\{\;  \begin{bmatrix}
        2\\
        1\\
        1\\
    \end{bmatrix}\;  \right\} \text{ .}} \]
\end{example}

\begin{exercise}
    Let \( A: \mathbb{R}^{4} \to \mathbb{R}^{3} \) be represented by the matrix: 
    \[ A= \begin{bmatrix}
        1 & 2 & 3 & 5\\
        2 & 4 & 8 & 12\\
        3 & 6 & 7 & 13\\
    \end{bmatrix} \]

\begin{enumerate}[label=\textbf{\roman*)}]
\item Convert the augmented matrix \( \left[ A \mid \va{b} \right] \) to an upper triangular system \( \left[ U \mid \va{c} \right] \). 
\item Convert \( \left[ U \mid \va{c} \right] \) to reduced row echelon form \( \left[ R \mid \va{d} \right] \).
\item What is \( \mathrm{col} \left( A \right) \)?
\item What is \( \mathrm{null} \left( A \right) \)?
\item Given a vector \( \va{b} \in \mathrm{col}(A) \), what is the general form of all solutions to \( A\va{x} = \va{b} \)?
\item Find a particular and then a general solution to the \( A \va{x}= \left< 0,6,-6 \right> \).
\end{enumerate}
\end{exercise}
\begin{solution} $ $
    \begin{enumerate}[label=\textbf{\roman*)}]
        \item We have 
        \begin{align*}
            \begin{bmatrix}[*3cr@{\quad}|@{\quad}>{\color{black}}r]
            1 & 2 & 3  &  5 &b_{1}\\
            2 & 4 & 8 & 12 & b_{2}\\
            3 & 6 & 7 & 13 &b_{3}
            \end{bmatrix} \quad  &\underrightarrow{(R_{2} - 2R_{1}) \to R_{2}} \quad  \begin{bmatrix}[*3cr@{\quad}|@{\quad}>{\color{black}}c]
            1 & 2 & 3  &  5 &b_{1}\\
            0 & 0 & 2 & 2 & b_{2}- 2b_{1}\\
            3 & 6 & 7 & 13 &b_{3}
            \end{bmatrix} \\
            \begin{bmatrix}[*3cr@{\quad}|@{\quad}>{\color{black}}c]
            1 & 2 & 3  &  5 &b_{1}\\
            0 & 0 & 2 & 2 & b_{2}- 2b_{1}\\
            3 & 6 & 7 & 13 &b_{3}
            \end{bmatrix}  \quad  &\underrightarrow{(R_{3} - 2R_{3}) \to R_{3}} \quad  \begin{bmatrix}[*3cr@{\quad}|@{\quad}>{\color{black}}c]
            1 & 2 & 3  &  5 &b_{1}\\
            0 & 0 & 2 & 2 & b_{2}- 2b_{1}\\
            0 & 0 & -2 & -2 &b_{3} - 3b_{1}
            \end{bmatrix} \\
             \begin{bmatrix}[*3cr@{\quad}|@{\quad}>{\color{black}}c]
            1 & 2 & 3  &  5 &b_{1}\\
            0 & 0 & 2 & 2 & b_{2}- 2b_{1}\\
            0 & 0 & -2 & -2 &b_{3} - 3b_{1}
            \end{bmatrix} \quad  &\underrightarrow{(R_{2} + R_{3}) \to R_{3}} \quad   \begin{bmatrix}[*3cr@{\quad}|@{\quad}>{\color{black}}c]
            1 & 2 & 3  &  5 &b_{1}\\
            0 & 0 & 2 & 2 & b_{2}- 2b_{1}\\
            0 & 0 & 0 & 0 &b_{2} +b_{3} - 5b_{1}
            \end{bmatrix} 
        \end{align*}
        So our upper triangular system becomes: 
        \begin{equation}\label{2025-10-11 11:02:53}
            \left[ U \mid \va{c} \right] = \begin{bmatrix}[*3cr@{\quad}|@{\quad}>{\color{black}}c]
            1 & 2 & 3  &  5 &b_{1}\\
            0 & 0 & 2 & 2 & b_{2}- 2b_{1}\\
            0 & 0 & 0 & 0 &b_{2} +b_{3} - 5b_{1}
            \end{bmatrix} 
        \end{equation}
        \item Working from our upper-triangular system, we have 
        \begin{align*}
            \begin{bmatrix}[*3cr@{\quad}|@{\quad}>{\color{black}}c]
            1 & 2 & 3  &  5 &b_{1}\\
            0 & 0 & 2 & 2 & b_{2}- 2b_{1}\\
            0 & 0 & 0 & 0 &b_{2} +b_{3} - 5b_{1}
            \end{bmatrix}  \quad  &\underrightarrow{\left( \frac{1}{2}R_{2} \right) \to R_{2}} \quad  \begin{bmatrix}[*3cr@{\quad}|@{\quad}>{\color{black}}c]
            1 & 2 & 3  &  5 &b_{1}\\
            0 & 0 & 1 & 1 &  \frac{1}{2}b_{2}- b_{1}\\
            0 & 0 & 0 & 0 &b_{2} +b_{3} - 5b_{1}
            \end{bmatrix}  \\
             \begin{bmatrix}[*3cr@{\quad}|@{\quad}>{\color{black}}c]
            1 & 2 & 3  &  5 &b_{1}\\
            0 & 0 & 1 & 1 &  \frac{1}{2}b_{2}- b_{1}\\
            0 & 0 & 0 & 0 &b_{2} +b_{3} - 5b_{1}
            \end{bmatrix} \quad  &\underrightarrow{\left( R_{1} - 3 R_{2} \right) \to R_{1}} \quad   \begin{bmatrix}[*3cr@{\quad}|@{\quad}>{\color{black}}c]
            1 & 2 & 0  &  2 &-3b_{1} - \frac{3}{2} b_{2}\\
            0 & 0 & 1 & 1 &  \frac{1}{2}b_{2}- b_{1}\\
            0 & 0 & 0 & 0 &b_{2} +b_{3} - 5b_{1}
            \end{bmatrix}  
        \end{align*}
        So the reduced row echelon form is: 
        \begin{equation}\label{2025-10-11 11:15:17}
            \left[ R \mid \va{d} \right] = \begin{bmatrix}[*3cr@{\quad}|@{\quad}>{\color{black}}c]
            1 & 2 & 0  &  2 &-3b_{1} - \frac{3}{2} b_{2}\\
            0 & 0 & 1 & 1 &  \frac{1}{2}b_{2}- b_{1}\\
            0 & 0 & 0 & 0 &b_{2} +b_{3} - 5b_{1}
            \end{bmatrix}  
        \end{equation}
        \item From \cref{2025-10-11 11:02:53} or \cref{2025-10-11 11:15:17}, we see that our system has a consistent solution only when 
        \begin{equation}\label{2025-10-11 11:34:08}
             -5b_{1}+b_{2}+b_{3}=0 
        \end{equation}
        which is the equation of a plane in \( \mathbb{R}^{3} \). \\ 
        We also see that our pivot columns from our original matrix are 
        \[ \left\{ \; \begin{bmatrix}
            1 \\
            2\\
            3\\
        \end{bmatrix}, \; \begin{bmatrix}
            3\\
            8\\
            7 \\
        \end{bmatrix} \;  \right\} \]
        So 
        \[ \mathrm{span} \left\{ \; \begin{bmatrix}
            1 \\
            2\\
            3\\
        \end{bmatrix}, \; \begin{bmatrix}
            3\\
            8\\
            7 \\
        \end{bmatrix} \;  \right\}  \]
        should also be a representation of the column space. To verify this, take any vector \( \va{v} \in \mathrm{span} \left\{ \left< 1,2,3 \right>, \left< 3,8,7 \right> \right\} \). Then for some scalars \( s,t \in \mathbb{R} \), we have \( \va{v} = s \left< 1,2,3 \right> +t \left< 3,8,7 \right>\) or \( \va{v} = \left< s +3t, 2s+8t, 3s+7t \right> \). Substituting the components of \( \va{v} \) into \cref{2025-10-11 11:34:08}, we have 
        \begin{align*}
            -5 v_{1}+ v_{2}+v_{3} &= -5 \left( s +3t \right) + \left( 2s+8t \right) + \left( 3s+7t \right) \\
            &= \left( -5s +2s +3s \right)  + \left( -15t +8t +7t \right) \\
            &= 0
        \end{align*}
        Since the dimensions of both descriptions of the column space are equal and we have shown that one is contained in the other, it follows that both descriptions are equivalent. 
        \item From  \cref{2025-10-11 11:02:53}, we want to solve 
        \[  \begin{bmatrix}[*3cr@{\quad}|@{\quad}>{\color{black}}c]
            1 & 2 & 3  &  5 &0\\
            0 & 0 & 2 & 2 & 0\\
            0 & 0 & 0 & 0 &0
            \end{bmatrix}  \]
            So 
            \begin{align*}
                x_{1} + 2x_{2} + 3x_{3} + 5x_{4} &=0\\
                x_{3} + x_{4} &=0
            \end{align*}
            If we set \( \boxed{x_{3}=1} \), we have \( \boxed{x_{4} =-1} \) through the second equation. Then our first equation becomes 
            \[ x_{1} + 2x_{2} =2 \]
            Setting \( \boxed{x_{2} =1} \) implies \( \boxed{x_{1}= 0} \) so the first vector in our nullspace is \( \boxed{ \left< 0,1,1,-1 \right>} \). Similarly setting \( \boxed{x_{1} -2} \) implies that \( \boxed{x_{2}=0} \) so the next vector in our nullspace is \( \boxed{\left< 2,0,1,-1 \right>} \). So 
            \[ \mathrm{null} \left( A \right) = \mathrm{span} \left\{ \; \begin{bmatrix}
                0\\
                1\\
                1\\
                -1\\
            \end{bmatrix}, \; \begin{bmatrix}
                2\\
                0\\
                1\\
                -1\\
            \end{bmatrix}  \; \right\} \]
            To show that this is indeed the nullspace of \( A \) pick any \( \va{v} \in \mathrm{null} \left( A \right) \). So \( \va{v} = x \left< 0,1,1, -1, \right> + y \left< 2, 0,1 -1 \right> \) for some \( x, y \in \mathbb{R} \). So \( \va{v} = \left< 2y, x, x+y, -x-y \right> \). Then 
            \begin{align*}
                A \va{v} &= \begin{bmatrix}
        1 & 2 & 3 & 5\\
        2 & 4 & 8 & 12\\
        3 & 6 & 7 & 13\\
    \end{bmatrix} \begin{bmatrix}
        2y \\
        x \\
        x + y \\
        -x-y\\
    \end{bmatrix} \\
    &= \begin{bmatrix}
        2y + 2x + 3 (x+y) + 5 (-x-y)\\
        4y +4x 8 \left( x+y \right) + 12 \left( -x-y \right)\\
        6y + 6x + 7 \left( x+y \right) + 13 \left( -x-y \right)\\
    \end{bmatrix}\\
    &= \begin{bmatrix}
        0\\
        0\\
        0\\
    \end{bmatrix}
            \end{align*}
    \item From \textbf{(iii)}, we know that output space is the span of the pivot columns and from \textbf{(iv)}, we have a description of the null space. So if \( \va{b} = s \left< 1,2,3 \right> + t \left< 3,8,7 \right> \). Then the \( \va{x} \) that solves \( A \va{x} = \va{b} \) is precisely of the form 
    \[ \va{x} = \begin{bmatrix}
        s + 2v \\
        u\\
        t + u +v\\
        -u-v\\
    \end{bmatrix} \quad \text{ for all } u, v \in \mathbb{R}. \]
    \item Clearly the vector \( \left< 0,6,-6 \right> \) is a member of the output space since it solves \( b_{2} + b_{3} - 5b_{1} =0 \). 
    \end{enumerate}
    
\end{solution}


