\begin{dfn}\label{def: Metric Space}
    Let \( M \) be a set. A \vocab{metric} on \( M \) is a map \( \rho: M \times M \to [0, \infty) \) such that the following criteria are held:
    \begin{enumerate}[label=\textbf{\roman*)}]
        \item \( \rho \left( x,y \right) =0 \) if and only if \( x=y \). 
        \item \( \rho \left( x,y \right) = \rho \left( y,x \right)\) for every \( x, y \in M \). 
        \item \( \rho \left( x,z \right) \le \rho \left( x,y \right) + \rho \left( y,z \right)\), this is called the \vocab{triangle inequality}.
    \end{enumerate}
    We call the pair \( \left( M, \rho \right) \) a \vocab{metric space}. We will often omit mention of \( \rho \) if it is understood in context or mention of \( \rho \) is unnecessary.
\end{dfn}

\begin{lemma}
    A metric is non-negative. That is; 
    \[ \rho \left(  x,y  \right) \ge 0 \] for all \( x,y \in M \).
\end{lemma}
\begin{proof}
    \begin{align*}
        \rho \left( x,x \right) &\le \rho \left( x,y \right) + \rho \left( y,x \right) \tag{By the triangle ineqality}\\
        \rho \left( x,x \right) & \le 2 \rho \left( x,y \right) \\
        0 & \le 2 \rho \left( x,y \right) \\
        0 &\le \rho \left( x,y \right)
    \end{align*}
    
\end{proof}


\begin{dfn}
    Let \( M \) be a metric space and define the \vocab{\( \epsilon \)-ball centered at \( x \)} to be
    \[ B_{\rho} \left( x; \epsilon \right) = \left\{ y \in M: \rho \left( x,y \right)< \epsilon \right\}\]
The \vocab{topology induced by the metric} is a topological space on \( M \) with a basis consisting of all possible \( \epsilon \)-balls centered at every \( x \in M \).
\end{dfn}

\begin{lemma}
    The basis described above is indeed a basis.
\end{lemma}
\begin{proof}
    We will verify that the conditions of \Cref{def: basis for a topology} hold.
    Suppose that \( M \) is a metric space with basis elements. \( \left\{ B_{\rho} \left( x; \epsilon \right) : x \in M , \epsilon \in \mathbb{R} \right\} \). By definition, every element of \( M \) belongs to some basis element. So we just need to verify the intersection condition of the basis. Suppose that we fix \( x_{1}, x_{2} \in M \), \( \epsilon_{1}, \epsilon_{2} \in \mathbb{R} \) such that 
    \[ B_{\rho} \left( x_{1}; \epsilon_{1} \right) \cap B_{\rho} \left( x_{2}; \epsilon_{2} \right) \neq \varnothing. \]
    Pick any \( y \in  B_{\rho} \left( x_{1}; \epsilon_{1} \right) \cap B_{\rho} \left( x_{2}; \epsilon_{2} \right) \) and let 
    \[ \epsilon < \min \left\{ \epsilon_{1}- \rho \left( x_{1}, y \right), \epsilon_{2} - \rho \left( x_{2},y \right) \right\} .\]
    I claim that \( B_{\rho} \left( y; \epsilon \right) \subseteq B_{\rho} \left( x_{1}; \epsilon_{1} \right) \cap B_{\rho} \left( x_{2}; \epsilon_{2} \right) \). To show this pick any \( z \in B_{\rho} \left( y; \epsilon \right) \), then 
    \begin{align*}
        \rho \left( x_{1},z \right)  & \le \rho \left( x_{1},y \right) + \rho \left( y,z \right)\\
        &< \rho\left( x_{1},y \right) + \epsilon_{1} - \rho \left( x_{1},y \right) \tag{Since $z \in B_{\rho} \left( y, \epsilon \right)$}\\
       &< \epsilon_{1}
    \end{align*}
    so \( z \in B_{\rho} \left( x_{1}; \epsilon_{1} \right) \). Showing \( z \in B_{\rho} \left( x_{2};\epsilon_{2} \right) \) is similar. So \( z \in   B_{\rho} \left( x_{1}; \epsilon_{1} \right) \cap B_{\rho} \left( x_{2}; \epsilon_{2} \right)\), which shows that \( B_{\rho} \left( y; \epsilon \right) \subseteq   B_{\rho} \left( x_{1}; \epsilon_{1} \right) \cap B_{\rho} \left( x_{2}; \epsilon_{2} \right) \)
\end{proof}

\begin{figure}[H]
    \centering
    \includegraphics[width=0.8\textwidth]{figures/analysis/basis.png}
    \caption{Given \( y \in B_\rho(x_1; \varepsilon_1) \cap B_\rho(x_2; \varepsilon_2) \), we choose \( \varepsilon \) small enough so that \( B_\rho(y; \varepsilon) \subseteq B_\rho(x_1; \varepsilon_1) \cap B_\rho(x_2; \varepsilon_2) \), verifying the basis intersection property for the topology induced by a metric.}
    \label{fig:basis}
\end{figure}
\FloatBarrier



\begin{lemma}[Generalized Triangle Inequality]
    For any \( x_{1}, \dots, x_{n} \) in a metric space \( M \), we have
    \[
        \rho(x_{1}, x_{n}) \le \sum_{j=1}^{n-1} \rho(x_{j}, x_{j+1}).
    \]
\end{lemma}
\begin{proof}
    We prove this by induction on \( n \).\\
Although the base cases of \( n =2 \) or \( n=3 \) trivially hold, we will set the base case to be \( n =4 \) for illustrative purposes. 
    \textbf{Base case:} Let \( n = 4 \). Suppose \( x_1, x_2, x_3, x_4 \in M \). Then:
    \begin{align}
        \rho(x_1, x_4) 
            &\le \rho(x_1, x_3) + \rho(x_3, x_4) \tag*{[triangle inequality on \( x_1, x_3, x_4 \)]} \\
        &\le \rho(x_1, x_2) + \rho(x_2, x_3) + \rho(x_3, x_4) \tag*{[triangle inequality on \( x_1, x_2, x_3 \)]}
    \end{align}
    \textbf{Inductive step:} Assume the inequality holds for \( n - 1 \), i.e.,
    \[
        \rho(x_1, x_{n-1}) \le \sum_{j=1}^{n-2} \rho(x_j, x_{j+1}).
    \]
    Then:
    \begin{align}
        \rho(x_1, x_n) 
            &\le \rho(x_1, x_{n-1}) + \rho(x_{n-1}, x_n) \tag*{[triangle inequality]} \\
        &\le \sum_{j=1}^{n-2} \rho(x_j, x_{j+1}) + \rho(x_{n-1}, x_n) \tag*{[by induction hypothesis]} \\
        &= \sum_{j=1}^{n-1} \rho(x_j, x_{j+1}) \tag*{[combine and reindex]}
    \end{align}

    Therefore, the inequality holds for all \( n \ge 2 \).
\end{proof}

\begin{corollary}
    \[ \abs{\rho \left( w,x \right) - \rho \left( y,z \right)} \le \rho \left( x,y \right) + \rho \left( w,z \right) \]
\end{corollary}
\begin{proof}
    Applying the generalized triangle inequality, we have 
    \begin{align*}
        \rho \left( w,x \right) \le \rho \left( w,z \right) &+ \rho \left( z,y \right) + \rho \left( y,x \right)\\
        \rho \left( w,x \right)- \rho \left( y,z \right) &\le \rho \left( x,y \right)+ \rho \left( w,z \right)\\
        \abs{ \rho \left( w,x \right)- \rho \left( y,z \right) } &\le \abs{\rho \left( x,y \right)+ \rho \left( w,z \right)}\\
        \abs{ \rho \left( w,x \right)- \rho \left( y,z \right) } &\le \rho \left( x,y \right) + \rho \left( w,z \right)
    \end{align*}
    
\end{proof}


\begin{example}[Euclidean Metric on \( \mathbb{R}^{n} \)]
    Let \( \vb{x} = \left( x_{1}, \dots, x_{n} \right), \vb{y} = \left( y_{1}, \dots , y_{n} \right) \in \mathbb{R}^{n}\). We define the standard Euclidean metric on \( \mathbb{R}^{n} \) to be
    \[ \rho\left( \vb{x}, \vb{y} \right) : = \sqrt{ \sum_{j=1}^{n } \left( x_{j}-y_{j} \right)^{2}} \]
    We need only to verify \textbf{iii} of \Cref{def: Metric Space} as \textbf{i} and \textbf{ii} are fairly easy to see. Let \( \vb{x} = \left( x_{1},\dots , x_{n} \right), \vb{y} = \left( y_{1}, \dots, y_{n} \right), \vb{z}= \left( z_{1}, \dots, z_{n} \right) \in \mathbb{R}^{n} \)
    \begin{align*}
        \sum_{j=1}^{n} \left( x_{j}-z_{j} \right)^{2} &= \sum_{j=1}^{n} \left( x_{j}-y_{j}+y_{j}-z_{j} \right)^{2}\\
        &= \sum_{j=1}^{n} \left( x_{j}- y_{j} \right) ^{2} + \sum_{j=1}^{n } \left( y_{j}-z_{j} \right)^{2} + 2 \sum_{j=1}^{n } \left( x_{j}-y_{j} \right)\left( y_{j}-z_{j} \right)\\
        & \le \sum_{j=1}^{n} \left( x_{j}- y_{j} \right) ^{2} + \sum_{j=1}^{n } \left( y_{j}-z_{j} \right)^{2} + 2 \sqrt{\sum_{j=1}^{n} \left( x_{j}-y_{j} \right)^{2} }\sqrt{\sum_{j=1}^{n} \left( y_{j}-z_{j} \right)^{2} } \tag{Applying the \nameref{thrm: The Cauchy-Schwarz Inequality}}
    \end{align*}
   This gives 
   \[ \sum_{j=1}^{n} \left( x_{j}-z_{j} \right)^{2} \le  \sum_{j=1}^{n} \left( x_{j}- y_{j} \right) ^{2} + \sum_{j=1}^{n } \left( y_{j}-z_{j} \right)^{2} + 2 \sqrt{\sum_{j=1}^{n} \left( x_{j}-y_{j} \right)^{2} }\sqrt{\sum_{j=1}^{n} \left( y_{j}-z_{j} \right)^{2} } \]  or 
   \[ \sum_{j=1}^{n} \left( x_{j}-z_{j} \right)^{2} \le \left( \sqrt{\sum_{j=1}^{n} \left( x_{j}-y_{j} \right)^{2} }\ + \sqrt{\sum_{j=1}^{n} \left( y_{j}-z_{j} \right)^{2} } \right)^{2} \]
   Taking the square root of both sides verifies the triangle inequality 
   \[ \sqrt{\sum_{j=1}^{n} \left( x_{j}-z_{j} \right)^{2}} \le \sqrt{\sum_{j=1}^{n} \left( x_{j}-y_{j} \right)^{2} }\ + \sqrt{\sum_{j=1}^{n} \left( y_{j}-z_{j} \right)^{2} } \]
\end{example}

\begin{example}
  Let \( S \) be the set of all complex-valued sequences. Define a metric on \( S \) by
  \[
    \rho \left( \vb{x}, \vb{y} \right) = \sum_{j=1}^{\infty} \frac{1}{2^{j}} \cdot \frac{\abs{x_{j} - y_{j}}}{1 + \abs{x_{j} - y_{j}}}.
  \]
  Much like the previous example, we will verify the triangle inequality. Consider the function
  \[
    f(t) = \frac{t}{1 + t},
  \]
  which is increasing on \( [0, \infty) \) since
  \[
    f'(t) = \frac{1}{(1 + t)^2} > 0.
  \]
  Therefore, by the triangle inequality \( \abs{a + b} \le \abs{a} + \abs{b} \), we have
  \[
    f\left( \abs{a + b} \right) \le f\left(  \abs{a}+ \abs{b} \right) = f\left( \abs{a} \right) + f \left( \abs{b} \right).
  \]
  Setting \( a = z_j - y_j \) and \( b = y_j - x_j \), we get:
  \begin{align*}
    f\left( \abs{z_j - x_j} \right)
    &\le f\left( \abs{z_j - y_j} \right) + f\left( \abs{y_j - x_j} \right) \\
    \frac{1}{2^j} f\left( \abs{z_j - x_j} \right)
    &\le \frac{1}{2^j} f\left( \abs{z_j - y_j} \right) + \frac{1}{2^j} f\left( \abs{y_j - x_j} \right)
  \end{align*}
  Summing over all \( j \in \mathbb{N} \), we obtain:
  \begin{align*}
    \sum_{j=1}^\infty \frac{1}{2^j} f\left( \abs{z_j - x_j} \right)
    &\le \sum_{j=1}^\infty \frac{1}{2^j} f\left( \abs{z_j - y_j} \right)
    + \sum_{j=1}^\infty \frac{1}{2^j} f\left( \abs{y_j - x_j} \right) \\
    \sum_{j=1}^\infty \frac{1}{2^j} \cdot \frac{\abs{z_j - x_j}}{1 + \abs{z_j - x_j}}
    &\le \sum_{j=1}^\infty \frac{1}{2^j} \cdot \frac{\abs{z_j - y_j}}{1 + \abs{z_j - y_j}}
    + \sum_{j=1}^\infty \frac{1}{2^j} \cdot \frac{\abs{y_j - x_j}}{1 + \abs{y_j - x_j}} \\
    \rho(\vb{z}, \vb{x}) &\le \rho(\vb{z}, \vb{y}) + \rho(\vb{y}, \vb{x}).
  \end{align*}
  Hence, the triangle inequality holds.
\end{example}


\begin{lemma}
    Let \( \mathbf{V} \) be a  \hyperref[def:norm]{normed space}.  Then \( \mathbf{V} \) has a \vocab{metric induced by the norm} defined by 
    \[ \rho \left( \vb{x}, \vb{y} \right) : = \norm{ \vb{x} - \vb{y}}\quad \text{for all } \vb{x}, \vb{y} \in \mathbf{V} .\]
\end{lemma}
\begin{proof}
    We want to show that \( \rho \left( \vb{x}, \vb{y} \right) = \norm{ \vb{x} - \vb{y}} \) defines a metric.\\ 
    Suppose that \( \vb{x} = \vb{y} \). Then 
    \begin{align*}
        \rho \left( \vb{x} ,\vb{y} \right) &= \norm{\vb{x} - \vb{y}}\\
        &= \norm{\vb{0}} \\
        &=0
    \end{align*}
    So \( \vb{x} =\vb{y}  \Rightarrow \rho \left( \vb{x},\vb{y} \right) =0\). Conversely, if \( \rho \left( \vb{x}, \vb{y} \right) =0\) then \( \norm{\vb{x} - \vb{y}} =0 \) so \( \vb{x} - \vb{y} = \vb{0} \) or \( \vb{x} = \vb{y} \). \\ 
    For symmetry, 
    \begin{align*}
        \rho \left( \vb{y},\vb{x} \right) &= \norm{\vb{y} - \vb{x}}\\
        &= \norm{-1 \left( \vb{x} -\vb{y} \right)} \\
        &= \abs{-1} \norm{\vb{x} - \vb{y} }\\
        &= \norm{ \vb{x} - \vb{y}} \\
        &= \rho \left( \vb{x}, \vb{y} \right)
    \end{align*}
    Finally, for the triangle inequality, 
    \begin{align*}
        \rho \left( \vb{x}, \vb{z} \right) &= \norm{\vb{x} - \vb{z}}\\
        &= \norm{ \left( \vb{x} - \vb{y} \right) + \left( \vb{y} - \vb{z} \right)} \\
        &\le \norm{\vb{x} - \vb{y}} + \norm{ \vb{y} - \vb{z}} \\
        & = \rho \left( \vb{x}, \vb{y} \right) + \rho \left( \vb{y}, \vb{z} \right)
    \end{align*}
    So \( \rho \left( \vb{x}. \vb{z} \right) \le \rho \left( \vb{x}, \vb{y} \right) + \rho \left( \vb{y}, \vb{z} \right) \).\\
    So \( \rho \left( \vb{x}, \vb{y} \right) = \norm{ \vb{x} - \vb{y}} \) defines a metric.
\end{proof}



\begin{lemma}[The Reverse Triangle Inequality]\label{thm:Reverse-Triangle}
    In a normed space \( \abs{ \; \norm{\vb{x}} - \norm{\vb{y}}\; } \le \norm{\vb{x} - \vb{y}}. \)
\end{lemma}
\begin{proof}
       We can define the metric \( \rho \) induced by the norm as \( \rho \left( \vb{x}, \vb{y} \right) := \norm{ \vb{x} -\vb{y}}\). As such, we wish to show that \( \abs{ \rho \left( \vb{x}, \vb{0} \right) - \rho \left( \vb{y}, \vb{0} \right)} \le \rho \left( \vb{x}, \vb{y} \right) \). This is straightforward. 
    \begin{align*}
        &\rho \left( \vb{x}, \vb{0} \right)  \le \rho \left( \vb{x},\vb{y} \right) + \rho \left(  \vb{y}, \vb{0} \right) \tag*{By the triangle inequality.} \\
        &\rho \left( \vb{x} , \vb{0} \right) - \rho \left( \vb{y}, \vb{0} \right) \le \rho \left( \vb{x},\vb{y} \right)
    \end{align*}
    Applying the absolute value to both sides finishes this proof.
\end{proof}
\begin{altproof}
    We can set \( \vb{x} = \vb{x}- \vb{y} + \vb{y} \). Then 
    \begin{align*}
        \norm{\vb{x}} &= \norm{\left( \vb{x}- \vb{y} \right) + \vb{y}} \\
        &\le \norm{\vb{x}- \vb{y}} + \norm{\vb{y}}
    \end{align*}
    so 
    \[ \norm{\vb{x}} - \norm{\vb{y}} \le \norm{\vb{x}- \vb{y}}\]. Taking the absolute value of both sides finishes this proof.
\end{altproof}