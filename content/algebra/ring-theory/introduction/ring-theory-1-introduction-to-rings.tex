\begin{dfn}
A \vocab{ring} \(R\) is a non-empty set equipped with two binary operations \(+\) and \(\times\), called addition and multiplication, respectively, such that:
\begin{enumerate}[label=\textbf{\roman*)}]
    \item \(R\) is an abelian group under addition. 
    \item Multiplication \(\times\) is associative.
    \item The left and right distributive laws hold: for all \(r,s,t \in R\),
    \[
        r \times (s+t) = (r \times s) + (r \times t) \quad \text{and} \quad (r+s) \times t = (r \times t) + (s \times t).
    \]
\end{enumerate}
The ring \(R\) is \vocab{commutative} if multiplication is commutative. It is \vocab{unitary} if there exists an identity element \(1_R \in R\) such that \(1_R \times r = r \times 1_R = r\) for all \(r \in R\). Multiplication may be written as
\[
r \times s, \quad r \cdot s, \quad \text{or simply } rs,
\]
with juxtaposition \(rs\) usually being the default.
\end{dfn}

\begin{theorem}
    Every finite integral domain is a field. 
\end{theorem}
\begin{proof}
    We need only show that every non-zero element of \( R \) has a multiplicative inverse. \\
    Suppose that \( R \) is a finite integral domain. Pick \( a \in R \) non-zero and define 
    \[ \varphi_{a}(x) := ax .\]
    Suppose that \( \varphi_{a}(x) = \varphi_{a}(y) \). Then 
    \begin{align*}
        \varphi_{a}(x) &= \varphi_{a}(y)\\
        ax &=ay  \\
        ax- ay &= 0\\
        a \left( x-y \right) &=0
    \end{align*}
    Since \( R \) is an integral domain, either \( a \) or \( x-y \) must be \( 0 \). Since \( a \) was chosen to be non-zero, \( x-y \) must be \( 0 \) or \( x=y \). Since \(    \varphi_{a}(x) = \varphi_{a}(y) \Rightarrow x =y\), \( \varphi_{a} \) is injective. Since \( R \) is finite, \( \varphi_{a} \) is also surjective and hence has a unique two sided inverse, call it \( \varphi_{b} \). Therefore 
    \[ \varphi_{b} \left( \varphi_{a} (1) \right) = \varphi_{a} \left( \varphi_{b} (1) \right) = 1 \Rightarrow ba =ab =1 \]
    So \( a  \) has a multiplicative inverse and we are done. 
\end{proof}


\section{Ideals}
Much like in group theory, we want to develop some quotient structure on rings. Since a ring is a group under addition, that can be taken care of with existing tools. So let us contend with multiplication. \\ 

Given a subset \( I \) of a ring \( R \). What conditions do we need for the quotient \( \faktor{R}{I} \) to be a ring? Let \( r+I \) and \( s+I \) be left additive cosets. We want 
\[ \left( r+I \right) \left( s+I \right) = rs +I .\]
Let us "expand" the left side. 
\[ \left( r+I \right) \left( s+I \right) = rs + rI + sI + I \cdot I \]
If we impose the conditions, that 
\begin{enumerate}[label=\textbf{\roman*)}]
    \item \( I \) is a subring of \( R \), 
    \item For all \( r \in R \), \( rI \subseteq I \),
\end{enumerate}
we quickly see that sets \( \left( r+I \right)\left( s+I \right)  \) and \( rs + I \) are equal. This is the motivation for the definition of an \vocab{ideal}. 

\begin{dfn}
    Let \( I \subseteq R \) be a subset of a ring \( R \). We say \( I \) is a \vocab{left ideal} of \( R \) if \( I \) is an additive subgroup of \( R \) closed under multiplication, and for all \( r \in R \) and \( x \in I \), we have \( rx \in I \). \\ 
    Similarly, \( I \) is a \vocab{right ideal} of \( R \) if \( I \) is an additive subgroup of \( R \) closed under multiplication, and for all \( r \in R \) and \( x \in I \), we have \( xr \in I \). \\ 
    A \vocab{two-sided ideal} of \( R \) is a subset that is both a left ideal and a right ideal. \\ 
    When the sidedness is clear from context or irrelevant, we will simply use the term \vocab{ideal}.
\end{dfn}