\begin{dfn}
    Let \( G \) be a non-empty set. \( \str: G \times G \to G \) is a \vocab{binary operation} and we write \(a  \str b \) instead of \( \str \left( a,b \right) \). \( G \) is called a \vocab{group} if the following hold:
    \begin{enumerate}[label=\textbf{\roman*)}]
        \item \( G  \) is \textbf{associative}. For every \( a,b,c \in G \), we have \(  \left(  a \str b \right) \str c = a \str \left( b \str c \right)\). 
        \item There exists an element \( e \in G\) called the \vocab{identity element} with the property that every \( g \in G \), we have \( e \str g = g \str e = g. \)
        \item For every \( g \in G \), there is an associated element \( g^{-1} \) called the \vocab{inverse} with the property that \( g \str g^{-1}= g^{-1} \str g = e. \)
    \end{enumerate}
If for every \( g,h \in G \), we have \( g \str h = h \str g \), we say the group is \vocab{abelian} and the operations is commutative.
\end{dfn}

\begin{example}
    \( \mathbb{Z} ,\mathbb{Q}, \mathbb{R},\) and \( \mathbb{C} \) are all groups under addition with \( e =0 \) and \( x^{-1}=-x \).
\end{example}

\begin{lemma}
    The identity of a group \( G \) is unique.
\end{lemma}
\begin{proof}
    Suppose that \( G \) is a group and \( e \) as well as \( e' \) are both identity elements of \( G \). Then 
    \begin{align*}
        e &= e \str e' \tag{Since $e'$ is an identity element.}\\
        &= e' \tag{Since $e$ is an identity element.}
    \end{align*}
    
\end{proof}

\begin{lemma}
    For every \( g \in G \), \( g^{-1} \) is unique.
\end{lemma}
\begin{proof}
    Very similar to the previous proof, we will assume that \( g^{-1} \) and \( g'^{-1} \) are both inverse to \( g \). So we have 
\begin{align*}
    g^{-1} &= g^{-1} \str e \\
    &= g^{-1} \str \left( g \str g'^{-1} \right)\\
    &= \left( g^{-1} \str g\right) \str g'^{-1}\\
    &= g'^{-1}
\end{align*}

\end{proof}

\begin{corollary}
        \( \left( g^{-1} \right)^{-1} =g.\)
\end{corollary}

\begin{lemma}
    For every \( g,h \in G \), \( \left( g \str h \right) ^{-1} = h^{-1} \str g^{-1}.\)
\end{lemma}
\begin{proof}
    \begin{align*}
        \left( g \str h \right) \str \left( g \str h \right) ^{-1} & = e \tag{By definition}\\
        g^{-1} \str  \left[ \left( g \str h \right) \str \left( g \str h \right) ^{-1} \right] & = g^{-1} \str e  \tag{Applying $g^{-1}$ to both sides}\\
        \left( g^{-1} \str g \right) \str h  \str \left( g \str h \right) ^{-1} & = g^{-1} \tag{By associativity}\\
        h  \str \left( g \str h \right) ^{-1} & = g^{-1} \\
        h^{-1} \str \left[ h  \str \left( g \str h \right) ^{-1}  \right] & =h^{-1} \str g^{-1}\\
        \left( h^{-1} \str h \right) \str \left( g \str h \right) ^{-1}   & =h^{-1} \str g^{-1}\\
        \left( g \str h \right) ^{-1} &= h^{-1} \str g^{-1}.
    \end{align*}
\end{proof}

\begin{exercise}
    Suppose that \( G \) and \( H \) are groups. Then \( G \times H \) can be made into a group with 
    \[ \left( g_{1}, h_{1} \right) \str_{G \times  H} \left( g_{2}, h_{2} \right) :=  \left( g_{1} \str_{G}g_{2}, h_{1} \str_{H} h_{2}\right).\]
\end{exercise}
\begin{solution}
    We will verify each of the group axioms. 
    \begin{enumerate}[label=\textbf{\roman*)}]
        \item For associativity, we have 
        \begin{align*}
            \left[ \left( g_{1}, h_{1} \right) \str_{G \times H} \left( g_{2}, h_{2} \right)\right] \str_{G \times H} \left( g_{3}, h_{3} \right) &=  \left( g_{1} \str_{G}g_{2}, h_{1} \str_{H} h_{2}\right)\str_{G \times H} \left( g_{3}, h_{3} \right)\\
            &= \left( \left(  g_{1} \str_{G}g_{2} \right) \str_{G} g_{3},  \left( h_{1} \str_{H} h_{2} \right) \str_{H} h_{3}\right)\\
            &= \left( g_{1} \str_{G} \left( g_{2} \str_{G}g_{3} \right), h_{1} \str_{H} \left( h_{2}\str_{H}h_{3} \right)  \right)\\
            &= \left( g_{1},h_{1} \right) \str_{G \times H} \left( g_{2} \str_{G} g_{3}, h_{2} \str_{H} h_{3} \right)\\
            &= \left( g_{1}, h_{1} \right) \str_{G \times H} \left[ \left( g_{2}, h_{2} \right) \str_{G \times H} \left( g_{3}, h_{3} \right)\right]
        \end{align*}
        \item We choose the identity element to be \( \left( e_{G},e_{H} \right) \). Then 
        \[ \left( g,h \right) \str_{G \times H} \left( e_{G}, e_{H} \right) =\left( g \str_{G} e_{G}, h \str_{H} e_{H} \right) = \left( g,h \right)\]
        \item For the inverse element of \( \left( g,h \right) \), we choose \( \left( g^{-1},h^{-1} \right) \). 
        \[ \left( g,h \right) \str_{G \times H} \left( g^{-1}, h^{-1} \right) =\left( g \str_{G} g^{-1}, h \str_{H} h^{-1} \right) = \left( e_{G},e_{H} \right). \]
    \end{enumerate}
\end{solution}

\begin{lemma}
    If \( G \) is a group and \( g,h \in G \). The equations \( g \str x =h \) and \( y \str g =h \) have unique solutions in \( G \).
\end{lemma}
\begin{proof}
    We can very quickly see 
    \[ x= g^{-1} \str h \quad \text{ and } \quad y = h \str g^{-1} .\]
\end{proof}

\begin{dfn}
    For a group \( G \), we define the \vocab{order} of an element \( g \in G\) to be the smallest positive integer \( n \) for which \( g^{n} =e \). We will denote this by \( \abs{g}=n \). If no such integer exists, we say that \( g \) has infinite order. 
\end{dfn}

\begin{exercise}
    If \( x^{2} =e \) for every \( x \in G \), then \( G \) is abelian.
\end{exercise}
\begin{solution}
    If \( G \) has one or two elements, the result is trivial. So we will assume \( G \) has at least three elements. Pick any non-identity elements \( x,y \in G \). 
    \begin{align*}
        \left( x \str y \right) \str \left( x \str y \right) &=e \\
        \left( x \str y \right) \str \left( x \str y \right) \str y &= e \str y\\
        \left( x \str y \right) \str x \str \left( y \str y \right) &= y\\
        \left( x \str y \right) \str x \str \left( e \right) &= y\\
        \left( x \str y \right) \str x &= y \\
         \left[ \left( x \str y \right) \str x   \right]\str x &= y \str x \\
         \left( x \str y \right) \str \left( x \str \right) &= y \str x\\
         x \str y &= y \str x
    \end{align*}
The last equation is what was required to prove.
\end{solution}
