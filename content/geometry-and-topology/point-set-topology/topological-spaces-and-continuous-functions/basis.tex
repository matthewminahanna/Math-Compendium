\begin{dfn}\label{def: basis for a topology}
    If \( X \) is a set, we define a \vocab{basis} \( \mathscr{B} \) for a topology to be a collection of subsets of \( X \) that satisfies the following criteria:
    \begin{enumerate}[label=\textbf{\roman*)}]
        \item For every \( x \in X \), there is some \( B \in \mathscr{B} \) such that \( x \in B \). 
        \item For every \( B_{1}, B_{2} \in \mathscr{B} \) and \( x \in B_{1} \cap B_{2} \) there is some \( B_{3} \in \mathscr{B} \) such that \( x \in B_{3} \) and \( B_{3} \subseteq B_{1} \cap B_{2} \). 
    \end{enumerate}
A subset \( U \) of \( X \) belongs to the topology  \( \mathscr{T} \) generated by \( \mathscr{B} \) if for each \( x \in U \), there is some \( B_{x} \in \mathscr{B} \) such that \( x \in B_{x} \) and \( B_{x} \subseteq U \).
\end{dfn}

\begin{theorem}
    The topology \( \mathscr{T} \) generated by \( \mathscr{B} \) is indeed a topology.
\end{theorem}
\begin{proof}
    \( \varnothing \in \mathscr{T} \) vacuously and \( X  \in \mathscr{T} \) by definition. Now let \( U_{\alpha \in A}  \) be an arbitrary collection of open sets. We wish to show that 
    \[ \bigcup_{\alpha \in A } U_{\alpha} \in \mathscr{T}. \]
Pick any \( x \in \bigcup_{\alpha \in A } U_{\alpha}  \). Then \( x \in U_{\beta} \) for at least one \( \beta \in A \). Since \( U_{\beta} \in \mathscr{T} \), there is some \( B \in \mathscr{B} \) for which \( x \in B \) and \( B \subseteq U_{\beta} \). So we can choose the same \( B \) to get 
\[ x \in B \subseteq  \bigcup_{\alpha \in A } U_{\alpha}.\]
Since this holds for all \( x \in \bigcup_{\alpha \in A } U_{\alpha}  \), this shows that \( \bigcup_{\alpha \in A } U_{\alpha} \in \mathscr{T} \).\\
Now let \( U_{1},\dots, U_{n} \) be a finite collection of open sets.  We wish to show that 
\[ \bigcap_{j=1}^{n} U_{j} \in \mathscr{T}. \]
We will proceed by induction. The case \( n=1 \) is trivial so our base case will be \( n=2 \). Suppose \( U_{1} \) and \( U_{2} \) are open sets. We wish to show that \( U_{1} \cap U_{2} \) is open. In other words, for any \( x \in U_{1} \cap U_{2} \), we wish to find a basis element that contains \( x \) and is contained in \( U_{1} \cap U_{2}. \) So pick any \( x \in U_{1} \cap U_{2} \). Then \( x \in U_{1} \) and \( x \in U_{2} \). Since \( U_{1} \) and \( U_{2} \) were already open, there exists \( B_{1}, B_{2} \in \mathscr{B} \) such that \( x \in B_{1} \subseteq U_{1} \) and \( x \in B_{2} \subseteq U_{2}\). Since \( \mathscr{B} \) is a basis, there is another basis element \( B_{3} \) containing \( x \) and contained in \( B_{1} \cap B_{2} \). It is clear that \( B_{3} \in U_{1} \cap U_{2} \), and hence\( U_{1} \cap U_{2} \) is open in \( \mathscr{T} \).\\
 Now for the inductive step, assume that we have shown that \( \bigcap_{j=1}^{n-1} U_{j} \) is open in \( \mathscr{T} \). We define \( V = \bigcap_{j=1}^{n-1} U_{j} \), which is open by the inductive hypothesis. Then \( V \cap U_{n} \) collapses to the base case The inductive step and the proof is completed.
\end{proof}

\begin{theorem}\label{thm:topology-equals-union-of-basis}
    Let \( \mathscr{B} \) be a basis for a topology \( \mathscr{T} \) on \( X \). Then \( \mathscr{T} \) equals the collection of all unions of elements of \( \mathscr{B} \).
\end{theorem}
\begin{proof}
    Suppose that \( \mathbf{B} \) is the collection of all unions of elements of \( \mathscr{B} \). We wish to show that \( \mathscr{T} = \mathbf{B} \).\\ \( \mathbf{B} \subseteq \mathscr{T} \) since each member of \( \mathscr{B} \) is open in \( \mathscr{T} \) and since \( \mathscr{T} \) is a topology, their unions are also members of \( \mathscr{T} \). \\ One the other hand, we pick any \( U \in \mathscr{T} \). Since \( \mathscr{B} \) generates the topology \( \mathscr{T} \), for every \( x \in U \), we may find \( B_{x} \in \mathscr{B} \) such that \( x \in B_{x} \subseteq U \). So we may write
    \[ U = \bigcup_{x \in U} B_{x}. \]
    This completes the proof.
\end{proof}

\begin{theorem}
    Let \( \left( X, \mathscr{T} \right) \) be a topological space. Suppose that \( \mathscr{C} \) is a collection of open sets of \( X \) such that for each open set \( U \) of  \( X \) and each element \( x \) of \( U \), there is a \( C \in \mathscr{C} \) such that \( x \in C \subseteq U \). This qualifies \( \mathscr{C} \) as a basis for the topology on \( X \).
\end{theorem}
\begin{proof}
    We need to first verify that the collection \( \mathscr{C} \) satisfies the conditions laid out in \cref{def: basis for a topology}. \\ 
    The first condition is easy, we simply take our open set to be \( X \) and the conditions of the statement of the theorem ensure that there is some element \( C \in \mathscr{C} \) for which \( x \in C \). \\ 
    For the second condition, pick \( C_{1}, C_{2} \in \mathscr{C} \) such that \( C_{1} \cap C_{2} \neq \varnothing \). Since \( C_{1} \) and \( C_{2} \) are open, it follows that \( C_{1} \cap C_{2} \) is also open. Hence for any \( x \in C_{1} \cap C_{3} \), there is some \( C_{3} \) for which \( x \in C_{3} \subseteq  C_{1} \cap C_{2} \). This shows that \( \mathscr{C} \) is a basis for a topology on \( X \). \\ \\ 
    All this shows is that \( \mathscr{C} \) generates \emph{some} topology \( \mathscr{T}' \). We want to show that \( \mathscr{T} = \mathscr{T}' \). If \( U \in \mathscr{T} \), then for each \( x \in U \), there is some \( C \in \mathscr{C} \) for which \( x \in C \subseteq U \). By the last sentence of \cref{def: basis for a topology}, \( U \) must belong to the topology generated \( \mathscr{C} \) and hence \( U \in \mathscr{T}' \). \\ 
    Conversely, if \( U' \in \mathscr{T}' \), by \cref{thm:topology-equals-union-of-basis}, 
    \[ U' = \bigcup_{x \in U'} C_{x} .\]
    Since each \( C_{x} \) is open in \( \mathscr{T} \), it follows that \( U' \) is also open in \( \mathscr{T} \) so \( U' \subseteq \mathscr{T} \). This completes the proof.
\end{proof}
