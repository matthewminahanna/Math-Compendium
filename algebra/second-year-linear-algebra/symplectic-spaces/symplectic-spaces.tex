While symplectic vector spaces are not usually taught in a first- or second-year linear algebra course, I have elected to include them in a chapter immediately following the chapter on inner products due to the parallels that many of the theorems and proofs share between these two spaces. However, symplectic vector spaces differ from inner product spaces in interesting ways. For example, a finite-dimensional symplectic vector space necessarily has even dimension. This chapter may be skipped without affecting comprehension of subsequent material, but students interested in differential geometry, classical mechanics, or quantum mechanics will benefit from this early exposure.

\begin{dfn}
    A \vocab{linear symplectic form} on a vector space \( \mathbf{V} \) is a map \( \omega: \mathbf{V} \times \mathbf{V} \to \mathbb{K} \) such that: 
    \begin{enumerate}[label=\textbf{\roman*)}]
        \item \( \omega \) is bilinear on \( \mathbf{V} \) 
        \item \( \omega \) is skew-symmetric, that is, \( \omega \left( \vb{v},\vb{w} \right) = -\omega \left( \vb{w},\vb{v} \right) \). 
        \item \( \omega \) is non-degenerate: if \( \omega \left( \vb{v},\vb{w} \right) =0\) for all \( \vb{w} \in \mathbf{V} \), then \( \vb{v}= \vb{0} \).
    \end{enumerate}
\end{dfn}

\begin{theorem}[All symplectic vector spaces have even dimension.]
    Let \( \left( \mathbf{V}, \omega \right) \) be a symplectic vector space. Then:
    \begin{enumerate}[label=\textbf{\roman*)}]
        \item\( \dim \left( \mathbf{V} \right) \) is even
        \item \( \mathbf{V} \) admits a \vocab{Darboux basis} (or \vocab{symplectic basis}) \( B = \left\{ \vb{e}_{1}, \vb{f}_{1}, \vb{e}_{2}, \vb{f}_{2}, \dots , \vb{e}_{n}, \vb{f}_{n} \right\} \) satisfying:
              \begin{enumerate}[label=\textbf{\alph*)}]
                  \item \( \omega \left( \vb{e}_{j}, \vb{f}_{k} \right) = \delta_{jk} \)
                  \item \( \omega \left(\vb{e}_{j}, \vb{e}_{k} \right) = \omega \left( \vb{f}_{j}, \vb{f}_{k} \right) =0 \)
              \end{enumerate}
    \end{enumerate}
\end{theorem}
\begin{proof}
    
\end{proof}
