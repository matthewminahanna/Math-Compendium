\chapter*{Why Linear Algebra?}
It is with very little exaggeration that I say that linear algebra is the singular most important area of mathematics. Our intuition for this subject rests largely on how we as humans naturally think about space, transformation, and relationships. Without prompt, we naturally organize data in arrays, think about directions and dimensions, and have well-developed intuition around addition and scaling, which are the fundamental operations of linear algebra. \\

What makes linear algebra so powerful is its remarkable ability to be both a highly effective practical tool and an interesting theoretical subject to study for its own sake. Systems of linear equations appear everywhere, from balancing chemical reactions to modeling physical phenomena to training neural networks. All this while the theoretical side is one of the best ways to introduce new mathematicians to proofs and its mastery and applications often illuminate other areas of math, like our understanding of symmetry (through representation theory) or spaces (through homology).  \\

Linear algebra bridges the concrete and the abstract, and through crossing this bridge many rewards await. A matrix can simultaneously represent a database of information, a system of equations, a geometric transformation, or an abstract linear operator. Much like life, a matrix is what you make it. This versatility explains linear algebra's ubiquity across quantitative fields: computer graphics, quantum mechanics, machine learning, optimization, statistics, differential equations, and far beyond. Your ability to understand applications will provide manifold examples to test your theoretical knowledge, while your mastery of theory helps you make deeper deductions in applications. Each side enriches the other.\\

This book will honor the canonical way of introducing linear algebra, recognizing that this pedagogical structure exists for good reason. The first-year course is mostly calculation-based and is meant for you to get comfortable with performing concrete computations. This computational fluency builds the intuition to appreciate the abstract theory that follows. The second-year course is more abstract, introducing vector spaces axiomatically, exploring the structure of linear transformations, and proving the fundamental theorems that explain why the calculations work. Students who rush to abstraction without computational experience often struggle to develop geometric intuition, while those who master only calculations miss the unifying principles that make linear algebra so powerful. By working through both stages thoughtfully, you'll develop both the technical facility to solve problems and the conceptual framework to understand the deeper structure of the subject.

\chapter{Vectors}
\section{Vectors and Linear Combinations}
The core object of study in linear algebra is the \emph{vector}.

\begin{dfn}
    By a \vocab{vector} (in this part), often denoted by \( \vb{v} \) or \( \va{v} \), we mean an ordered list of \( n \) numbers:
    \[
        \vb{v} = \left< v_1, v_2, \dots, v_n \right>.
    \]
    The numbers \( v_1, v_2, \dots, v_n \) are called the \vocab{components} or \vocab{entries} of the vector. 
    
    We will restrict our attention to vectors whose components come from either the real numbers \( \mathbb{R} \) or the complex numbers \( \mathbb{C} \). That is, we write \( \vb{v} \in \mathbb{R}^n \) or \( \vb{v} \in \mathbb{C}^n \) depending on context.
    
    Two basic operations on vectors are:
    \begin{itemize}
        \item \textbf{Vector addition:} Given \( \vb{v} = \left< v_1, \dots, v_n \right> \) and \( \vb{w} = \left< w_1, \dots, w_n \right> \), their sum is
        \[
            \vb{v} + \vb{w} = \left< v_1 + w_1, \dots, v_n + w_n \right>.
        \]
        
        \item \textbf{Scalar multiplication:} Given a scalar \( c \in \mathbb{R} \) or \( \mathbb{C} \) and a vector \( \vb{v} = \left< v_1, \dots, v_n \right> \), their product is
        \[
            c \vb{v} = \left< c v_1, \dots, c v_n \right>.
        \]
    \end{itemize}
Throughout linear algebra, we will often represent vectors with \( n \) components as an \vocab{\( n \times 1 \) matrix}, also called a column vector. That is,
\[
    \vb{v} = \left< v_1, \dots, v_n \right> = \begin{bmatrix}
        v_1 \\
        \vdots \\
        v_n
    \end{bmatrix}.
\]
\end{dfn}

This definition carries some immediate consequences on which the rest of linear algebra is built.

For any vector \( \vb{v} = \left< v_1, \dots, v_n \right> \), we can write:
\[
\vb{v} = \begin{bmatrix}
    v_1\\
    v_2\\
    \vdots\\
    v_n
\end{bmatrix} = 
v_1 \begin{bmatrix}
    1\\
    0\\
    \vdots\\
    0
\end{bmatrix} +
v_2 \begin{bmatrix}
    0\\
    1\\
    \vdots\\
    0
\end{bmatrix} +
\cdots +
v_n \begin{bmatrix}
    0\\
    0\\
    \vdots\\
    1
\end{bmatrix}
= \sum_{j=1}^n v_j \vb{e}_j,
\]
where \( \vb{e}_j \) is the vector with a \( 1 \) in the \( j \)th component and \( 0 \) elsewhere.

\begin{example}
    Consider the two-dimensional vector \( {\color[HTML]{7200fa} \vb{v} = \left< 3, 4 \right>} \). We can decompose \( {\color[HTML]{7200fa} \vb{v}} \) as
    \[
        {\color[HTML]{7200fa} \vb{v}} = 3\, {\color[HTML]{ff0000} \vb{e}_1} + 4\, {\color[HTML]{0000ff} \vb{e}_2}
    \]
    or equivalently,
    \[
        {\color[HTML]{7200fa} \left< 3, 4 \right>} = 3\, {\color[HTML]{ff0000} \left< 1, 0 \right>} + 4\, {\color[HTML]{0000ff} \left< 0, 1 \right>}.
    \]

    Geometrically, this means:
    \begin{itemize}
        \item Take 3 copies of the {\color[HTML]{ff0000} unit vector in the \( x \)-direction} and place them tip-to-tail.
        \item Then take 4 copies of the {\color[HTML]{0000ff} unit vector in the \( y \)-direction} and continue placing them tip-to-tail.
        \item The resulting vector \( {\color[HTML]{7200fa} \left< 3, 4 \right>} \) is the diagonal of the resulting "L" shape — the vector sum.
    \end{itemize}

    \begin{center}
        \includegraphics[width=0.5\textwidth]{figures/algebra/linearalgebra/vector34.png}
    \end{center}
\end{example}

This example highlights a subtle but important point about vectors: when we write the vector \( \left< 3,4 \right> \), we have implicitly chosen a basis. In this case, we have chosen the unit vector in the \( x \)-direction to be \( \left< 1,0 \right> \), and the unit vector in the \( y \)-direction to be \( \left< 0,1 \right> \). But this was a \textbf{choice} we made, nature did not hand us this grid.

We could just as well have chosen a different pair of linearly independent vectors and called those our new \( \left< 1,0 \right> \) and \( \left< 0,1 \right> \). This change of basis would then alter the meaning of \( \left< 3,4 \right> \), since that vector is \textbf{defined} as “3 copies of \( \left< 1,0 \right> \)” plus “4 copies of \( \left< 0,1 \right> \).” The numbers 3 and 4 are coordinates relative to a basis, not intrinsic properties of the vector itself. 

\begin{center}
    \includegraphics[width=0.5\textwidth]{figures/algebra/linearalgebra/another34.png}
\end{center}

This figure shows an equally valid way to define \( {\color[HTML]{ff0000} \left< 1,0 \right>} \) and \( {\color[HTML]{0000ff} \left< 0,1 \right>} \), leading to a vector \( {\color[HTML]{7200fa} \left< 3,4 \right>} \) that is distinct from the one in our earlier example.

\vspace{1em}

The key idea is this: mathematics, and any rigorous quantitative discipline, should not depend on how we choose to measure things. If we're describing something objective and external to us, then the way we draw our grid should not affect the truth of what we're describing.

In linear algebra, we will develop tools to \emph{compare grids}. That is, if you and your colleague make observations using different measurement systems (different bases), we need a way to transform your measurements into theirs, and vice versa, without losing the underlying geometric or physical meaning.

\begin{dfn}
    The \vocab{length} or \vocab{norm} of a vector \( \vb{v} = \left< v_{1}, v_{2 }, \dots, v_{n} \right> \) in \( \mathbb{R}^{n} \) is given by 
    \[ \abs{\vb{v}} = \norm{ \vb{v}} = \sqrt{\sum_{j=1}^{n}  \left( v_{j} \right)^{2}} = \sqrt{ \left( v_{1} \right)^{2} + \left( v_{2} \right)^{2} + \cdots + \left( v_{n} \right)^{2}} \]
\end{dfn}

\begin{example}\label{ex:sphere-closest-furthest-vector}
       What are the points on the sphere \( x^{2}+ y^{2}+ z^{2}=4 \) that are closest and furthest from the point \( (x,y,z) = (3,1, -1) \)? \\
       
       For the point closest to \( (3,1,-1) \), we want a vector that points in the same direction as \( \vb{v} = \left< 3,1,-1 \right>\) but has length \( 2 \) (the radius of the sphere). To achieve this, we simply scale the components of \( \vb{v}\) by \( \frac{2}{\norm{\vb{v}}} = \frac{2}{\sqrt{(3)^{2}+ (1)^{2} + (-1)^{2}}} = \frac{2}{\sqrt{11}} = \frac{2 \sqrt{11}}{11}\). So we have the point 
       \[ P_{1} =  \frac{2 \sqrt{11}}{11} \left( 3,1,-1 \right) = \boxed{\left( \frac{6\sqrt{11}}{11}, \frac{2\sqrt{11}}{11}, -\frac{2\sqrt{11}}{11} \right)}  \]
       
       And to find the point furthest from \( (3,1,-1) \), we want a vector pointing in the opposite direction, so we simply reverse the sign:
       \[ P_{2} = -\frac{2 \sqrt{11}}{11} \left( 3,1,-1 \right) = \boxed{\left( -\frac{6\sqrt{11}}{11}, -\frac{2\sqrt{11}}{11}, \frac{2\sqrt{11}}{11} \right)} \]
\end{example}



\section{Dot Products and Cross Products}
\subsection{Dot Products}
\begin{dfn}
An \( n \)-dimensional \vocab{row vector} is a \( 1 \times n \) matrix:
\[
    \vb{\upsilon} = \begin{bmatrix}
        \upsilon_1 & \cdots & \upsilon_n
    \end{bmatrix}.
\]
\end{dfn}

While row and column vectors contain the same components, they behave differently in matrix operations. For now, we will set aside these distinctions and treat them informally.

\begin{dfn}
Given two real vectors 
\[
    \vb{v}=  \begin{bmatrix}
        v_1 \\
        \vdots \\
        v_n
    \end{bmatrix}, \quad
    \vb{w} =  \begin{bmatrix}
        w_1 \\
        \vdots \\
        w_n
    \end{bmatrix},
\]
we define the \vocab{dot product} of \( \vb{v} \) and \( \vb{w} \), denoted \( \vb{v} \cdot \vb{w} \), as
\[
    \vb{v} \cdot \vb{w} = \sum_{j=1}^{n} v_j w_j.
\]
For those familiar with matrix multiplication, the dot product can also be viewed as the product of the row vector corresponding to \( \vb{v} \) and the column vector \( \vb{w} \). \\
We will sometimes denote the dot product of \( \vb{v} \) and \( \vb{w} \) as \( \left< \vb{v}, \vb{w} \right> \).
\end{dfn}

Before we investigate the geometric properties of the dot product, we first verify a key algebraic property.

\begin{theorem}
The dot product is linear in each argument.
\end{theorem}
\begin{proof}
We begin by noting that the dot product is symmetric in its arguments: \( \vb{v} \cdot \vb{w} = \vb{w} \cdot \vb{v} \). This means it suffices to show linearity in the first argument. Let \( \vb{u}, \vb{v}, \vb{w} \in \mathbb{R}^n \) and \( \lambda \in \mathbb{R} \). Then
\[
    \vb{u} + \lambda \vb{v} = \begin{bmatrix}
        u_1 + \lambda v_1 \\
        \vdots \\
        u_n + \lambda v_n
    \end{bmatrix}.
\]
So,
\begin{align*}
    (\vb{u} + \lambda \vb{v}) \cdot \vb{w} 
    &= \sum_{j=1}^n (u_j + \lambda v_j) w_j \\
    &= \sum_{j=1}^n u_j w_j + \lambda \sum_{j=1}^n v_j w_j \\
    &= \vb{u} \cdot \vb{w} + \lambda\, \vb{v} \cdot \vb{w}.
\end{align*}
\end{proof}

Note that we can express the length \( \abs{\vb{v}} \) of a vector \( \vb{v} \) in terms of the dot product:
\[
    \abs{\vb{v}}^2 = \vb{v} \cdot \vb{v}.
\]

Armed with this identity, we can now give a geometric interpretation of what the dot product measures. Consider the following figure:

\begin{center}
    \includegraphics[width=0.25\textwidth]{figures/algebra/linearalgebra/dotproduct1.png}
    \label{fig:dot-product-interpretation}
\end{center}

Applying the Law of Cosines to the triangle formed by the vectors, we obtain:
\[
    \abs{{\color[HTML]{2f21a1} \vb{v} - \vb{w}}}^2 
    = \abs{{\color[HTML]{3a8939} \vb{v}}}^2 
    + \abs{{\color[HTML]{a1212e} \vb{w}}}^2 
    - 2 \abs{{\color[HTML]{3a8939} \vb{v}}} \abs{{\color[HTML]{a1212e} \vb{w}}} \cos(\theta).
\]

We now rewrite both sides using the dot product:
\begin{align*}
    (\vb{v} - \vb{w}) \cdot (\vb{v} - \vb{w}) 
    &= \vb{v} \cdot \vb{v} + \vb{w} \cdot \vb{w} - 2\, \vb{v} \cdot \vb{w} \tag*{by bilinearity and symmetry} \\
    &= \vb{v} \cdot \vb{v} + \vb{w} \cdot \vb{w}  - 2 \abs{\vb{v}} \abs{\vb{w}} \cos(\theta).
\end{align*}

Comparing both expressions, we conclude
\[
    \vb{v} \cdot \vb{w} = \abs{\vb{v}} \abs{\vb{w}} \cos(\theta).
\]

This final expression tells us that the dot product measures how much the vector \( {\color[HTML]{3a8939} \vb{v}} \) points in the direction of \( {\color[HTML]{a1212e} \vb{w}} \), and vice versa. Here is the first consequence of this fact. 

\begin{theorem}[The dot product detects orthogonality]\label{thm: dot product detects orthogonality}
    Two nonzero vectors \( \vb{v} \) and \( \vb{w} \) are perpendicular if and only if \( \vb{v} \cdot \vb{w} = 0 \).
\end{theorem}
\begin{proof}
    \( \Rightarrow \) Suppose \( \vb{v} \) and \( \vb{w} \) are perpendicular. Then the angle \( \theta \) between them satisfies \( \cos(\theta) = 0 \). By the geometric definition of the dot product,
    \[
        \vb{v} \cdot \vb{w} = \abs{\vb{v}} \abs{\vb{w}} \cos(\theta) = \abs{\vb{v}} \abs{\vb{w}} \cdot 0 = 0.
    \]
    \( \Leftarrow \) Conversely, suppose \( \vb{v} \cdot \vb{w} = 0 \). Then again using the geometric definition,
    \[
        \vb{v} \cdot \vb{w} = \abs{\vb{v}} \abs{\vb{w}} \cos(\theta) = 0.
    \]
    Since \( \vb{v} \) and \( \vb{w} \) are nonzero, \( \abs{\vb{v}} \neq 0 \) and \( \abs{\vb{w}} \neq 0 \). Therefore, it must be that \( \cos(\theta) = 0 \), which implies that \( \theta = \frac{\pi}{2} \). In other words, the vectors are perpendicular.
\end{proof}


\begin{theorem}[The dot product detects colinearity or parallelism]
    Two nonzero vectors \( \vb{v} \) and \( \vb{w} \) are colinear (parallel or antiparallel) if and only if
    \[
        \vb{v} \cdot \vb{w} = \lvert \vb{v} \rvert \lvert \vb{w} \rvert
        \quad\text{or}\quad
        \vb{v} \cdot \vb{w} = -\lvert \vb{v} \rvert \lvert \vb{w} \rvert.
    \]
\end{theorem}
\begin{proof}
    \(\Rightarrow\)  
    Suppose \( \vb{v} \) and \( \vb{w} \) are colinear. Then the angle \(\theta\) between them is either \(0\) (parallel, same direction) or \(\pi\) (parallel, opposite direction).  
    By the dot product formula,
    \[
        \vb{v} \cdot \vb{w} = \lvert\vb{v}\rvert \lvert\vb{w}\rvert \cos\theta,
    \]
    so if \(\theta = 0\), \(\cos\theta = 1\) and \(\vb{v} \cdot \vb{w} = \lvert\vb{v}\rvert \lvert\vb{w}\rvert\).  
    If \(\theta = \pi\), \(\cos\theta = -1\) and \(\vb{v} \cdot \vb{w} = -\lvert\vb{v}\rvert \lvert\vb{w}\rvert\).

    \(\Leftarrow\) 
    Conversely, suppose
    \[
        \vb{v} \cdot \vb{w} = \pm \lvert\vb{v}\rvert \lvert\vb{w}\rvert.
    \]
    Then by the dot product formula,
    \[
        \cos\theta = \pm 1,
    \]
    which means \(\theta = 0\) or \(\theta = \pi\) radians.  
    In either case, \( \vb{v} \) and \( \vb{w} \) are colinear.
\end{proof}

\begin{exercise}
    Let \( \vb{x} = \left< 1,1 \right> \). Find all vectors \( \vb{y} \in \mathbb{R}^{2} \) such that \( \vb{x} \cdot \vb{y} =0 \) and \( \norm{\vb{x}} = \norm{\vb{y}} \).
\end{exercise}
\begin{solution}
    We require that \( \vb{y} \) is perpendicular to \( \vb{x} \). So \( \vb{y} = \left< \lambda, - \lambda \right> \) for some \( \lambda \in \mathbb{R}. \) Since \( \norm{\vb{x}} = \norm{\vb{y}} \), we have 
    \begin{align*}
        \sqrt{2} &= \sqrt{ \left( \lambda  \right)^{2} + \left( -\lambda \right)^{2}} \\
        2 &= 2 \left( \lambda \right)^{2}\\
        1&= \lambda^{2}
    \end{align*}
    This gives that \( \lambda =\pm 1 \) so \( \boxed{ \vb{y} = \left< 1,-1 \right>} \) and \( \boxed{\vb{y} = \left< -1,1 \right>} \; \).
\end{solution}


Another useful application of the dot product is that it helps us define lines, planes, hyperplanes etc. 


Consider a line \( a x+by =c \). We know that this line is parallel to \( ax+by=0 \), we will focus on this form for the time being. We can express \( ax+by=0 \) in terms of the dot product. Namely,
\[ \begin{bmatrix}
    a & b\\
\end{bmatrix} \cdot \begin{bmatrix}
    x \\
    y\\
\end{bmatrix} =0 .\]

By \Cref{thm: dot product detects orthogonality}, this means that the line \( ax+by=0 \) consists of all vectors \( \left< x,y \right> \) that are perpendicular to \( \left< a,b \right> \) and that adding \( c \) to the right hand side simply shifts this line without change its slope/direction.

This is a preferable interpretation because it generalizes quite easily to higher dimensions. In particular, if we have the equation 

\[ \sum_{j=1}^{n} a_{j} x_{j} =0  \]

Then the solution consists of all vectors \( \left< x_{1}, x_{2}, \dots, x_{n} \right> \) that are perpendicular to \( \left< a_{1}, a_{2}, \dots, a_{n} \right> \). Adding a constant to the right hand side simply shifts this hyperplane in \( \mathbb{R}^{n} \), again, without changing slope or direction. 

\subsubsection{Vector Projections}
Another very useful application of the dot product is the calculation of how much one vector points in the direction of the other.

We denote the \vocab{vector projection of \( \vb{w} \) onto \( \vb{v} \)} as \( \mathrm{proj}_{\vb{v}}\vb{w} .\) Now since \( \vb{w} \) projects onto \( \vb{v} \), \( \mathrm{proj}_{\vb{v}}\vb{w} \) must point in the direction of \( \vb{v} \) or more specifically, \( \mathrm{proj}_{\vb{v}}\vb{w} \) must point in the direction of \( \frac{\vb{v}}{\abs{\vb{v}}} \), the unit vector in the direction of \( \vb{v} \). So 
\[ \mathrm{proj}_{\vb{v}}\vb{w} =\abs{\mathrm{proj}_{\vb{v}}\vb{w}} \frac{\vb{v}}{\abs{\vb{v}}} .\] So all we need to do is find \( \abs{\mathrm{proj}_{\vb{v}}\vb{w}}\). Luckily our friend, the dot product, can help. If we consider the right triangle with \( \vb{w} \) as the hypotenuse and \( \mathrm{proj}_{\vb{v}}\vb{w} \) as one of the legs, then 
\[ \cos{ \left( \theta \right) }= \frac{\abs{\mathrm{proj}_{\vb{v}}\vb{w}}}{\abs{\vb{w}}} \]
Now we can multiply both sides by \( \abs{\vb{w}} \) to get 
\[ \abs{\vb{w}} \cos{ \left( \theta  \right) } = \abs{\mathrm{proj}_{\vb{v}}\vb{w}} .\] While it might seem like we are done, I did promise you that the dot product will show up. In that spirit, let us multiply and divide the left hand side by \( \abs{\vb{v}} \) to get 
\[ \frac{\abs{\vb{v}} \abs{\vb{w}} \cos{ \left( \theta \right) }}{\abs{\vb{v}}}  = \abs{\mathrm{proj}_{\vb{v}}\vb{w}}\] so 
\[ \boxed{\abs{\mathrm{proj}_{\vb{v}}\vb{w}} = \frac{\vb{v} \cdot \vb{w}}{\abs{\vb{v}}}} \]
Substituting this into our earlier expression, we get 
\[ \boxed{\mathrm{proj}_{\vb{v}} \vb{w} = \frac{\vb{v} \cdot \vb{w}}{\vb{v} \cdot \vb{v}} \vb{v}} \]

\subsection{Square Matrices and the Determinant}

\begin{dfn}
    By an \vocab{\( m \times n \) matrix}, we mean an \( m \)-by-\( n \) rectangular array of numbers:
    \[
        A = \begin{bmatrix}
            a_{11} & a_{12} & \dots & a_{1n} \\
            a_{21} & a_{22} & \dots & a_{2n} \\
            \vdots & \vdots & \ddots & \vdots \\
            a_{m1} & a_{m2} & \dots & a_{mn}
        \end{bmatrix}.
    \]
    The entry \(a_{ij}\) denotes the element in the \(i\)th row and \(j\)th column.  
    Unless otherwise stated, we assume the entries \(a_{ij}\) are real numbers.  
    If \(m = n\), we say \(A\) is a \vocab{square matrix}.  
\end{dfn}

Recall that a \(1 \times n\) matrix is a \emph{row vector} and an \(m \times 1\) matrix is called a \emph{column vector}. \\

If we are given two vectors \(  {\color[HTML]{FF0000} \vb{v} = \left< v_{1},v_{2} \right>}\) and \(  {\color[HTML]{0000FF} \vb{w} = \left< w_{1},w_{2} \right>}\), we may arrange them as \textbf{columns of a matrix \( A \)}.
\[ A = \begin{bmatrix}
    {\color[HTML]{FF0000} v_{1}} & {\color[HTML]{0000FF} w_{1}}\\
     {\color[HTML]{FF0000} v_{2}} &  {\color[HTML]{0000FF} w_{2}}\\
\end{bmatrix}. \]
Furthermore there is a special operation called the \textbf{determinant} of \( A \) that returns the signed area spanned by the vectors \( {\color[HTML]{FF0000} \vb{v}} \) and \(  {\color[HTML]{0000FF} \vb{w}} \). We will denote this as \( {\color[HTML]{7E00FF} \det \left( A \right)} \).
\begin{center}
    \includegraphics[width=0.25\textwidth]{figures/algebra/linearalgebra/determinant.png}
    \label{fig:cross-product picture}
\end{center}
One of the reasons the dot product is so attractive is that it can be computed using components only. The determinant can be similarly calculated from components only. Here is how.\\ 
Recall that the {\color[HTML]{7E00FF}area of a parallelogram} is {\color[HTML]{FF0000}base} times {\color[HTML]{3B00FF} height}. We can find the height by taking the length of the component of {\color[HTML]{0000FF} \( \vb{w} \)} which does not point in direction of {\color[HTML]{FF0000} \( \vb{v} \)}. This is simply {\color[HTML]{3B00FF} \( \abs{\vb{w}} \abs{\sin{ \left( \theta \right) }} \)}, where \( \theta \) is the angle formed between {\color[HTML]{FF0000}\( \vb{v} \)} and {\color[HTML]{0000FF} \( \vb{w} \)}. To summarize, 
\begin{align*}
    \text{{\color[HTML]{7E00FF}area of a parallelogram}} &= \text{{\color[HTML]{FF0000}base} } \text{ times }\text{ {\color[HTML]{3B00FF} height}} \\
     {\color[HTML]{7E00FF} \abs{\det \left( A \right)}} &= {\color[HTML]{FF0000} \abs{\vb{v}}}  {\color[HTML]{3B00FF} \abs{\vb{w}} \abs{\sin{ \left( \theta \right) }}}
\end{align*}
\textbf{Note:} Here we are assuming $ \theta \in \left[ 0, \frac{\pi}{2} \right]$ for visualization's sake. Of course, $\theta \in \left[ 0, 2 \pi \right)$ would explain why the area is \emph{signed}. 
\begin{center}
    \includegraphics[width=0.25\textwidth]{figures/algebra/linearalgebra/areaofparallelogram.png}
    \label{fig:cross-product picture}
\end{center}
Now, we can use a useful identity \( \sin{ \left( \theta \right) } = \cos{ \left( \frac{\pi}{2} - \theta \right) } .\) Substituting this in, we have 
\[ {\color[HTML]{7E00FF} \abs{\det \left( A \right)}} = {\color[HTML]{FF0000} \abs{\vb{v}}} {\color[HTML]{3B00FF} \abs{\vb{w}} \cos{ \left( \frac{\pi }{2} -\theta \right) }} \]
This is looking like a dot product. To get us over the finish line, we will define a new vector {\color[HTML]{3B00FF}\( \vb{w}' \)} with the same length as {\color[HTML]{0000FF} \( \vb{w} \)}, forms an angle of \( \varphi= \frac{\pi}{2}- \theta \) with {\color[HTML]{FF0000} \( \vb{v} \)}, and forms a \( \frac{\pi}{2} \) angle with {\color[HTML]{0000FF} \( \vb{w} \)}. Therefore 
\begin{align*}
    {\color[HTML]{7E00FF} \det \left( A \right)} &=  {\color[HTML]{FF0000} \abs{\vb{v}}} \ {\color[HTML]{3B00FF} \abs{\vb{w}'} \cos{ \left( \varphi\right) }} \\
    &= {\color[HTML]{FF0000} \vb{v}} \cdot {\color[HTML]{3B00FF} \vb{w}'}
\end{align*}
\begin{center}
    \includegraphics[width=0.25\textwidth]{figures/algebra/linearalgebra/dettodot.png}
    \label{fig:dettodot}
\end{center}
By rotating \( {\color[HTML]{0000FF} \vb{w}} \) counterclockwise by 90°, we obtain \( {\color[HTML]{3B00FF} \vb{w}' = \left< w_{2},-w_{1} \right>} \). This choice ensures that \( \vb{w}' \) is perpendicular to \( \vb{w} \), has the same magnitude, and preserves the correct sign for the determinant based on the orientation of the vectors. So 
\begin{align*}
    {\color[HTML]{7E00FF} \det \left( A \right)} &= {\color[HTML]{FF0000} \vb{v}} \cdot {\color[HTML]{3B00FF} \vb{w}'} \\
    &=  {\color[HTML]{FF0000} \left< v_{1}, v_{2} \right>} \cdot {\color[HTML]{3B00FF} \left< w_{2}, - w_{1} \right>} \\
    &= {\color[HTML]{FF0000} v_{1}}{\color[HTML]{3B00FF} w_{2}} -{\color[HTML]{FF0000}  v_{2}}{\color[HTML]{3B00FF} w_{1}} 
\end{align*}
\subsection{The Cross-Product}

\begin{dfn}
    The \vocab{cross product} is a function \( C: \mathbb{R}^{2} \times \mathbb{R}^{3} \to \mathbb{R}^{3} \) that takes two vectors \( \vb{x} = \left< x_{1}, x_{2 }, x_{3} \right> \) and \( \vb{y} = \left< y_{1 }, y_{2 }, y_{3} \right> \) and produces the vector:
    \begin{align*}
         \vb{x} \times \vb{y} &= \det \begin{bmatrix}
        \hat{\vb{i}} & \hat{\vb{j}} & \hat{\vb{k}}\\
        x_{1} & x_{2} & x_{3}\\
        y_{1} & y_{2} & y_{3}\\
    \end{bmatrix}\\ 
    &= \det \begin{bmatrix}
        x_{2 } & x_{3}\\
         y_{2} & y_{3}\\
    \end{bmatrix}  \hat{\vb{i}} - \det \begin{bmatrix}
        x_{1 } & x_{3}\\
         y_{1} & y_{3}\\
    \end{bmatrix}  \hat{\vb{j}} + \det \begin{bmatrix}
        x_{1 } & x_{2}\\
         y_{1} & y_{2}\\
    \end{bmatrix}  \hat{\vb{k}} \\
    &= \left( x_{2}y_{3} - x_{3}y_{2} \right) \hat{\vb{i}} + \left( x_{3}y_{1}-x_{1}y_{3} \right) \hat{\vb{j}} + \left( x_{1}y_{2}- x_{2}y_{1} \right) \hat{\vb{k}} \\
    &= \left< x_{2}y_{3} - x_{3}y_{2} , \;  x_{3}y_{1}-x_{1}y_{3}, \; x_{1}y_{2}- x_{2}y_{1}  \right>.
    \end{align*}
    It is trivial to check that \( \vb{x} \times  \vb{y} = - (\vb{y} \times \vb{x}). \)
\end{dfn}

\begin{lemma}
    The vector \( \vb{x} \times  \vb{y} \) is perpendicular to both \( \vb{x} \) and \( \vb{y} \).
\end{lemma}
\begin{proof}
    
\end{proof}
\begin{exercise}
    Suppose that \( \vb{A} = \left< -1, 0,1  \right> \) and \( \vb{B} = \left< 1, -2, 2 \right> \). Find 
    \begin{enumerate}[label=\textbf{\roman*)}]
        \item A vector perpendicular to both \( \vb{A} \) and \( \vb{B} \) whose \( y \)-component is \( 6 \). 
        \item A vector perpendicular to both \( \vb{A} \) and \( \vb{B} \) whose length is \( 6 \). 
    \end{enumerate}
\end{exercise}
\begin{solution} For both parts of the problem, we need to compute \( \vb{A} \times \vb{B} \). 
    \begin{align*}
        \vb{A} \times \vb{B} &= \mathrm{det} \begin{bmatrix}
            \hat{\vb{i}} & \hat{\vb{j}} & \hat{\vb{k}}\\
            -1 & 0 & 1\\
            1 & -2 & 2\\
        \end{bmatrix} \\
        &= \mathrm{det} \begin{bmatrix}
            0  & 1 \\
            -2  & 2 \\
        \end{bmatrix} \hat{\vb{i}} - \mathrm{det} \begin{bmatrix}
            -1  & 1 \\
            1  & 2 \\
        \end{bmatrix} \hat{\vb{j}} + \mathrm{det} \begin{bmatrix}
            -1  & 0 \\
            1 & -2 \\
        \end{bmatrix} \hat{\vb{k}} \\
        &= 2 \hat{\vb{i}} + \hat{\vb{j}} + 2 \hat{\vb{k}}\\
        &= \left< 2,3,2 \right>
    \end{align*}
    We can verify that we have the correct vector since \( \left( \vb{A} \times \vb{B} \right) \cdot \vb{A} = 0\) and e \( \left( \vb{A} \times \vb{B} \right) \cdot \vb{B} = 0\).
    \begin{enumerate}[label=\textbf{\roman*)}]
        \item $ $\\ 
        To find a vector perpendicular to both \( \vb{A} \) and \( \vb{B} \) whose \( y \)-component is \( 6 \), we just need to take our \( \vb{A} \times \vb{B} \) vector and scale it so that its y-component is \( 6 \). This gives us \[ \boxed{\vb{v} = \left< 4,6,4 \right>} .\]
        \item $ $\\ 
        To find a vector perpendicular to both \( \vb{A} \) and \( \vb{B} \) whose length is \( 6 \), we first need to calculate \( \abs{\vb{A} \times \vb{B}} \), which is 
        \[  \abs{\vb{A} \times \vb{B}}= \sqrt{2^{2} + 3^{2} + 2^{2}} = \sqrt{17} \]
        Now, we just need to scale \(  \vb{A} \times \vb{B} \) by the quantity \( \frac{6 \sqrt{17}}{17} \) (since \( \frac{\sqrt{17 }}{17} \) normalizes \( \vb{A} \times \vb{B}   \) and multiplying by \( 6 \) will scale this new unit vector ) so we have 
        \[ \boxed{\vb{w} = \left<  \frac{12 \sqrt{17}}{17},\frac{18 \sqrt{17}}{17}, \frac{12 \sqrt{17}}{17}  \right> \text{ .}} \]
    \end{enumerate}    
\end{solution}
Recalling our discussion of how the dot product 
\[ a \cdot \left( \vb{x} - p \right)=0 \]
determines a line in \( \mathbb{R}^{n} \), we now have the tools to find a plane in \( \mathbb{R}^{3} \) given \( 3 \) points on it. 
\begin{example}
    Find the equation of a plane that contains the points. \( A = \left( 4,1,3 \right) \), \( B = (1,5,4) \), and \( C = (-3,2,6) \). \\ 
    We first find \( \va{CA} \) and \( \va{CB} \). 
    \[ \va{CA} = \left< 4 +3, 1-2, 2-6  \right> = \boxed{\left< 7, -1, -4  \right>} \quad \text{and} \quad \va{CB} = \left< 1+3, 5-2 ,4-6 \right> = \boxed{\left< 4,3,-2 \right>} .\]
    To find the normal, we have \
    \begin{align*}
        \hat{\vb{n}} &= \va{CA} \times \va{CB} \\
        &= \mathrm{det} \begin{bmatrix}
            \hat{\vb{i}} & \hat{\vb{j}}& \hat{\vb{k}}\\
            7 & -1 & -4\\
            4 & 3 & -2\\
        \end{bmatrix}\\
        &= \mathrm{det} \begin{bmatrix}
            -1 & -4\\
            3 & -2\\
        \end{bmatrix} \hat{\vb{i}} - \mathrm{det} \begin{bmatrix}
            7 & -4\\
            4 & -2 \\
        \end{bmatrix}\hat{\vb{j}} + \mathrm{det}\begin{bmatrix}
            7 & -1\\
            4 & 3\\
        \end{bmatrix} \hat{\vb{k}} \\
        &= \left< 14, -2, 25 \right>
    \end{align*}
    So the equation of our plane is 
    \begin{align*}
        14(x-1) - 2 \left( y-5 \right) + 25(z-4) &=0 \\
        14x -2y + 25z &= 104
    \end{align*}
    We can check that \( A \) and \( C \) also lie on the plane.
\end{example}

Using the cross and dot products, we can find the volume of a parallelepiped to be 
\[ \text{volume of a parallelepiped} = \abs{ \left( A \times B  \right) \cdot C} \]
where \( A, B \), and \( C \) are vectors that make up its side. 



\chapter{Solving Linear Equations}
\section{Vectors and Linear Equations}
\subsection{Introduction to the Row and Column Picture: Two Equations, Two Unknowns}


    Suppose that we are given the following system of equations
    \begin{align}
2x + 3y &= 12 \label{6/4/25/1} \\
x - y &= 1 \label{6/4/25/2}
\end{align}
We wish to find values for \( x \) and \( y \) that simultaneously solve \cref{6/4/25/1} and \cref{6/4/25/2}. \\
We can first view this system by the rows; that is, we wish to find the point of intersection of the lines \(  2x + 3y = 13 \) and \( x - y = 1 \), which is shown in \Cref{fig:FirstLinearSystem}. This picture is quite familiar.\\
The novel idea is to now consider the column picture. We can combine the above system into a single vector equation 
\[ x\begin{pmatrix}2\\1\end{pmatrix} +y \begin{pmatrix}
3\\
-1
\end{pmatrix}= \begin{pmatrix}
12\\
1
\end{pmatrix} .\]
Now, we wish to find the correct scalars \( x \) and \( y \) that makes this equation true, this is highlighted in \Cref{fig:FirstScalars}. \\



\begin{figure}[ht]
  \centering

  \begin{subfigure}[b]{0.48\textwidth}
    \centering
    \fbox{\includegraphics[width=\textwidth]{figures/algebra/firstlinearsystem}}
    \caption{The lines \(2x+3y=12\) (blue) and \(x-y=1\) (green) intersect at \((3,2)\).}
    \label{fig:FirstLinearSystem}
  \end{subfigure}
  \hfill
  \begin{subfigure}[b]{0.48\textwidth}
    \centering
    \fbox{\includegraphics[width=\textwidth]{figures/algebra/firstscalars}}
    \caption{\(c_1 = x = 3\) copies of \(u = \begin{pmatrix}2\\1\end{pmatrix}\) plus \(c_2 = y = 2\) copies of \(v = \begin{pmatrix}3\\-1\end{pmatrix}\) yields \(w = \begin{pmatrix}12\\1\end{pmatrix}\).}
    \label{fig:FirstScalars}
  \end{subfigure}

  \caption{Geometric and algebraic views of solving a linear system.}
  \label{fig:GeometricAndAlgebgraicViews}
\end{figure}

Now to solve this equation, we observe that we can add \( 3 \) copies of \Cref{6/4/25/2} to \Cref{6/4/25/1} to get 
\begin{equation}
	2x+3y+ 3(x-y)= 12 + 3(1) \Rightarrow 5x =15 \label{6/4/25/3}
\end{equation}

From \Cref{6/4/25/3}, we see that \( x=3 \) and then substitution into either \Cref{6/4/25/1} or \Cref{6/4/25/2} (If you are new to this, substitute into both to verify consistency) and solving for \( y \), we get that \( y=2 \), which is consistent with both of our pictures.\\
The standard way to write this equation in linear algebra is collect all of our left coefficients in a \vocab{coefficient matrix} \( A \) where 
\[ A = \begin{pmatrix}
2 & 3\\
1 & -1
\end{pmatrix} .\] Then we collect our variables 
\[ \vb{x}= \begin{pmatrix}
x \\
y
\end{pmatrix} \] and finally our right hand side 
\[ \vb{b} = \begin{pmatrix}
12\\
1
\end{pmatrix} .\]
Combining everything, we have 
\[ A \vb{x}= \vb{b} \quad \text{or} \quad  \begin{pmatrix}
2 & 3\\
1 & -1
\end{pmatrix} \begin{pmatrix}
x \\
y
\end{pmatrix} = \begin{pmatrix}
12\\
1
\end{pmatrix} . \]

\subsection{Three Equations, Three Unknowns}


\begin{example}
  Consider the following system of three equations with three unknowns:
\[
\begin{aligned}
  3x+y-z &= 2 \quad\quad&(1)\\
  4x+2y +3z &=23 \quad\quad&(2)\\
  x-3y+2z &=19 \quad\quad&(3)
\end{aligned}
\]

Using the row/plane picture, the normals to the three planes are
\[
\mathbf{n}_1=(3,1,-1),\qquad \mathbf{n}_2=(4,2,3),\qquad \mathbf{n}_3=(1,-3,2).
\]
None of these is a scalar multiple of another so no two planes are parallel. This means that the first two planes intersect along a line and that line will intersect the third plane at a point. That point will be the solution to our system.

Now solve the system by eliminating \(y\).  One convenient approach is to form combinations that cancel the \(y\)-terms.

First, add (1), (2), and (3):
\[
(3x+y-z)+(4x+2y+3z)+(x-3y+2z)=8x+0y+4z=44,
\]
so
\[
8x+4z=44 \quad\Longrightarrow\quad 2x+z=11. \tag{A}
\]

Next, take \(-1\) times (1), plus \(2\) times (2), plus \(1\) times (3):
\[
-1\cdot(3x+y-z)+2\cdot(4x+2y+3z)+1\cdot(x-3y+2z)
\]
Compute coefficients:
\[
x:\ -3+8+1=6,\qquad y:\ -1+4-3=0,\qquad z:\ 1+6+2=9,
\]
RHS: \(-2+46+19=63\). Thus
\[
6x+9z=63 \quad\Longrightarrow\quad 2x+3z=21. \tag{B}
\]

Now solve the \(2\times2\) system (A) and (B):
\[
\begin{cases}
2x+z=11\\[4pt]
2x+3z=21
\end{cases}
\]
Subtract (A) from (B) to eliminate \(x\):
\[
(2x+3z)-(2x+z)=2z=21-11=10 \quad\Longrightarrow\quad z=5.
\]
Substitute \(z=5\) into (A):
\[
2x+5=11 \quad\Longrightarrow\quad 2x=6 \quad\Longrightarrow\quad x=3.
\]
Finally substitute \(x=3,z=5\) into equation (1) to find \(y\):
\[
3(3)+y-(5)=2 \quad\Longrightarrow\quad 9+y-5=2 \quad\Longrightarrow\quad y=-2.
\]
So the solution is
\[
\boxed{(x,y,z)=(3,-2,5)}.
\]

Writing the system in matrix form \(A\mathbf{x}=\mathbf{b}\):
\[
A=\begin{bmatrix}
3 & 1 & -1\\[4pt]
4 & 2 & 3\\[4pt]
1 & -3 & 2
\end{bmatrix},\qquad
\mathbf{x}=\begin{bmatrix}x\\y\\z\end{bmatrix},\qquad
\mathbf{b}=\begin{bmatrix}2\\23\\19\end{bmatrix}.
\]
\end{example}


\section{Elimination}
We want a systematic way of solving systems of linear equations. Gaussian elimination provides this. 
We first note that the system
\begin{align*}
  3x+y-z &= 2 \\
  4x+2y +3z &=23 \\
  x-3y+2z &=19 
\end{align*}
can be written more compactly as an \vocab{augmented matrix}.
\[ \begin{bmatrix}[*2cr@{\quad}|@{\quad}>{\color{red}}r]
  3 & 1 & -1  &  2 \\
  4 & 2 & 3 & 23\\
  1 & -3 & 2 & 19
\end{bmatrix} \] 
Here each equation gets assigned a row. We will dedicate this section to motivating the row operations needed for Gaussian elimination. \\
Let us take, for example the first two equations
\begin{align*}
  3x+y-z &= 2 \\
  4x+2y +3z &=23 \\
\end{align*}
We can take arbitrary linear combinations of these two equations and \emph{replace one equation}. For example, we can take three copies of the second equation and subtract from it four copies of the first. This gives us 
\[ 3 \left( 4x+2y +3z =23 \right) - 4 \left(  3x+y-z = 2 \right) \Rightarrow \left( 12x + 6y + 9z =69 \right) - \left( 12x +4y -4z =8 \right) \Rightarrow \boxed{0x +2y +13z = 61} \]
Notice that because we are taking linear combinations of these two equations, the solution we found earlier of \( \left( x,y,z \right) = \left( 3,-2,5 \right) \) is a solution to \[ 0x +2y +13z = 61. \] 
We can verify this: \( 2(-2) + 13(5) = -4 + 65 = 61 \). As such the system 
\begin{align*}
  3x+y-z &= 2 \\
  0x +2y +13z &= 61 \\
  x-3y+2z &=19 
\end{align*}
still has the same solution as above. So we can write our augmented system as
\[ \begin{bmatrix}[*2cr@{\quad}|@{\quad}>{\color{red}}r]
  3 & 1 & -1  &  2 \\
  0 & 2 & 13 & 61\\
  1 & -3 & 2 & 19
\end{bmatrix} \]
Finally note that row exchange is trivially acceptable since our solution shouldn't depend on the order in which we write out equations.\\
Notice that if we have an augmented matrix of the form
\[ \begin{bmatrix}[*2cr@{\quad}|@{\quad}>{\color{red}}r]
  a & b & c  &  s  \\
  0 & d &e  & t \\
  0 & 0 & f & u
\end{bmatrix} \]
This corresponds to the system of equations
\begin{align*}
  ax+by+ cz &= s \\
  dy +ez &=t  \\
  fz &=u
\end{align*} 
So then we can simply solve for \( z \) in the third equation, then back substitute our solution for \( z \) into the second equation to solve for \( y \), and finally solve for \( x \) by substituting the solutions for \( y \) and \( z \). \emph{This is the goal for Gaussian elimination}. We wish to convert an arbitrary linear system into an \vocab{upper triangular matrix.}

\begin{example}
  Let us try Gaussian elimination with a system we are very familiar with 
  \begin{align*}
  3x+y-z &= 2 \\
  4x+2y +3z &=23 \\
  x-3y+2z &=19 
\end{align*}
Converting to an augmented matrix, we get
\[ \begin{bmatrix}[*2cr@{\quad}|@{\quad}>{\color{red}}r]
  3 & 1 & -1  &  2 \\
  4 & 2 & 3 & 23\\
  1 & -3 & 2 & 19
\end{bmatrix} \] 

First, we eliminate the $x$-term in the second row by replacing $R_2$ with $(3R_2 - 4R_1)$:
\[\begin{bmatrix}[*2cr@{\quad}|@{\quad}>{\color{red}}r]
  3 & 1 & -1  &  2 \\
  4 & 2 & 3 & 23\\
  1 & -3 & 2 & 19
\end{bmatrix}\quad  \underrightarrow{(3R_{2} - 4R_{1}) \to R_{2}} \quad \begin{bmatrix}[*2cr@{\quad}|@{\quad}>{\color{red}}r]
  3 & 1 & -1  &  2 \\
  0 & 2 & 13 & 61\\
  1 & -3 & 2 & 19
\end{bmatrix} \]

Next, we eliminate the $x$-term in the third row by replacing $R_3$ with $(3R_3 - R_1)$:
\[  \begin{bmatrix}[*2cr@{\quad}|@{\quad}>{\color{red}}r]
  3 & 1 & -1  &  2 \\
  0 & 2 & 13 & 61\\
  1 & -3 & 2 & 19
\end{bmatrix} \quad  \underrightarrow{(3R_{3} - R_{1}) \to R_{3}} \quad  \begin{bmatrix}[*2cr@{\quad}|@{\quad}>{\color{red}}r]
  3 & 1 & -1  &  2 \\
  0 & 2 & 13 & 61\\
  0 & -10 &7 & 55
\end{bmatrix} \]

Finally, we eliminate the $y$-term in the third row by replacing $R_3$ with $(R_3 + 5R_2)$:
\[ \begin{bmatrix}[*2cr@{\quad}|@{\quad}>{\color{red}}r]
  3 & 1 & -1  &  2 \\
  0 & 2 & 13 & 61\\
  0 & -10 &7 & 55
\end{bmatrix}  \quad  \underrightarrow{(R_{3} + 5R_{2}) \to R_{3}} \quad \begin{bmatrix}[*2cr@{\quad}|@{\quad}>{\color{red}}r]
  3 & 1 & -1  &  2 \\
  0 & 2 & 13 & 61\\
  0 & 0 & 72 & 360
\end{bmatrix}\]

We now have the upper triangular system 
\begin{align*}
  3x +y -z &= 2\\
  2y + 13z &= 61\\
  72z &= 360
\end{align*}

Now we use back substitution. From the third equation, we see that $z = \frac{360}{72} = 5$, so $\boxed{z=5}$. 

From the second equation:
\begin{align*}
  2y + 13(5) &= 61\\
  2y + 65 &= 61\\
  2y &= -4\\
  y &= -2
\end{align*}
so $\boxed{y=-2}$. 

Finally, from the first equation:
\begin{align*}
  3x + (-2) - 5 &= 2\\
  3x - 7 &= 2\\
  3x &= 9\\
  x &= 3
\end{align*}
so $\boxed{x=3}$.

Therefore, our solution is $(x,y,z) = (3,-2,5)$.
\end{example}

\begin{example}
  Use Gaussian elimination to solve the system 
  \begin{align*}
    3w +2x +11y + 5z &=25 \\
    -2 w +7x -8y +z &=13\\
    12w-3x+9y + 4z &=-21 \\
    -6w +x -4y -3z &= 15
  \end{align*}
  We have the following augmented matrix 
  \[ \begin{bmatrix}[*3rr@{\quad}|@{\quad}>{\color{black}}r]
  3 & 2& 11  &  5 & 25\\
  -2 & 7 & -8 & 1 & 13\\
  12 & -3 & 9 & 4 & -21 \\
  -6 & 1 & -4 & -3 & 15
\end{bmatrix} \]
For the first column, we have 
  \[ \begin{bmatrix}[*3rr@{\quad}|@{\quad}>{\color{black}}r]
  3 & 2& 11  &  5 & 25\\
  -2 & 7 & -8 & 1 & 13\\
  12 & -3 & 9 & 4 & -21 \\
  -6 & 1 & -4 & -3 & 15
\end{bmatrix} \quad  \underrightarrow{(3R_{2} + 2R_{1}) \to R_{2}} \quad \begin{bmatrix}[*3rr@{\quad}|@{\quad}>{\color{black}}r]
  3 & 2& 11  &  5 & 25\\
  0 & 25 & -2 & 13 & 89\\
  12 & -3 & 9 & 4 & -21 \\
  -6 & 1 & -4 & -3 & 15
\end{bmatrix}\]
\[ \begin{bmatrix}[*3rr@{\quad}|@{\quad}>{\color{black}}r]
  3 & 2& 11  &  5 & 25\\
  0 & 25 & -2 & 13 & 89\\
  12 & -3 & 9 & 4 & -21 \\
  -6 & 1 & -4 & -3 & 15
\end{bmatrix}  \quad  \underrightarrow{(R_{3} -4 R_{1}) \to R_{3}} \quad \begin{bmatrix}[*3rr@{\quad}|@{\quad}>{\color{black}}r]
  3 & 2& 11  &  5 & 25\\
  0 & 25 & -2 & 13 & 89\\
  0 & -11 & -35 & -16 & -121 \\
  -6 & 1 & -4 & -3 & 15
\end{bmatrix} \]
\[ \begin{bmatrix}[*3rr@{\quad}|@{\quad}>{\color{black}}r]
  3 & 2& 11  &  5 & 25\\
  0 & 25 & -2 & 13 & 89\\
  0 & -11 & -35 & -16 & -121 \\
  -6 & 1 & -4 & -3 & 15
\end{bmatrix}\quad  \underrightarrow{(R_{4} +2 R_{1}) \to R_{3}} \quad  \begin{bmatrix}[*3rr@{\quad}|@{\quad}>{\color{black}}r]
  3 & 2& 11  &  5 & 25\\
  0 & 25 & -2 & 13 & 89\\
  0 & -11 & -35 & -16 & -121 \\
  0 & 5 & 18 & 7 & 65
\end{bmatrix} \]
For the second column, we have 
\[  \begin{bmatrix}[*3rr@{\quad}|@{\quad}>{\color{black}}r]
  3 & 2& 11  &  5 & 25\\
  0 & 25 & -2 & 13 & 89\\
  0 & -11 & -35 & -16 & -121 \\
  0 & 5 & 18 & 7 & 65
\end{bmatrix} \quad  \underrightarrow{(25R_{3} +11 R_{2}) \to R_{3}} \quad  \begin{bmatrix}[*3rr@{\quad}|@{\quad}>{\color{black}}r]
  3 & 2& 11  &  5 & 25\\
  0 & 25 & -2 & 13 & 89\\
  0 & 0 & -897 & -257 & -2046 \\
  0 & 5 & 18 & 7 & 65
\end{bmatrix} \]
\[ \begin{bmatrix}[*3rr@{\quad}|@{\quad}>{\color{black}}r]
  3 & 2& 11  &  5 & 25\\
  0 & 25 & -2 & 13 & 89\\
  0 & 0 & -897 & -257 & -2046 \\
  0 & 5 & 18 & 7 & 65
\end{bmatrix}  \quad  \underrightarrow{(5R_{4} - R_{2}) \to R_{4}} \quad \begin{bmatrix}[*3rr@{\quad}|@{\quad}>{\color{black}}r]
  3 & 2& 11  &  5 & 25\\
  0 & 25 & -2 & 13 & 89\\
  0 & 0 & -897 & -257 & -2046 \\
  0 & 0 & 92 & 22 & 236
\end{bmatrix} \]
\[ \begin{bmatrix}[*3rr@{\quad}|@{\quad}>{\color{black}}r]
  3 & 2& 11  &  5 & 25\\
  0 & 25 & -2 & 13 & 89\\
  0 & 0 & -897 & -257 & -2046 \\
  0 & 0 & 92 & 22 & 236
\end{bmatrix} \quad  \underrightarrow{ \left( \frac{1}{2}R_{4} \right) \to R_{4}} \quad \begin{bmatrix}[*3rr@{\quad}|@{\quad}>{\color{black}}r]
  3 & 2& 11  &  5 & 25\\
  0 & 25 & -2 & 13 & 89\\
  0 & 0 & -897 & -257 & -2046 \\
  0 & 0 & 46 & 11 & 118
\end{bmatrix}\]
Finally for the third column, we have 
\[  \begin{bmatrix}[*3rr@{\quad}|@{\quad}>{\color{black}}r]
  3 & 2& 11  &  5 & 25\\
  0 & 25 & -2 & 13 & 89\\
  0 & 0 & -897 & -257 & -2046 \\
  0 & 0 & 46 & 11 & 118
\end{bmatrix}\quad  \underrightarrow{ \left( 897R_{4} + 46 R_{3} \right) \to R_{4}} \quad \begin{bmatrix}[*3rr@{\quad}|@{\quad}>{\color{black}}r]
  3 & 2& 11  &  5 & 25\\
  0 & 25 & -2 & 13 & 89\\
  0 & 0 & -897 & -257 & -2046 \\
  0 & 0 & 0 & -1955 & 11730
\end{bmatrix} \]
We now have the upper triangular system 
\begin{align*}
  3w + 2x + 11y + 5z &= 25\\
  25x - 2y + 13z &= 89\\
  -897y - 257z &= -2046\\
  -1955z &= 11730
\end{align*}

Now we use back substitution. From the fourth equation, we see that $z = \frac{11730}{-1955} = -6$, so $\boxed{z=-6}$. 

From the third equation:
\begin{align*}
  -897y - 257(-6) &= -2046\\
  -897y + 1542 &= -2046\\
  -897y &= -3588\\
  y &= 4
\end{align*}
so $\boxed{y=4}$. 

From the second equation:
\begin{align*}
  25x - 2(4) + 13(-6) &= 89\\
  25x - 8 - 78 &= 89\\
  25x - 86 &= 89\\
  25x &= 175\\
  x &= 7
\end{align*}
so $\boxed{x=7}$. 

Finally, from the first equation:
\begin{align*}
  3w + 2(7) + 11(4) + 5(-6) &= 25\\
  3w + 14 + 44 - 30 &= 25\\
  3w + 28 &= 25\\
  3w &= -3\\
  w &= -1
\end{align*}
so $\boxed{w=-1}$.

Therefore, our solution is $(w,x,y,z) = (-1,7,4,-6)$.
\end{example}

\subsection{Elimination Matrices}

Much like how in \( \mathbb{R}^{3} \) every vector can be written as a linear combination of the standard basis vectors
\( \vb{e}_{1}, \vb{e}_{2}, \vb{e}_{3} \),
we can describe the steps of Gaussian elimination using linear combinations of elimination matrices.

\begin{dfn}
By an \vocab{elimination matrix}, denoted \( \vb{E}_{ji} \) with \( j > i \), we mean a matrix that is equal to the identity except for a scalar \( m \) in the \( (j,i) \) entry.
\end{dfn}


\section{LU Factorization}
This section relies on the key observation that the steps behind elimination are linear. As such, we can record the elimination steps in a matrix. 

\begin{example}
    Suppose that we want to transform the matrix 
    \( A = \begin{bmatrix}
        2 & 1\\
        6 & 8\\
    \end{bmatrix} \) into an upper triangular matrix \( U \). To do this, we need to subtract \( 3 \) copies of row 1 from row 2. We record this operation in a matrix \( E_{21} = \begin{bmatrix}
        1 & 0\\
        -3 & 1\\
    \end{bmatrix} \) The ones on the diagonal indicate that we are not scaling any rows and the \( -3 \) in the \( (i,j) = (2,1) \) entry records the operation of subtracting \( 3 \) copies of row 1 from row 2. Sure enough, 
    \[ \begin{bmatrix}
        1 & 0\\
        -3 & 1\\
    \end{bmatrix} \begin{bmatrix}
        2 & 1\\
        6 & 8\\
    \end{bmatrix} = \begin{bmatrix}
        (1 \cdot 2) + (0 \cdot 6) & (1 \cdot 1) + (0 \cdot 8)\\
        (-3 \cdot 2) + (1 \cdot 6) & (-3 \cdot 1) + (1 \cdot 8)\\
    \end{bmatrix} =\begin{bmatrix}
        2 & 1\\
        0 & 5\\
    \end{bmatrix} \]
  Moreover, since we know what \( E_{21} \) represents, we can calculate \( E^{-1}_{21} \) in our heads. Namely, to undo \( R_{2} - 3R_{1} \to R_{2} \), we simply apply the operation \( R_{2} + 3 R_{1} \to R_{2} \). So 
  \[ E^{-1}_{21} = \begin{bmatrix}
        1 & 0\\
        3 & 1\\
    \end{bmatrix}. \]
    It easy to check that \( E^{-1}_{21} E_{21} = I \). Moreover, 
    \begin{align*}
        E_{21} A &= U \\
        E^{-1}_{21} E_{21} A &= E^{-1}_{21} U \\
        A &= E^{-1}_{21} U
    \end{align*}
    This is the goal. Letting \( E^{-1}_{21} = L \), we have factored \( A \) into a lower triangular matrix times an upper triangular matrix. This is an example of\vocab{LU factorization}. 
\end{example}





\chapter{Subspaces}
\section{The Nullspace}
\begin{dfn}
    Let \( A \) be an \( m \times n \) matrix. The \vocab{nullspace} of \( A \), denoted by \( \mathrm{null}\left( A \right) \) is the set of vectors in \(\vb{v} \in \mathbb{R}^{n} \) such that \( A\vb{v} =\vb{0} \in \mathbb{R}^{m} .\)
\end{dfn}

\begin{example}
    What is the nullspace of the matrix 
    \[ A = \begin{bmatrix*}[r]
        3  & 1 & -7\\
        1 & 0 & -2\\
        2 & 1 & -5\\
    \end{bmatrix*} .\]
    Gaussian elimination comes to the rescue!
    \begin{align*}
         \begin{bmatrix}
        3  & 1 & -7\\
        1 & 0 & -2\\
        2 & 1 & -5\\
    \end{bmatrix} \quad  &\underrightarrow{(3R_{2} - R_{1}) \to R_{2}} \quad  \begin{bmatrix*}[r]
        3  & 1 & -7\\
        0 & -1 & 1\\
        2 & 1 & -5\\
    \end{bmatrix*} \\
    \begin{bmatrix*}[r]
        3  & 1 & -7\\
        0 & -1 & 1\\
        2 & 1 & -5\\
    \end{bmatrix*}  \quad  &\underrightarrow{(3R_{3} - 2R_{1}) \to R_{3}} \quad \begin{bmatrix*}[r]
        3  & 1 & -7\\
        0 & -1 & 1\\
        0 & 1 & -1\\
    \end{bmatrix*}\\
    \begin{bmatrix*}[r]
        3  & 1 & -7\\ 
        0 & -1 & 1\\
        0 & 1 & -1\\
    \end{bmatrix*} \quad  &\underrightarrow{(R_{3} + R_{2}) \to R_{3}}  \quad  \begin{bmatrix*}[r]
        3  & 1 & -7\\
        0 & -1 & 1\\
        0 & 0 & 0\\
    \end{bmatrix*}
    \end{align*}
    Reading off the equations from this reduced matrix, we have the system 
    \begin{align*}
        3x_{1} + x_{2} - 7x_{3} &=0\\
        -x_{2} + x_{3} &= 0
    \end{align*}
    From the second equation, \( x_{2} = x_{3} \). Since \( x_{3} \) is a free variable, we can set it to any value. Let's choose \( x_{3} = 1 \) for convenience, so \( x_{2} = 1 \) as well. Then 
    \begin{align*}
        3x_{1} +1 -7 &=0 \\
        3x_{1} &= 6\\
        x_{1} &= 2
    \end{align*}
    So 
    \[ \boxed{\mathrm{null} \left( A \right) = \mathrm{span} \left\{\;  \begin{bmatrix}
        2\\
        1\\
        1\\
    \end{bmatrix}\;  \right\} \text{ .}} \]
\end{example}

\begin{exercise}
    Let \( A: \mathbb{R}^{4} \to \mathbb{R}^{3} \) be represented by the matrix: 
    \[ A= \begin{bmatrix}
        1 & 2 & 3 & 5\\
        2 & 4 & 8 & 12\\
        3 & 6 & 7 & 13\\
    \end{bmatrix} \]

\begin{enumerate}[label=\textbf{\roman*)}]
\item Convert the augmented matrix \( \left[ A \mid \va{b} \right] \) to an upper triangular system \( \left[ U \mid \va{c} \right] \). 
\item Convert \( \left[ U \mid \va{c} \right] \) to reduced row echelon form \( \left[ R \mid \va{d} \right] \).
\item What is \( \mathrm{col} \left( A \right) \)?
\item What is \( \mathrm{null} \left( A \right) \)?
\item Given a vector \( \va{b} \in \mathrm{col}(A) \), what is the general form of all solutions to \( A\va{x} = \va{b} \)?
\item Find a particular and then a general solution to the \( A \va{x}= \left< 0,6,-6 \right> \).
\end{enumerate}
\end{exercise}
\begin{solution} $ $
    \begin{enumerate}[label=\textbf{\roman*)}]
        \item We have 
        \begin{align*}
            \begin{bmatrix}[*3cr@{\quad}|@{\quad}>{\color{black}}r]
            1 & 2 & 3  &  5 &b_{1}\\
            2 & 4 & 8 & 12 & b_{2}\\
            3 & 6 & 7 & 13 &b_{3}
            \end{bmatrix} \quad  &\underrightarrow{(R_{2} - 2R_{1}) \to R_{2}} \quad  \begin{bmatrix}[*3cr@{\quad}|@{\quad}>{\color{black}}c]
            1 & 2 & 3  &  5 &b_{1}\\
            0 & 0 & 2 & 2 & b_{2}- 2b_{1}\\
            3 & 6 & 7 & 13 &b_{3}
            \end{bmatrix} \\
            \begin{bmatrix}[*3cr@{\quad}|@{\quad}>{\color{black}}c]
            1 & 2 & 3  &  5 &b_{1}\\
            0 & 0 & 2 & 2 & b_{2}- 2b_{1}\\
            3 & 6 & 7 & 13 &b_{3}
            \end{bmatrix}  \quad  &\underrightarrow{(R_{3} - 2R_{3}) \to R_{3}} \quad  \begin{bmatrix}[*3cr@{\quad}|@{\quad}>{\color{black}}c]
            1 & 2 & 3  &  5 &b_{1}\\
            0 & 0 & 2 & 2 & b_{2}- 2b_{1}\\
            0 & 0 & -2 & -2 &b_{3} - 3b_{1}
            \end{bmatrix} \\
             \begin{bmatrix}[*3cr@{\quad}|@{\quad}>{\color{black}}c]
            1 & 2 & 3  &  5 &b_{1}\\
            0 & 0 & 2 & 2 & b_{2}- 2b_{1}\\
            0 & 0 & -2 & -2 &b_{3} - 3b_{1}
            \end{bmatrix} \quad  &\underrightarrow{(R_{2} + R_{3}) \to R_{3}} \quad   \begin{bmatrix}[*3cr@{\quad}|@{\quad}>{\color{black}}c]
            1 & 2 & 3  &  5 &b_{1}\\
            0 & 0 & 2 & 2 & b_{2}- 2b_{1}\\
            0 & 0 & 0 & 0 &b_{2} +b_{3} - 5b_{1}
            \end{bmatrix} 
        \end{align*}
        So our upper triangular system becomes: 
        \begin{equation}\label{2025-10-11 11:02:53}
            \left[ U \mid \va{c} \right] = \begin{bmatrix}[*3cr@{\quad}|@{\quad}>{\color{black}}c]
            1 & 2 & 3  &  5 &b_{1}\\
            0 & 0 & 2 & 2 & b_{2}- 2b_{1}\\
            0 & 0 & 0 & 0 &b_{2} +b_{3} - 5b_{1}
            \end{bmatrix} 
        \end{equation}
        \item Working from our upper-triangular system, we have 
        \begin{align*}
            \begin{bmatrix}[*3cr@{\quad}|@{\quad}>{\color{black}}c]
            1 & 2 & 3  &  5 &b_{1}\\
            0 & 0 & 2 & 2 & b_{2}- 2b_{1}\\
            0 & 0 & 0 & 0 &b_{2} +b_{3} - 5b_{1}
            \end{bmatrix}  \quad  &\underrightarrow{\left( \frac{1}{2}R_{2} \right) \to R_{2}} \quad  \begin{bmatrix}[*3cr@{\quad}|@{\quad}>{\color{black}}c]
            1 & 2 & 3  &  5 &b_{1}\\
            0 & 0 & 1 & 1 &  \frac{1}{2}b_{2}- b_{1}\\
            0 & 0 & 0 & 0 &b_{2} +b_{3} - 5b_{1}
            \end{bmatrix}  \\
             \begin{bmatrix}[*3cr@{\quad}|@{\quad}>{\color{black}}c]
            1 & 2 & 3  &  5 &b_{1}\\
            0 & 0 & 1 & 1 &  \frac{1}{2}b_{2}- b_{1}\\
            0 & 0 & 0 & 0 &b_{2} +b_{3} - 5b_{1}
            \end{bmatrix} \quad  &\underrightarrow{\left( R_{1} - 3 R_{2} \right) \to R_{1}} \quad   \begin{bmatrix}[*3cr@{\quad}|@{\quad}>{\color{black}}c]
            1 & 2 & 0  &  2 &-3b_{1} - \frac{3}{2} b_{2}\\
            0 & 0 & 1 & 1 &  \frac{1}{2}b_{2}- b_{1}\\
            0 & 0 & 0 & 0 &b_{2} +b_{3} - 5b_{1}
            \end{bmatrix}  
        \end{align*}
        So the reduced row echelon form is: 
        \begin{equation}\label{2025-10-11 11:15:17}
            \left[ R \mid \va{d} \right] = \begin{bmatrix}[*3cr@{\quad}|@{\quad}>{\color{black}}c]
            1 & 2 & 0  &  2 &-3b_{1} - \frac{3}{2} b_{2}\\
            0 & 0 & 1 & 1 &  \frac{1}{2}b_{2}- b_{1}\\
            0 & 0 & 0 & 0 &b_{2} +b_{3} - 5b_{1}
            \end{bmatrix}  
        \end{equation}
        \item From \cref{2025-10-11 11:02:53} or \cref{2025-10-11 11:15:17}, we see that our system has a consistent solution only when 
        \begin{equation}\label{2025-10-11 11:34:08}
             -5b_{1}+b_{2}+b_{3}=0 
        \end{equation}
        which is the equation of a plane in \( \mathbb{R}^{3} \). \\ 
        We also see that our pivot columns from our original matrix are 
        \[ \left\{ \; \begin{bmatrix}
            1 \\
            2\\
            3\\
        \end{bmatrix}, \; \begin{bmatrix}
            3\\
            8\\
            7 \\
        \end{bmatrix} \;  \right\} \]
        So 
        \[ \mathrm{span} \left\{ \; \begin{bmatrix}
            1 \\
            2\\
            3\\
        \end{bmatrix}, \; \begin{bmatrix}
            3\\
            8\\
            7 \\
        \end{bmatrix} \;  \right\}  \]
        should also be a representation of the column space. To verify this, take any vector \( \va{v} \in \mathrm{span} \left\{ \left< 1,2,3 \right>, \left< 3,8,7 \right> \right\} \). Then for some scalars \( s,t \in \mathbb{R} \), we have \( \va{v} = s \left< 1,2,3 \right> +t \left< 3,8,7 \right>\) or \( \va{v} = \left< s +3t, 2s+8t, 3s+7t \right> \). Substituting the components of \( \va{v} \) into \cref{2025-10-11 11:34:08}, we have 
        \begin{align*}
            -5 v_{1}+ v_{2}+v_{3} &= -5 \left( s +3t \right) + \left( 2s+8t \right) + \left( 3s+7t \right) \\
            &= \left( -5s +2s +3s \right)  + \left( -15t +8t +7t \right) \\
            &= 0
        \end{align*}
        Since the dimensions of both descriptions of the column space are equal and we have shown that one is contained in the other, it follows that both descriptions are equivalent. 
        \item From  \cref{2025-10-11 11:02:53}, we want to solve 
        \[  \begin{bmatrix}[*3cr@{\quad}|@{\quad}>{\color{black}}c]
            1 & 2 & 3  &  5 &0\\
            0 & 0 & 2 & 2 & 0\\
            0 & 0 & 0 & 0 &0
            \end{bmatrix}  \]
            So 
            \begin{align*}
                x_{1} + 2x_{2} + 3x_{3} + 5x_{4} &=0\\
                x_{3} + x_{4} &=0
            \end{align*}
            If we set \( \boxed{x_{3}=1} \), we have \( \boxed{x_{4} =-1} \) through the second equation. Then our first equation becomes 
            \[ x_{1} + 2x_{2} =2 \]
            Setting \( \boxed{x_{2} =1} \) implies \( \boxed{x_{1}= 0} \) so the first vector in our nullspace is \( \boxed{ \left< 0,1,1,-1 \right>} \). Similarly setting \( \boxed{x_{1} -2} \) implies that \( \boxed{x_{2}=0} \) so the next vector in our nullspace is \( \boxed{\left< 2,0,1,-1 \right>} \). So 
            \[ \mathrm{null} \left( A \right) = \mathrm{span} \left\{ \; \begin{bmatrix}
                0\\
                1\\
                1\\
                -1\\
            \end{bmatrix}, \; \begin{bmatrix}
                2\\
                0\\
                1\\
                -1\\
            \end{bmatrix}  \; \right\} \]
            To show that this is indeed the nullspace of \( A \) pick any \( \va{v} \in \mathrm{null} \left( A \right) \). So \( \va{v} = x \left< 0,1,1, -1, \right> + y \left< 2, 0,1 -1 \right> \) for some \( x, y \in \mathbb{R} \). So \( \va{v} = \left< 2y, x, x+y, -x-y \right> \). Then 
            \begin{align*}
                A \va{v} &= \begin{bmatrix}
        1 & 2 & 3 & 5\\
        2 & 4 & 8 & 12\\
        3 & 6 & 7 & 13\\
    \end{bmatrix} \begin{bmatrix}
        2y \\
        x \\
        x + y \\
        -x-y\\
    \end{bmatrix} \\
    &= \begin{bmatrix}
        2y + 2x + 3 (x+y) + 5 (-x-y)\\
        4y +4x 8 \left( x+y \right) + 12 \left( -x-y \right)\\
        6y + 6x + 7 \left( x+y \right) + 13 \left( -x-y \right)\\
    \end{bmatrix}\\
    &= \begin{bmatrix}
        0\\
        0\\
        0\\
    \end{bmatrix}
            \end{align*}
    \item From \textbf{(iii)}, we know that output space is the span of the pivot columns and from \textbf{(iv)}, we have a description of the null space. So if \( \va{b} = s \left< 1,2,3 \right> + t \left< 3,8,7 \right> \). Then the \( \va{x} \) that solves \( A \va{x} = \va{b} \) is precisely of the form 
    \[ \va{x} = \begin{bmatrix}
        s + 2v \\
        u\\
        t + u +v\\
        -u-v\\
    \end{bmatrix} \quad \text{ for all } u, v \in \mathbb{R}. \]
    \item Clearly the vector \( \left< 0,6,-6 \right> \) is a member of the output space since it solves \( b_{2} + b_{3} - 5b_{1} =0 \). 
    \end{enumerate}
    
\end{solution}






\chapter{The Einstein Summation Convention}

Although this topic is normally outside the scope of a first-year course in linear algebra, this is as good a time as any to begin planting the seeds for understanding tensors. This chapter may be skipped on a first reading.

\section{Repeated Indices in Sums}
We begin with the familiar definition of the dot product:
\[
    \sum_{j=1}^{n} a_{j}x_{j} := a_{1}x_{1} + a_{2}x_{2} + \cdots + a_{n}x_{n}.
\]

However, if the limits of the sum and the range of indices are clear from context, then writing out the summation symbol provides little additional information. Thus, in many areas of mathematics and physics, we adopt the following convention:

\[
    a_{j}x_{j} := \sum_{j=1}^{n} a_{j}x_{j}.
\]

This is known as the \vocab{Einstein summation convention}: whenever an index variable appears exactly twice in a single term, it is understood to be summed over.

% Optional continuation:
This not only saves notation, but also prepares us to work with more complex indexed expressions, especially when dealing with tensors in differential geometry, general relativity, or continuum mechanics.
\begin{dfn} 
    In an indexed expression, an index is called a \vocab{dummy index} (or \vocab{summation index}) if it appears exactly twice in a single term and is implicitly summed over by the Einstein summation convention. 

    An index is called a \vocab{free index} if it is not summed over and appears exactly once in each term of an equation. Free indices determine the components of the resulting expression and must match on both sides of an equation.
\end{dfn}

Some immediate consequences of the above definition are:
\begin{enumerate}[label=\textbf{\roman*)}]
    \item \( a_{ij}x_{j} \neq a_{kj}x_{j} \) because \( i \neq k \)
    \item No index can appear three or more times in an expression.
\end{enumerate}

\begin{example}
    Suppose \( n = 4 \). Consider the expressions \( a_{ii}x_{k} \) and \( a_{ij}x_{j} \). By the Einstein summation convention, we interpret:
    \[
        a_{ii}x_{k} = a_{11}x_{k} + a_{22}x_{k} + a_{33}x_{k} + a_{44}x_{k},
    \]
    since the repeated index \( i \) is implicitly summed over, while \( k \) remains a free index.

    Similarly,
    \[
        a_{ij}x_{j} = a_{i1}x_{1} + a_{i2}x_{2} + a_{i3}x_{3} + a_{i4}x_{4},
    \]
    where the index \( j \) is summed over and \( i \) is free.
\end{example}

\begin{example}
    If \( n=3 \), write down the equations represented by \( y_{i}=a_{ij}x_j \).\\
    Since \( j \) is the summation index, we have 
    \[ y_{i}=a_{i1}x_{1}+a_{i2}x_{2}+a_{i3}x_{3}\]
    which in turn become the equations
    \begin{align*}
        y_{1}&=a_{11}x_{1}+a_{12}x_{2}+a_{13}x_{3}\\
        y_{2}&=a_{21}x_{1}+a_{22}x_{2}+a_{23}x_{3}\\
        y_{3}&=a_{31}x_{1}+a_{32}x_{2}+a_{33}x_{3}
    \end{align*}
    
\end{example}

\subsection{Double Sums}

Suppose we wish to substitute \( y_i = a_{ij}x_j \) into the expression \( Q = b_{ij}y_i x_j \). A careless substitution yields:
\[
Q = b_{ij}a_{ij}x_j x_j,
\]
which is problematic: the index \( j \) appears \textbf{four} times on the right-hand side, violating the rules of summation convention. Recall that a dummy index should appear exactly twice—once per term being summed over. 

To remedy this, we take advantage of the fact that dummy indices are arbitrary labels. We can safely rename one pair of repeated \( j \) indices to a new label, say \( k \). Here's the general procedure:

\begin{enumerate}[label=\textbf{\roman*)}]
    \item Identify the overused dummy index. In this case, \( j \) appears too many times:
    \[
    y_i = a_{ij}x_j, \qquad Q = b_{ij}y_i x_j.
    \]
    
    \item Rename one pair of the repeated \( j \)'s using a new index (e.g., \( k \)):
    \[
    y_i = a_{ik}x_k, \qquad Q = b_{ij}y_i x_j.
    \]
    
    \item Now the substitution is well-formed:
    \[
    Q = b_{ij}a_{ik}x_k x_j.
    \]
\end{enumerate}

\begin{example}
    If \( n = 3 \), write out explicitly the equation given by 
    \[
    Q = b_{ij}a_{ik}x_k x_j.
    \]
    
    First, sum over the \( k \) index:
    \[
    Q = b_{ij}(a_{i1}x_1 + a_{i2}x_2 + a_{i3}x_3)x_j
    = b_{ij}a_{i1}x_1 x_j + b_{ij}a_{i2}x_2 x_j + b_{ij}a_{i3}x_3 x_j.
    \]
    Now sum over the \( j \) index:
    \begin{align*}
    Q &= b_{i1}a_{i1}x_1 x_1 + b_{i2}a_{i1}x_1 x_2 + b_{i3}a_{i1}x_1 x_3 \\
      &\quad + b_{i1}a_{i2}x_2 x_1 + b_{i2}a_{i2}x_2 x_2 + b_{i3}a_{i2}x_2 x_3 \\
      &\quad + b_{i1}a_{i3}x_3 x_1 + b_{i2}a_{i3}x_3 x_2 + b_{i3}a_{i3}x_3 x_3.
    \end{align*}
    Finally, expand over \( i \):
    \begin{align*}
    Q &= b_{11}a_{11}x_1 x_1 + b_{12}a_{11}x_1 x_2 + b_{13}a_{11}x_1 x_3 
      + b_{11}a_{12}x_2 x_1 + b_{12}a_{12}x_2 x_2 + b_{13}a_{12}x_2 x_3 
      + b_{11}a_{13}x_3 x_1 + b_{12}a_{13}x_3 x_2 + b_{13}a_{13}x_3 x_3 \\
      &\quad + b_{21}a_{21}x_1 x_1 + b_{22}a_{21}x_1 x_2 + b_{23}a_{21}x_1 x_3 
      + b_{21}a_{22}x_2 x_1 + b_{22}a_{22}x_2 x_2 + b_{23}a_{22}x_2 x_3 
      + b_{21}a_{23}x_3 x_1 + b_{22}a_{23}x_3 x_2 + b_{23}a_{23}x_3 x_3 \\
      &\quad + b_{31}a_{31}x_1 x_1 + b_{32}a_{31}x_1 x_2 + b_{33}a_{31}x_1 x_3 
      + b_{31}a_{32}x_2 x_1 + b_{32}a_{32}x_2 x_2 + b_{33}a_{32}x_2 x_3 
      + b_{31}a_{33}x_3 x_1 + b_{32}a_{33}x_3 x_2 + b_{33}a_{33}x_3 x_3.
    \end{align*}
    This expansion shows how rapidly the number of terms grows. The compact form
    \[
    Q = b_{ij}a_{ik}x_k x_j
    \]
    is far more efficient.
\end{example}

\begin{example}
Suppose \( y_i = a_{ik}x_k \), and we wish to compute
\[
Q = g_{ij}y_i y_j.
\]

If we substitute directly, we get:
\[
Q = g_{ij}(a_{ik}x_k)(a_{jk}x_k),
\]
which is invalid since the index \( k \) appears four times.

We fix this by renaming one of the repeated \( k \)'s:
\[
Q = g_{ij}(a_{ik}x_k)(a_{jm}x_m),
\]
which now respects the summation convention.
\end{example}

\subsection{Kronecker Delta and Algebraic Rules}

\begin{dfn}
The \vocab{Kronecker delta} is the tensor defined by
\[
\delta_{ij} = \delta^i_j = \delta^{ij} = 
\begin{cases}
    1 & \text{if } i = j, \\
    0 & \text{if } i \neq j.
\end{cases}
\]
\end{dfn}
