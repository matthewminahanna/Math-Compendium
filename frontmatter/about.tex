This compendium contains my solutions to exercises and reformulations of proofs from various mathematics textbooks. It serves as a demonstration of my mathematical abilities and self-directed learning for graduate programs and potential employers.

\subsection*{Background and Purpose}
I completed my master's degree and am currently self-studying mathematics in preparation for a PhD program. This project reflects my commitment to deepening my mathematical understanding while showcasing my problem-solving abilities. My goal is straightforward: to improve at mathematics and demonstrate proficiency in the subject.

\subsection*{Scope and Content}
The topics covered range in difficulty from precalculus to graduate-level mathematics, organized roughly in the order I learned them and according to my interests. This is an ongoing, perpetual project—topics are not set in stone and will continue to evolve as I progress through new material. I include essentially every problem I work through, providing a comprehensive record of my mathematical development.

\subsection*{Methodology}
Each problem is first attempted by hand in \href{https://www.goodnotes.com}{Goodnotes}\footnote{For the love of God, Goodnotes, please stop trying to add A"I" features. Just make your app better.} for iPad before being typeset. All proofs included here have been memorized and recreated from memory rather than copied directly. When I develop an original solution or proof that works, I use that instead of reproducing the textbook's approach.

I also have a deep appreciation for mathematical exposition as an art form—writers like John Lee, John Stillwell, and Tristan Needham exemplify this craft. While my goal isn't to surpass such masters of exposition, I believe that writing substantial amounts of mathematics in \LaTeX{} can only help develop that skill.

\subsection*{For Students}
While this document is optimized as a portfolio piece rather than a pedagogical resource, students may find value in using it as a supplementary exercise collection alongside the original textbooks. If you are a student looking to learn mathematics, I suggest the following approach:
\begin{enumerate}[label=\textbf{\roman*)}]
    \item Locate the textbooks referenced in each section. The boxed ISBN numbers in the bibliography can be searched on library catalogs, book databases, or academic archives to help you find copies.
    \item Use those textbooks as your primary learning resource.
    \item Treat this document as a collection of worked exercises for practice and verification.
\end{enumerate}