

\begin{dfn}
    A \vocab{topological manifold} of dimension $n$ is a second-countable Hausdorff topological space \( M \) that is locally homeomorphic to \( \mathbb{R}^n \), together with a collection of open sets \( U_{\alpha} \) and corresponding homeomorphisms \( \varphi_{\alpha}: U_{\alpha} \to \mathbb{R}^{n} \) where \( \widetilde{U}_{\alpha} = \varphi_{\alpha}(U_{\alpha}) \) denotes the image of \( U_{\alpha} \) under \( \varphi_{\alpha} \) such that 
    \begin{enumerate}[label=\textbf{\roman*)}]
        \item The collection \( \left\{ U_{\alpha} \right\} \) covers \( M \). 
        \item For any \( \alpha, \beta \) such that \( U_{\alpha} \cap U_{\beta} \neq \emptyset \), the \vocab{transition maps} \( \varphi_{\alpha} \circ \varphi_{\beta}^{-1}: \varphi_{\beta}(U_{\alpha} \cap U_{\beta}) \to \varphi_{\alpha}(U_{\alpha} \cap U_{\beta})\) are continuous.
    \end{enumerate}
    The pairs \( (U_{\alpha}, \varphi_{\alpha}) \) are called \vocab{charts}, and the collection \( \{(U_{\alpha}, \varphi_{\alpha})\} \) forms an \vocab{atlas} for \( M \).
    
    Applying additional conditions to the second criterion will give us different types of manifolds. 
    \begin{itemize}
        \item If the transition maps are differentiable, we say that the manifold is differentiable. 
        \item[!] If the transition maps are smooth, then the manifold is smooth. 
        \item If the transition maps are analytic, then the manifold is analytic. 
        \item If the transition maps are holomorphic, then the manifold is complex.
    \end{itemize}
    When we say manifold without any other adjective, we will typically mean that it is a smooth manifold.
\end{dfn}

\begin{exercise}
    Show that the following three definitions of topological manifold are equivalent:
    \begin{enumerate}[label=\textbf{\roman*)}]
        \item The definition given above (where $\widetilde{U}_{\alpha} = \varphi_{\alpha}(U_{\alpha})$ can be any open subset of $\mathbb{R}^{n}$).
        \item The same definition, but requiring each $\varphi_{\alpha}: U_{\alpha} \to \mathbb{R}^{n}$ to be a homeomorphism onto all of $\mathbb{R}^{n}$.
        \item The same definition, but requiring each $\widetilde{U}_{\alpha} = \varphi_{\alpha}(U_{\alpha})$ to be an open ball in $\mathbb{R}^{n}$.
    \end{enumerate}
\end{exercise}
\begin{solution}
     It is sufficient to just show \textbf{(i)} \( \iff \) \textbf{(iii)} and \textbf{(ii)} \( \iff \) \textbf{(iii)}. \\
    \textbf{(iii)} \( \Rightarrow \)  \textbf{(i)} Trivial: open balls are themselves open subsets of \( \mathbb{R}^{n} \).\\
    \textbf{(i)} \( \Rightarrow \) \textbf{(iii)} Suppose \( p \in M \) and pick \( \varphi_{p} \) and \( U_{p} \) containing \( p \) compliant with the conditions of \textbf{(i)}. Given 
    \[ \varphi_{p}^{-1} : \widetilde{U_{p}} \to U_{p}  \]
    we may restrict \( \varphi_{p}^{-1}\) to an an open ball \( B_{\varphi_{p}(p)} \) containing \( \varphi_{p}(p) \) since \( \widetilde{U_{p}} \) is open in \( \mathbb{R}^{n} \). The restriction map 
    \[ \varphi_{p}^{-1} \eval_{B_{\varphi_{p}(p)}} \hspace{-2.5em}: B_{\varphi_{p} \left( p \right)} \to  \varphi_{p}^{-1} \left( B_{\varphi_{p}(p)} \right) \]
    is a local homeomorphism that satisfies the goal conditions. \\ 
    \textbf{(iii)} \( \Rightarrow \) \textbf{(ii)} Suppose that \( \varphi_{\alpha} \) maps the open set \( U_{\alpha} \) to \( B\left( \vb{x} ; \epsilon \right) \) with \( \varphi_{\alpha} (p) = \vb{x} \). We then further define a function \( f: B \left( \vb{x}; \epsilon \right) \to \mathbb{R}^{n} \) by 
    \[ f \left( \vb{y} \right) = \begin{cases}
        \tan{ \left( \frac{\pi}{2 \epsilon} \cdot \norm{\vb{y} -\vb{x}} \right) } \cdot \frac{\vb{y} -\vb{x}}{\norm{\vb{y} - \vb{x}}} \quad  & \text{if } \vb{y} \neq \vb{x} \\
        \vb{0}\quad & \text{if } \vb{y} = \vb{x}
    \end{cases} \]
Although the definition of \( f \) looks a bit unwieldy, all it is really doing is shifting the epsilon ball so it is centered at \( \vb{0}\), then stretching it outward along each ray with a tangent-based scaling that sends the boundary to infinity. The center is set to map to \( \vb{0} \) by hand.  We leave it to the reader to verify that \( f \) is doing what we claim and that it is a homeomorphism between \( B \left( \vb{x} ; \epsilon \right) \) and \( \mathbb{R}^{n} \). The composition \( f \circ \varphi_{\alpha} \) is the desired homeomorphism between \( U_{\alpha} \) and \( \mathbb{R}^{n} \).\\
\textbf{(ii)} \( \Rightarrow \) \textbf{(iii)} Assume the conditions of \textbf{(ii)}. Composing \( \varphi_\alpha \) with \( f^{-1} \) yields a chart whose image is an open ball.
\end{solution}
