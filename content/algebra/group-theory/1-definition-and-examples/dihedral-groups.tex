\begin{example}
Consider an equilateral triangle with vertices colored {\color[HTML]{0000ff} \textbf{blue}}, {\color[HTML]{ff0000} \textbf{red}}, and {\color[HTML]{008000} \textbf{green}}, positioned at labels \( 1, 2 \), and \( 3 \), respectively.

\begin{center}
    \includegraphics[width=0.2\textwidth]{figures/algebra/group_theory/initialtriangle.png}
    \label{fig:initial-triangle}
\end{center}

Using the previously established cycle notation, we impose a group structure on the symmetries of the triangle. The cycle \( \left( a \ \ b \ \ c \right) \) means that the vertex at position \( a \) moves to position \( b \), the vertex at position \( b \) moves to position \( c \), and the vertex at position \( c \) moves to position \( a \).

For example, applying the cycle \( \left( 1 \ \ 2 \right) \) to the triangle:
\begin{itemize}
    \item the {\color[HTML]{0000ff} \textbf{blue vertex}} at position 1 moves to position 2,
    \item the {\color[HTML]{ff0000} \textbf{red vertex}} at position 2 moves to position 1,
    \item the {\color[HTML]{008000} \textbf{green vertex}} at position 3 remains fixed (as it is omitted from the cycle).
\end{itemize}

The resulting triangle appears as follows:
\begin{center}
    \includegraphics[width=0.2\textwidth]{figures/algebra/group_theory/triangle(12).png}
\end{center}

This visual framework aligns naturally with the group operation in the symmetric group. Suppose that after applying \( \left( 1 \ \ 2 \right) \), we then apply the cycle \( \left( 1 \ \ 3 \ \ 2 \right) \). Then:
\begin{itemize}
    \item the {\color[HTML]{ff0000} \textbf{red vertex}}, now at position 1, moves to position 3,
    \item the {\color[HTML]{008000} \textbf{green vertex}} at position 3 moves to position 2,
    \item the {\color[HTML]{0000ff} \textbf{blue vertex}} at position 2 moves back to position 1.
\end{itemize}

The final configuration is:
\begin{center}
    \includegraphics[width=0.2\textwidth]{figures/algebra/group_theory/triangle(23).png}
\end{center}

Note that this is the same result as applying the cycle \( \left( 2 \ \ 3 \right) \) to the \hyperref[fig:initial-triangle]{original triangle}. Indeed,
\[
\left( 1 \ \ 3 \ \ 2 \right)\left( 1 \ \ 2 \right) = \left( 2 \ \ 3 \right).
\]
We can continue this with every possible symmetry of the equilateral triangle to get the following multiplication table:
\begin{center}
    \includegraphics[width=\textwidth]{figures/algebra/group_theory/Dihedral3Table.png}
    \label{fig:Dihedral3Table}
\end{center}
\end{example}


Before we explore the dihedral group in general, let us work through another example: the symmetries of the square. 

\begin{example}
Consider a square with vertices colored 
{\color[HTML]{0b4dff} \textbf{blue}}, 
{\color[HTML]{ff0b33} \textbf{red}}, 
{\color[HTML]{34ff0b} \textbf{green}}, and 
{\color[HTML]{ff8b39} \textbf{orange}} 
and positioned at labels \( 1, 2, 3, 4 \) respectively.
\begin{center}
    \includegraphics[width=0.2\textwidth]{figures/algebra/group_theory/initialsquare.png}
\end{center}

As we will show in generality later, we can generate every element of the dihedral group using a rotation and a flip. It does not matter which flip we choose, so let us pick the \emph{vertical flip} (swapping the left and right vertices), represented by the permutation \( (1\ 4)(2\ 3) \). For our rotation, we will use the \(90^\circ\) clockwise rotation (sending each vertex to the next one in clockwise order), represented by \( (1\ 2\ 3\ 4) \).\\

Applying \( (1\ 2\ 3\ 4) \) to our square has the following effect:
\begin{itemize}
    \item the {\color[HTML]{0b4dff} \textbf{blue}} vertex at position 1 moves to position 2,
    \item the {\color[HTML]{ff0b33} \textbf{red}} vertex at position 2 moves to position 3,
    \item the {\color[HTML]{34ff0b} \textbf{green}} vertex at position 3 moves to position 4, and
    \item the {\color[HTML]{ff8b39} \textbf{orange}} vertex at position 4 moves to position 1.
\end{itemize}
\begin{center}
    \includegraphics[width=0.7\textwidth]{figures/algebra/group_theory/square1234.png}
\end{center}

Similarly, applying \( (1\ 4)(2\ 3) \) (the vertical flip) gives:
\begin{itemize}
    \item the {\color[HTML]{0b4dff} \textbf{blue}} vertex at position 1 moves to position 4,
    \item the {\color[HTML]{ff0b33} \textbf{red}} vertex at position 2 moves to position 3,
    \item the {\color[HTML]{34ff0b} \textbf{green}} vertex at position 3 moves to position 2, and
    \item the {\color[HTML]{ff8b39} \textbf{orange}} vertex at position 4 moves to position 1.
\end{itemize}
\begin{center}
    \includegraphics[width=0.7\textwidth]{figures/algebra/group_theory/square1423.png}
\end{center}

Much like the previous example, we can compose symmetries.  
If we apply a rotation and then a flip, we have  
\( (1\ 4)(2\ 3) \cdot (1\ 2\ 3\ 4) = (1\ 3) \),  
a reflection across the diagonal from vertex 1 to vertex 3.\\
If we first apply a flip followed by a rotation, we have  
\( (1\ 2\ 3\ 4) \cdot (1\ 4)(2\ 3) = (2\ 4) \),  
a reflection across the diagonal from vertex 2 to vertex 4.
\begin{center}
    \includegraphics[width=0.9\textwidth]{figures/algebra/group_theory/squarescompare.png}
\end{center}

By combining the rotation and the flip in different orders, we can generate all eight symmetries of the square. The complete multiplication table is shown on the next page:
\begin{itemize}
    \item The identity element has a white background. 
    \item The rotation elements have grey backgrounds with \( (1\ 2\ 3\ 4) \) having {\color[HTML]{71797e} \textbf{this color background}}, \( (1\ 3)(2\ 4) \) having {\color[HTML]{848884} \textbf{this color background}}, and \( (1\ 4\ 3\ 2) \) having {\color[HTML]{c0c0c1} \textbf{this color background}}.
    \item The reflection elements have brown backgrounds with \( (1\ 4)(2\ 3) \) having {\color[HTML]{d2b48c} \textbf{this color background}}, \( (1\ 2)(3\ 4) \) having {\color[HTML]{c2b280} \textbf{this color background}}, \( (2\ 4) \) having {\color[HTML]{988558} \textbf{this color background}}, and \( (1\ 3) \) having {\color[HTML]{6f4e37} \textbf{this color background}}.
\end{itemize}
\begin{center}
    \includegraphics[width=0.9\textwidth]{figures/algebra/group_theory/D8table.png}
\end{center}
\end{example}
