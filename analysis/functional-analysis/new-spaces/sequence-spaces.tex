\begin{dfn}
    The space \( \ell_{0} \) (or \( c_{00} \left( \mathbb{K} \right) \)) consists of all \( \mathbb{K} \)-valued sequences \( (x_n)_{n=1}^{\infty} \) such that \( x_n = 0 \) for all but finitely many \( n \in \mathbb{N} \).
\end{dfn}

\begin{lemma}
    \( \ell_{0} \) is a \( \mathbb{K} \)-vector space where for all \( \vb{x}=  (x_n)_{n=1}^{\infty}, \vb{y} = (y_n)_{n=1}^{\infty} \in \ell_{0} \) and \( \lambda \in \mathbb{K} \), we have 
    \[ \vb{x} + \vb{y} := (x_n +y_{n})_{n=1}^{\infty}  \quad  \text{ and } \quad  \lambda \vb{x} : = ( \lambda x_n)_{n=1}^{\infty}.\]
\end{lemma}
\begin{proof}
    We need verify that the conditions laid out in \Cref{def:vector-space} hold.
    \begin{description}
        \item[\textbf{Closure under addition:}] Suppose that  \( \vb{x}=  (x_n)_{n=1}^{\infty}, \; \vb{y} = (y_n)_{n=1}^{\infty} \in \ell_{0} \). Then there exists \( k_{1}, k_{2} \in \mathbb{N} \) such that if \( n_{1} > k_{1} \), \( x_{n_{1}} =0 \) and if \( n_{2} > k_{2} \), \( y_{n_{2}} =0 \). So pick \( k = \max \left\{ k_{1}, k_{2} \right\} \). It is clear that if \( n >k \), we have \( x_{n}+y_{n}  =0\) so \( \vb{x}+\vb{y} \in \ell_{0} .\)
        \item[\textbf{Closure under scalar multiplication:}] Again, if \(  \vb{x}=  (x_n)_{n=1}^{\infty} \in \ell_{0} \), there exists a \( k \in \mathbb{N} \) such that if \( n >k \), \( x_{n} =0 \). So for any \( \lambda \in \mathbb{K} \), we can choose the same \( k \) so that if \( n >k \), \( \lambda x_{n} =0 \). So \( \lambda \vb{x} \in \ell_{0} \). 
        \item[\textbf{Commutativity:}] We have 
        \begin{align*}
            \vb{x} + \vb{y} &= (x_n +y_{n})_{n=1}^{\infty} \\
            &= (y_n +x_{n})_{n=1}^{\infty} \tag{Since $x_{n}$ and $y_{n}$ are elements of $\mathbb{K}.$} \\
            &= \vb{y} + \vb{x}
        \end{align*}
        \item[\textbf{Associativity:}] If \( \vb{x}=  (x_n)_{n=1}^{\infty}, \vb{y} = (y_n)_{n=1}^{\infty},  \vb{z}=  (z_n)_{n=1}^{\infty} \in \ell_{0} \), then 
        \begin{align*}
            \left( \vb{x} + \vb{y} \right) + \vb{z} &= (x_n +y_{n})_{n=1}^{\infty} + \vb{z} \\
            &= \left( (x_{n}+y_{n}) + z_{n} \right)_{n=1}^{\infty} \\
            &= \left( x_{n} + (y_{n} +z_{n}) \right)_{n=1}^{\infty} \tag{Inheritance of associativity from $\mathbb{K}$} \\
            &= \vb{x} + \left( y_{n} +z_{n} \right)_{n=1}^{\infty}\\
            &= \vb{x} + \left( \vb{y} + \vb{z} \right)
        \end{align*}
    Similarly if \( \alpha, \beta \in \mathbb{K} \), we have 
    \begin{align*}
        \left( \alpha \beta \right) \vb{x} &= \left( (\alpha \beta)x_{n} \right)_{n=1}^{\infty} \\
        &= \left( \alpha (\beta x_{n}) \right)_{n=1}^{\infty}\\
        &= \alpha \left( \beta x_{n} \right)_{n=1}^{\infty} \\
        &= \alpha \left( \beta \vb{x} \right)
    \end{align*}
    \item[\textbf{Distributive Properties:}] For \( \alpha \in \mathbb{K} \) and \( \vb{x}, \vb{y} \in \ell_{0} \), we have 
    \begin{align*}
        \alpha \left[ \vb{x} + \vb{y} \right] &= \alpha \left[ (x_n +y_{n})_{n=1}^{\infty} \right] \\
        &= \left( \alpha \left( x_{n} +y_{n} \right) \right)_{n=1}^{\infty} \\
        &= \left( \alpha x_{n} + \alpha y_{n}\right)_{n=1}^{\infty} \\
        &= \alpha \vb{x} + \alpha \vb{y}
    \end{align*}
    Similarly, if \( \beta \in \mathbb{K} \), we have 
    \begin{align*}
        \left( \alpha + \beta \right) \vb{x} &= \left( \alpha + \beta \right) \left( x_{n} \right)_{n=1}^{\infty} \\
        &= \left( (\alpha + \beta) x_{n} \right)_{n=1}^{\infty} \\
        &= \left( \alpha x_{n} + \beta x_{n} \right)_{n=1}^{\infty} \\
        &= \alpha \vb{x} + \beta \vb{x}
    \end{align*}
    
    \end{description}
\end{proof}

\begin{exercise}
    Find a basis for \( \ell^{0} \).
\end{exercise}

\begin{solution}
    For each \( k \in \mathbb{N} \), define the sequence 
    \[
        \vb{e}_{k} = (e_{n})_{n=1}^{\infty}, \quad 
        e_{n} = 
        \begin{cases}
            1, & n = k, \\[6pt]
            0, & n \neq k.
        \end{cases}
    \]
    Let 
    \[
        B = \{ \vb{e}_{k} \mid k \in \mathbb{N} \}.
    \]

    Each \(\vb{e}_{k}\) has exactly one nonzero coordinate, hence belongs to \(\ell_{0}\).

 
    Suppose
    \[
        \sum_{j=1}^{m} a_{j} \vb{e}_{k_{j}} = \vb{0}, 
        \qquad a_{j} \in \mathbb{K}, \ k_{j} \in \mathbb{N}.
    \]
    Looking at the \(k_{i}\)-th coordinate, we obtain \(a_{i} = 0\) for each \(i\).  
    Hence the family \(B\) is linearly independent.

    Let \(\vb{x} = (x_{n}) \in \ell^{0}\).  
    By definition, only finitely many \(x_{n}\) are nonzero. If \(x_{j} \neq 0\), then
    \[
        \vb{x} = \sum_{j : \, x_{j} \neq 0} x_{j} \vb{e}_{j},
    \]
    which is a finite linear combination of elements of \(B\). Thus, \(\vb{x}\) lies in the span of \(B\).

\end{solution}


\begin{dfn}
    A \( \mathbb{K} \) sequence \( \vb{x} = \left( x_{n} \right)_{n=1}^{\infty}\) is a member of \( \ell_{p} \) for \( 1 \le p < \infty \) if the sum 
    \[ \sum_{j=1}^{\infty} \abs{x_{j}}^{p} < \infty. \]
\end{dfn}

\begin{example}
    Let \( H \) denote the harmonic sequence \( \left( \frac{1}{n} \right)_{n=1}^\infty \). It is well-known that 
    \[ \sum_{n=1}^{\infty} \frac{1}{n^{2}} = \frac{\pi}{6}  \]
    so \( H \in \ell_{2} \). However, it is also know that 
    \[ \sum_{n =1}^{\infty} \frac{1}{n} \quad \text{ diverges} \]
    so \( H \not \in \ell_{1} \).
\end{example}

\begin{exercise}
    Let \( \mathbf{V} \) be any inner product space. If \( \norm{\vb{x}_{n}} \) converges to \( \norm{\vb{x}} \) and \( \left< \vb{x}_{n}, \vb{x} \right> \) converges to \( \left< \vb{x},\vb{x} \right> \), then \( \vb{x}_{n} \) converges to \( \vb{x} \).
\end{exercise}
\begin{solution}
    If \( \vb{x}_{n} \) is an eventually zero sequence and \( \vb{x} =0 \), the result is trivial so let us assume otherwise.\\
    Since \( \norm{\vb{x}_{n}} \) converges to \( \norm{\vb{x}} \) and and \( \left< \vb{x}_{n}, \vb{x} \right> \) converges to \( \left< \vb{x},\vb{x} \right> \), we can simultaneously chose a sufficiently large \( n \) such that 
    \[ \abs{ \; \norm{\vb{x}_{n}} - \norm{\vb{x}} \; } < \frac{\epsilon^{2}}{2 \left( \norm{\vb{x}_{n}} + \norm{\vb{x}} \right)}\quad \text{and} \quad \abs{\left< \vb{x},\vb{x} \right> -\left< \vb{x}_{n}, \vb{x} \right> } < \frac{\epsilon^{2}}{4}\]
    Now we apply the result from \Cref{exc:equality-for-norm-squared}.
\begin{align*}
    \norm{\vb{x}_{n} - \vb{x}}^{2} &= \norm{\vb{x}_{n}}^{2} + \norm{\vb{x}}^{2} - 2 \Re \left( \left< \vb{x}_n , \vb{x} \right> \right) \\
    &= \norm{\vb{x}_{n}}^{2} + \norm{\vb{x}}^{2} -2 \norm{\vb{x}}^{2}  + 2 \norm{\vb{x}}^{2}  - 2 \Re \left( \left< \vb{x}_n , \vb{x} \right> \right) \tag{add and subtract $2\norm{\vb{x}}^2$}\\
    &= \norm{\vb{x}_{n}}^{2} - \norm{\vb{x}}^{2} + 2 \norm{\vb{x}}^{2} - 2 \Re \left( \left< \vb{x}_n , \vb{x} \right> \right) \\
    & = \norm{\vb{x}_{n}}^{2} - \norm{\vb{x}}^{2} + 2 \left< \vb{x}, \vb{x} \right> - 2 \Re \left( \left< \vb{x}_n , \vb{x} \right> \right) \\
     & = \norm{\vb{x}_{n}}^{2} - \norm{\vb{x}}^{2} + 2 \Re \left( \left< \vb{x}, \vb{x} \right> \right)- 2 \Re \left( \left< \vb{x}_n , \vb{x} \right> \right) \tag{since $\left< \vb{x}, \vb{x} \right>$ is real}\\
     &= \norm{\vb{x}_{n}}^{2} - \norm{\vb{x}}^{2} + 2 \Re \left( \left< \vb{x}, \vb{x} \right> - \left< \vb{x}_n , \vb{x} \right>\right) \tag{linearity of $\Re$}\\
     &= \left( \norm{\vb{x}_{n}} + \norm{\vb{x}} \right) \left( \norm{\vb{x}_{n}} - \norm{\vb{x}} \right) + 2 \Re \left( \left< \vb{x}, \vb{x} \right> - \left< \vb{x}_n , \vb{x} \right>\right) \tag{difference of squares}\\
     & \le  \left( \norm{\vb{x}_{n}} + \norm{\vb{x}} \right)  \abs{ \; \norm{\vb{x}_{n}} - \norm{\vb{x}} \; }+ 2  \Re \left( \left< \vb{x}, \vb{x} \right> - \left< \vb{x}_n , \vb{x} \right>\right) \tag{since $a \le |a|$ for real $a$} \\
     & \le  \left( \norm{\vb{x}_{n}} + \norm{\vb{x}} \right)  \abs{ \; \norm{\vb{x}_{n}} - \norm{\vb{x}} \; } + 2\abs{\left< \vb{x},\vb{x} \right> -\left< \vb{x}_{n}, \vb{x} \right> } \tag{since $|\Re(z)| \le |z|$}\\
     & <  \left( \norm{\vb{x}_{n}} + \norm{\vb{x}} \right) \frac{\epsilon^{2}}{2 \left( \norm{\vb{x}_{n}} + \norm{\vb{x}} \right)} + 2 \frac{\epsilon^{2}}{4} \tag{by choice of $n$}\\
     &= \frac{\epsilon^{2}}{2} + \frac{\epsilon^{2}}{2} \\
    &= \epsilon^{2}
\end{align*}
    So \(  \norm{\vb{x}_{n} - \vb{x}}^{2} < \epsilon^{2} \) for sufficiently large \( n \) or \(  \norm{\vb{x}_{n} - \vb{x}} < \epsilon \), which is what we wanted.
\end{solution}
