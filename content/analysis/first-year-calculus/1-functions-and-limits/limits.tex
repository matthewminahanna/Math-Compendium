\begin{lemma}
    \( \lim_{h \to 0} \frac{e^{h}-1}{h} =1\)
\end{lemma}
\begin{proof}
    Since \( \left( 1 + \frac{h }{n} \right)^{n} < e^{h} < \left( 1 + \frac{ h }{n } \right)^{n+1} \) for every natural number \( n \) and real number \( h \), we have 
  \begin{equation}\label{eq:e-bound-ineq-pos}
\frac{\left( 1 + \tfrac{h}{n} \right)^{n}}{h} 
   < \frac{e^{h} - 1}{h} 
   < \frac{\left( 1 + \tfrac{h}{n} \right)^{n+1} - 1}{h},
   \quad \text{if } h > 0.
\end{equation}

\begin{equation}\label{eq:e-bound-ineq-neg}
\frac{\left( 1 + \tfrac{h}{n} \right)^{n}}{h} 
   > \frac{e^{h} - 1}{h} 
   > \frac{\left( 1 + \tfrac{h}{n} \right)^{n+1} - 1}{h},
   \quad \text{if } h < 0.
\end{equation}

    First looking at \( \frac{\left( 1 + \frac{h }{n} \right)^{n} }{h}  \), we have 
    \begin{align*}
       \frac{\left( 1 + \frac{h }{n} \right)^{n} }{h}  &= \frac{\displaystyle \sum_{j=0}^{n} {n \choose j} \left( \frac{h }{n} \right)^{j} -1 }{h} \\
       &= \frac{\displaystyle \sum_{j=1}^{n} {n \choose j} \left( \frac{h }{n} \right)^{j} }{h} \\
       &=  \sum_{j=1}^{n} {n \choose j} \frac{h^{j-1}}{n^{j}}\\
       &= {n \choose 1} \frac{1}{n} + \sum_{j=2}^{n} {n \choose j} \frac{h^{j-1}}{n^{j}} \\
       &= 1 +  \sum_{j=2}^{n} {n \choose j} \frac{h^{j-1}}{n^{j}}
    \end{align*}
    So taking the limit as \( h \to 0 \), we have 
    \begin{align*}
         \lim_{ h \to 0}  \frac{\left( 1 + \frac{h }{n} \right)^{n} }{h} &= \lim_{h \to 0 } \left(  1 +  \sum_{j=2}^{n} {n \choose j} \frac{h^{j-1}}{n^{j}} \right) \\
         &=1
    \end{align*}
For \( \frac{\left( 1 + \tfrac{h}{n} \right)^{n+1} - 1}{h} \), we will similarly get 
\begin{align*}
         \lim_{ h \to 0}  \frac{\left( 1 + \frac{h }{n} \right)^{n+1} -1}{h} &= \lim_{h \to 0 } \left(  1 + \sum_{j=2}^{n+1} {n+1 \choose j} \frac{h^{j-1}}{n^{j}} \right) \\
         &=1
    \end{align*}
    Applying the squeeze theorem to inequalities \eqref{eq:e-bound-ineq-pos} and \eqref{eq:e-bound-ineq-neg}, we see that
    \[ \boxed{\lim_{h \to 0} \frac{e^{h}-1}{h} =1}. \]
\end{proof}

\begin{exercise}
    Find 
    \( \lim_{x \to \infty} \left( x - x \cos{ \left( \frac{1}{\sqrt{x}} \right) } \right) \sin{ \left( \frac{1}{\sqrt{x}} \right) }\).
\end{exercise}
\begin{solution}
    Let \( t = \frac{1}{\sqrt{x}} \) then 
    \begin{align*}
     \lim_{x \to \infty} \left( x - x \cos{ \left( \frac{1}{\sqrt{x}} \right) } \right) \sin{ \left( \frac{1}{\sqrt{x}} \right) } &= \lim_{t \to 0+} \left( \frac{1}{t^{2}} - \frac{1}{t^{2}} \cos{ \left( t \right) } \right) \sin{ \left( t \right) } \\
     &= \lim_{t \to 0+} \frac{ \sin{ \left( t  \right) } \left( 1- \cos{ \left( t \right) } \right)}{t^{2}} \\
     &= \left[ \lim_{t \to 0+}\frac{ \sin{ \left( t \right) }}{t} \right] \left[ \lim_{t \to 0^{+}}  \frac{1- \cos{ \left( t \right) }}{t}\right] \\
     &= 1 \cdot 0
    \end{align*}
    So 
    \[ \boxed{\lim_{x \to \infty} \left( x - x \cos{ \left( \frac{1}{\sqrt{x}} \right) } \right) \sin{ \left( \frac{1}{\sqrt{x}} \right) } =0 \text{ .}} \]
\end{solution}

\subsection{Sequences} 

\begin{dfn}
    By a \vocab{sequence}, we mean a function 
    \[
        a: \{ 1, 2, \dots, n, \dots \} \to \mathbb{R},
    \]
    which we usually write as 
    \[
        \{ a_{1}, a_{2}, \dots, a_{n}, \dots \},
    \]
    where
    \begin{align*}
        a_{1} &= a(1), \\
        a_{2} &= a(2), \\
              &\ \ \vdots \\
        a_{n} &= a(n), \\
              &\ \ \vdots
    \end{align*}
    We say that a sequence \( \{ a_n \} \) is \vocab{convergent}, or that it \vocab{converges to the real number} \( L \), if for every \( \epsilon > 0 \) there exists a sufficiently large \( N \) such that whenever \( n > N \),
    \[
        \abs{a_{n} - L} < \epsilon.
    \]
    A sequence is \vocab{divergent}, or said to \vocab{diverge}, if no such \( L \) exists.
\end{dfn}

\begin{example}
    Suppose a sequence is given by \( a_{n} = \frac{2n-1}{n} \). Writing out the first members of this sequence, we have 
    \begin{align*}
        a_{1} &= \frac{2(1)-1}{\left( 1 \right)} = 1 \\
        a_{2} &= \frac{2(2)-1}{(2)} = \frac{3}{2} \\
        a_{3} &= \frac{2(3)-1}{(3)} = \frac{5}{3} \\
        a_{4} &= \frac{2(4)-1}{4} = \frac{7}{4} \\
        & \ \ \vdots
    \end{align*}
    Notice that the difference between each successive terms decreases. Indeed if we define \( b_{n} = a_{n+1} -a_{n}\) then 
    \begin{align*}
        b_{1} &= a_{2}- a_{1} = \frac{3}{2}-1 = \frac{1}{2} \\
        b_{2} &= a_{3} - a_{2} = \frac{5}{3} - \frac{3}{2} = \frac{1}{6} \\
        b_{3} &= a_{4} - a_{3} = \frac{7}{4} - \frac{5}{3} = \frac{1}{12} \\\
        &\ \ \vdots
    \end{align*}
    While the tendency for \( b_{n} \) to go zero does not itself guarantee convergence, it is a good sign and indeed \( a_{n} \) converges. We can write 
    \[ a_{n} = \frac{2n}{n} - \frac{1}{n} = 2 - \frac{1}{n} \text{ .}\]
    This suggests that \( a_{n} \) converges to \( 2 \). To show this, we can pick any \( \epsilon >0 \) and can pick an \( n \) sufficiently large such that \( n \epsilon > 1 \) or \( \frac{1}{n} < \epsilon \) Then 
    \begin{align*}
        \abs{2 - a_{n}} &= \abs{ 2 - \left( 2- \frac{1}{n} \right)} \\\
        &= \abs{\frac{1}{n}} \\
        &= \frac{1}{n } \\
        &< \epsilon
    \end{align*}
    We can do something similar to show that \( b_{n} \) converges to \( 0 \) since 
    \begin{align*}
        b_{n} &= a_{n+1} - a_{n} \\
        &= \frac{2 (n+1) -1}{n+1} - \frac{2n -1}{n} \\
        &= \frac{n \left( 2n+1  \right) - \left( n+1  \right)(2n-1)}{n(n+1)} \\
        &= \frac{1}{n(n+1)}
    \end{align*}
    As such the same \( n \) chosen for \( a_{n} \) is overkill since 
    \begin{align*}
        \abs{0 - b_{n}} &= \abs{-\frac{1}{n(n+1)}} \\
        &= \frac{1}{n(n+1)} \\
        &= \frac{1}{n+1} \frac{1}{n} \\
        & < \frac{1}{n^{2}}\\
        & < \epsilon^{2} \\
        &< \epsilon \tag{If $\epsilon < 1$}
    \end{align*}
    So \( b_{n} \) converges "faster" than \( a_{n} \).
\end{example}

\begin{example}
    Let 
    \[ H_{n} = 1 + \frac{1}{2} + \frac{1}{3} + \cdots + \frac{1}{n} = \sum_{k=1}^{n} \frac{1}{n}  \]
    This is called the \vocab{harmonic series}. 
    Writing out the first couple of terms,we have 
  \begin{alignat*}{2}
    H_{1} &= 1 \qquad & &= 1 \\
    H_{2} &= 1 + \frac{1}{2} & &= \frac{3}{2} \\
    H_{3} &= 1 + \frac{1}{2} + \frac{1}{3} & &= \frac{11}{6} \\
    H_{4} &=  1 + \frac{1}{2} + \frac{1}{3} + \frac{1}{4} & &= \frac{25}{12} \\
    H_{5} &= 1 + \frac{1}{2} + \frac{1}{3} + \frac{1}{4}  & &= \frac{137}{60} \\
    & \ \ \vdots & & \ \ \vdots
\end{alignat*}
    If we define \( I_{n}= H_{n+1}-H_{n} \), it is clear that \( I_{n} = \frac{1}{n+1} \), which converges to \( 0 \). But it is \emph{not} the case that \( H_{n} \) converge to any real number \( L \). To see this, 
    \[ 1 + \frac{1}{2} + \underbrace{\frac{1}{3} + \frac{1}{4}}_{\displaystyle >  \frac{1}{4} + \frac{1}{4} = \frac{1}{2}} + \underbrace{ \frac{1}{5} + \frac{1}{6} + \frac{1}{7} + \frac{1}{8}}_{\displaystyle > \frac{1}{8} + \frac{1}{8} + \frac{1}{8} + \frac{1}{8} = \frac{1}{2}} + \cdots \]
    This generalizes 
    \[ 1 + \sum_{k=0}^{n} \frac{1}{2} < H_{2^{n+1}} \]
    This shows that the harmonic series diverges (albeit very slowly) since I can make \( H_{n} \) as large as I want.
\end{example}

\begin{exercise}
    The sequence 
    \[ x_{n} = \frac{4n^{3} -n^{2} +5n}{2n^{3} + 6n^{2} -11} \]
    converges. Find its limit.
\end{exercise}
\begin{solution}
    We can multiply the top and bottom by \( \frac{1}{n^{3}} \) so 
    \begin{align*}
        \frac{4n^{3} -n^{2} +5n}{2n^{3} + 6n^{2} -11} \cdot \frac{ \displaystyle\frac{1}{n^{3}} }{ \displaystyle\frac{1}{n^{3}}} &= \frac{ \displaystyle\frac{4n^{3}}{n^{3}} -  \frac{n^{2}}{n^{3}} +  \frac{5n}{n^{3}}}{ \displaystyle\frac{2n^{3}}{n^{3}} +  \frac{6n^{2}}{n^{3}} -  \frac{11}{n^{3}}} \\
        &= \frac{\displaystyle 4 - \frac{1}{n } + \frac{5}{n^{2}}}{\displaystyle2 + \frac{6}{n} - \frac{11}{n^{3}}}
    \end{align*}
    So as \( n \to \infty \), the terms with \( n \) in the denominator get arbitrarily small so we can just ignore them so then all we have is 
    \[ \boxed{\lim_{n \to \infty}  \frac{4n^{3} -n^{2} +5n}{2n^{3} + 6n^{2} -11} = 2 \text{ .}}\]
\end{solution}

\begin{exercise}
    Find the limit of the sequence 
    \[ x_{n} = \sqrt{n + k} - \sqrt{n}  \] where \( k \) is a fixed constant. 
\end{exercise}
\begin{solution}
    We have 
    \begin{align*}
        \sqrt{n+k} - \sqrt{n} \cdot \frac{\sqrt{n+k} + \sqrt{n}}{\sqrt{n+k} + \sqrt{n}} &= \frac{n+k -n}{\sqrt{n+k} + \sqrt{n}} \\
        &= \frac{k}{ \sqrt{n+k} + \sqrt{n}}
    \end{align*}
    Since \( n \) can be made arbitrarily large and \( k \) is fixed, we have 
    \[ \boxed{\lim_{n \to \infty}\sqrt{n + k} - \sqrt{n} = 0 \text{ .}} \]
\end{solution}

\begin{exercise}
    Show that the sequence 
    \[ x_{n} = \frac{1 \cdot 3 \cdot 5 \cdots (2n-1)}{2 \cdot 4 \cdot 6 \cdots (2n)} \]
    converges.
\end{exercise}
\begin{solution}
    Notice that \( x_{n} < x_{n-1} \) for every \( n \ge 2 \) since 
    \[ x_{n} = x_{n-1} \cdot \frac{2n-1}{2n} \text{ .}\]
    Since the sequence is decreasing and bounded below by \( 0 \), it has no choice but to converge. 
\end{solution}

\begin{exercise}
    Show that the sequence 
    \[ x_{n} = \left( -1  \right)^{n} \frac{1}{n^{2}} \]
    converges.
\end{exercise}
\begin{solution}
    For any \( \epsilon > 0 \), we can choose a sufficiently large \( n \) such that 
    \[ \abs{\frac{1}{n^{2}}} < \epsilon \]
    the \( (-1)^{n} \) does not change this. 
\end{solution}

