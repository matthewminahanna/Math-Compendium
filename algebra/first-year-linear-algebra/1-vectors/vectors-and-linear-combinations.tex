The core object of study in linear algebra is the \emph{vector}.

\begin{dfn}
    By a \vocab{vector} (in this part), often denoted by \( \vb{v} \) or \( \va{v} \), we mean an ordered list of \( n \) numbers:
    \[
        \vb{v} = \left< v_1, v_2, \dots, v_n \right>.
    \]
    The numbers \( v_1, v_2, \dots, v_n \) are called the \vocab{components} or \vocab{entries} of the vector. 
    
    We will restrict our attention to vectors whose components come from either the real numbers \( \mathbb{R} \) or the complex numbers \( \mathbb{C} \). That is, we write \( \vb{v} \in \mathbb{R}^n \) or \( \vb{v} \in \mathbb{C}^n \) depending on context.
    
    Two basic operations on vectors are:
    \begin{itemize}
        \item \textbf{Vector addition:} Given \( \vb{v} = \left< v_1, \dots, v_n \right> \) and \( \vb{w} = \left< w_1, \dots, w_n \right> \), their sum is
        \[
            \vb{v} + \vb{w} = \left< v_1 + w_1, \dots, v_n + w_n \right>.
        \]
        
        \item \textbf{Scalar multiplication:} Given a scalar \( c \in \mathbb{R} \) or \( \mathbb{C} \) and a vector \( \vb{v} = \left< v_1, \dots, v_n \right> \), their product is
        \[
            c \vb{v} = \left< c v_1, \dots, c v_n \right>.
        \]
    \end{itemize}
Throughout linear algebra, we will often represent vectors with \( n \) components as an \vocab{\( n \times 1 \) matrix}, also called a column vector. That is,
\[
    \vb{v} = \left< v_1, \dots, v_n \right> = \begin{bmatrix}
        v_1 \\
        \vdots \\
        v_n
    \end{bmatrix}.
\]
\end{dfn}

This definition carries some immediate consequences on which the rest of linear algebra is built.

For any vector \( \vb{v} = \left< v_1, \dots, v_n \right> \), we can write:
\[
\vb{v} = \begin{bmatrix}
    v_1\\
    v_2\\
    \vdots\\
    v_n
\end{bmatrix} = 
v_1 \begin{bmatrix}
    1\\
    0\\
    \vdots\\
    0
\end{bmatrix} +
v_2 \begin{bmatrix}
    0\\
    1\\
    \vdots\\
    0
\end{bmatrix} +
\cdots +
v_n \begin{bmatrix}
    0\\
    0\\
    \vdots\\
    1
\end{bmatrix}
= \sum_{j=1}^n v_j \vb{e}_j,
\]
where \( \vb{e}_j \) is the vector with a \( 1 \) in the \( j \)th component and \( 0 \) elsewhere.

\begin{example}
    Consider the two-dimensional vector \( {\color[HTML]{7200fa} \vb{v} = \left< 3, 4 \right>} \). We can decompose \( {\color[HTML]{7200fa} \vb{v}} \) as
    \[
        {\color[HTML]{7200fa} \vb{v}} = 3\, {\color[HTML]{ff0000} \vb{e}_1} + 4\, {\color[HTML]{0000ff} \vb{e}_2}
    \]
    or equivalently,
    \[
        {\color[HTML]{7200fa} \left< 3, 4 \right>} = 3\, {\color[HTML]{ff0000} \left< 1, 0 \right>} + 4\, {\color[HTML]{0000ff} \left< 0, 1 \right>}.
    \]

    Geometrically, this means:
    \begin{itemize}
        \item Take 3 copies of the {\color[HTML]{ff0000} unit vector in the \( x \)-direction} and place them tip-to-tail.
        \item Then take 4 copies of the {\color[HTML]{0000ff} unit vector in the \( y \)-direction} and continue placing them tip-to-tail.
        \item The resulting vector \( {\color[HTML]{7200fa} \left< 3, 4 \right>} \) is the diagonal of the resulting "L" shape — the vector sum.
    \end{itemize}

    \begin{center}
        \includegraphics[width=0.5\textwidth]{figures/algebra/linearalgebra/vector34.png}
    \end{center}
\end{example}

This example highlights a subtle but important point about vectors: when we write the vector \( \left< 3,4 \right> \), we have implicitly chosen a basis. In this case, we have chosen the unit vector in the \( x \)-direction to be \( \left< 1,0 \right> \), and the unit vector in the \( y \)-direction to be \( \left< 0,1 \right> \). But this was a \textbf{choice} we made, nature did not hand us this grid.

We could just as well have chosen a different pair of linearly independent vectors and called those our new \( \left< 1,0 \right> \) and \( \left< 0,1 \right> \). This change of basis would then alter the meaning of \( \left< 3,4 \right> \), since that vector is \textbf{defined} as “3 copies of \( \left< 1,0 \right> \)” plus “4 copies of \( \left< 0,1 \right> \).” The numbers 3 and 4 are coordinates relative to a basis, not intrinsic properties of the vector itself. 

\begin{center}
    \includegraphics[width=0.5\textwidth]{figures/algebra/linearalgebra/another34.png}
\end{center}

This figure shows an equally valid way to define \( {\color[HTML]{ff0000} \left< 1,0 \right>} \) and \( {\color[HTML]{0000ff} \left< 0,1 \right>} \), leading to a vector \( {\color[HTML]{7200fa} \left< 3,4 \right>} \) that is distinct from the one in our earlier example.

\vspace{1em}

The key idea is this: mathematics, and any rigorous quantitative discipline, should not depend on how we choose to measure things. If we're describing something objective and external to us, then the way we draw our grid should not affect the truth of what we're describing.

In linear algebra, we will develop tools to \emph{compare grids}. That is, if you and your colleague make observations using different measurement systems (different bases), we need a way to transform your measurements into theirs, and vice versa, without losing the underlying geometric or physical meaning.

\begin{dfn}
    The \vocab{length} or \vocab{norm} of a vector \( \vb{v} = \left< v_{1}, v_{2 }, \dots, v_{n} \right> \) in \( \mathbb{R}^{n} \) is given by 
    \[ \abs{\vb{v}} = \norm{ \vb{v}} = \sqrt{\sum_{j=1}^{n}  \left( v_{j} \right)^{2}} = \sqrt{ \left( v_{1} \right)^{2} + \left( v_{2} \right)^{2} + \cdots + \left( v_{n} \right)^{2}} \]
\end{dfn}

\begin{example}\label{ex:sphere-closest-furthest-vector}
       What are the points on the sphere \( x^{2}+ y^{2}+ z^{2}=4 \) that are closest and furthest from the point \( (x,y,z) = (3,1, -1) \)? \\
       
       For the point closest to \( (3,1,-1) \), we want a vector that points in the same direction as \( \vb{v} = \left< 3,1,-1 \right>\) but has length \( 2 \) (the radius of the sphere). To achieve this, we simply scale the components of \( \vb{v}\) by \( \frac{2}{\norm{\vb{v}}} = \frac{2}{\sqrt{(3)^{2}+ (1)^{2} + (-1)^{2}}} = \frac{2}{\sqrt{11}} = \frac{2 \sqrt{11}}{11}\). So we have the point 
       \[ P_{1} =  \frac{2 \sqrt{11}}{11} \left( 3,1,-1 \right) = \boxed{\left( \frac{6\sqrt{11}}{11}, \frac{2\sqrt{11}}{11}, -\frac{2\sqrt{11}}{11} \right)}  \]
       
       And to find the point furthest from \( (3,1,-1) \), we want a vector pointing in the opposite direction, so we simply reverse the sign:
       \[ P_{2} = -\frac{2 \sqrt{11}}{11} \left( 3,1,-1 \right) = \boxed{\left( -\frac{6\sqrt{11}}{11}, -\frac{2\sqrt{11}}{11}, \frac{2\sqrt{11}}{11} \right)} \]
\end{example}
