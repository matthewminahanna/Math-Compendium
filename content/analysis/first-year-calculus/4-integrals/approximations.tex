\subsection{Riemann Sums}
\subsubsection{Left and Right Riemann Sums}
The idea behind a Riemann sum is very simple: we approximate the area underneath a function using rectangles.

\textbf{Setting Up the Problem:} Suppose we want to find the area under a continuous function \( f(x) \) between \( x = a \) and \( x = b \). The first step is to divide this interval into \( n \) equal pieces.

\textbf{Width of each rectangle:} Since we're dividing the interval \([a,b]\) into \( n \) equal pieces, each piece has width
\[ \Delta x = \frac{b-a}{n} \]

\textbf{The subintervals:} This divides \([a,b]\) into the subintervals
\[ [a, a+\Delta x], \quad [a+\Delta x, a+2\Delta x], \quad [a+2\Delta x, a+3\Delta x], \quad \ldots, \quad [a+(n-1)\Delta x, b] \]

Notice that the left endpoints of these subintervals are \( a, a+\Delta x, a+2\Delta x, \ldots, a+(n-1)\Delta x \), which we can write as \( a + k\Delta x \) for \( k = 0, 1, 2, \ldots, n-1 \).

\textbf{Left Riemann Sum:} To form rectangles, we need both a width and a height. We already have the width (\( \Delta x \)). For the height, we use the function value at the \textbf{left endpoint} of each subinterval.

For the \( k \)-th rectangle (where \( k \) goes from 0 to \( n-1 \)):
\begin{itemize}
    \item \textbf{Width:} \( \Delta x \)
    \item \textbf{Height:} \( f(a + k\Delta x) \)
    \item \textbf{Area:} \( f(a + k\Delta x) \cdot \Delta x \)
\end{itemize}

Adding up all \( n \) rectangles, the total approximate area is
\[ \mathrm{Area}(f) \approx \sum_{k=0}^{n-1} f(a + k \Delta x) \cdot \Delta x \]
where \( \Delta x = \frac{b-a}{n} \).


\begin{figure}[h]
    \centering
    \includegraphics[width=0.9\textwidth]{figures/analysis/calculus/LHRS.png}
    \caption{Left Riemann Sum with $n$ subdivisions. Each rectangle has width $\Delta x = \frac{b-a}{n}$ and height $f(a + k\Delta x)$ for $k = 0, 1, \ldots, n-1$.}
    \label{fig:LHRS}
\end{figure}
