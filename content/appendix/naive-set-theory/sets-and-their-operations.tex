\subsection{Sets}
For much of modern math, the "bottom turtle" for the objects we study are sets.
\begin{dfn}
    A \vocab{set}, roughly speaking, is a collection of elements. We typically denote a set with a capital letter \( X \) and their \vocab{elements} are often denoted with a lowercase letter. \( x \). \\
    If some element \( x  \) belongs to \( X  \), we will write this as 
    \[ x \in X. \] 
Otherwise, if the element \( x  \) is \textbf{not} a member of the set \( X  \), we will denote this by 
\[ x \not \in X. \] 
\end{dfn}
There are multiple ways to describe a set. 
\\For example, we may explicitly list out the elements. 
\begin{example}
    Let 
    \[ X = \{a,b,c \} .\] 
This is precisely the set \( X \) which contains only the elements \( a,b,c \).
\end{example}
We might begin with an existing set and then apply a filter to generate a new set. In other words, if we are given a set \( Y \), we may construct a new set \( X \) of the form 
\[ X = \{x \in Y \mid P(x)\} \] 
where \( P(x) \) is some statement involving the element \( x \) and whenever \( P(x) \) is true, we include the element \( x \) in the set \( X  \).
\begin{example}
Let \( \mathbb{R} \) denote the collection of real numbers. Then we may construct \( \mathbb{R}^{+} \) to be 
\[ \mathbb{R}^{+} = \{x \in \mathbb{R} \mid x > 0\} \] 
which is the set of all positive real numbers.
\end{example}
\begin{dfn}
Let \( X  \) and \( Y \) each be sets. If \( X  \) and \( Y \) each contain the exact same elements, we say that they are the same set and write \( X =Y \).
\end{dfn}
\begin{example}
    If \( X = \left\{ 1,2,3 \right\} \) and \( Y = \{1,2,3\} \), then \( X  \) and \( Y  \) are the same set and we write \( X=Y. \) \\
    If \( Z = \{1,2\} \), it is clear that \( X \neq Z \) and \( Y \neq Z \) since \( 3 \in X \) and \( 3 \in Y \) but \( 3 \not \in Z. \) 
\end{example}

\begin{dfn}
Let \( X  \) and \( Y \) each be sets. We say that \( X \)  is a \vocab{subset} of \( Y \) if \( x \in Y \) whenever \( x \in X \). We write this as 
\[ X \subseteq Y \]. If \( Y \) contains at least one element not contained in \( X \), we say that \( X \) is a \vocab{subset}  subset of \( Y \) and denote it by \( X \subset Y \).
\end{dfn}
\begin{example}
    Using our previous example of \( X,Y \) and \( Z \), it is clear that 
    \[ Z \subseteq X \quad \text{and} \quad  Z \subseteq Y \] 
\end{example}

 \begin{theorem}
For any two sets \( X  \) and \( Y \),
\[ X=Y \iff X \subseteq Y \text{ and } Y \subseteq X. \] 
 \end{theorem}
 \begin{proof}
     \( \Rightarrow \) Suppose that \( X =Y \). Then, by definition, \( X  \) and \( Y \) contain exactly the same elements. Therefore, each element of \( X \) must also be in \(  Y  \) and vice-versa. So it must be the case that \( X \subseteq Y \) and \( Y \subseteq X \). \\
     \( \Leftarrow \) Now suppose that \( X \subseteq Y \) and \( Y \subseteq X \). If \( x \in X \), then, by the fact that \( X \subseteq Y \), we have that \( x \in Y \). In other words, \( X \) contains nno elements that are not also in \( Y \). We can just as easily conclude that \( Y \) contains no elements that cannot be found in \( X \). Since \( X \) and \( Y \) contain exactly the same elements, it follows that \( X=Y \).
 \end{proof}
\begin{dfn}
    The \vocab{empty set}, denoted as \( \varnothing \), is the set with no elements. 
\end{dfn}
\begin{lemma}
    The empty set is unique and it is a subset of every set.
\end{lemma}
\begin{proof}
    It is not hard to show that any other set that contains no elements must be identical to the empty set. To show for any set \( X \), that \( \varnothing \subseteq X \), we will write out explicitly what it means for \( \varnothing \subseteq X \). We have 
    \[ x \in \varnothing \Rightarrow x \in X. \] 
However, the statement \( x \in \varnothing \) is false ,by definition. So the entire implication becomes true. 
\end{proof}
\subsection{Unions, Intersections, and Compliments}
\begin{dfn}
    Let \( X  \) and \( Y  \) be any sets. Then we define the \vocab{union} of the sets \( X \) and \( Y \), denoted by \( X \cup Y \) is the new set defined by 
    \[ X \cup Y = \{z \mid z \in X \text{ or } z \in Y\} \] 
\end{dfn}

\begin{example}
    Let \( X = \left\{ 1,2,3 \right\} \) and \( Y = \left\{ 3,4,5 \right\} \). Then,
    \[ X \cup Y = \left\{ 1,2,3,4,5 \right\} \] 
\end{example}

\begin{dfn}
    Let \( X \) and \( Y \) be sets. Then the \vocab{intersection} of \( X \) and \( Y \), denoted by \( X \cap Y \) is the set 
    \[ X \cap Y = \{z \mid z \in X \text{ and } z \in Y\} \] 
\end{dfn}

\begin{example}
    Again letting \( X = \left\{ 1,2,3 \right\} \) and \( Y = \left\{ 3,4,5 \right\} \). Then,
    \[ X \cap Y = \left\{ 3 \right\} \] 
\end{example}

\begin{theorem}
    For any sets \( X \), \( Y \), and \( Z \), we have 
    \[ X \cap \left( Y \cup Z \right) = \left( X \cap Y \right) \cup \left( X \cap Z \right).\]
\end{theorem}
\begin{proof}
Suppose \( x \in X \cap \left( Y \cup Z \right)\) . Then \(  x \in X \) and \( x \in Y \cup Z \). Since \(  x \in Y \cup Z \), \( x \in Y \) or \( x \in Z \). Without loss of generality, we can assume \( x \in Y \). Since \( x \in X \) and \( x \in Y \), we have \( x \in X \cap Y \). So \( x \in  \left( X \cap Y \right) \cup \left( X \cap Z \right) \). This demonstrates that 
\[ X \cap \left( Y \cup Z \right) \subseteq  \left( X \cap Y \right) \cup \left( X \cap Z \right)\]
For the reverse direction, assume that \( x \in  \left( X \cap Y \right) \cup \left( X \cap Z \right) \). The \( x \in \left( X \cap Y \right) \) or \( x \in \left( X \cap Z \right) \). Again, without loss of generality, assume that \( x \in X \cap Z \). Then \( x \in X \) and \( x \in Z \). Therefore, \( x \in Y \cup Z \). So we can conclude that \(  x \in X \cap \left( Y \cup Z \right) \). Together with the previous conclusion, we have 
\[ X \cap \left( Y \cup Z \right) = \left( X \cap Y \right) \cup \left( X \cap Z \right),\]
as desired.
\end{proof}

\begin{theorem}
    For any sets \( X \), \( Y \), \(  Z \), we have 
    \[ X \cup \left( Y \cap Z \right) = \left( X \cup Y \right) \cap \left( X \cup Z \right)\]
\end{theorem}
\begin{proof}
    Suppose \( x \in  X \cup \left( Y \cap Z \right).\) Then \( x \in X \) or \( x \in Y \cap Z \). Take the case that \( x \in X \). Then clearly \( x \in X \cup Y \) and \( x \in X \cup Z \). So \( x \in \left(  X \cup Y \right) \cap \left( X \cup Z \right) \). If we take the case that \( x \in Y \cap Z \). Then \( x \in Y \) and \( x \in Z \). So clearly \( x \in X \cup Y \) and \( x \in X \cup Z \). Therefore, \( x \in \left(  X \cup Y \right) \cap \left( X \cup Z \right) \). We have shown that 
\[ X \cup \left( Y \cap Z \right) \subseteq \left( X \cup Y \right) \cap \left( X \cup Z \right)\]
Now suppose that \( x \in \left( X \cup Y \right) \cap \left( X \cup Z \right).\) Then \( x \in \left( X \cup Y \right) \) and \( x \in \left( X \cup Z \right) \). Now if \( x \in X \), we trivially have that \(  x \in X \in X \cup \left( Y \cap Z \right) \). So we take the case that \( x \not \in X \). So \( x \) must be contained in \( Y \) and in \( Z \). So \(  x \in Y \cap Z \). So \( x \in X \cup \left(  Y \cap Z \right) \). With the previous conclusion, we have 
 \[ X \cup \left( Y \cap Z \right) = \left( X \cup Y \right) \cap \left( X \cup Z \right),\]
 as desired.
\end{proof}

\begin{dfn}
    Suppose that \( X \) and \( Y \) are sets such that \( Y \subseteq X \). Then the \vocab{set compliment} of \( Y \) relative to \( X \) is the set \( X \minus Y \) (or \( Y^{C} \) if the set \( X \) is understood in context), defined by 
    \[ X-Y = Y^{C }= \{x \in X \mid x \not \in Y\} \] 
\end{dfn}

\begin{example}
    Let \( X = \left\{ 1,2,3,4 \right\} \) and \( Y = \{1.3\} \). Then 
    \[ X \minus Y = \{2,4\} \] 
\end{example}

\begin{exercise}
    For any set \( X \), we have
    \[ X \cap \varnothing = \varnothing \text{ and } X \cup \varnothing = X\] 
\end{exercise}
\begin{solution}
Trivial.
\end{solution}

\begin{theorem}[DeMorgan's Laws]
    Suppose \( X,Y \), and \( Z \) are sets with \( X \subseteq Z \) and \( Y \subseteq Z \). Then 
    \[ \left( X \cup Y \right)^{C}= X^{C} \cap Y^{C} \] and 
    \[ \left( X \cap Y \right)^{C}= X^{C} \cup Y^{C} \] 
\end{theorem}
\begin{proof}
If either \( X \) or \( Y \) are the empty set or the universal set \( Z \), then the conclusion is trivial so we will assume otherwise.\\
We will first show 
\[ \left( X \cup Y \right)^{C}= X^{C} \cap Y^{C} \]
Suppose \( x \in \left( X \cup Y \right)^{C} \). Then, by definition, \( x \in Z \) but \( x \not \in X \cup Y \). Therefore, \( x \not \in X \) and \( x \not \in Y \). This gives us \( x \in X^{C } \) and \( x \in Y^{C} \). So \( x \in X^{C } \cap Y^{C}. \) So 
 \[ \left( X \cup Y \right)^{C}  \subseteq X^{C} \cap Y^{C} \] 
Now assume \( x \in X^{C} \cap Y^{C} \). By definition, \( x \in X^{C} \) and \( x \in Y^{C} \). In other words, \( x \in Z \) but \( x \not \in X \) and \( x \not \in Y \). So \( x \not \in X \cup Y \) or \( x \in \left( X \cup Y \right)^{C} \). Taken with the earlier conclusion, we have 
\[ \left( X \cup Y \right)^{C}= X^{C} \cap Y^{C}. \]
Now we will show 
\[ \left( X \cap Y \right)^{C}= X^{C} \cup Y^{C} \]
Assume that \( x \in \left( X \cap Y \right)^{C} \). Then \( x \not \in X \cap Y\). So it must be the case that \( x \not \in X \) or \( x \not \in Y \) or both. Without loss of generality, we will assume that \( x \not  \in X  \).  Then \( x \in X^{C} \). So then, \( x \in X^{C} \cup Y^{C} \). This gives us 
\[ \left( X \cap Y \right)^{C} \subseteq  X^{C} \cup Y^{C}. \] 
Now assume \( x \in X^{C} \cup Y^{C} \). Without loss of generality, we may assume that \( x \in Y^{C} \). This gives us that \( x \not \in Y \) so \( x \not \in X \cap Y \). So then \( x \in \left( X \cap Y \right)^{C} \). Together with the previous conclusion, we have
\[ \left( X \cap Y \right)^{C}= X^{C} \cup Y^{C} \] 
which concludes this proof.
\end{proof}