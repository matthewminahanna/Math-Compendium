\subsection*{A Method for Memorizing Proofs}
All proofs recreated in this document have been memorized and written from memory rather than copied directly from the source material. One effective technique that I use for proof memorization is to work in pairs:
\begin{enumerate}
    \item Select two theorems, A and B.
    \item Write and annotate (fill in the {gaps in logic}\footnote{\cite{ref:rudin_pma} is good practice for this skill.} or compute specific examples) the proof for theorem A. Review if needed before you proceed to the next step. 
    \item Completely move on from theorem A. Then write and annotate the proof for theorem B.
    \item Close the book or file (or look away) and attempt to recreate the proof for A, then the proof for B.
\end{enumerate}
The key insight: proving B ``purges'' your short-term memory of proof A, and vice versa. This creates a more rigorous test of whether you've truly memorized the material rather than simply holding it in working memory.\\
\textbf{Note:} This technique works equally well for practice problems. Work through problem A, then problem B, then attempt to solve both from scratch without referring to your previous work.

\subsection*{Bloom's Taxonomy in Pure Mathematics}
Memorization is the lowest level of learning in Bloom's Taxonomy, but it's still learning. The hierarchy progresses through six levels:

\begin{enumerate}
    \item \textbf{Memorization:} Recall definitions, theorems, and formulas. For example, memorizing the definition of continuity, the statement of the Intermediate Value Theorem, or the formula for Taylor series.  
    
   \item \textbf{Understand:} Explain concepts in your own words and grasp why theorems are true. For instance, understanding why continuous functions on compact sets attain their extrema, or why the proof of a theorem requires certain hypotheses. You can implement this by omitting certain hypotheses in the statement of a theorem and attempting to prove it—you'll discover exactly why those conditions are needed.\footnote{See: Reverse mathematics.}
    
    \item \textbf{Apply:} Use theorems and techniques to solve problems. This means applying the Mean Value Theorem to prove inequalities, using the definition of convergence to show a sequence converges, or employing integration techniques to evaluate integrals.
    
    \item \textbf{Analyze:} Break down complex problems and recognize which tools are needed. This involves identifying which theorem applies to a given situation, understanding the structure of a proof, or determining why a particular approach fails and what conditions are missing.
    
    \item \textbf{Evaluate:} Assess the validity of proofs, compare different approaches, and judge whether solutions are correct or complete. For example, finding gaps in arguments, determining whether a proof generalizes to other contexts, or critiquing the elegance and efficiency of different solution methods.
    
    \item \textbf{Create:} Construct original proofs, develop new problem-solving approaches, or formulate conjectures. This is the highest level—producing novel mathematics, whether it's finding your own proof of a known theorem, solving an unseen problem, or discovering new patterns and relationships.
\end{enumerate}

Understanding should always be the goal, but sometimes memorizing key results, techniques, or proof structures provides scaffolding that deeper understanding can build upon later. Don't dismiss memorization as worthless—it's the foundation upon which higher-order thinking is built.

\subsection*{The Struggle Hierarchy}
When working on problems, resist the urge to immediately look at solutions. Productive struggle is where real learning happens. Follow this hierarchy:
\begin{enumerate}
    \item \textbf{Struggle independently.} Give yourself substantial time (20-30 minutes minimum for challenging problems or even an hour if you're in the zone) to think, try approaches, and explore dead ends.
    \item \textbf{Struggle with a peer.} Discuss approaches and share insights, but continue working toward the solution together rather than looking it up.
    \item \textbf{Struggle with a mentor.} Seek guidance from someone more experienced who can point you in the right direction without giving away the answer.
    \item \textbf{Look at the first step only.} If still stuck, reveal just the initial approach or first line of the solution, then close it and try to continue on your own.
    \item \textbf{Repeat.} If needed, look at the next step, then try again independently. Continue this cycle until you complete the problem.
\end{enumerate}
Looking at a full solution should be a last resort. When you do, study it carefully, then close it and recreate the entire solution from memory to ensure you've internalized the reasoning.
