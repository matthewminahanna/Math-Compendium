\begin{example}\label{example: line integral of x4xy over triangle}
    Evaluate the line integral of \( \vb{F} \left< x,y \right>= \left( x^{4} , xy\right) \) along the triangular path with vertices \( \left( 0,0 \right) , \left( 0,1 \right)\) and \( \left( 1,0 \right) \). \\
    We have 3 paths to integrate over 
\begin{align*}
    \gamma_{1}(t) &= \left< t, 0 \right> & t \in &\left[ 0,1 \right] \\
    \gamma_{2}(t) &= \left< 1-t,\,t \right> & t \in & \left[ 0,1 \right] \\
    \gamma_{3}(t) &= \left< 0,\,1-t \right> & t \in & \left[ 0,1 \right]
\end{align*}
So we have 
\begin{align*}
    I_{1} &= \int_{0}^{1} \vb{F} \left( \gamma_{1} \left( t \right) \right) \cdot \gamma_{1}'(t) \dd{t}\\
    &= \int_{0}^{1} \left< t^{4}, \left( t \right) \left( 0 \right) \right> \cdot \left< 1,0 \right> \dd{t} \\
    &= \int_{0}^{1} t^{4} \dd{t}\\
    &= \frac{1}{5}t^{5} \eval_{0}^{1}
\end{align*}
So 
\[ \boxed{I_{1} = \frac{1}{5}}. \]
Next we have 
\begin{align*}
    I_{2} &= \int_{0}^{1} \vb{F} \left( \gamma_{2} \left( t \right) \right) \cdot \gamma_{2}'(t) \dd{t} \\
    &= \int_{0}^{1} \left< \left( 1-t \right)^{4}, t-t^{2}\right> \cdot \left< -1,1 \right> \dd{t}\\
    &= \int_{0}^{1} - \left( 1-t\right)^{4} +t-t^{2}\dd{t} \\
    &= \left( \frac{1}{5} \left( 1-t \right)^{5} +\frac{1}{2}t^{2}- \frac{1}{3}t^{3} \right) \eval_{0}^{1}
\end{align*}
So 
\[ \boxed{I_{2} = - \frac{1}{30}} .\]
Finally, 
\begin{align*}
    I_{3} &=  \int_{0}^{1} \vb{F} \left( \gamma_{3} \left( t \right) \right) \cdot \gamma_{3}'(t) \dd{t}  \\
    &= \int_{0}^{1} \left< 0,0 \right> \cdot \left< 0,-1 \right> \dd{t} \\
    &= \int_{0}^{1} 0 \dd{t}
\end{align*}
So 
\[ \boxed{I_{3}=0}. \]
Finally, we have 
\begin{align*}
    I &= I_{1}+I_{2}+I_{3}\\
    &= \frac{1}{5} - \frac{1}{30} + 0
\end{align*}
So 
\[ \boxed{I = \frac{1}{6}} .\]
An easier method of performing this calculation is showcased in \Cref{example: Green's theorem integral of x4xy over triangle}
\end{example}