\subsection{Two Variable Case}


\begin{dfn}
    Let \( T: \mathbb{R}^{2} \to \mathbb{R}^{2} \) be a transformation defined by
    \[
        x = \vb{x}(u, v), \quad y = \vb{y}(u, v),
    \]
    where both \( \vb{x} \) and \( \vb{y} \) are of class \( \mathcal{C}^{1} \) with respect to the variables \( u \) and \( v \). \\
     The \vocab{Jacobian determinant} of \( T \), denoted by \( \abs{\pdv{(x, y)}{(u, v)}} \), is the absolute value of the determinant (orientation is irrelevant for our purposes) of the derivative matrix \( \mathbf{D}[T(u, v)] \). That is,
    \[
       \abs{ \pdv{(x, y)}{(u, v)}} =
        \left|
        \begin{array}{cc}
        \pdv{x}{u} & \pdv{x}{v} \\[1ex]
        \pdv{y}{u} & \pdv{y}{v}
        \end{array}
        \right|.
    \]
\end{dfn}

\begin{example}
    Suppose \( P \) is the parallelogram bounded by \( y= 2x \), \( y=2x-2 \), \( y=x \), and \( y=x+1 \). Evaluate \( \int \int_{P} xy \dd{x} \dd{y} \).\\
    The easiest type of surface to integrate over is a rectangle. With that in mind, notice that 
    \[ y-2x=0 \quad y-2x=-2 \]
    and 
    \[ y-x=0 \quad y-x =1 \]
    lend themselves to a nice substitution of 
    \[ u = y-2x \quad v =y-x .\]
    Now to find \( x \) and \( y \) in terms of \( u \) and \( v \), notice that 
    \[ v-u = (y-x)-(y-2x) \]
    or 
    \[ \boxed{x=v-u}. \]
    Back-substitution gives us 
   \[ v=y-x \Rightarrow v=y- (v-u) \Rightarrow \boxed{y= 2v-u} .\]
   By construction, our limits of integration are 
   \[ -2 \le u \le 0 ,\quad  0 \le v \le 1. \]
   The Jacobian determinant then becomes 
   \begin{align*}
    \abs{ \pdv{(x, y)}{(u, v)}} &=  \left|
        \begin{array}{cc}
        \pdv{x}{u} & \pdv{x}{v} \\[1ex]
        \pdv{y}{u} & \pdv{y}{v}
        \end{array}
        \right|\\
        &=\left|
        \begin{array}{cc}
        \pdv{u}\left( v-u \right)  & \pdv{v} \left( v-u \right) \\[1ex]
        \pdv{u} \left(2v-u \right) & \pdv{v} \left( 2v-u \right)
        \end{array}
        \right|\\
        &=\left|
        \begin{array}{cc}
        -1  & 1 \\[1ex]
       -1&2
        \end{array}
        \right|
   \end{align*}
   So 
   \[ \boxed{ \abs{ \pdv{(x, y)}{(u, v)}} =1} \]
   With everything in place, we are now ready to evaluate 
   \begin{align*}
    \int \int_{P} xy \dd{x} \dd{y} &= \int \int _{P^{*}} \vb{x} \left( u,v \right) \vb{y} \left( u,v \right)  \abs{ \pdv{(x, y)}{(u, v)}} \dd{u} \dd{v}\\
    &= \int_{0}^{1} \int_{-2}^{0} \left( v-u \right) \left( 2v-u \right) \dd{u} \dd{v}\\
    &=\int_{0}^{1} \int_{-2}^{0} 2v^{2}-3vu+u^{2} \dd{u} \dd{v}\\
    &=\int_{0}^{1}\left( 2v^{2}u- \frac{3}{2}v u^{2} + \frac{u^{3}}{3} \eval_{-2}^{0} \right) \dd{v}\\
     &=\int_{0}^{1} 4v^{2}+6v+\frac{8}{3} \dd{v}\\
     &= \frac{4}{3} v^{3} + 3v^{2} + \frac{8}{3}v \eval_{0}^{1}\\
     &= 7
   \end{align*}
So 
\[ \boxed{\int \int_{P} xy \dd{x} \dd{y} =7}. \]   
\end{example}

\subsection{Three Variable Case}

\begin{dfn}
    Let \( T: \mathbb{R}^{3} \to \mathbb{R}^{3} \) be a transformation defined by 
    \[ x = \vb{x} \left( u,v,w \right), \quad y = \vb{y} \left( u,v,w \right) , \quad z = \vb{z} \left( u,v,w \right)  \]
    where \( \vb{x}, \vb{y} \), and \( \vb{z} \) are of class \( \mathcal{C}^{1} \) with respect to \( u,v \) and \( w \).
    The Jacobian determinant of \( T \) is denoted by \( \abs{\pdv{(x,y,z)}{(u,v,w)}} \) and is similarly defined to be 
    \[ \abs{\pdv{(x,y,z)}{(u,v,w)}} =  \left|
        \begin{array}{ccc}
        \pdv{x}{u} & \pdv{x}{v} & \pdv{x}{w} \\[1ex]
        \pdv{y}{u} & \pdv{y}{v}& \pdv{y}{w} \\[1ex]
        \pdv{z}{u} & \pdv{z}{v}& \pdv{z}{w} 
        \end{array}
        \right|.\]
\end{dfn}

\begin{example}
    Calculate the Jacobian determinant for cylindrical coordinates. \\
    We have 
    \[ x = r \cos{ \left( \theta \right) }, \quad  y = r \sin{ \left( \theta \right) }, \quad z=z \]
    So 
    \begin{align*}
         \abs{\pdv{(x,y,z)}{(r,\theta,z)}} &=  \left|
        \begin{array}{ccc}
        \pdv{x}{r} & \pdv{x}{\theta} & \pdv{x}{z} \\[1ex]
        \pdv{y}{r} & \pdv{y}{\theta}& \pdv{y}{z} \\[1ex]
        \pdv{z}{r} & \pdv{z}{\theta}& \pdv{z}{z} 
        \end{array}
        \right|\\
        &=  \left|
        \begin{array}{ccc}
        \pdv{r} \left( r \cos{ \left( \theta \right) } \right) & \pdv{\theta} \left( r \cos{ \left( \theta \right) } \right)& \pdv{z} \left( r \cos{ \left( \theta \right) } \right)\\[1ex]
        \pdv{r} \left( r \sin{ \left( \theta \right) } \right) & \pdv{\theta}  \left( r \sin{ \left( \theta \right) } \right)& \pdv{z} \left( r \sin{ \left( \theta \right) } \right) \\[1ex]
        \pdv{r} \left( z \right) & \pdv{\theta} \left( z \right)& \pdv{z} \left( z \right)
        \end{array}
        \right|\\
        &= \left|
        \begin{array}{ccc}
        \cos{ \left( \theta \right) }& -r \sin{ \left( \theta \right) } & 0 \\[1ex]
       \sin{ \left(  \theta \right) } & r \cos{ \left( \theta \right) }& 0\\[1ex]
       0& 0& 1
        \end{array}
        \right|\\
        &= r
    \end{align*}
    This gives us
    \[ \boxed{ \abs{\pdv{(x,y,z)}{(r,\theta,z)}} =r} \]
\end{example}

\begin{example}
    Calculate the Jacobian determinant for spherical coordinates. \\
    We have 
    \[ x = \rho \sin{ \left( \varphi \right) } \cos{ \left( \theta \right) }, \quad  y= \rho \sin{ \left( \varphi \right) } \sin{ \left( \theta \right) }, \quad z = \rho \cos{ \left( \varphi \right) }.\]
So 
\begin{align*}
    \abs{\pdv{(x,y,z)}{(\rho, \varphi ,\theta)}} &=  \left|
        \begin{array}{ccc}
        \pdv{x}{\rho} & \pdv{x}{\varphi} & \pdv{x}{\theta} \\[1ex]
        \pdv{y}{\rho} & \pdv{y}{\varphi}& \pdv{y}{\theta} \\[1ex]
        \pdv{z}{\rho} & \pdv{z}{\varphi}& \pdv{z}{\theta} 
        \end{array}
        \right|\\\\
        &= \left|
        \begin{array}{ccc}
        \pdv{\rho} \left(  \rho \sin{ \left( \varphi \right) } \cos{ \left( \theta \right) } \right) & \pdv{\varphi}\left(  \rho \sin{ \left( \varphi \right) } \cos{ \left( \theta \right) } \right) & \pdv{\theta} \left(  \rho \sin{ \left( \varphi \right) } \cos{ \left( \theta \right) } \right)\\[1ex]
        \pdv{\rho} \left( \rho \sin{ \left( \varphi \right) } \sin{ \left( \theta \right) } \right) & \pdv{\varphi} \left( \rho \sin{ \left( \varphi \right) } \sin{ \left( \theta \right) } \right)& \pdv{\theta} \left( \rho \sin{ \left( \varphi \right) } \sin{ \left( \theta \right) } \right) \\[1ex]
        \pdv{\rho}\left(  \rho \cos{ \left( \varphi \right) } \right) & \pdv{\varphi} \left(  \rho \cos{ \left( \varphi \right) } \right)& \pdv{\theta} \left(  \rho \cos{ \left( \varphi \right) } \right)
        \end{array}
        \right| \\\\
        &= \left|
        \begin{array}{ccc}
            \sin{ \left( \varphi \right) } \cos{ \left( \theta \right) } & \rho \cos{ \left( \varphi \right) } \cos{ \left( \theta \right) } & - \rho \sin{ \left( \varphi \right) } \sin{ \left( \theta \right) } \\[1ex]
            \sin{ \left( \varphi \right) } \sin{ \left( \theta \right) } & \rho \cos{ \left( \varphi \right) } \sin{ \left(  \theta \right) } & \rho \sin{ \left( \varphi \right) } \cos{ \left( \theta \right) }\\[1ex]
            \cos{ \left( \varphi \right) } & - \rho \sin{ \left( \varphi \right) } & 0
        \end{array}
        \right|
\end{align*}
Unlike the previous example, we elect to explicitly show the determinant calculation here. 
\begin{align*}
    \left|
        \begin{array}{ccc}
            \sin{ \left( \varphi \right) } \cos{ \left( \theta \right) } & \rho \cos{ \left( \varphi \right) } \cos{ \left( \theta \right) } & - \rho \sin{ \left( \varphi \right) } \sin{ \left( \theta \right) } \\[1ex]
            \sin{ \left( \varphi \right) } \sin{ \left( \theta \right) } & \rho \cos{ \left( \varphi \right) } \sin{ \left(  \theta \right) } & \rho \sin{ \left( \varphi \right) } \cos{ \left( \theta \right) }\\[1ex]
            \cos{ \left( \varphi \right) } & - \rho \sin{ \left( \varphi \right) } & 0
        \end{array}
        \right| &= \sin{ \left( \varphi \right) } \cos{ \left( \theta \right) } \left|
        \begin{array}{cc}
     \rho \cos{ \left( \varphi \right) } \sin{ \left(  \theta \right) } & \rho \sin{ \left( \varphi \right) } \cos{ \left( \theta \right) }\\[1ex]
        - \rho \sin{ \left( \varphi \right) } & 0
        \end{array}
        \right|\\
        &- \rho \cos{ \left( \varphi \right) } \cos{ \left( \theta \right) } \left|
        \begin{array}{cc}
    \sin{ \left( \varphi \right) } \sin{ \left( \theta \right) } & \rho \sin{ \left( \varphi \right) } \cos{ \left( \theta \right) }\\[1ex]
        \cos{ \left( \varphi \right) } & 0
        \end{array}
        \right|\\
        &- \rho \sin{ \left( \varphi \right) } \sin{ \left( \theta \right) } \left|
        \begin{array}{cc}
    \sin{ \left( \varphi \right) } \sin{ \left( \theta \right) } & \rho \cos{ \left( \varphi \right) } \sin{ \left(  \theta \right) }\\[1ex]
        \cos{ \left( \varphi \right) } & - \rho \sin{ \left( \varphi \right) }
        \end{array}
        \right|\\
        &= \rho^{2} \sin^{3}{ \left( \varphi \right) } \cos^{2}{ \left( \theta \right) } + \rho^{2} \cos^{2}{ \left( \varphi \right) } \sin{ \left( \varphi \right) } \cos^{2}{ \left( \theta \right) }+ \rho^{2} \sin{ \left(  \varphi \right) }  \sin^{2}{ \left( \theta \right) }\\
        &= \rho^{2} \sin{ \left(  \varphi \right) } \left( \sin^{2}{ \left( \varphi \right) } \cos^{2}{ \left( \theta \right) } + \cos^{2}{ \left( \varphi \right) } \cos^{2}{ \left( \theta \right) } + \sin^{2}{ \left( \theta \right) }\right)\\
        &=\rho^{2} \sin{ \left(  \varphi \right) } \left( \left(  \sin^{2}{ \left( \varphi \right) }  + \cos^{2}{ \left( \varphi \right) } \right)  \cos^{2}{ \left( \theta \right) } + \sin^{2}{ \left( \theta \right) }\right) \\
        &= \rho^{2} \sin{ \left(  \varphi \right) } \left( \cos^{2}{ \left( \theta \right) } + \sin^{2}{ \left( \theta \right) }\right) 
\end{align*}
So 
\[ \boxed{\abs{\pdv{(x,y,z)}{(\rho, \varphi ,\theta)}}  = \rho^{2} \sin{ \left( \varphi \right) }} .\]
\end{example}

\begin{example}
    Evaluate 
    \[ \int \int \int_{W} \exp \left( x^{2}+y^{2}+ z^{2} \right)^{\frac{3}{2}} \dd{x} \dd{y} \dd{z} \]
    where \( W \) is unit ball in \( \mathbb{R}^{3} \).\\
    This question is \emph{begging} to be integrated in spherical coordinates and we will oblige it. \\
    Set \( x^{2} + y^{2}+z^{2} = \rho^{2} \) and the limits of integration become 
    \[ 0 \le \rho \le 1, \quad 0 \le \varphi \le \pi, \quad 0 \le \theta \le 2 \pi . \]
    This gives us 
    \begin{align*}
        \int \int \int_{W} \exp \left( x^{2}+y^{2}+ z^{2} \right)^{\frac{3}{2}} \dd{x} \dd{y} \dd{z}  &= \int_{0}^{1} \int_{0}^{\pi} \int_{0}^{2 \pi} \exp \left( \rho^{2} \right)^{\frac{3}{2}} \abs{\pdv{(x,y,z)}{(\rho, \varphi ,\theta)}} \dd{\theta} \dd{\varphi}  \dd{\rho} \\
        &= \int_{0}^{1} \int_{0}^{\pi} \int_{0}^{2 \pi} \exp \left( \rho^{3} \right) \rho^{2} \sin{ \left( \varphi \right) } \dd{\theta} \dd{\varphi} \dd{\rho} \\
        &= \int_{0}^{1} \exp \left( \rho^{3} \right) \rho^{2} \dd{\rho} \cdot \int_{0}^{\pi} \sin{ \left( \varphi \right) } \dd{\varphi} \cdot \int_{0}^{2 \pi} \dd{\theta}\\
        &= \frac{1}{3} \left(  e-1 \right) \cdot 2 \cdot 2 \pi
    \end{align*}
    So 
    \[ \boxed{\int \int \int_{W} \exp \left( x^{2}+y^{2}+ z^{2} \right)^{\frac{3}{2}} \dd{x} \dd{y} \dd{z} = \frac{4 \pi \left( e-1 \right)}{3}} \]
\end{example}

