
\begin{dfn}
A \vocab{ring} \(R\) is a non-empty set equipped with two binary operations \(+\) and \(\times\), called addition and multiplication, respectively, such that:
\begin{enumerate}[label=\textbf{\roman*)}]
    \item \(R\) is an abelian group under addition. 
    \item Multiplication \(\times\) is associative.
    \item The left and right distributive laws hold: for all \(r,s,t \in R\),
    \[
        r \times (s+t) = (r \times s) + (r \times t) \quad \text{and} \quad (r+s) \times t = (r \times t) + (s \times t).
    \]
\end{enumerate}
The ring \(R\) is \vocab{commutative} if multiplication is commutative. It is \vocab{unitary} if there exists an identity element \(1_R \in R\) such that \(1_R \times r = r \times 1_R = r\) for all \(r \in R\). Multiplication may be written as
\[
r \times s, \quad r \cdot s, \quad \text{or simply } rs,
\]
with juxtaposition \(rs\) usually being the default.
\end{dfn}
