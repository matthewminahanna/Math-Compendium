\begin{example}
    Suppose a drug X has a half-life of \( 45 \) minutes in the blood stream and that it is not safe to have more than \( 175 \)mg of the drug in a patient's blood. If Dr. Alice administered a \( 120 \)mg dose to her patient, how long does she need to wait before she can safely administer another \( 120 \)mg dose?\\
    Dr. Alice needs to wait until the patient's current drug level drops below \( 55 \) mg (since \( 55 + 120 = 175 \)). We need to determine the time when the patient has less than \( 55 \) mg of drug X left in their system, so we have 
\[  120 \left( \frac{1}{2} \right)^{\frac{t }{45}} <55.  \]
Dividing both sides by \( 120 \), 
   \begin{equation}
     \left( \frac{1}{2}  \right)^{ \frac{t }{45}} < \frac{55}{120} = \frac{11}{24}. \label{eqn:6/21/25/1} 
\end{equation}
Taking the natural logarithm of both sides, 
\[ \frac{t}{45} \ln \left( \frac{1}{2} \right) < \ln \left( \frac{11}{24} \right). \]
Since \( \ln \left( \frac{1}{2} \right) < 0 \), dividing both sides by this negative number flips the inequality:
\[ \frac{t}{45} > \frac{\ln \left( \frac{11}{24} \right)}{ \ln \left( \frac{1}{2} \right)}. \]
Therefore,
\[ t >   45 \cdot \frac{\ln \left( \frac{11}{24} \right)}{ \ln \left( \frac{1}{2} \right)} \approx 51 \text{ minutes}. \]
Dr. Alice must wait at least 51 minutes before she can safely administer drug X again.
\textbf{Common Mistake:} We can take a different route following from \Cref{eqn:6/21/25/1}. Namely, we might try taking \( \log_{\frac{1}{2}} \) of both sides:
\[  \frac{t}{45} \log_{\frac{1}{2}} \left( \frac{1}{2} \right) < \log_{\frac{1}{2}} \left( \frac{11}{24} \right). \]
Since \( \log_{\frac{1}{2}} \left( \frac{1}{2} \right) = 1 \), this would give us
\[ \frac{t}{45} < \log_{\frac{1}{2}} \left(  \frac{11}{24} \right), \] 
or \[ t < 45 \log_{\frac{1}{2}} \left(  \frac{11}{24} \right) \approx 51 \text{ minutes}. \]
This solution would incorrectly suggest that Dr. Alice should administer drug X \textit{before} 51 minutes have passed, which would be unsafe for the patient.
\textbf{What went wrong?} The function \( f(x) = \log_{\frac{1}{2}}(x) \) is decreasing since its base \( \frac{1}{2} < 1 \). For any decreasing function \( f \), we have 
\[ x <y \iff f(x) > f(y). \]
When we applied the decreasing function \( \log_{\frac{1}{2}} \) to both sides of the inequality, we should have flipped the inequality sign:
\[ \frac{t}{45} \log_{\frac{1}{2}} \left( \frac{1}{2} \right) > \log_{\frac{1}{2}} \left( \frac{11}{24} \right), \]
which gives us the correct answer: \( t > 51 \) minutes.
\end{example}

\begin{exercise}
   In the popular RPG \textbf{Pokémon}, there are different colorations of pocket monsters often referred to as "shiny" Pokémon. In the current games, the base odds of encountering a shiny Pokémon is \( \frac{1}{4096} \). This can be optimized to \( \frac{1}{683} \) through various in-game methods.
   \begin{enumerate}[label=\textbf{\roman*)}]
    \item How many encounters are required to have a \( 50\% \) chance of encountering a shiny at the base odds of \( \frac{1}{4096} \)? 
    \item How many encounters are required to have a \( 50\% \) chance of encountering a shiny at the optimized odds of \( \frac{1}{683} \)? 
   \end{enumerate}
\end{exercise}
\begin{solution}
    We can tackle both \textbf{(i)} and \textbf{(ii)} simultaneously. Let \( p \) be the probability of encountering a shiny Pokémon in a single encounter, so \( 1-p \) is the probability of not encountering a shiny. The probability of not encountering a shiny in \( n \) independent encounters is \( (1-p)^n \). We want this to equal \( 0.5 \):
    \[ \left( 1-p \right)^{n} = 0.5 \]
    Taking the natural logarithm of both sides:
    \[ n \ln(1-p) = \ln(0.5) \]
    \[ n = \frac{\ln(0.5)}{\ln(1-p)} \]
    
    For \textbf{(i)}, with \( p = \frac{1}{4096} \):
    \[ n = \frac{\ln(0.5)}{\ln(1-\frac{1}{4096})} \approx \frac{-0.693}{-0.000244} \approx 2839 \text{ encounters} \]
    
    For \textbf{(ii)}, with \( p = \frac{1}{683} \):
    \[ n = \frac{\ln(0.5)}{\ln(1-\frac{1}{683})} \approx \frac{-0.693}{-0.00147} \approx 473 \text{ encounters} \]

This exercise shows that the median number of encounters to find a shiny at base odds is approximately 2839, not 4096. The value \( \frac{1}{4096} \) represents the probability per encounter, not the expected number of encounters for a 50\% success rate.
\end{solution}


