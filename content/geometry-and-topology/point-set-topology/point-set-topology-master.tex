\chapter*{Why Topology?}
It's 6th period geometry. Your teacher, yet un-assaulted by chalk dust (but give it time), is talking about when we know two triangles are congruent. SSS, SAS, ASA... wait what? Doesn't she mean equal? Why are we just making up new words for equality? Is Big Math™ just making stuff up for the sake of confusion? \\

You're picking up on a subtle point. Two triangles might \emph{seem} equal if they have the exact same side lengths and angles but they aren't for what seems to be a very nit-picky but important reason, they don't occupy the same points in space. However, we have an easy fix. We can find a map called an \vocab{isometry} that maps the set of points that make one triangle onto the set of points that make up the other. Under this rephrasing, the SSS, SAS, ASA conditions are sufficient conditions that allow us to declare the existence of an isometry that maps one triangle onto another without having to explicitly find one. \\

The idea of finding a "useful" class of maps and conditions that guarantee their existence cannot possibly be more prevalent than it is in topology. However, isometries, while useful, are too rigid, they preserve too much structure. That is why, in topology, we begin with the most general class of functions that preserve spatial (topological) information called \vocab{homeomorphisms}. We will discuss conditions that guarantee their existence and spatial information that they preserve. \\ 

While we will (usually) just stick to homeomorphisms in this part, it is important to mention that homeomorphisms are \emph{too} general. A complete classification of spaces up to homeomorphism will elude us as long as Sisyphus remains at his task. So, in practice, we impose additional structure on topological spaces and restrict to maps that preserve this structure, allowing us to detect much finer invariants.\\ 

While homeomorphisms are the maps, the primary actors for which homeomorphisms act on are \vocab{topological spaces}, which are supposed to capture the notion of closeness. It has taken mathematicians many years to pick out the right abstractions of closeness. A primitive approach might be "two points are close if the distance between them is small." But that just kicks the can down the road. "The distance between two points is small if they are close together." The stroke of genius here is that closeness is a contextual property. If I give you one point, you cannot tell me anything about closeness or farness. This is the first hint that sets are the natural tool to talk about closeness. But what \emph{types} of sets? If I just hand you the full collection of points, there just isn't enough information to construct a notion of closeness. We need to pick the \emph{right} sets. \\ 

Let's see how sets can encode closeness through a game. Suppose that we play a game of darts but the bar hasn't ordered the dartboard. Like the hooligans we are, we proceed anyways. We mark a bulls-eye on the blank wall and each throw our dart a single round before the bar owner gets mad at us. Before we are kicked out of the bar, we must determine a winner. The bar owner obliges us and hands us a bunch of empty cardboard in which we can cut out circles. We make a big circle with the first cardboard and place the center on our imagined bullseye. It covers the two holes in the wall we made so we conclude that our darts landed in the interior of our circle. We make a second circle but this time it is far too small. Both holes we made lie in the exterior of our circle. The third circle is our Goldilocks circle, we see that the hole you made lies in the interior and the hole I made lies in the exterior. You win again (how do you keep winning?). Notice that we arrived at a notion of closeness \emph{without measuring or even assigning a number to anything}. Each circle defines an \vocab{open set}: the collection of all points that would fall strictly inside it. By declaring which sets are "open," we encode the information about closeness we need.

\chapter{Topological Spaces and Continuous Functions}
\section{Definition, Open Sets}
\begin{dfn}
    Let \( X \) be any set. A \vocab{topology} on \( X \) is a collection \( \mathscr{T} \subseteq \mathcal{P}(X) \) such that the following criteria hold:
    \begin{enumerate}[label=\textbf{\roman*)}]
        \item \( \varnothing, X \in \mathscr{T} \).
        \item For any collection of sets \( \left\{ U_{\alpha} \right\}_{\alpha \in \mathcal{A}} \subseteq \mathscr{T} \), we have \[ \bigcup_{\alpha \in \mathcal{A}} U_{\alpha} \in \mathscr{T}. \]
        \item For any finite collection of sets \( \left\{ U_{j} \right\}_{j=1}^n \subseteq \mathscr{T}\), we have \[ \bigcap_{j=1}^{n} U_{j} \in \mathscr{T}. \]
    \end{enumerate}
If a set \( U \subseteq X \) belongs to \( \mathscr{T} \), we call \( U \) an \vocab{open} set. If \( x \in U \) and \( U \) is open, we sometimes refer to \( U \) as an \vocab{open neighborhood} of \( x \).
\end{dfn}

The topology on the empty set is not interesting, so from now on we will assume that \( X \) is non-empty.

\begin{dfn}
    Let \( X \) be a set and let \( \mathscr{T} \) and \( \mathscr{T}' \) be two topologies on \( X \).  If \( \mathscr{T} \subseteq \mathscr{T}' \), we say that \( \mathscr{T}' \) is \vocab{finer} than \( \mathscr{T} \), or equivalently that \( \mathscr{T} \) is \vocab{coarser} than \( \mathscr{T}' \). If \( \mathscr{T} \subset \mathscr{T}' \), we say that \( \mathscr{T}' \) is \vocab{strictly finer} than \( \mathscr{T} \), or that \( \mathscr{T} \) is \vocab{strictly coarser} than \( \mathscr{T}' \). We call the topologies \( \mathscr{T} \) and \( \mathscr{T}' \) \vocab{comparable} if one is finer than the other.
\end{dfn}

\begin{example}
    For any set \( X \), there are two obvious topologies. The \vocab{indiscrete} topology which is just 
    \[ \mathscr{T}_{\mathrm{indiscrete}} =\left\{ \varnothing, X \right\}\]
    and the \vocab{discrete} topology which is just 
    \[ \mathscr{T}_{\mathrm{discrete}} = \mathcal{P} \left( X \right).\]
\end{example}

\begin{exercise}
    Let \( X = \left\{ a,b,c \right\} \). What are all the possible topologies on \( X \)?
\end{exercise}
\begin{solution}
    We have the discrete and indiscrete topologies on \( X \). 
    \[ \mathscr{T}_{\mathrm{discrete}} = \left\{ \varnothing, \left\{ a \right\}, \left\{ b \right\}, \left\{ c \right\}, \left\{ a,b  \right\}, \left\{ a,c  \right\}, \left\{ b,c \right\}, \left\{ a,b,c \right\} \right\} \]
    \[ \mathscr{T}_{\mathrm{indiscrete}} = \left\{ \varnothing, \left\{ a,b,c \right\} \right\} \]
Then we have the topologies that augment the indiscrete topology with a singleton set 
\[ \mathscr{T}_{1} = \left\{ \varnothing, \left\{ a \right\}, \left\{ a ,b, c \right\} \right\},\  \mathscr{T}_{2} = \left\{ \varnothing, \left\{ b \right\}, \left\{ a ,b, c \right\} \right\}, \  \mathscr{T}_{3} = \left\{ \varnothing, \left\{ c \right\}, \left\{ a ,b, c \right\} \right\} \]
\[ \]
We can fill out the rest 
\[ \mathscr{T}_{4} = \left\{ \varnothing, \left\{ a \right\}, \left\{ a,b \right\}, \left\{ a ,b, c \right\} \right\}, \ \mathscr{T}_{5} = \left\{ \varnothing, \left\{ a \right\}, \left\{ a,c \right\},\left\{ a ,b, c \right\} \right\}, \  \mathscr{T}_{6} = \left\{ \varnothing, \left\{ a \right\}, \left\{ b,c \right\}, \left\{ a ,b, c \right\} \right\}\]
\[ \mathscr{T}_{7} = \left\{ \varnothing, \left\{ b \right\}, \left\{ a,b \right\}, \left\{ a ,b, c \right\} \right\}, \  \mathscr{T}_{8} = \left\{ \varnothing, \left\{ b \right\}, \left\{ a,c \right\},\left\{ a ,b, c \right\} \right\}, \ \mathscr{T}_{9} = \left\{ \varnothing, \left\{ b \right\}, \left\{ b,c \right\}, \left\{ a ,b, c \right\} \right\}  \]
\[ \mathscr{T}_{10} = \left\{ \varnothing, \left\{ c \right\}, \left\{ a,b \right\}, \left\{ a ,b, c \right\} \right\}, \ \mathscr{T}_{11} = \left\{ \varnothing, \left\{ c \right\}, \left\{ a,c \right\},\left\{ a ,b, c \right\} \right\}, \  \mathscr{T}_{12} = \left\{ \varnothing, \left\{ c \right\}, \left\{ b,c \right\}, \left\{ a ,b, c \right\} \right\} \]


\[ \mathscr{T}_{13} = \left\{ \varnothing, \left\{ a \right\}, \left\{ b \right\}, \left\{ a,b \right\}, \left\{ a ,b, c \right\} \right\}, \ \mathscr{T}_{14} = \left\{ \varnothing, \left\{ a \right\}, \left\{ c \right\}, \left\{ a,c \right\}, \left\{ a ,b, c \right\} \right\}, \  \mathscr{T}_{15} = \left\{ \varnothing, \left\{ b  \right\}, \left\{ c \right\}, \left\{ b,c \right\}, \left\{ a ,b, c \right\} \right\}\]
\[ \mathscr{T}_{16} = \left\{ \varnothing, \left\{ a \right\}, \left\{a, b \right\}, \left\{ a,c \right\}, \left\{ a ,b, c \right\} \right\}, \ \mathscr{T}_{17} = \left\{ \varnothing, \left\{ b \right\}, \left\{ a,b \right\}, \left\{ b,c \right\}, \left\{ a ,b, c \right\} \right\}, \  \mathscr{T}_{18} = \left\{ \varnothing, \left\{ c  \right\}, \left\{ a,c \right\}, \left\{ b,c \right\}, \left\{ a ,b, c \right\} \right\}\]

\[ \mathscr{T}_{19} = \left\{ \varnothing, \left\{ a \right\}, \left\{ b  \right\}, \left\{a, b \right\} ,\left\{ a,c \right\}, \left\{ a ,b, c \right\} \right\}, \ \mathscr{T}_{20} = \left\{ \varnothing, \left\{ a \right\}, \left\{ c  \right\}, \left\{a, b \right\} ,\left\{ a,c \right\}, \left\{ a ,b, c \right\} \right\}\]
\[ \mathscr{T}_{21} =  \left\{ \varnothing, \left\{ a \right\}, \left\{ b  \right\}, \left\{a, b \right\} ,\left\{ b,c \right\}, \left\{ a ,b, c \right\} \right\}, \ \mathscr{T}_{22} = \left\{ \varnothing, \left\{ a \right\}, \left\{ c  \right\}, \left\{a, c \right\}, \left\{ b,c \right\}, \left\{ a ,b, c \right\} \right\}\]
\[ \mathscr{T}_{23}= \left\{ \varnothing, \left\{ b \right\}, \left\{ c \right\}, \left\{ b,c \right\}, \left\{ a,b \right\}, \left\{ a,b,c \right\} \right\}, \mathscr{T}_{24}= \left\{ \varnothing, \left\{ b \right\}, \left\{ c \right\}, \left\{ b,c \right\}, \left\{ a,c \right\}, \left\{ a,b,c \right\} \right\} \]

\[ \mathscr{T}_{25} = \left\{ \varnothing, \left\{ a,b \right\}, \left\{ a,b,c \right\} \right\}, \  T_{26} = \left\{ \varnothing, \left\{ a,c \right\}, \left\{ a,b,c \right\} \right\}, \  \mathscr{T}_{27} = \left\{ \varnothing, \left\{ b,c \right\}, \left\{ a,b,c \right\} \right\}  \]


\end{solution}

\begin{exercise}
    Let \( X \) be a set and we define the co-finite topology \( \mathscr{T}_{\mathrm{cf}} \) as follows: \( U \) is open in \( \mathscr{T}_{\mathrm{cf}} \) if and only if \( X-U \) is finite or all of \( X \). Show that this is indeed a topology.
\end{exercise}
\begin{solution}
    Clearly \( \varnothing \) and \( X \) each belong to \( \mathscr{T}_{\mathrm{cf}} \), so we will jump right into verifying closure under arbitrary unions and finite intersections. \\
    Let \( U_{\alpha \in A} \in \mathscr{T}_{\mathrm{cf}} \). We want to show that 
    \[ \bigcup_{\alpha \in A} U_{\alpha} \in \mathscr{T}_{\mathrm{cf}} \]
    or that 
    \[ X - \left(  \bigcup_{\alpha \in A} U_{\alpha} \right) \]
    is finite. We can apply one of DeMorgan's laws to the above expression to get 
    \[ X - \left(  \bigcup_{\alpha \in A} U_{\alpha} \right) = \bigcap_{\alpha \in A } \left( X - U_{\alpha} \right)\]
Since each \( U_{\alpha} \) belongs to \( \mathscr{T}_{\mathrm{cf}} \), each \( X- U_{\alpha} \) is finite. Therefore \( \bigcap_{\alpha \in A } \left( X - U_{\alpha} \right) \) is certainly finite. This establishes that \( \bigcup_{\alpha \in A} U_{\alpha} \in \mathscr{T}_{\mathrm{cf}} \).\\
Now for finite intersections, suppose that \( \{U_{1}, \dots U_{n}\} \subseteq \mathscr{T}_{\mathrm{cf}} \). We want to show that 
\[ \bigcap_{j=1}^{n} U_{n} \in \mathscr{T}_{\mathrm{cf}}\] or 
\[ X - \left( \bigcap_{j=1}^{n} U_{j} \right) \]
is finite. Again, we apply one of DeMorgan's laws to get 
\[ X - \left( \bigcap_{j=1}^{n} U_{j} \right) = \bigcup_{j=1}^{n } \left( X- U_{j}\right).\]
Since each \( U_{j} \in \mathscr{T}_{\mathrm{cf}} \), each \( X-U_{j} \) is finite. This implies that \( \bigcup_{j=1}^{n }\left( X-U_{j} \right) \) is finite so \( \bigcap_{j=1}^{n }U_{j} \in \mathscr{T}_{\mathrm{cf}}\). This concludes the proof.
\end{solution}

\begin{dfn}
    Let \( X \) be a topological space. The \vocab{interior} of a set \( A \) of \( X \), denoted by \( \mathrm{Int} \left( A \right) \) is defined to be the set: 
    \[ \mathrm{Int} \left( A \right) = \bigcup \left\{ U \subseteq A \  \middle| \ U \text{ is an open set.} \right\} \]
    In other words, \( \mathrm{Int}(A) \) is the union of all open sets contained in \( A \). \\ 
    A point \( x \) of \( A \) is called an \vocab{interior point} of \( A \) it is a member of \( \mathrm{Int} \left( A \right) \).
\end{dfn}

\begin{dfn}
    Let \( X \) be a topological space. The \vocab{exterior} of a set \( A \), denoted by \( \mathrm{Ext}(A) \) is defined to be 
    \[ \mathrm{Ext} \left( A \right) = \mathrm{Int} \left( X-A \right) .\]
\end{dfn}

\begin{dfn}
    The \( X \) be a topological space. The \vocab{boundary} of a set \( A \), denoted as \( \partial \left( A \right) \), is the set 
    \[ \partial \left( A \right) = X - \left( \mathrm{Int} \left( A \right) \cup \mathrm{Ext} \left( A \right) \right).\]
\end{dfn}

\begin{lemma}\label{thm:open-set-equals-its-interior}
    A set \( U \) of a topological space \( X \) is open if and only if \( U = \mathrm{Int}(U) \).
\end{lemma}
\begin{proof}
    \( (\Rightarrow) \) Suppose that \( U \) is open. We want to show that \( U = \mathrm{Int}(U) \). \( \mathrm{Int}(U) \subseteq U \) is obvious by definition so we just need to show that \( U \subseteq \mathrm{Int}(U) \). Since \( \mathrm{Int}(U) \) is the union of all open subsets of \( U \) and \( U \) is open, it follows that \(U \subseteq  \mathrm{Int}(U) \). \\ 
    \( (\Leftarrow )\) Suppose \( \mathrm{Int}(U) = U \). Since \( \mathrm{Int}(U) \) is the union of all open subsets of \( U \), \( \mathrm{Int} \left( U \right) \) is open and hence, \( U \) is open.
\end{proof}

\begin{corollary}
    The exterior of a set \( A \) in a topological space is an open set. 
\end{corollary}












\section{Closed Sets}
\begin{dfn}
    Let \( X \) be a topological space. A set \( C \subseteq X \) is said to be \vocab{closed} in \( X \) if \( X - C \) is open in \( X \).
\end{dfn}

\begin{theorem}
    The following are equivalent: 
    \begin{enumerate}[label=\textbf{\Roman*)}]
        \item There is a set \( \mathscr{C} \subseteq \mathcal{P} \left( X \right) \) such that: 
        \begin{enumerate}[label=\textbf{\roman*)}]
            \item \( \varnothing, X \in \mathscr{C} \)
            \item For any arbitrary collection \( \left\{ C_{\alpha} \right\}_{\alpha \in A} \subseteq  \mathscr{C} \), \( \bigcap_{\alpha \in A } C_{\alpha} \in \mathscr{C} \)
            \item For any finite collection \( \left\{ C_{k} \right\}_{k=1}^{n} \), \( \bigcup_{k=1 }^{n } C_{k} \in \mathscr{C} \)
        \end{enumerate}
        \item There is a topology on \( X \) whose collection of closed sets is precisely \( \mathscr{C} \).
    \end{enumerate}
    This is to say that we could have defined a topology on \( X \) in terms of closed sets.
\end{theorem}
\begin{proof}
   \( \left(  \Rightarrow  \right) \) Assume the conditions of \textbf{(I)} and define 
   \[ \mathscr{T} = \left\{ U \in \mathcal{P} \left( X \right) \ \middle| \ X-U \in \mathscr{C} \right\} \]
   We want to show that \( \mathscr{T} \) is indeed a topology. \\ 
   \( \varnothing \in \mathscr{T}\) since \( X = X- \varnothing \in \mathscr{C} \) and similarly \( X \in \mathscr{T} \) since \( \varnothing = X-X \in \mathscr{C} \). \\ 
   Pick some arbitrary subcollection \( \left\{ U_{\alpha} \right\}_{\alpha \in A} \) of \( \mathscr{T} \). We want to show that \(  \bigcup_{\alpha \in A} U_{\alpha} \in \mathscr{T} \) or, equivalently, \( X - \left(  \bigcup_{\alpha \in A} U_{\alpha} \right) \in \mathscr{C} \)
   \begin{align*}
    X - \left(  \bigcup_{\alpha \in A} U_{\alpha} \right)  &= \bigcap_{\alpha \in A} \left( X-U_{\alpha} \right) \\
    & \in \mathscr{C}
   \end{align*}
   Similarly, if \( \left\{ U_{k} \right\}_{k=1}^{n} \) is a finite subcollection of \( \mathscr{T} \), we want to show that \( \bigcap_{k=1 }^{n} U_{k} \in \mathscr{T} \) or, equivalently, \( X - \left( \bigcap_{k=1}^{n} U_{k}\right) \in \mathscr{C} \)
   \begin{align*}
    X - \left( \bigcap_{k=1}^{n} U_{k}\right) &= \bigcup_{k=1}^{n} \left( X- U_{k} \right) \\
    & \in \mathscr{C}
   \end{align*}
   So \( \mathscr{T} \) is a topology defined on \( X \) whose collection of closed sets is, by construction, \( \mathscr{C} \). \\ 
 \(
\left( \Leftarrow \right)
\)
Assume \( X \) is equipped with a topology \( \mathscr{T} \), and let
\[
\mathscr{C} = \{\, X - U \mid U \in \mathscr{T} \,\}
\]
be the collection of closed sets. We check that \(\mathscr{C}\) satisfies \textbf{(i)}–\textbf{(iii)}.\\
\textbf{(i)} Since \( X \) and \( \varnothing \) are open, their complements
\[
X - X = \varnothing, \qquad X - \varnothing = X
\]
are closed, so \( \varnothing, X \in \mathscr{C} \).\\
\textbf{(ii)} Let \( \{ C_{\alpha} \}_{\alpha \in A} \subseteq \mathscr{C} \). For each \( \alpha \), choose an open set \( U_{\alpha} \) with
\( C_{\alpha} = X - U_{\alpha} \). Then
\[
\bigcap_{\alpha \in A} C_{\alpha}
= \bigcap_{\alpha \in A} (X - U_{\alpha})
= X - \bigcup_{\alpha \in A} U_{\alpha},
\]
and since an arbitrary union of open sets is open, the right-hand side is in \( \mathscr{C} \).\\
\textbf{(iii)} For a finite collection \( C_1, \dots, C_n \in \mathscr{C} \), write
\( C_k = X - U_k \) with each \( U_k \) open. Then
\[
\bigcup_{k=1}^n C_k
= \bigcup_{k=1}^n (X - U_k)
= X - \bigcap_{k=1}^n U_k,
\]
and because a finite intersection of open sets is open, this lies in \( \mathscr{C} \) as well.\\
Thus \(\mathscr{C}\) satisfies the three conditions, completing the proof.
\end{proof}


\begin{dfn}
    A point \( x \) of a topological space \( X \) is said to be a \vocab{limit point} of a set \( A \subseteq X \) if every deleted open neighborhood of \( x \) contains a point of \( A \). That is to say; for every open set \( U \) that contains \( x \): 
    \[ \left( U - \left\{ x \right\} \right) \cap A \neq \varnothing.\]
    The collection of limit points of \( A \) is denoted by \( A' \).
\end{dfn}

\begin{theorem}
    A subset \( C \) of a topological space \( X \) is closed if and only if it contains all its limit points. In other words,
    \[ C \text{ is closed} \iff C' \subseteq C .\]
\end{theorem}
\begin{proof}
    \( (\Rightarrow) \) Suppose that \( C \) is closed. Pick some \( x \notin C \). So \( x \in X - C \). Since \( C \) is closed, \( X - C \) is open. But that means we have found an open neighborhood, particularly \( X - C \), of \( x \) that is disjoint from \( C \). So \( x \) cannot be a limit point of \( C \). Since the assumption \( x \notin C \) leads to the conclusion that \( x \) is not a limit point of \( C \), it follows that \( C \) must contain all its limit points.\\
    \( (\Leftarrow) \) Suppose that \( C \) contains all its limit points. Consider the set \( X - C \). Since \( C \) contains all its limit points, for every element \( x \in X - C \), we know that \( x \) is not a limit point of \( C \). Therefore, there exists a neighborhood \( U \) of \( x \) for which \( (U - \{x\}) \cap C = \varnothing \). Since \( x \in X - C \) (so \( x \notin C \)), this implies that \( U \cap C = \varnothing \), and hence \( U \subseteq X - C \). This shows that every point of \( X - C \) is an interior point of \( X - C \), or equivalently \( X - C = \mathrm{Int}(X - C) \). By \cref{thm:open-set-equals-its-interior}, this implies \( X - C \) is open and hence \( C \) is closed.
\end{proof}

\begin{dfn}
    The \vocab{closure} of a set \( A \) in a topological space \( X \), denoted as \( \overline{A} \), is the set 
    \[ \overline{A} = \bigcap \; \left\{ A \subseteq C \; \middle| \; C \text{ is a closed set.} \right\} \]
\end{dfn}


\section{Basis for a Topology}
\begin{dfn}\label{def: basis for a topology}
    If \( X \) is a set, we define a \vocab{basis} \( \mathscr{B} \) for a topology to be a collection of subsets of \( X \) that satisfies the following criteria:
    \begin{enumerate}[label=\textbf{\roman*)}]
        \item For every \( x \in X \), there is some \( B \in \mathscr{B} \) such that \( x \in B \). 
        \item For every \( B_{1}, B_{2} \in \mathscr{B} \) and \( x \in B_{1} \cap B_{2} \) there is some \( B_{3} \in \mathscr{B} \) such that \( x \in B_{3} \) and \( B_{3} \subseteq B_{1} \cap B_{2} \). 
    \end{enumerate}
A subset \( U \) of \( X \) belongs to the topology  \( \mathscr{T} \) generated by \( \mathscr{B} \) if for each \( x \in U \), there is some \( B_{x} \in \mathscr{B} \) such that \( x \in B_{x} \) and \( B_{x} \subseteq U \).
\end{dfn}

\begin{theorem}
    The topology \( \mathscr{T} \) generated by \( \mathscr{B} \) is indeed a topology.
\end{theorem}
\begin{proof}
    \( \varnothing \in \mathscr{T} \) vacuously and \( X  \in \mathscr{T} \) by definition. Now let \( U_{\alpha \in A}  \) be an arbitrary collection of open sets. We wish to show that 
    \[ \bigcup_{\alpha \in A } U_{\alpha} \in \mathscr{T}. \]
Pick any \( x \in \bigcup_{\alpha \in A } U_{\alpha}  \). Then \( x \in U_{\beta} \) for at least one \( \beta \in A \). Since \( U_{\beta} \in \mathscr{T} \), there is some \( B \in \mathscr{B} \) for which \( x \in B \) and \( B \subseteq U_{\beta} \). So we can choose the same \( B \) to get 
\[ x \in B \subseteq  \bigcup_{\alpha \in A } U_{\alpha}.\]
Since this holds for all \( x \in \bigcup_{\alpha \in A } U_{\alpha}  \), this shows that \( \bigcup_{\alpha \in A } U_{\alpha} \in \mathscr{T} \).\\
Now let \( U_{1},\dots, U_{n} \) be a finite collection of open sets.  We wish to show that 
\[ \bigcap_{j=1}^{n} U_{j} \in \mathscr{T}. \]
We will proceed by induction. The case \( n=1 \) is trivial so our base case will be \( n=2 \). Suppose \( U_{1} \) and \( U_{2} \) are open sets. We wish to show that \( U_{1} \cap U_{2} \) is open. In other words, for any \( x \in U_{1} \cap U_{2} \), we wish to find a basis element that contains \( x \) and is contained in \( U_{1} \cap U_{2}. \) So pick any \( x \in U_{1} \cap U_{2} \). Then \( x \in U_{1} \) and \( x \in U_{2} \). Since \( U_{1} \) and \( U_{2} \) were already open, there exists \( B_{1}, B_{2} \in \mathscr{B} \) such that \( x \in B_{1} \subseteq U_{1} \) and \( x \in B_{2} \subseteq U_{2}\). Since \( \mathscr{B} \) is a basis, there is another basis element \( B_{3} \) containing \( x \) and contained in \( B_{1} \cap B_{2} \). It is clear that \( B_{3} \in U_{1} \cap U_{2} \), and hence\( U_{1} \cap U_{2} \) is open in \( \mathscr{T} \).\\
 Now for the inductive step, assume that we have shown that \( \bigcap_{j=1}^{n-1} U_{j} \) is open in \( \mathscr{T} \). We define \( V = \bigcap_{j=1}^{n-1} U_{j} \), which is open by the inductive hypothesis. Then \( V \cap U_{n} \) collapses to the base case The inductive step and the proof is completed.
\end{proof}

\begin{theorem}\label{thm:topology-equals-union-of-basis}
    Let \( \mathscr{B} \) be a basis for a topology \( \mathscr{T} \) on \( X \). Then \( \mathscr{T} \) equals the collection of all unions of elements of \( \mathscr{B} \).
\end{theorem}
\begin{proof}
    Suppose that \( \mathbf{B} \) is the collection of all unions of elements of \( \mathscr{B} \). We wish to show that \( \mathscr{T} = \mathbf{B} \).\\ \( \mathbf{B} \subseteq \mathscr{T} \) since each member of \( \mathscr{B} \) is open in \( \mathscr{T} \) and since \( \mathscr{T} \) is a topology, their unions are also members of \( \mathscr{T} \). \\ One the other hand, we pick any \( U \in \mathscr{T} \). Since \( \mathscr{B} \) generates the topology \( \mathscr{T} \), for every \( x \in U \), we may find \( B_{x} \in \mathscr{B} \) such that \( x \in B_{x} \subseteq U \). So we may write
    \[ U = \bigcup_{x \in U} B_{x}. \]
    This completes the proof.
\end{proof}

\begin{theorem}
    Let \( \left( X, \mathscr{T} \right) \) be a topological space. Suppose that \( \mathscr{C} \) is a collection of open sets of \( X \) such that for each open set \( U \) of  \( X \) and each element \( x \) of \( U \), there is a \( C \in \mathscr{C} \) such that \( x \in C \subseteq U \). This qualifies \( \mathscr{C} \) as a basis for the topology on \( X \).
\end{theorem}
\begin{proof}
    We need to first verify that the collection \( \mathscr{C} \) satisfies the conditions laid out in \cref{def: basis for a topology}. \\ 
    The first condition is easy, we simply take our open set to be \( X \) and the conditions of the statement of the theorem ensure that there is some element \( C \in \mathscr{C} \) for which \( x \in C \). \\ 
    For the second condition, pick \( C_{1}, C_{2} \in \mathscr{C} \) such that \( C_{1} \cap C_{2} \neq \varnothing \). Since \( C_{1} \) and \( C_{2} \) are open, it follows that \( C_{1} \cap C_{2} \) is also open. Hence for any \( x \in C_{1} \cap C_{3} \), there is some \( C_{3} \) for which \( x \in C_{3} \subseteq  C_{1} \cap C_{2} \). This shows that \( \mathscr{C} \) is a basis for a topology on \( X \). \\ \\ 
    All this shows is that \( \mathscr{C} \) generates \emph{some} topology \( \mathscr{T}' \). We want to show that \( \mathscr{T} = \mathscr{T}' \). If \( U \in \mathscr{T} \), then for each \( x \in U \), there is some \( C \in \mathscr{C} \) for which \( x \in C \subseteq U \). By the last sentence of \cref{def: basis for a topology}, \( U \) must belong to the topology generated \( \mathscr{C} \) and hence \( U \in \mathscr{T}' \). \\ 
    Conversely, if \( U' \in \mathscr{T}' \), by \cref{thm:topology-equals-union-of-basis}, 
    \[ U' = \bigcup_{x \in U'} C_{x} .\]
    Since each \( C_{x} \) is open in \( \mathscr{T} \), it follows that \( U' \) is also open in \( \mathscr{T} \) so \( U' \subseteq \mathscr{T} \). This completes the proof.
\end{proof}




\chapter{Connectedness and Compactness}


\section{Connectedness}
\begin{dfn}
    A topological space \( X \) is said to be \vocab{disconnected} if there exists non-empty open sets \( U \) and \( V \) of \( X \) such that 
    \[ U \cup V =X \]
    and 
    \[ U \cap V = \varnothing .\]
    A topological space is \vocab{connected} if it is not disconnected.
\end{dfn}

Here is a useful equivalent condition for connectedness. 

\begin{theorem}
    A topological space \( X \) is connected if and only if every continuous characteristic function \( \chi_{U}: X \to \left\{ 0,1 \right\} \) is constant.
\end{theorem}
\begin{proof}
    For a given subset \( U \subseteq X \), recall that the characteristic function \( \chi_{U} \) is defined by
    \[
        \chi_{U}(x) = \begin{cases}
            1 & \text{if } x \in U \\
            0 & \text{if } x \notin U
        \end{cases}.
    \]
    \( \Rightarrow \) Suppose there exists a continuous, non-constant characteristic function \( \chi_U \) for some \( U \subset X \). Then by definition,
    \[
        \chi_U^{-1}(1) = U \quad \text{and} \quad \chi_U^{-1}(0) = X - U.
    \]
    Since \( \chi_U \) is non-constant, both \( U \) and \( X - U \) are nonempty. And since \( \chi_U \) is continuous and \( \{0,1\} \) has the discrete topology, both \( U \) and \( X - U \) are open in \( X \).\\
    Thus, \( X \) is the union of two disjoint, nonempty open sets  so \( X \) is disconnected.\\
    \(\Leftarrow\) Now assume that \( X \) is disconnected. Then there exist disjoint, non-empty open sets \( U \) and \( V \) in \( X \) such that \( U \cup V = X \). \\
Consider the characteristic function \( \chi_U : X \to \{0,1\} \). Since \( U \) and \( V = X - U \) are both open, and \( \{0,1\} \) has the discrete topology, the preimages
\[
\chi_U^{-1}(1) = U \quad \text{and} \quad \chi_U^{-1}(0) = V
\]
are open. Hence, \( \chi_U \) is continuous. Moreover, it is non-constant, since both \( U \) and \( V \) are non-empty. This completes the proof.

\end{proof}

\begin{corollary}
    Suppose \( X \) is a connected topological space and \( Y \) is any topological space. If \( f: X \to Y \) is continuous, then the image \( f(X) \), equipped with the subspace topology from \( Y \), is connected.
\end{corollary}
\begin{proof}
   If \( f(X) \) contains exactly one point, the result holds trivially, since a one-point space is connected. So assume \( f(X) \) contains more than one point.\\
Suppose, for a contradiction, that \( f(X) \) is not connected. Then, by the previous theorem, there exists a non-constant continuous characteristic function \( \chi_U : f(X) \to \{0,1\} \) for some proper, nonempty clopen subset \( U \subset f(X) \).\\
Now consider the composition \( \chi_U \circ f : X \to \{0,1\} \). This map is continuous, since it is the composition of continuous functions. Moreover, since \( f \) is surjective onto \( f(X) \), the composition is also non-constant.\\
But this contradicts the connectedness of \( X \), since we've constructed a non-constant continuous characteristic function on \( X \). Hence, \( f(X) \) must be connected.
\end{proof}

\begin{exercise}
    Show that a finite set of points in a \( T_2 \) (Hausdorff) space is not connected.
\end{exercise}
\begin{solution}
    Let \( S = \{x_1, x_2, \dots, x_n\} \subseteq X \), where \( X \) is a \( T_2 \)  space and \( n \geq 2 \). We aim to show that \( S \), with the subspace topology inherited from \( X \), is not connected.\\
    Since \( X \) is Hausdorff, we can find pairwise disjoint open sets \( U_{x_1}, \dots, U_{x_n} \) in \( X \), each containing \( x_i \).\\
    Now consider the subspace topology on \( S \). Let
    \[
        U = U_{x_1} \cap S, \qquad V = \bigcup_{j = 2}^{n} \left( U_{x_j} \cap S \right).
    \]
    Then \( U \) and \( V \) are open in the subspace topology on \( S \), disjoint by construction, and cover \( S \), since each \( x_i \in S \) is contained in some \( U_{x_i} \).\\
    Thus, \( S = U \cup V \) is a separation of \( S \) into two nonempty, disjoint, open sets. Therefore, \( S \) is disconnected.
\end{solution}



