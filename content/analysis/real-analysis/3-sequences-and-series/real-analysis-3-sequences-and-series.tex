\section{Cauchy Sequences}

\begin{dfn}
    A sequence \( \left\{ x_{n} \right\} \) in a metric space \( \left( M, \rho \right) \) is said to be a \vocab{Cauchy sequence} if for all \( \epsilon >0 \), there exists an \( N  \in \mathbb{N}\) such that if \( m,n > N \) then 
    \[ \rho \left( x_{m}, x_{n} \right) < \epsilon \]
\end{dfn}

\begin{lemma}
    If \( \left\{ x_{n} \right\} \) is a Cauchy sequence in a metric space \( \left( M, \rho \right) \), then for all \( y \in M \), there exists an \( \epsilon_{y} \in \mathbb{R} \) such that every element of the sequence is contained in the closed ball \( \overline{B_{\rho} \left( y ; \epsilon_{y} \right)} \).
\end{lemma}
\begin{proof}
    
\end{proof}



\begin{dfn}\label{def:complete-metric-space}
    A metric space \( M \) is said to be \vocab{complete} if every Cauchy sequence converges to a point in \( M \).
\end{dfn}