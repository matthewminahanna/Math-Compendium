\begin{theorem}[Green's theorem]\label{theorem: Green's theorem}
    Let \( D \) be a simply connected region in \( \mathbb{R}^{2} \) whose boundary \( \partial D \) is a piecewise smooth, simple closed curve oriented counterclockwise. Let \( P \) and \( Q \) be continuously differentiable functions on an open set containing \( D \cup \partial D \). Then 
    \[ \oint_{\partial D} P \, dx + Q \, dy = \iint_{D} \left(\frac{\partial Q}{\partial x} - \frac{\partial P}{\partial y}\right) \, dA . \]
\end{theorem}
\begin{proof}
    
\end{proof}

\begin{example}\label{example: Green's theorem integral of x4xy over triangle}
    Suppose that \( D \) is the triangular region with vertices given by \( (0,0), (1,0) \) and \( (0,1) \). Apply \nameref{theorem: Green's theorem} to \( P = x^{4} \) and \( Q = xy \).\\ 
    We have 
    \begin{align*}
        \oint_{\partial D} P \, dx + Q \, dy &= \iint_{D}  \left(\frac{\partial Q}{\partial x} - \frac{\partial P}{\partial y}\right) \, dA  \\
        &= \int_{0}^{1} \int_{0}^{1-x}  \pdv{x} \left( xy \right) - \pdv{y} \left( x^{4} \right) \dd{y} \dd{x} \\
        &= \int_{0}^{1} \int_{0}^{1-x} y \dd{y} \dd{x}\\
        &= \frac{1}{2} \int_{0}^{1} y^{2} \eval_{0}^{1-x} \dd{x} \\
        &= \frac{1}{2} \int_{0}^{1} \left( 1-x \right)^{2} \dd{x} \\
        &= -\frac{1}{6} \left( 1-x \right)^{3} \eval_{0}^{1}
    \end{align*}
    So 
    \[ \boxed{ \oint_{\partial D} P \, dx + Q \, dy = \frac{1}{6}}.   \]
    Notice that his method is easier than explicitly computing the line integral as shown in \Cref{example: line integral of x4xy over triangle}.
\end{example}


If we know only the boundary of a region, it is natural to ask whether we can compute the area enclosed. Green's theorem provides a direct method.

Recall that the area of a region \(D\) is
\[
\text{Area}(D) = \iint_{D} 1 \, dA.
\]
If we choose functions \(P(x,y), Q(x,y)\) such that
\[
\pdv{Q}{x} - \pdv{P}{y} = 1,
\]
then Green's theorem tells us
\[
\text{Area}(D) = \oint_{\partial D} P\,dx + Q\,dy.
\]

For instance, taking \(P=0, Q=x\) yields
\[
\text{Area}(D) = \oint_{\partial D} x \, dy,
\]
while taking \(P=-y, Q=0\) gives
\[
\text{Area}(D) = -\oint_{\partial D} y \, dx.
\]
or taking \( P = - \frac{1}{2} y \), \( Q = \frac{1}{2} x \) yields
\[ \text{Area} \left( D  \right) = \frac{1}{2} \oint_{\partial D} x \dd{y} - y \dd{x} \]

\begin{example}
    Calculate the area of an ellipse given by the equation 
    \[
        \frac{x^{2}}{a^{2}} + \frac{y^{2}}{b^{2}} = 1.
    \]
    Let \(E\) denote the region bounded by the ellipse. 
    Choose \(P = -\tfrac{1}{2}y\), \(Q = \tfrac{1}{2}x\), so that 
    \[
        \text{Area}(E) 
        = \oint_{\partial E} P\,dx + Q\,dy
        = \frac{1}{2}\oint_{\partial E} x\,dy - y\,dx.
    \]
    The ellipse can be parametrized by 
    \[
        x = a\cos\theta, 
        \quad 
        y = b\sin\theta,
        \quad \theta \in [0,2\pi].
    \]
    Substituting, we obtain
    \[
        \text{Area}(E) 
        = \frac{1}{2} \int_{0}^{2\pi} 
        \Big( a\cos\theta\, d(b\sin\theta) 
        - b\sin\theta\, d(a\cos\theta)\Big).
    \]
    Simplifying,
    \[
        \text{Area}(E) 
        = \frac{1}{2}\int_{0}^{2\pi} 
        ab(\cos^{2}\theta + \sin^{2}\theta)\,d\theta
        = \frac{1}{2}\int_{0}^{2\pi} ab\,d\theta
        = \pi ab.
    \]
    Thus the area of the ellipse is 
    \[
        \boxed{ab \pi}.
    \]
\end{example}