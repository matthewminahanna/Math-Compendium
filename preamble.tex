\usepackage{makeidx}
\makeindex
\usepackage[utf8]{inputenc}
\usepackage[T1]{fontenc}


\usepackage{amsmath}
\usepackage{amsfonts}
\usepackage{amsthm}
\usepackage{mathtools}
\usepackage{amssymb}
\usepackage{enumitem}


\usepackage{tikz-cd,tikz}
    \usetikzlibrary{hobby, calc, intersections, decorations.markings, decorations.pathreplacing} %libraries
    \tikzset{>=latex}

\usepackage{xifthen}
\usepackage{pdfpages}
\usepackage{transparent}
\usepackage{import}
\usepackage{subfiles}
\usepackage[sfdefault,lining,book]{FiraSans}
\usepackage{FiraMono}
\usepackage{sansmathfonts}


\usepackage{faktor}
\DeclareMathSizes{10.95}{11.2}{7}{7}
%\boldmath
\everymath{\displaystyle}
\usepackage{hyperref}
\usepackage{nameref,cleveref}
\usepackage[a4paper, 
left=0.05in,
right=0.1in,
top=0.70in,
bottom=0.70in
]{geometry}
\usepackage[mathscr]{eucal}
\PassOptionsToPackage{usenames,svgnames,dvipsnames,table,xcdraw}{xcolor}
\usepackage{xcolor}

\usepackage[tight]{minitoc}
\mtcsetfont{parttoc}{chapter}{\sffamily\bfseries}
\mtcsetfont{parttoc}{section}{\footnotesize\upshape\mdseries}
\mtcsetfont{parttoc}{subsection}{\footnotesize\upshape\mdseries}
\mtcsetfont{parttoc}{subsubsection}{\footnotesize\upshape\mdseries}
%\mtcsetdepth{parttoc}{1}
\setcounter{parttocdepth}{3}
\renewcommand*{\partheadstartvskip}{\vspace*{-7em}}
\renewcommand*{\partheadendvskip}{}
%\noptcrule
\renewcommand\beforeparttoc{\noindent{}}
%\hspace{\fill}\rule{0.95\linewidth}{2pt}\hspace{\fill}
\doparttoc[n]

\usepackage{tikz}
\usepackage{tikz-cd}
\usetikzlibrary{intersections, angles, quotes, calc, positioning}
\usetikzlibrary{arrows.meta}
\usepackage{pgfplots}
\pgfplotsset{compat=1.13}
\tikzset{
    force/.style={thick, {Circle[length=2pt]}-stealth, shorten <=-1pt}
}
\makeatother
\usepackage{thmtools}
\usepackage[framemethod=TikZ]{mdframed}
\mdfsetup{skipabove=1em,skipbelow=0em}
\usepackage{wasysym}

\newenvironment{ignore}{}{}

\theoremstyle{definition}

\newenvironment{solutiont}
  {\begin{proof}[Solution]}
  {\end{proof}}

% Catppuccin Latte - Blue for theorems
\definecolor{theoremcolor}{HTML}{1e66f5}
\declaretheoremstyle[
    headfont=\bfseries\sffamily\color{theoremcolor!70!black}, bodyfont=\normalfont,
     postheadspace=\newline,
    mdframed={
        linewidth=4pt,
        rightline=false, topline=false, bottomline=false,
        linecolor=theoremcolor, backgroundcolor=theoremcolor!20,
    }
]{theoremst}

\declaretheorem[style=theoremst, name=Theorem,numberwithin=section]{theorem}
\declaretheorem[style=theoremst, name=Lemma,sibling=theorem]{lemma}
\declaretheorem[style=theoremst, name=Proposition,numberwithin=section]{proposition}
\declaretheorem[style=theoremst, name=Corollary, numbered=no]{corollary}

% Catppuccin Latte - Lavender for ideas
\definecolor{ideacolor}{HTML}{7287fd}
\declaretheoremstyle[
    headfont=\bfseries\sffamily\color{ideacolor!70!black}, bodyfont=\normalfont,
    mdframed={
        linewidth=2pt,
        rightline=false, topline=false, bottomline=false,
        linecolor=ideacolor, backgroundcolor=ideacolor!20,
    }
]{ideast}

\declaretheorem[style=ideast, name=Idea]{idea}

\declaretheoremstyle[
	numbered = no,
    headfont=\bfseries\sffamily\color{theoremcolor!70!black}, bodyfont=\normalfont,
    postheadspace=\newline,
    mdframed={
        linewidth=2pt,
        linewidth=4pt,
        skipabove = 0pt,
        rightline=false, topline=false, bottomline=false,
        linecolor=theoremcolor, backgroundcolor=theoremcolor!10,
    },
    qed = $\blacksquare$
]{proofst}

\declaretheorem[style=proofst, name=Proof]{replacementproof}
\renewenvironment{proof}[1][\proofname]{\begin{replacementproof}}{\end{replacementproof}}

% Catppuccin Latte - Red for definitions
\definecolor{definitioncolor}{HTML}{d20f39}
\declaretheoremstyle[
    headfont=\bfseries\sffamily\color{definitioncolor!70!black}, bodyfont=\normalfont,
    postheadspace=\newline,
    mdframed={
        linewidth=3pt,
        rightline=false, topline=false, bottomline=false,
        linecolor=definitioncolor, backgroundcolor=definitioncolor!20,
    }
]{definitionst}

\declaretheorem[style=definitionst, name=Definition, numberwithin=section]{dfn}

% Catppuccin Latte - Teal for examples
\definecolor{examplecolor}{HTML}{179299}
\declaretheoremstyle[
    headfont=\bfseries\sffamily\color{examplecolor!70!black}, bodyfont=\normalfont,
    postheadspace=\newline,
    mdframed={
        linewidth=3pt,
        rightline=false, topline=false, bottomline=false,
        linecolor=examplecolor, backgroundcolor=examplecolor!20,
    }
]{examplest}

\declaretheorem[style=examplest, name=Example, numberwithin=section]{example}

% Catppuccin Latte - Lavender for exercises
\definecolor{exercisecolor}{HTML}{7287fd}
\declaretheoremstyle[
    headfont=\bfseries\sffamily\color{exercisecolor!70!black}, bodyfont=\normalfont,
    postheadspace=\newline,
    mdframed={
        linewidth=3pt,
        rightline=false, topline=false, bottomline=false,
        linecolor=exercisecolor, backgroundcolor=exercisecolor!20,
    }
]{exercisest}

\declaretheorem[style=exercisest, name=Exercise, numberwithin=section]{exercise}


\declaretheoremstyle[
    numbered= no,
    headfont=\bfseries\sffamily\color{exercisecolor!70!black}, bodyfont=\normalfont,
    postheadspace=\newline,
    mdframed={
        linewidth=3pt,
        skipabove = 0pt,
        rightline=false, topline=false, bottomline=false,
        linecolor=exercisecolor, backgroundcolor=exercisecolor!10,
    },
qed = $\newmoon$
]{solutionst}

\declaretheorem[style=solutionst, name=Solution]{solution}
\renewenvironment{solutiont}[1][\solutionstname]{\vspace{-20pt}\begin{solution}}{\end{solution}}

% Catppuccin Latte - Peach for recall
\definecolor{recallcolor}{HTML}{fe640b}
\declaretheoremstyle[
    headfont=\bfseries\sffamily\color{recallcolor!70!black}, bodyfont=\normalfont,
    numbered = no,
    mdframed={
        linewidth=3pt,
        rightline=false, topline=false, bottomline=false,
        linecolor=recallcolor, backgroundcolor=recallcolor!10,
    }
]{recallst}

\declaretheorem[style=recallst, name=Recall]{recall}

% Catppuccin Latte - Maroon for vocabulary
\definecolor{vocabb}{HTML}{e64553}

\newcommand{\vocab}[1]{\textbf{\color{vocabb} #1}}

\newcommand{\comment}[1]{%
  \text{\phantom{(#1)}} \tag{#1}
}

\usepackage{physics}

\DeclareMathOperator*{\str}{\ast}

% Catppuccin Latte - Subtext0 for section colors
\definecolor{sectcolor}{HTML}{6c6f85}

\renewcommand*{\sectionformat}{\color{sectcolor}\S\firaextrabold\thesection\autodot\enskip}
\renewcommand*{\subsectionformat}{\color{sectcolor}\S\thesubsection\autodot\enskip}



\addtokomafont{chapterprefix}{}{\raggedleft}
\RedeclareSectionCommand[beforeskip=0.5em]{chapter}
\renewcommand*{\chapterformat}{%
\mbox{\scalebox{1.5}{\chapappifchapterprefix{\nobreakspace}}%
\scalebox{1.5}{\color{sectcolor}\firaheavy\thechapter\autodot}\enskip}}


\addtokomafont{partprefix}{\firaheavy}
\renewcommand*{\partformat}{}


\setkomafont{chapter}{\Huge\firaheavy}
\setkomafont{section}{\Large\firaextrabold}
\setkomafont{subsection}{\large\firasemibold}
% for smarter referencing (optional but helpful)

% Custom equation numbering format: Part.Chapter.Section.Equation
\renewcommand{\theequation}{\thesection.\arabic{equation}}

% Reset equation counter at each new section
\makeatletter
\@addtoreset{equation}{section}
\makeatother

\usepackage{subcaption}

% *** quiver ***
% A package for drawing commutative diagrams exported from https://q.uiver.app.
%
% This package is currently a wrapper around the `tikz-cd` package, importing necessary TikZ
% libraries, and defining new TikZ styles for curves of a fixed height and for shortening paths
% proportionally.
%
% Version: 1.6.0
% Authors:
% - varkor (https://github.com/varkor)
% - AndréC (https://tex.stackexchange.com/users/138900/andr%C3%A9c)
% - Andrew Stacey (https://tex.stackexchange.com/users/86/andrew-stacey)

\NeedsTeXFormat{LaTeX2e}
\ProvidesPackage{quiver}[2025/09/20 quiver]

% `tikz-cd` is necessary to draw commutative diagrams.
\RequirePackage{tikz-cd}
% `amssymb` is necessary for `\lrcorner` and `\ulcorner`.
\RequirePackage{amssymb}
% `calc` is necessary to draw curved arrows.
\usetikzlibrary{calc}
% `pathmorphing` is necessary to draw squiggly arrows.
\usetikzlibrary{decorations.pathmorphing}
% `spath3` is necessary to draw shortened edges.
\usetikzlibrary{spath3}

% A TikZ style for curved arrows of a fixed height, due to AndréC.
\tikzset{curve/.style={settings={#1},to path={(\tikztostart)
    .. controls ($(\tikztostart)!\pv{pos}!(\tikztotarget)!\pv{height}!270:(\tikztotarget)$)
    and ($(\tikztostart)!1-\pv{pos}!(\tikztotarget)!\pv{height}!270:(\tikztotarget)$)
    .. (\tikztotarget)\tikztonodes}},
    settings/.code={\tikzset{quiver/.cd,#1}
        \def\pv##1{\pgfkeysvalueof{/tikz/quiver/##1}}},
    quiver/.cd,pos/.initial=0.35,height/.initial=0}

% A TikZ style for shortening paths without the poor behaviour of `shorten <' and `shorten >'.
\tikzset{between/.style n args={2}{/tikz/execute at end to={
    \tikzset{spath/split at keep middle={current}{#1}{#2}}
}}}

% TikZ arrowhead/tail styles.
\tikzset{tail reversed/.code={\pgfsetarrowsstart{tikzcd to}}}
\tikzset{2tail/.code={\pgfsetarrowsstart{Implies[reversed]}}}
\tikzset{2tail reversed/.code={\pgfsetarrowsstart{Implies}}}
% TikZ arrow styles.
\tikzset{no body/.style={/tikz/dash pattern=on 0 off 1mm}}

\endinput

%end quiver


\usepackage{tocbasic}
\makeatletter
\renewcommand*{\l@part}[2]{%
  \ifnum \scr@tocdepth >-2\relax
    \addpenalty\@secpenalty
    \addvspace{1.0em plus\p@}%
    \begingroup
      \parindent \z@
      \rightskip \@tocrmarg
      \parfillskip -\rightskip
      \leavevmode
      \advance\leftskip\@tempdima
      \hskip -\leftskip
      \llap{\thepart\enskip \quad}% ↠ADD SPACE HERE
      #1\nobreak\hfil\nobreak\hb@xt@\@pnumwidth{\hss #2}%
      \par
    \endgroup
  \fi}
\makeatother

\usepackage{xpatch}

