\subsection{The Integral Test and the Divergence Test}
\subsection{Direct Comparison Test}

Suppose we are given two positive sequences, \( a_{n} \) and \( b_{n} \) with the property that if \( a_{n} \le  b_{n}\) for all \( n \ge N \) for some \( N \in \mathbb{N} \). If \({\displaystyle \sum_{k =1}^{\infty} b_{n}} \) converges, then \( {\displaystyle \sum_{k =1}^{n} a_{n}} \) also converges. \\ 

Similarly, if we are given two positive sequences, \( a_{n} \) and \( b_{n} \) with the property that if  \( a_{n} \; {\color{red} \ge}\;  b_{n} \) for all \( n \ge N \) for some \( N \in \mathbb{N} \). If \({\displaystyle \sum_{k =1}^{\infty} b_{n}} \) diverges, then \( {\displaystyle \sum_{k =1}^{n} a_{n}} \) also diverges.

\subsection{The Limit Comparison Test} 

Suppose we are given two positive sequences, \( a_{n} \) and \( b_{n} \) such that 
\[ \lim_{n \to \infty} \frac{a_{n }}{b_{n}} = L < \infty .\]
Then \( {\displaystyle \sum_{k =1}^{\infty} a_{n}} \) and \( {\displaystyle \sum_{k =1}^{\infty} b_{n}} \) both converge or both diverge.

\subsection{Alternating Series and Absolute Convergence} 

\begin{dfn}
    A series is \vocab{alternating} if every term is of the form \( (-1)^{n}a_{n} \) or \( (-1)^{{n+1}}a_{n} \).
\end{dfn}

Suppose that \( A= \sum_{k =1}^{\infty} \left( -1  \right)^{k}a_{k} \) is an alternating series. If 
\begin{enumerate}[label=\textbf{\roman*)}]
    \item \( a_{n+1} \le a_{n} \) for all \( n \ge N \), \( N \in \mathbb{N} \)
    \item \( \lim_{n \to \infty} a_{n} = 0 \)
\end{enumerate}
Then the series converges.

\begin{dfn}
    A series \( {\displaystyle \sum_{k =1}^{\infty} a_{k}} \) is \vocab{absolutely convergent} or \vocab{converges absolutely} if 
    \[ \sum_{k =1}^{\infty} \abs{a_{k}} \text{ converges.} \]
\end{dfn}

\begin{lemma}
    If a series converges absolutely, then it converges.
\end{lemma}
\begin{proof}
    Suppose that the series \( \sum_{k=1}^{\infty} a_{k} \) converges absolutely. 
    Then the series \( \sum_{k=1}^{\infty} \lvert a_{k} \rvert \) converges. 
    Since \( \sum_{k=1}^{\infty} -\lvert a_{k} \rvert = - \sum_{k=1}^{\infty} \lvert a_{k} \rvert \), it also converges.
    
    For each \( n \in \mathbb{N} \), define the partial sums
    \[
        S_{n} = \sum_{k=1}^{n} a_{k}
        \quad\text{and}\quad
        T_{n} = \sum_{k=1}^{n} \lvert a_{k} \rvert .
    \]
    Since \( -\lvert a_{k} \rvert \le a_{k} \le \lvert a_{k} \rvert \) for all \( k \),
    we have
    \[
        -T_{n} \le S_{n} \le T_{n} \quad \text{for all } n.
    \]
    As \( T_{n} \to \sum_{k=1}^{\infty} \lvert a_{k} \rvert \),
    it follows that \( -T_{n} \to -\sum_{k=1}^{\infty} \lvert a_{k} \rvert \).
    Hence \( S_{n} \) is squeezed between two convergent sequences and therefore converges.
    Thus \( \sum_{k=1}^{\infty} a_{k} \) converges.
\end{proof}



\begin{dfn}
    If a series \( {\displaystyle \sum_{k =1}^{\infty} a_{n}} \) converges but \( \displaystyle \sum_{k =1}^{\infty} \abs{a_{n}} \) does not, we say that it is \vocab{conditionally convergent} or \vocab{converges conditionally}.
\end{dfn}

\subsection{The Root and Ratio Tests}
 \subsubsection{The Ratio Test}
\( \bullet \) If 
\[ \lim_{n \to \infty} \abs{\frac{a_{n+1 }}{a_{n}}} <1 \]
then the series \( \displaystyle \sum_{k =1}^{\infty} a_{n} \) converges absolutely. \\ 

\( \bullet \) If 
\[ \lim_{n \to \infty} \abs{\frac{a_{n+1 }}{a_{n}}} >1 \]
then the series \( \displaystyle \sum_{k =1}^{\infty} a_{n} \) diverges.

\( \bullet \) If 
\[ \lim_{n \to \infty} \abs{\frac{a_{n+1 }}{a_{n}}} =1 \]
then the ratio test provides no information about the series \( \displaystyle \sum_{k =1}^{\infty} a_{n} \) .

\begin{exercise}
    Determine if 
    \[ \sum_{k =1}^{\infty} \frac{n}{5^{n}} \] converges or diverges. 
\end{exercise}
\begin{solution}
       Let \( a_{n} \) denote the \( n \)-th term in the given series. Then applying the ratio test, we have 
       \begin{align*}
        \frac{\abs{a_{n+1}}}{a_{n}} &=   \frac{{\displaystyle \frac{n+1}{5^{n+1}}}}{{\displaystyle \frac{n}{5^{n}}}} \\ 
        &= \frac{n+1}{5^{n+1}} \cdot \frac{5^{n}}{n} \\
        &= \frac{n+1}{5^{n} \cdot 5}  \cdot \frac{5^{n}}{n} \\
        &=\frac{n+1}{\Ccancel[DeepRed]{5^{n}}  \cdot 5}  \cdot \frac{\Ccancel[DeepRed]{5^{n}}}{n}\\
        &= \frac{1}{5} \cdot \frac{n+1}{n}
       \end{align*}
     Now taking the limit, we have 
     \[ \lim_{n \to \infty} \frac{1}{5} \frac{n+1}{n} = \frac{1}{5} <1 \] 
     so 
     \[ \boxed{ \sum_{k =1}^{\infty} \frac{n}{5^{n}} \text{ converges absolutely.}} \]
\end{solution}

\begin{exercise}
     Determine if 
    \[ \sum_{k =1}^{\infty} \left( -1 \right)^{n-1} \frac{3^{n}}{2^{n}n^{3}} \] converges or diverges. 
\end{exercise}
\begin{solution}
    Let \( a_{n} \) denote the \( n \)-th term in the given series. Then applying the ratio test, we have 
\begin{align*}
    \frac{\abs{a_{n+1}}}{\abs{a_{n}}} &=  \frac{{\displaystyle \frac{3^{n+1}}{2^{n+1} (n+1)^{3}}}}{{\displaystyle \frac{3^{n}}{2^{n}n^{3}}}} \\
    &= \frac{3^{n+1}}{2^{n+1} (n+1)^{3}} \cdot \frac{2^{n} n^{3}}{3^{n}} \\
    &= \frac{3 \cdot 3^{n}}{2 \cdot 2^{n} \left( n+1 \right)^{3}} \cdot \frac{2^{n} n^{3}}{3^{n}}\\
    &=\frac{3 \cdot \Ccancel[DeepRed]{3^{n}} }{2 \cdot  \Ccancel[AmberOrange]{2^{n}}  \left( n+1 \right)^{3}} \cdot \frac{\Ccancel[AmberOrange]{2^{n}}  n^{3}}{\Ccancel[DeepRed]{3^{n}}} \\
    &= \frac{3}{2} \frac{n^{3}}{\left( n+1 \right)^{3}}
\end{align*}
 Now taking the limit, we have 
     \[ \lim_{n \to \infty}\frac{3}{2} \cdot \frac{n^{3}}{\left( n+1 \right)^{3}} = \frac{3}{2} >1 \] 
     so 
     \[ \boxed{\sum_{k =1}^{\infty} \left( -1 \right)^{n-1} \frac{3^{n}}{2^{n}n^{3}} \text{ diverges.}} \]
\end{solution}



\begin{exercise}
    Determine if the following series converges or diverges
    \[ \sum_{n =1}^{\infty} \frac{ \prod_{k=1}^{n } 2k}{n!} \text{ .}\]
\end{exercise}
\begin{solution}
    Let \( a_{n} \) denote the \( n \)-th term in the given series. Then applying the ratio test, we have 
    \begin{align*}
        \frac{\abs{a_{n+1}}}{\abs{a_{n}}} &=   \frac{{\displaystyle  \frac{\prod_{k =1}^{n+1} 2k}{\left( n+1  \right)!}} }{ {\displaystyle \frac{\prod_{k =1}^{n} 2k}{n!}}} \\
        &= \frac{\prod_{k =1}^{n+1} 2k}{\left( n+1  \right)!} \cdot \frac{n!}{\prod_{k =1}^{n} 2k} \\
        &= \frac{\left( \prod_{k =1}^{n} 2k  \right) \cdot 2(n+1)}{n! \left( n+1 \right)} \cdot \frac{n!}{\prod_{k =1}^{n} 2k} \\
        &= \frac{\Ccancel[DeepRed]{\left( \prod_{k =1}^{n} 2k  \right) }  \cdot 2(n+1) }{\Ccancel[AmberOrange]{n!} \cdot \left( n+1 \right) } \cdot \frac{\Ccancel[AmberOrange]{n!}}{\Ccancel[DeepRed]{ \prod_{k =1}^{n} 2k  }} \\
        &= \frac{2 (n+ 1 )}{n+1} \\
        &=2
    \end{align*}
    Taking the limit 
    \[ \lim_{n \to \infty} 2 =2 >1\]
    so 
    \[ \boxed{ \sum_{n =1}^{\infty} \frac{ \prod_{k=1}^{n } 2k}{n!} \text{ diverges .}} \]
\end{solution}

\subsubsection{The Root Test}

\( \bullet \) If 
\[ \lim_{n \to \infty} \sqrt[n]{\abs{a_{n}}} <1 ,\] then the series \( \sum_{k =1}^{\infty} a_{n} \) is absolutely convergent. 

\( \bullet \) If \[ \lim_{n \to \infty} \sqrt[n]{\abs{a_{n}}} >1 ,\] then the series \( \sum_{k =1}^{\infty} a_{n} \) is divergent. 

\( \bullet \) If \[ \lim_{n \to \infty} \sqrt[n]{\abs{a_{n}}} =1 ,\] then no information about the series \( \sum_{k =1}^{\infty} a_{n} \) can be extracted from the root test.
