\begin{dfn}\label{def:polynomial}
    A \vocab{(real) polynomial} is a function of the form:
    \[ p(x) = a_{0} + a_{1}x+ a_{2}x^{2} + \cdots + a_{n}x^{n} = \sum_{k=0}^{n}a_{k}x^{k} \] 
    where the \vocab{coefficients} \( a_{k} \in \mathbb{R} \). The \vocab{degree} of a polynomial is the largest \( n \) for which \( a_{n} \neq 0 \).
    
    A polynomial of degree: 
    \begin{enumerate}[label=\textbf{\arabic*)},start=0]
    \item is called a \vocab{constant function}. (This is literally just the function that sends every \( x \) to the same number.) 
    \item is called a \vocab{linear function}. 
    \item is called a \vocab{quadratic function}.
    \item is called a \vocab{cubic function}. 
    \item is called a \vocab{quartic function}. 
    \item is called a \vocab{quintic function}.   
    \end{enumerate}
    Those are the polynomials with special names. Any polynomial can be referred to as a \vocab{polynomial of degree \( n \)}.
\end{dfn}