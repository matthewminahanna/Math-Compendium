A useful source of examples of Lie groups involves the quaternions. We will dedicate this section to defining them and exploring some of their properties.

\subsection{Basic Definition}
\begin{dfn}
    The \vocab{quaternions}, denoted \( \mathbb{H} \), are defined to be the associative real algebra
    \[
        \mathbb{H} = \left\{ a + b\vb{i} + c\vb{j} + d\vb{k} \ \middle\vert\  a, b, c, d \in \mathbb{R} \right\},
    \]
    where the fundamental quaternion units satisfy
    \[
        \vb{i}^{2} = \vb{j}^{2} = \vb{k}^{2} = \vb{i} \vb{j} \vb{k} = -1.
    \]
\end{dfn}

\begin{lemma}
    The quaternion units obey the following multiplication rules:
    \begin{align*}
        \vb{i}\vb{j} &= \vb{k}, & \vb{j}\vb{i} &= -\vb{k}, \\
        \vb{j}\vb{k} &= \vb{i}, & \vb{k}\vb{j} &= -\vb{i}, \\
        \vb{k}\vb{i} &= \vb{j}, & \vb{i}\vb{k} &= -\vb{j}.
    \end{align*}
\end{lemma}

\begin{proof}
    We will just show one of these equalities. The rest are very similar.  \\
    We have 
    \[ \vb{i}\vb{j}\vb{k}=-1 \]
    Multiplying on the right by \( \vb{k} \) gives us 
    \[ \vb{i}\vb{j}\vb{k}\vb{k}=-\vb{k} \]
    \[ -1\vb{i}\vb{j}=-\vb{k} \text{ since } \vb{k}^{2}=-1 \]
    \[ \vb{i}\vb{j}=\vb{k} \]
\end{proof}
The products of \( \vb{i}, \vb{j} \) and \( \vb{k} \) can be memorized with the following diagram:
\[\begin{tikzcd}
	& i \\
	\\
	k && j
	\arrow[curve={height=-6pt}, from=1-2, to=3-3]
	\arrow[curve={height=-6pt}, from=3-1, to=1-2]
	\arrow[curve={height=-6pt}, from=3-3, to=3-1]
\end{tikzcd}\]

\begin{dfn}
    The \vocab{real quaternions} is the set 
    \[ \Re (\mathbb{H}) = \left\{  a =a1 \in \mathbb{H} \middle\vert a \in \mathbb{R}\right\}. \]
    The \vocab{imaginary quaternions}  are similarly defined 
    \[ \Im \left( \mathbb{H} \right) = \left\{ b \vb{i} + c \vb{j} + d \vb{k} \middle\vert b,c,d \in \mathbb{R} \right\} \]
\end{dfn}

\begin{dfn}
    The \vocab{conjugate} of a quaternion \( w = a + b \vb{i} + c \vb{j} + d \vb{k} \) is 
    \[ \overline{w}= a - b \vb{i} - c \vb{j} - d\vb{k} \]
    and the \vocab{norm squared} is 
    \[ \norm{w}^{2} =a^{2}+b^{2}+c^{2}+d^{2} \]
\end{dfn}

\begin{lemma}
    For a quaternion \( w = a + b\vb{i} + c\vb{j} + d\vb{k} \), we have:
    \[
    \norm{w}^2 = w \overline{w} = \overline{w} w.
    \]
\end{lemma}

\begin{proof}
    Let \( w = a + b\vb{i} + c\vb{j} + d\vb{k} \). Then the conjugate of \( w \) is:
    \[
    \overline{w} = a - b\vb{i} - c\vb{j} - d\vb{k}.
    \]
    Now compute:
    \begin{align*}
        w \overline{w} &= (a + b\vb{i} + c\vb{j} + d\vb{k})(a - b\vb{i} - c\vb{j} - d\vb{k}) \\
        &= a^2 - ab\vb{i} - ac\vb{j} - ad\vb{k} 
        + ab\vb{i} - b^2\vb{i}^2 - bc\vb{i}\vb{j} - bd\vb{i}\vb{k} \\
        &\quad + ac\vb{j} - bc\vb{j}\vb{i} - c^2\vb{j}^2 - cd\vb{j}\vb{k} \\
        &\quad + ad\vb{k} - bd\vb{k}\vb{i} - cd\vb{k}\vb{j} - d^2\vb{k}^2
    \end{align*}

    Substituting in:
    \begin{align*}
        w \overline{w} &= a^2 + b^2 + c^2 + d^2 \\
        &\quad - bc\vb{k} - bd(-\vb{j}) - bc(-\vb{k}) - cd\vb{i} \\
        &\quad + bd\vb{j} + cd(-\vb{i}) \\
        &= a^2 + b^2 + c^2 + d^2 \quad \text{(all imaginary terms cancel out)}.
    \end{align*}

    Therefore:
    \[
    w \overline{w} = \norm{w}^2 = a^2 + b^2 + c^2 + d^2.
    \]
    Similarly, \( \overline{w}w = \norm{w}^2 \), since quaternion norm is preserved under conjugation order.

    \qedhere
\end{proof}


\subsection{Quaternionic Matrices}

Since any quaternion \( q \in \mathbb{H} \) can be written uniquely as \( q=q_{1}+\vb{j}q_{2} \) for \( q_{1}, q_{2} \in \mathbb{C} \), any \( n \times n \) quaternionic matrix \( A \in \mathrm{Mat} \left( \mathbb{H}, n \times n \right) \) can be written uniquely as \( A = A_{1}+\vb{j}A_{2} \) for \( A_{1}, A_{2} \in \mathrm{Mat} \left( \mathbb{C}, n \times n \right) \). This allows us to define the following. 

\begin{dfn}
    Let \( A \in \mathrm{Mat} \left( \mathbb{H}, n \times n \right) \) and \( A_{1}, A_{2} \in \mathrm{Mat} \left( \mathbb{C}, n \times n \right) \) such that \( A = A_{1}+\vb{j}A_{2} \). The \vocab{adjoint} of \( A \) is the matrix \( \chi_{A} \in \mathrm{Mat} \left( \mathbb{C} , 2n \times 2n\right) \) defined to be 
    \[ \chi_{A} = \begin{pmatrix}
    A_{1}  & - \overline{A_{2}}\\
    A_{2} & \overline{A_{1}}
    \end{pmatrix} \]
\end{dfn}

